%%%%%%%%%%%%%%%%%%%%%%%%%%%%%%%%%%%%%%%%%%%%%%%%%%%%%%%%%%%
%%%%%%%%%%%%%%%%%%%%%%%%%%%%%%%%%%%%%%%%%%%%%%%%%%%%%%%%%%%
% 这里用来暂时储存一些乱七八糟的杂稿,
% 笔者暂时没有理解的内容,或者是零散、不成体系的东西
%%%%%%%%%%%%%%%%%%%%%%%%%%%%%%%%%%%%%%%%%%%%%%%%%%%%%%%%%%%
%%%%%%%%%%%%%%%%%%%%%%%%%%%%%%%%%%%%%%%%%%%%%%%%%%%%%%%%%%%


%%%%%%%%%%%%%%%%%%%%%%%%%%%%%%%%%%%%%%%%%%%%%%%%%%%
% 第五周 周一
% 以下内容位于 位于Kontsevich量子化公式之后
% 李思讲的“故事”:大概是在辛流形的特殊情况下,
%  Kontesvich量子化与Atiyah-Singer指标定理有联系
\section{指标定理}
index theorem:

Symplectic case: $(X,\omg)$ symplectic mfd,
$P=\omg^{-1}$

Fedosov: a single geometric way
to get deformation quantization and the classification by
$$\omg_{\hbar}=-\omg+\hbar\omg_1+\hbar^2\omg_2+\cdots$$
where
$$\omg_i\in H^2(X,\bbR)$$

\begin{definition}
 the trace map
$$\tau: c^{\infty}(X)\to \bbR\Ls{\hbar}$$
satisfying
$$\tau(f\star g)=\tau(g\star f)$$
$$\Longrightarrow \tau\in \HH^0(C^{\infty}(x)\fps{\hbar})$$

Fedosov/Nest Tsygar: canonical trace

such that
$$\tau(f)=\frac{1}{(2\pi\hbar)^n}
\int_X
  \frac{\omg^n}{n!}
  (f+O(\hbar))$$
\end{definition}

注意记号:
$$\HH^i(A):=H^i(A,A^*)$$


\begin{thm}(Fedosov/Nest-Tsygan)

$$\tau(1)=\int_X e^{w_{\hbar}/\hbar}\Ahat(X)$$
(algebraic index thm)
where $\Ahat$ is the genus of $X$,
$$\Ahat(X):=
\det\nolimits^{1/2}\frac{R/2}{\sinh(R/2)}$$
$R$ is the curvature of $T_X$.
\end{thm}

Goal: explain both 1 and 2 from

non-commutative Hochschild theory

quantum- field theory 

%%%%%%%%%%%%%%%%%%%%%%%%%%%%%%%%%%%%%%%%%%%%%%%%%%%
%%%%%%%%%%%%%%%%%%%%%%%%%%%%%%%%%%%%%%%%%%%%%%%%%%%
%%%以上内容位于【phi^4场论、重整化】之后%%%%%%%%%%%
\begin{example}Quantum mechanics (QFT in dimension 1)

field:
$$\gamma:\bbR\to\text{a space}$$
$$t\mapsto \gamma(t)$$

$$\mcalS[\gma]=\frac{1}{2}\int\bbR\td t|\gma'(t)|^2$$
is called energy...

\end{example}

"Physics fact": For $x,y\in\bbR^d$,consider
$$\int_{\gma:[0,t]\to\bbR^d\atop\gma(0)=x,\quad\gma(t)=y}
[D\gma]e^{-\mcalS[\gma]/\hbar}
=h_t(x,y)$$

"the second story":first order formula,
$$\mcalS\to\mcalS[\cdots]$$

还是下一节课再讲吧。。。