\chapter{Hochschild理论}

\section{结合代数的双模、余中心}

我们需要\textbf{代数拓扑}、\textbf{同调代数}的预备知识,
并且采用同调代数的标准术语、记号,
诸如链复形、上同调、导出函子等等。
首先介绍基本的记号与概念。

在本课,我们给定一个特征$0$的含幺交换环$K$(例如一个域),
考虑含幺结合$K$-代数$A$(注意$A$未必是交换代数),
并且$A$作为交换环$K$上的模是投射模(projective module)。
\index{projective module\kong 投射模}
$A$的$K$-代数结构给出如下$K$-模同态:
\begin{eqnarray*}
A\otimes_KA        &\rightarrow& A\\
(a_1,a_2)          &\mapsto    & a_1a_2
\end{eqnarray*}
由$A$的结合性,$(a_1a_2)a_3=a_1(a_2a_3)$
对$A$中任意元素$a_1,a_2,a_3$成立.

对于含幺结合$K$-代数$A$,
回顾$A$的\textbf{反代数}(opposite algebra)
\index{opposite algebra\kong 反代数}
$A\op$.反代数$A\op$作为$K$-模与$A$完全相同,记号如下:
\begin{eqnarray*}
\text{id}:A &\rightarrow & A\op\\
x&\mapsto&x\op
\end{eqnarray*}
但是$A\op$具有与$A$“相反”的乘法,
具体地,对于$A\op$中的元素$x\op,y\op$,
成立
$$x\op y\op:=(yx)\op$$

\begin{definition}
对于含幺结合$K$-代数$A$,我们定义$K$-代数$A^e$为
$$A^e:=A\ten_K A\op$$
即$A$与$A\op$的$K-$代数张量积。
\end{definition}

容易验证对于任何两个含幺结合$K$-代数$A,B$,总有
$$(A\ten_K B)\op=A\op\ten_{K}B\op$$
从而容易得到
$$(A\op)^e=(A^e)\op$$

对于$K-$代数$A$,回顾
\textbf{双$A-$模}($A$-bimodule)的概念如下:

\begin{definition}对于$K$-代数$A$,
双$A$-模是指如下资料:

(1)$K$-模$M$;

(2)$A$在$M$上的左、右$K$-线性作用,

并且满足相容性:$(a.m).b=a.(m.b)$对任意$m\in M$以及$a,b\in A$成立。
\end{definition}

例如,$A$本身自然有双$A$-模结构,
$A$在其上的左、右作用即为左乘、右乘。
再比如$K$-模张量积$A\ten_KA$具有如下双$A$-模结构:
$$b.(a_1\ten a_2):=(ba_1)\ten a_2$$
$$(a_1\ten a_2).b:=a_1\ten(a_2b)$$
其中$a_1,a_2,b\in A$.

我们不再回顾左模、右模的概念了,
也不去回顾右模与左模的平衡张量积。

\begin{prop}设$M$为双$A$-模,

(1)$M$可自然地视为左$A^e$-模:
$$(a_1\ten a_2\op).m=a_1.m.a_2$$

(2)$M$可自然地视为右$A^e$-模:
$$m.(a_1\ten a_2\op)=a_2.m.a_1$$
反之,左(右)$A^e$-模也可视为双$A$-模。
\end{prop}

\begin{proof}
容易验证。
\end{proof}

特别地如果$M,N$都是双$A$-模,那么考虑平衡张量积
$M\ten_{A^e}N$,它的双$A$-模结构具体如下:
$$a.(m\ten n)=(a.m)\ten n=m\ten(n.a)$$
$$(m\ten n).b=m\ten (n.b)=(b.m)\ten n$$
对于任何$m\in M,n\in N,a,b\in A$成立。

\begin{definition}(余中心 cocenter)
\index{cocenter\kong 余中心}
对于双$A$-模$M$,称双$A$-模
$$M\ten_{A^e}A$$
为$M$的\textbf{余中心}(cocenter)。
\end{definition}

容易看出,对任意的$m\in M,a\in A$,
在余中心$M\ten_{A^e}A$当中,成立
$$(m.a)\ten 1=m\ten(a.1)=m\ten a=m\ten(1.a)=(a.m)\ten 1$$
从而$(m.a-a.m)\ten 1=0$.
事实上,$M$的余中心具有如下结构:

\begin{prop}
对于双$A$-模$M$,则有如下双$A$-模同构
$$M\ten_{A^e}A\cong M/\{(m.a-a.m)|a\in A,m\in M\}$$
\label{双模的余中心的结构prop}
\end{prop}

\begin{proof}
考虑如下的双$A$-模链复形
$$\p\downdot:A\ten A\ten A\ra A\ten A\ra A\ra 0$$
其中
\begin{eqnarray*}
\p:a_1\ten a_2\ten a_3 &\mapsto& a_1a_2\ten a_3-a_1\ten a_2a_3\\
a_1\ten a_2&\mapsto&a_1a_2
\end{eqnarray*}
容易验证$\p^2$=0(由$A$的结合性),
从而$\p\downdot$为双$A$-模链复形。
并且显然$\p:A\ten A\ra A$是满同态。

断言链复形$\p\downdot$为正合(exact)的。
\index{exact\kong 正合}
事实上,$\p\downdot$到其自身的恒等链映射与零链映射是链同伦的。
我们构造如下的链同伦$h\downdot$:
\begin{eqnarray*}
h:a_1&\mapsto&1\ten a_1\\
a_1\ten a_2&\mapsto&1\ten a_1\ten a_2
\end{eqnarray*}
容易验证,对于任意的$\varphi=a_1\ten a_2\in A\ten A$,成立
\begin{eqnarray*}
   (\p h+h\p)\varphi
&=&(\p h+h\p)(a_1\ten a_2)\\
&=&\p(1\ten a_1\ten a_2)+h(a_1a_2)\\
&=&a_1\ten a_2-1\ten a_1a_2+1\ten a_1a_2\\
&=&a_1\ten a_2=\varphi\\
\end{eqnarray*}
从而对于$\varphi\in A\ten A$,
如果$\p\varphi=0$,那么
$$\varphi=(\p h+h\p)\varphi=\p(h\varphi)$$
这说明链复形$\p\downdot$在$A\ten A$处正合,
因此$\p\downdot$是正合的。\vs

接下来,将函子$M\ten_{A^e}-$作用于链复形
$\p\downdot$,得到如下的双$A$-模链复形:
$$M\ten_{A^e}\p\downdot:
M\ten A\ra M\ra M\ten_{A^e}A\ra 0$$
由张量函子的右正合性,上述链复形也是正合的。
其中注意到双$A$-模同构
\begin{eqnarray*}
M\ten_{A^e}(A\ten A\ten A)&\cong& M\ten A\\
m\ten(a_1\ten a_2\ten a_3)&\mapsto&(a_3.m.a_1)\ten a_2
\end{eqnarray*}
以及双$A$-模同构
\begin{eqnarray*}
M\ten_{A^e}(A\ten A)&\cong& M\\
m\ten(a_1\ten a_2)&\mapsto& a_2.m.a_1
\end{eqnarray*}

于是正合列$M\ten_{A^e}\p\downdot$的边界映射有如下具体表达式:

\begin{eqnarray*}
M\ten_{A^e}\p:
M\ten A&\to& M\\
m\ten A&\mapsto&m.a-a.m
\end{eqnarray*}

从而由正合性,易知
$$M\ten_{A^e}A\cong M/\{(m.a-a.m)|a\in A,m\in M\}$$
\end{proof}

可见,$M$的余中心无非是商掉$M$当中“非交换的部分”
所得到的“交换的部分”,如此望文生义。
例如,如果$A$为交换$K$-代数,
那么$A$本身作为双$A$-模,其余中心为$A$本身.

\section{Hochschild同调}

\begin{definition}(Hochschild 同调)
\index{Hochschild 同调}

对于双$A$-模$M$,以及非负整数$n$,记
$$H_n(A,M):=\Tor_n^{A^e}(M,A)$$
称为$M$的第$n$个Hochschild同调。特别地,我们记
$$\HH_n(A):=H_n(A,A)$$
\end{definition}

由定义以及导出函子的基础知识,容易知道
双$A-$模$M$的第$0$个Hochschild同调
$$H_0(A,M)=M\ten_{A^e}A=M/\{(m.a-a.m)|a\in A,m\in M\}$$
正是$M$的余中心。
注意Hochschild同调一般并不是环,
仅仅能保证它是双$A$-模。

%%%%%%%%%%%%%%%%2019.2.26 - 第1周第2次%%%%%%%%%%%%%%%%%%%%%%

具体地,由导出函子的定义,
我们采用投射消解(projective resolution)
\index{projective resolution\kong 投射消解}
来计算Hochschild同调。
若双$A$-模链复形
$$P\downdot\to A:=
...\to P_3\to P_2\to P_1\to P_0\to A\to 0$$
为双$A$-模$A$的投射消解
(正合,并且每个$P_i(i\geq 0)$作为$K$-模是投射的),那么
$$H_n(A,M)\cong H_n(M\ten_{A^e}P\downdot)$$
由同调代数的事实,它与投射消解$P\downdot$的选取无关。\vs

事实上Hochschild同调可以与空间上的微分形式类比。
作为一个具体计算例子,我们考虑$\bbC$上的$n$元多项式代数
$$A:=\bbC[x^1,x^2,...,x^n]$$
注意到$A$作为$\bbC$-代数是交换的,从而$A=A\op$.我们记
$$A\op=\bbC[y^1,y^2,...,y^n]\,\,\,\,\,
A^e=\bbC[x^1,x^2,...,x^n;y^1,y^2,...,y^n]$$

\begin{prop}
\label{C[x^i]的HH同调}
考虑$\bbC$-代数$A:=\bbC[x^1,x^2,...,x^n]$,
则其第$k$个Hochschild同调
$$\HH_k(A)\cong\Omg^k_A:=A\ten\wedgeform k(\bbC^n)$$
是以$A$为系数的$k$-形式。
\end{prop}

\begin{proof}

我们给出$A$的投射消解,比如众所周知
的Koszul消解$$\mcalK_A\to A\to 0$$

具体地,引入$n$个新的独立变元
$\eta^1,\eta^2,...,\eta^n$
(视为复线性空间$\bbC^n$的一组基),考虑环
$$\mcalK:=\frac{A^e[\eta^1,\eta^2,...,\eta^n]}
               {\{(\eta^i\eta^j+\eta^j\eta^i)|i\neq j\}}
         =A^e\ten\wedgeform *(\bbC^n)$$
为以$A^e$为系数的外代数。

注意$\mcalK$有自然的分次:
$$\deg\eta^i=1\,\,\,\,\,\deg x^i=\deg y^i=\deg 1=0$$
记$\mcalK_l$为$\mcalK$的$l$次分量($0\leq l\leq n$),即
$$
\mcalK_l=\bigoplus_{1\leq {i_1}<{i_2}<...<{i_l}\leq n}
          A^e\eta^{i_1}\wedge\eta^{i_2}\wedge...\wedge\eta^{i_l}
        =A^e\ten\wedgeform l(\bbC^n)$$
此时$K=\bbC$是域,因此$\mcalK$
(作为$K$-模,即复线性空间)的投射性显然。
我们定义Koszul复形$(\mcalK_A,\p)$如下:
$$\mcalK_A:...\xra{\p} \mcalK_n\xra{\p}\mcalK_{n-1}
\xra{\p}...\xra{\p}\mcalK_1\xra{\p}\mcalK_0$$
其中边缘算子$\p$(首先是$A^e$-模同态)满足

$$\p\eta^i= x^i-y^i$$
以及与外微分相同的莱布尼茨法则:
对任意$\omega\in\mcalK$,成立

$$\p(\eta^i\wedge\omega)=
\p\eta^i\wedge\omega-\eta^i\wedge\p\omega$$

再考虑连接映射
\begin{eqnarray*}
\veps:\mcalK_0=A^c&\to& A\\
x^i&\mapsto&x^i\\
y^i&\mapsto&x^i\\
\end{eqnarray*}
则众所周知,Koszul复形
$$\mcalK_A\xra{\veps} A\to 0$$
为$A$的投射消解(证明从略)。
我们以此计算$\HH\updot(A)$.
我们注意到以下两个简单事实:

其一:对任何$1\leq l\leq n$,
成立双$A$-模同构
$$A\ten_{A^e}\mcalK_l=
A\ten_{A^e}A^e\ten\wedgeform l(\bbC^n)
\cong A\ten\wedgeform l(\bbC^n)$$

其二:函子$A\ten_{A^e}-$作用于Koszul复形
$\mcalK_A$之后,成立
$$A\ten_{A^e}\p=0$$
这是因为,对于任意$f\in A$,在$A\ten_{A^e}A^e$
当中总成立
$$f\ten x^i=x^if\ten 1
=fx^i\ten 1=f\ten (x^i)\op=f\ten y^i$$
因此
$$f\ten(x^i-y^i)=0\in
A\ten_{A^e}A^e$$
从而由$\p$的定义,容易看出$A\ten_{A^e}\p=0$.\vs

综上两方面,直接计算之,
\begin{eqnarray*}
\HH_k(A)&=&H_k(A\ten_{A^e}^LA)\\
&=&H_k(A\ten_{A^e}\mcalK_A)\\
&=&A\ten_{A^e}\mcalK_k\\
&=&A\ten\wedgeform k(\bbC^n)\\
&=&\Omg^k_A
\end{eqnarray*}
从而得证。
\end{proof}

事实上对于一般的含幺结合$K$-代数$A$,$\HH\downdot(A)$
扮演了“微分形式”的角色。这是Hochschild同调的一种几何解释。\vs

对于一般的$A$,
$A$作为双$A$-模,由一种典范的投射消解,
称之为\textbf{Bar-复形}:

\begin{definition}(Bar-复形)

对于含幺结合$K$-代数$A$,定义以下双$A$-模链复形
$$\cdots\to B_2A\xra{b}B_1A\xra{b}B_0A\xra{b}A\to 0$$
如下:
$$B_nA:=A\ten A^{\ten n}\ten A\,\,\,(n\geq 0)$$
\begin{eqnarray*}
b:\,\,a_0\ten a_1\ten...\ten a_n
&\mapsto&
\sum_{k=0}^{n-1}(-1)^k
      a_0\ten a_1\ten...\ten(a_{k}a_{k+1})\ten...\ten a_n
\end{eqnarray*}
称之为\textbf{Bar-复形}。
\index{Bar-复形}
\end{definition}

首先容易验证$b^2=0$,从而
$B\downdot A\xra{b} A\to 0$确实是链复形。
对于$n\geq 1$,具体验证如下:
\begin{eqnarray*}
    b^2(a_0\ten a_1\ten...\ten a_n)
&=& b\left(
         \sum_{k=0}^{n-1}(-1)^k
             a_0\ten a_1
             \ten...\ten(a_{k}a_{k+1})
             \ten...\ten a_n
     \right)\\
&=&  \sum_{k=0}^{n-1}(-1)^k
             b(a_0\ten a_1
             \ten...\ten(a_{k}a_{k+1})
             \ten...\ten a_n)\\
&=&  \sum_{k=0}^{n-1}(-1)^k
         \left[
             \sum_{l=0}^{k-2}(-1)^l
                  a_0\ten...\ten(a_{l}a_{l+1})
                  \ten...\ten(a_{k}a_{k+1})
                  \ten...\ten a_n
         \right.\\
&&+  (-1)^{k-1}
      a_0\ten...\ten(a_{k-1}a_{k}a_{k+1})
                  \ten...\ten a_n\\
&&+
     (-1)^{k}
      a_0\ten...\ten(a_{k}a_{k+1}a_{k+2})
                  \ten...\ten a_n\\
&&-\left.
             \sum_{l=k+2}^{n-1}(-1)^l
                  a_0\ten...\ten(a_{k}a_{k+1})
                  \ten...\ten(a_{l}a_{l+1})
                  \ten...\ten a_n
    \right]\\
&&\\
&=&  \sum_{\substack{0\leq k<l\leq n-1\\ l-k\geq 2}}
         \left(-(-1)^{k+l}+(-1)^{k+l}\right)
         a_0\ten...\ten(a_{k}a_{k+1})
                  \ten...\ten(a_{l}a_{l+1})
                  \ten...\ten a_n\\
&&+  \sum_{0\leq k\leq n-2}
         \left((-1)^{2k+1}+(-1)^{2k}\right)
             a_0\ten...\ten(a_{k}a_{k+1}a_{k+2})
                  \ten...\ten a_n\\
&=&0
\end{eqnarray*}
从而验证完毕。

我们可以把$a_0\ten...\ten a_n$想象为直线上依次排列的$n+1$
个质点,将算子$b$想象为相邻质点两两“碰撞”。

\begin{prop}记号同之前,则Bar-复形
$$B\downdot A\to A\to 0$$
是$A$的投射消解。
\end{prop}

\begin{proof}
对任意$n\geq0$,$B_nA=A\ten A^{\ten n}\ten A$
是投射$K$-模(这是因为由最初的假定,
$A$是投射$K$-模,从而其张量积也投射)
于是我们只需再验证该链复形是正合的。

为此,我们构造链同伦
\begin{eqnarray*}
h:B_{n-2}A&\to& B_{n-1}A
\,\,\,\,\,\,\,(n\geq1,\,B_{-1}A=A)\\
a_0\ten...\ten a_n
&\mapsto&
1\ten a_0\ten...\ten a_n
\end{eqnarray*}

只需验证$hb+bh$=1,
之后与性质\ref{双模的余中心的结构prop}的证明类似。

注意到对于任意$n\geq 0$,成立
\begin{eqnarray*}
    bh(a_0\ten...\ten a_n)
&=& b(1\ten a_0\ten...\ten a_n)\\
&=& a_0\ten...\ten a_n-
    \sum_{k=0}^{n-1}
        1\ten a_0\ten...\ten(a_ka_{k+1})\ten...\ten a_n\\
&=& a_0\ten...\ten a_n-
    1\ten b(a_0\ten...\ten a_n)\\
&=&(1-hb)a_0\ten...\ten a_n
\end{eqnarray*}
因此$bh+hb=1$,证毕。
\end{proof}

\begin{definition}
设$M$为双$A$-模,定义
\textbf{Hochschild链复形}
\index{Hochschild链复形}
\label{Hochschild链复形-def}
$$C\downdot(A,M):=M\ten_{A^e} B\downdot A$$
$$\cdots M\ten A^{\ten 3}\to M\ten A^{\ten 2}\to M\ten A\to M$$
方便起见,该链复形的边缘算子仍记作$b$.
%%%%%%005%%%%%%%
\end{definition}

则易知$M$的Hochdchild同调无非是Hochschlid链复形的同调:
$$H_n(A,M)=H_n(C\downdot(A,M))$$

注意到有双$A$-模同构
$$C_n(A,M)=M\ten_{A^e}
\left(A\ten A^{\ten n}\ten A\right)
\cong M\ten A^{\ten n}$$
在此同构意义下,容易验证$C\downdot(A,M)$
的边缘算子$b$有如下显示表达:

对任意$m\in M$,以及$a_1,a_2,...,a_n\in A$,成立
\begin{eqnarray*}
    b\left(m\ten (a_1\ten...\ten a_n)\right)
&=& m\ten_{A^e}
      \left(
        b(1\ten a_1\ten...\ten a_n\ten 1)
      \right)\\
&=& m\ten_{A^e}
      \large[
        a_1\ten...\ten a_n\ten 1
      \\
&&+  \sum_{k=1}^{n-1}(-1)^k
           1\ten a_1\ten...\ten(a_ka_{k+1})\ten...\ten a_n\ten 1\\
&&+
        (-1)^n1\ten a_1\ten...\ten a_n
    \large]\\
&=& (m.a_1)\ten a_2\ten...\ten a_n\\
&&+ \sum_{k=1}^{n-1}(-1)^k
      m\ten a_1\ten...\ten(a_ka_{k+1})\ten...\ten a_n\\
&&+ (-1)^n(a_n.m)\ten a_1\ten...\ten a_{n-1}
\end{eqnarray*}
Hochschlid链复形的边缘算子的显式表达
与Bar-复形非常相似,从上式最右边的前两项可以看出;
区别在于上式最右边的第三项。

\section{Hochschlid上同调}

对于双$A$-模$M$,既然我们已经考虑
余中心$M\ten_{A^e}A$,那么我们自然也会去考虑
$\Hom_{A^e}(A,M)$.我们称
双$A$-模$\Hom_{A^e}(A,M)$为$M$的\textbf{导出中心}
(derived center)。
\index{derived center\kong 导出中心}

\begin{prop}(导出中心的结构)
\label{双模的导出中心的结构prop}
对于双$A$-模$M$,则有双$A$-模同构
$$\Hom_{A^e}(A,M)
\cong\{m\in M|a.m-m.a=0\,\,\,\forall a\in A\}$$
\end{prop}

容易验证$\{m\in M|a.m-m.a=0\,\,\,\forall a\in A\}$
为$M$的双$A$-子模。粗俗地说,
该子模由“与$A$中所有元素交换”的元素构成,
故谓之“中心”。

\begin{proof}对于任意的$\fai\in\Hom_{A^e}(A,M)$
以及$a\in A$,则$\fai(a)$的取值由$\fai(1)$完全决定:
$$\fai(a)=\fai(a.1)=a.\fai(1)$$
而另一方面,
$$\fai(a)=\fai(1.a)=\fai(1).a$$
从而有$a.\fai(1)=\fai(1).a$.
于是我们可以构造如下双$A-$模同态:
\begin{eqnarray*}
\Hom_{A^e}(A,M)
&\to&\{m\in M|a.m-m.a=0\,\,\,\forall a\in A\}\\
\fai&\mapsto&\fai(1)
\end{eqnarray*}
容易验证该模同态为同构。证毕。
\end{proof}

然后我们考虑$\Hom(-,M)$的导出函子,
自然地去定义如下:

\begin{definition}(Hochschild上同调)
\index{Hochschild上同调}
\label{Hochschild上同调-def}

对于双$A$-模$M$,以及$n\geq 0$,
定义$M$的第$n$个Hochschild上同调
$$H^n(A,M)=\Ext_{A^e}^n(A,M)$$
特别地,我们记
$$H^n(A)=\Ext_{A^e}^n(A,A)$$
\end{definition}

由定义知,$M$的第$0$个Hochschild上同调为
$\Hom_{A^e}(A,M)$,是$M$的导出中心。
回顾Bar-复形,我们考虑如下的
\textbf{Hochschild上链复形}
\index{Hochschild上链复形}
$$C\updot(A,M)=\Hom_{A^e}(B\downdot A,M)$$
该上链复形的微分算子$\p$
由Bar-复形$B\downdot A$的边缘算子$b$所诱导。
则$M$的Hochschild上同调满足
$$H^n(A,M)=H^n(C\updot(A,M),\p)
=H^n(\Hom_{A^e}(B\downdot A,M),\p)$$

注意有自然的双$A$-模同构
$$C^n(A,M)
=\Hom_{A^e}(A\ten A^{\ten n}\ten A,M)
\cong\Hom(A^{\ten n},M)$$
(即取值于$M$的$n$重$K$-线性映射)
于是对于任意的$\fai\in C^n(A,M)=\Hom(A^{\ten n},M)$,
容易知道$\p\fai\in\Hom(A^{\ten n+1},M)$具有如下显式表达:
对任意$a_0,a_1,...,a_m\in A$,
\begin{eqnarray*}
    \p\varphi(a_0,a_1,...,a_n)
&=& a_0.\varphi(a_1,a_2,...,a_n)\\
&&- \sum_{k=0}^{n-1}(-1)^k
       \fai(a_0,...;(a_ka_{k+1});...,a_n)\\
&&-  (-1)^n\fai(a_0,a_1,...,a_{n-1}).a_n
\end{eqnarray*}


接下来讨论Hochschild上同调的几何意义。
我们已经知道第0个Hochschild上同调为$M$的导出中心;
现在我们看$H^1(A,M)$,
我们将发现它是$A$的取值于$M$的外导子。

回顾\textbf{导子}(derivation)的概念如下:

\begin{definition}(导子)
对于双$A$-模$M$,$K$-线性映射
$$D:A\to M$$
称为$A$的取值于$M$的\textbf{导子}(derivation),
\index{derivation\kong 导子}
如果对任意的$a_1,a_2\in A$,成立
$$D(a_1a_2)=D(a_1).a_2+a_1.D(a_2)$$
\end{definition}

对于$m\in M$我们定义
\begin{eqnarray*}
\ad_m:A&\to& M\\
a&\mapsto& [m,a]:=m.a-a.m
\end{eqnarray*}
则容易验证$\ad_m$为$A$的取值于$M$的导子,
称形如这样的导子为\textbf{内导子}(inner derivation)。
\index{inner derivation\kong 内导子}

我们记
$$\Der(A,M):=\{D:A\to M|D\text{为导子}\}$$
$$\Inn(A,M):=\{\ad_m|m\in M\}\subseteq \Der(A,M)$$

注意$\Inn(A,M)$与$\Der(A,M)$都有显然的$K$-模结构,
且前者是后者的$K$-子模。

\begin{prop}($\HH^1(A,M)$的结构)

对于双$A$-模$M$,成立
$$\HH^1(A,M)\cong\frac{\Der(A,M)}{\Inn(A,M)}$$
\end{prop}
我们称上式右边的集合当中的元素
为$A$的取值于$M$的\textbf{外导子}(outer derivation)。
\index{outer derivation\kong 外导子}

\begin{proof}
只需考虑Hochschild上链复形
$$C^0(A,M)\xra{\p^0}
C^1(A,M)\xra{\p^1}C^2(A,M)\ra\cdots$$

我们只需具体计算之。对于$\fai\in C^1(A,M)\cong\Hom(A,M)$,
则$\p^1\fai\in C^2(A,M)\cong\Hom(A^{\ten 2},M)$满足:
对任意$a_1,a_2\in A$,成立
$$\p^1\fai(a_1,a_2)=
a_1.\fai(a_2)-\fai(a_1a_2)+\fai(a_1).a_2$$
可见$\fai\in\ker\p^1$当且仅当$\fai\in\Der(A,M)$.
也就是说$\ker\p^1=\Der(A,M)$.

另一方面,对于$m\in C^0(A,M)\cong M$,以及$a\in A$,成立
$$(\p^0m)(a)=a.m-m.a=-\ad_m(a)$$
因此$\ker\p^0\cong\Inn(A,M)$.
从而
$$\HH^1(A,M)=\frac{\ker\p^1}{\im\p^0}
\cong\frac{\Der(A,M)}{\Inn(A,M)}$$
\end{proof}

特别地,当$M=A$时,
$$\HH^1(A)=\Der(A,A)/\text{Inn}(A,A)$$
注意到$\Der(A,A)$上面还有更多的结构:对于
$\forall D_1,D_2\in\Der(A,A)$,定义
$$[D_1,D_2]:=D_1\circ D_2-D_2\circ D_1:A\to A$$
容易验证$[D_1,D_2]$仍然为$A$的导子,并且
$[-,-]$为$\Der(A,A)$上的李括号(Lie bracket)。
\index{Lie bracket\kong 李括号}

另外容易验证
$$[\Der(A,A),\Inn(A,A)]\subseteq\Inn(A,A)$$
具体地,对于$D\in\Der(A,A)$以及$m\in M$,成立
$$[D,\ad_m]=\ad_{D(m)}$$
也就是说$\Inn(A,A)$是$\Der(A,A)$的理想。
于是$[-,-]$诱导了
$\HH^1(A)=\frac{\Der(A,A)}{\text{Inn}(A,A)}$
上的李括号结构.

如果$A$是交换$K$-代数,则$\Inn(A,A)=0$。于是
$$\HH^1(A)\cong\Der(A,A)$$
可被认为是“切向量场”(此时$A$被认为是“函数环”)。

\vsp

我们再去考虑$\HH^2(A,M)$.对于任意的
$$\varphi\in C^2(A,M)=\Hom(A^{\ten 2},M)$$
则对$a_0,a_1,a_2\in A$,成立
$$\p\varphi(a_0,a_1,a_2)=
a_0.\varphi(a_1,a_2)-\varphi(a_0a_1,a_2)
+\varphi(a_0,a_1a_2)
-\varphi(a_0,a_1).a_2$$

\begin{lemma}对于双$A$-模$M$,以及
$\varphi\in C^2(A,M)=\Hom(A^{\ten 2},M)$,
我们令
$$\Ahat:=A\oplus M$$
并赋以如下的$K$-代数结构:
对于任意$a_1,a_2\in A$以及$m_1,m_2\in M$,
规定$\Ahat$的乘法$\hat{\bullet}_{\fai}$为
$$(a_1\oplus m_1)\hat{\bullet}_{\fai}(a_2\oplus m_2)
:=a_1a_2\oplus [a_1.m_2+m_1.a_2+\fai(a_1,a_2)]$$
那么$(\Ahat,\hat{\bullet}_{\fai})$为结合代数,
当且仅当$\p\fai=0$.
\end{lemma}

\begin{proof}
这是简单的计算验证。对于任意的$a_0,a_1,a_2\in A$
以及$m_0,m_1,m_2\in M$,直接计算之,
\begin{eqnarray*}
&&  [(a_0\oplus m_0)\hat{\bullet}_{\fai}(a_1\oplus m_1)]
    \hat{\bullet}_{\fai}(a_2\oplus m_2)\\
&=& a_0a_1a_2\oplus
    [a_0a_1.m_2+a_0.m_1.a_2+m_0.a_1a_2+
    \fai(a_0,a_1).a_2+\fai(a_0a_1,a_2)]
\end{eqnarray*}
以及
\begin{eqnarray*}
&&  (a_0\oplus m_0)\hat{\bullet}_{\fai}[(a_1\oplus m_1)
    \hat{\bullet}_{\fai}(a_2\oplus m_2)]\\
&=& a_0a_1a_2\oplus
    [a_0a_1.m_2+a_0.m_1.a_2+m_0.a_1a_2+
    a_0.\fai(a_1,a_2)+\fai(a_0,a_1a_2)]
\end{eqnarray*}
因此$\hat{\bullet}_{\fai}$满足结合性,当且仅当
$$\fai(a_0,a_1).a_2+\fai(a_0a_1,a_2)
=a_0.\fai(a_1,a_2)+\fai(a_0,a_1a_2)$$
而此式等价于$\p\fai=0$.
\end{proof}

注意到在$\Ahat$当中,对任意的$m_1,m_2\in M$,
以及任意$\fai\in C^2(A,M)$,
总有$m_1\hat{\bullet}_{\fai}m_2=0$.于是我们不妨将
“$A\oplus M$”当中的“$M$”理解为“一阶小量”。
我们考虑$\fai=0$时$\Ahat_0:=A\oplus M$的代数结构
$$(a_1\oplus m_1)\bullet(a_2\oplus m_2)
:=a_1a_2\oplus (a_1.m_2+m_1.a_2)$$
显然$(\Ahat_0,\bullet)$为结合代数。
若$\p\fai=0$,则结合代数$(\Ahat,\hat{\bullet}_{\fai})$
为$(\Ahat_0,\bullet)$的\textbf{一阶形变},
而$\fai$为其“形变参数”。

从而$M$的第2个Hochschild上同调

$$
    H^2(A,M)\cong
  \frac{
         \{\fai|(\Ahat,\hat{\bullet}_{\fai})
         \text{是结合代数}\}
       }{
       \im(\p:C^1(A,M)\to C^2(A,M))
       }
$$
商掉的东西($\im\p$)为
形如以下的一类特殊的一阶形变:
\begin{eqnarray*}
\fai_f:A\ten A&\to& M\\
a_1\ten a_2 &\mapsto&
a_1.f(a_2)+f(a_1).a_2-f(a_1a_2)
\end{eqnarray*}
其中$f\in C^1(A,M)=\Hom(A,M)$,$\fai_f=\p f$.

\vs

我们考察一个Hochschild上同调的具体算例。

\begin{prop}若$A=\bbC[x^1,...,x^n]$为
$\bbC$上的$n$元多项式环,则

$$\HH^k(A)\cong
\Hom\left(\wedgeform{k}(\bbC^n),A\right)$$
\end{prop}

\begin{proof}
对于这个特例,采用Koszul复形计算更佳简便。
有关记号同性质\ref{C[x^i]的HH同调}的证明过程.
考虑Koszul复形
$$
\mcalK_A:\cdots
\xra{\p}A^e\ten\wedgeform{k+1}(\bbC^n)
\xra{\p}A^e\ten\wedgeform{k}(\bbC^n)
\xra{\p}A^e\ten\wedgeform{k-1}(\bbC^n)
\xra{\p}\cdots
$$
然后将函子$\Hom_{A^e}(-,A)$作用于之上。
注意到有$\bbC$-线性同构
\begin{eqnarray*}
&&      \Hom_{A^e}\left(A^e\ten\wedgeform{k}(\bbC^n),A\right)\\
&\cong& \Hom\left(\wedgeform{k}(\bbC^n),\Hom_{A^e}(A^e,A)\right)\\
&\cong& \Hom\left(\wedgeform{k}(\bbC^n),A\right)
\end{eqnarray*}

此外再注意到,上链复形$\Hom_{A^e}(\mcalK_A,A)$的
微分算子$\td:=\Hom_{A^e}(\p,A)=0$.这是因为对于
$\fai\in\Hom_{A^e}\left(A^e\ten\wedgeform{k}(\bbC^n),A\right)$,
$\omega\in\wedgeform{k+1}(\bbC^n)$以及$f\in A^e$,成立
$$\td\fai(f\ten\omega)=\fai(\p(f\ten\omega))$$
回顾Koszul复形边缘算子运算规则
$$\p:\eta^i\mapsto x^i-y^i\in A^e$$
又由于$\fai$为$A^e$-模同态,
从而对于任意$\tilde{\omega}\in\wedgeform{k}(\bbC^n)$,成立
$$
\fai(x^i\ten\tilde{\omega})
=x^i.\fai(1\ten\tilde{\omega})
=\fai(1\ten\tilde{\omega}).x^i
=\fai((x^i)\op\ten\tilde{\omega})
=\fai(y^i\ten\tilde{\omega})$$
也就是说$\fai((x^i-y^i)\ten\tilde{\omega})=0$.
由此可见$\td=0$.综上可知

$$\HH^k(A)\cong
\Hom\left(\wedgeform{k}(\bbC^n),A\right)$$

\end{proof}

注意到$\Hom\left(\wedgeform{k}(\bbC^n),A\right)$
之中的元素形如
$$
\sum_{1\leq i_1<...<i_k\leq n}
f_{i_1...i_k}\p_{i_1}\wedge...\wedge\p_{i_k}
$$
回顾$\HH\downdot(A)$中的元素可被认为是“微分形式”,
可见$\HH\updot(A)$中的元素则是“多重切向量场”。

%%%%%%%%%%%2019.3.4 Mon 第2周%%%%%%%%%%%%

\section{一些例子}

如果$K\inj A$为嵌入,那么我们可以更加方便地计算Hochschild
(上)同调:

\begin{definition}(约化Bar-复形)(reduced Bar-complex)
\index{reduced Bar-complex\kong 约化Bar-复形}
\label{约化Bar复形-def}

对于$K$-代数$A$,如果$K\inj A$,那么考虑$K$-模

$$\overline{A}:=A/K$$
我们定义如下的约化Bar-复形$(\overline{B}\downdot A,b)$:
$$\overline{B}_n A:=A\ten \overline{A}^{\ten n}\ten A
\quad \forall \,i\geq 0$$
边缘算子$b:\overline{B}_n A\to \overline{B}_{n-1} A$如下定义:
\begin{eqnarray*}
b\Big(a_0\ten(\overline{a_1}\ten\cdots\ten \overline{a_n})\ten a_{n+1}\Big)
&:=&
 (a_0a_1)\ten(\overline{a_2}\ten\cdots\ten\overline{a_n})\ten a_{n+1}\\
&+ &
 \sum_{i=1}^{n-1}(-1)^ia_0\ten
    (\overline{a_1}\ten\cdots
    (\overline{a_ia_{i+1}})\ten\cdots\ten \overline{a_n})
    \ten a_{n+1}\\
&+ &(-1)^n
 a_0\ten(\overline{a_1}\ten\cdots\ten \overline{a_{n-1}})
   \ten(a_na_{n+1})
\end{eqnarray*}
\end{definition}

注意到$\overline{B}\downdot A$是$B\downdot A$的商模:
$$\overline{B}_nA=
     \frac{B_n A}
     {\{a_0\ten(a_1\ten\cdots
     \ten a_{i-1}\ten 1\ten a_{i+1}
     \ten\cdots\ten a_n)\ten a_{n+1}\}
     }
$$
%%%%%%%%%崩殂%%%%%%!

容易发现约化Bar-复形的“$b$”正是Bar-复形的$b$.
但是我们要验证$b$的良定性,即与代表元选取无关。
这是容易验证的。于是我们得到以下链复形:

$$\overline{B}\downdot A\to A\to 0$$

与之前$Bar$-复形完全类似,我们容易验证此复形也是正合的。
只需构造同伦算子
\begin{eqnarray*}
h:\overline{B}_{n-1}A&\to&\overline{B}_{n}A\\
a_0\ten(\overline{a_1}\ten\cdots\ten \overline{a_{n-1}})\ten a_{n}
&\mapsto&
1\ten(\overline{a_0}\ten\overline{a_1}\ten\cdots\ten \overline{a_n})\ten a_{n+1}
\end{eqnarray*}
验证$bh+hb=1$即可。
%%%%%%天下三分%%%%%%%!

\begin{definition}(约化Hochschild(上)链复形)

对于双$A$-模$M$,我们令
\begin{eqnarray*}
\overline{C}\downdot(A,M)&:=&
M\ten_{A^e}\overline{B}\downdot A\cong M\ten\overline{A}^{\ten\bullet}\\
\overline{C}\updot(A,M)&:=&
\Hom_{A^e}(\overline{B}\downdot A,M)
\cong\Hom(\overline{A}^{\ten\bullet},M)
\end{eqnarray*}
称之为关于$M$的约化Hochschild(上)链复形。
\end{definition}

事实上,约化Hochschild(上)
链复形的(上)同调自然同构于Hochschild(上)同调——
这是由以下代数引理保证的:

\begin{lemma}条件同上,则商映射
$$\pi\downdot:C\downdot(A,M)
\surj\overline{C}\downdot(A,M)$$
所诱导的链映射
$$\pi\downdot:(C\downdot(A,M),d)
\surj(\overline{C}\downdot(A,M),d)$$
为拟同构。
\label{约化Hoschild复形与拟同构-lemma}
\end{lemma}

\begin{proof}
注意链映射$\pi\downdot$为满态射,
只需再证明其核复形
$$\ker\pi\downdot$$
是正合的即可。我们承认之(似乎不太好证)。
\end{proof}

注意上述引理也适用于Hochschild上链复形的情形,
完全类似,不再赘述。
从而我们立刻有如下推论:

\begin{cor}
对于$K$-代数$A$,如果$K\inj A$为嵌入,则有自然同构:
\begin{eqnarray*}
H\downdot(A,M)&\cong&H\downdot(\overline{C}\downdot(A,M))\\
H\updot  (A,M)&\cong&H\updot  (\overline{C}\updot  (A,M))
\end{eqnarray*}
\end{cor}

关于(约化)Bar-复形,我们还有另一种理解方式:
关于$A$的(约化)Bar-复形是$A$与某个微分分次代数的自由乘积。
%Another way to understand $B\downdot A,\bar{B}\downdot A$

\begin{definition}(微分分次代数)

%A differential graded algebra
\index{differential graded algebra\kong 微分分次代数}
$\bbZ$-分次$K$-代数
$$A:=\bigoplus_{n\in\mathbb{Z}}A_n$$
称为\textbf{微分分次代数}(differential graded algebra),
若它配以$K$-线性算子$\td:A\to A$,并且满足:
$$\left\{\begin{array}{l}
\td(A_n)\subseteq A_{n+1}\quad\forall n\in\bbZ\\
\td^2=0\\
\td(\alpha\beta)
=(\td\alpha)\beta+(-1)^{\deg\alpha}\alpha(\td\beta)
\quad\forall\alpha,\beta\in A\text{,并且$\alpha$是齐次元}
\end{array}\right.$$
\end{definition}

对于微分分次代数$(A,\td)$,由于$A$的分次以及$\td^2=0$,
从而自然有上链复形
$$\cdots\to A_{-1}\xra{\td}A_0\xra{\td}A_1\to\cdots$$
我们将此上链复形也记为$(A,\td)$.

微分分次代数最直接的例子是,对于光滑流形$X$,
考虑$A:=\Omg\updot(X)$为$X$上的微分形式。
$A$上的乘法即为微分形式的外积$\wedge$,
微分结构即为外微分$\td$.

我们可以适当修改微分分次代数的定义,
将条件“$\td(A_n)\subseteq A_{n+1}$”改为
“$\td(A_n)\subseteq A_{n-1}$”,
此时的微分算子我们习惯记为“$\p$”.
对于这样的微分分次代数$(A,\p)$,它可以被视为链复形。

\begin{example}我们考虑如下$K$-代数:
$$A:=K[\veps]:=K\oplus K\veps\oplus K\veps^2\oplus\cdots$$
其中$\veps$为形式变量,并且规定$\deg\veps=1$,
由此诱导出$K[\veps]$的分次结构。其微分算子$\p_{\veps}$由以下诱导:
$$\p_{\veps}(1)=0\quad\p_{\veps}(\veps)=1$$
\label{典型微分分次代数-example}
\end{example}

注意$\deg\veps=1$,按照微分代数的定义可计算出
$$\p_{\veps}(\veps^2)=
  \p_{\veps}(\veps)\veps
  +(-1)^{\deg\veps}\veps\p_{\veps}(\veps)
=\veps-\veps=0$$
一般地,对于非负整数$n$我们有
$$
  \p_{\veps}(\veps^n)
= \left\{
    \begin{array}{ll}
      0           &\text{$n$为偶数}\\
      \veps^{n-1} &\text{$n$为奇数}
    \end{array}
  \right.
$$
从而链复形$(K[\veps],\p_{\veps})$:
$$\cdots\to K\veps^4\xra{0}K\veps^3\xra{1}
K\veps^2\xra{0}K\veps\xra{1}K\to 0$$
是正合的。其中$1:K\veps^{2n+1}\to K\veps^{2n}$
将$\veps^{2n+1}$映为$\veps^{2n}$.

%%%%微分分次代数的例子%%%%%%%!

众所周知,对于两个$K$-代数$A,B$,
我们可以谈论它们的\textbf{自由乘积}(free product)
$A*B$.若$A=\bigoplus\limits_{i\in\bbZ}A_n$是微分分次代数,
其微分算子为$\td$,则容易知道$A*B$自然也有微分分次代数结构:
$$
\left\{
    \begin{array}{rcll}
       \deg b&=&0&\forall{b\in B}\\
       \deg a&=&n&\forall{a\in A_n\subseteq A}\\
       \td b&=&0 &\forall{b\in B}
    \end{array}
\right.
$$
容易知道$A*B$中的$N$次齐次元必形如以下元素的有限和:
$$b_1a_1b_2a_2\cdots b_ma_mb_{m+1}\quad
(b_i\in B\,,\,a_i\in A_{n_i}\,,\,\sum_{i=1}^mn_i=N)$$

\begin{prop}
对于$K$-代数$A$,则有链复形的同构
$$(B\downdot A\to A,b)\cong (A*K[\veps],\p_{\veps})$$
其中$(K[\veps],\p_{\veps})$为例子
\ref{典型微分分次代数-example}
当中的微分分次代数,视为链复形;
同构映射为
\begin{eqnarray*}
\fai_n:B_nA&\to&(A*K[\veps])_n\\
a_0\ten(a_1\ten\cdots\ten a_n)\ten a_{n+1}&\mapsto&
a_0\veps a_1\veps a_2\cdots a_n\veps a_{n+1}
\end{eqnarray*}
%the bar resolution $[B\downdot\to A]\cong A*K[\veps]$
%(free product),and $b=\p_{\veps}$
\end{prop}

这给出了Bar-复形的又一种理解方式。

\begin{proof}
容易验证$\fai_n$为$K$-模同构,且逆映射$\fai_n^{-1}$由以下诱导:
$$\veps^n\mapsto\underbrace
{1\veps 1\veps 1\cdots 1\veps 1}_{n\text{个}\veps}$$
然后只需验证$\fai\downdot:(B\downdot\to A,b)\to(A*K[\veps],\p_{\veps})$:
是链映射,也就是要验证交换关系$\fai\circ b=\p_{\veps}\circ\fai$
$$
\xymatrix{
B_nA \ar[d]^{\fai} \ar[r]^b         &B_{n-1}A\ar[d]^{\fai}\\
(K[\veps]*A)_n \ar[r]^{\p_{\veps}}  &(K[\veps]*A)_{n-1}
}$$
而这容易验证,验证如下:
\begin{eqnarray*}
&&   \fai\circ b(a_0\ten a_1\ten\cdots\ten a_n\ten a_{n+1})\\
&= &\fai\left(
          \sum_{k=0}^n(-1)^k
            a_0\ten\cdots\ten(a_k a_{k+1})\ten\cdots\ten a_{n+1}
       \right)\\
&= &\sum_{k=0}^n(-1)^k
      a_0\veps a_1\veps\cdots\veps a_{n+1}\\
&= &\p_{\veps}(a_0\veps a_1\veps\cdots a_n\veps a_{n+1})\\
&= &\p_{\veps}\circ\fai(a_0\ten a_1\ten\cdots\ten a_n\ten a_{n+1})
\end{eqnarray*}

%$$[B\downdot A\to A]\mapsto A*K[\veps]$$
%$$a_0\ten...\ten a_n\mapsto
%a_0\veps a_1\veps...a_{n-1}\veps a_n$$
%this is bijective.

%$$\p_{\veps}:a_0\veps a_1\veps...a_{n-1}\veps a_n\mapsto
%a_0a_1\veps.....-a_0\veps a_1a_2...$$
%...

%check

\end{proof}

我们还可以考虑$(K[\veps],\p_{\veps})$的商代数
$K[\veps]/\veps^2$,易知
$(K[\veps]/\veps^2,\p_{\veps})$也构成微分分次代数,
从而也通过微分算子$\p_{\veps}$视为链复形。
在此代数中,$\veps^2=0$.\vs

类似地,我们可以给出约化Bar-复形的另一种理解方式:

\begin{prop}
对于$K$-代数$A$,则有链复形同构
$$(\overline{B}\downdot A\to A,b)\cong (A*K[\veps]/\veps^2,\p_{\veps})$$
\end{prop}

只需注意到$A*K[\veps]/\veps^2$
当中的$n$次齐次元必形如以下元素的有限和:
$$a_0\veps a_1\veps\cdots a_n\veps a_{n+1}\quad (a_i\in A)$$

\begin{proof}
完全类似。事实上此链复形同构映射由
$\fai_n:B_nA \to (A*K[\veps])_n$诱导,
其良定性由下式保证:对任意$1\leq i\leq n$,
\begin{eqnarray*}
&&  \fai_n(a_0\ten \cdots\ten a_{i-1}
    \ten 1\ten a_{i+1}\ten\cdots\ten a_{n+1})\\
&=& a_0\veps a_1\cdots a_{i-1}\veps1\veps a_{i+1}\cdots\veps a_{n+1}\\
&=& a_0\veps a_1\cdots a_{i-1}1\veps^2a_{i+1}\cdots\veps a_{n+1}\\
&=& 0\mod\veps^2
\end{eqnarray*}
\end{proof}
%Similarly, reduced Bar-complex
%$$[\bar{B}\downdot A\to A]\cong A*K[\veps]/\veps^2$$

本节最后简单介绍以下Hochschild(上)
同调与其它常见的(上)同调理论的关系。

\begin{example}(群的上同调)%(Group (co)homology)

设$G$是一个群,$M\in\Rep(G)$为群$G$的一个左$K$-表示,
则有$G$-模链复形
%Let $G$ be a group, $M\in\text{Rep}(G)$ is a left module,
%then we obtain a complex
$$0\to M\xra{\delta}C^1(G,M)\xra{\delta}C^2(G,M)\xra{\delta}...$$
其中%where
$$C^n(G,M):=\Hom(G^n,M)=\{f:G^n\to M\}$$
并且微分算子$\delta$满足%the differential $\delta$ s.t.
$$\left\{\begin{array}{rcl}
\delta(m)(g)&=&g.m-m\\
(\delta f)(g_0,g_1,...,g_n)&=&
g_0.f(g_1,g_2,...,g_n)\\
&&-\sum\limits_{k=1}^{n-1}(-1)^kf(g_1,...,g_kg_{k+1},...,g_n)\\
&&-(-1)^nf(g_0,g_1,...,g_{n-1})
\end{array}\right.
$$
%check $\delta^2=0$,this cohomology
容易验证$\delta^2$=0.此链复形的上同调
$$H\updot(G,M):=H\updot(C\updot(G,M),\delta)$$
称之为\textbf{群的上同调}(group cohomology)
\index{group cohomology\kong 群的上同调}
\end{example}

由$\delta$的表达式容易看出,
群的上同调与Hochschild上同调有以下关系:

\begin{prop}设$G$是一个群,$M$为群$G$的一个左$K$-模,
考虑群代数$A:=K[G]$,于是$M$自然有左$A$-模结构。那么有同构:
%Let $A=K[G]$ group algebra with right $A$-action.then
$$H\updot(G,M)\cong H\updot(K[G],M)$$
%hochschild cohomology.
其中左边为群$G$关于$M$的上同调,
右边为群代数$K[G]$关于$M$的Hichschild上同调。
\end{prop}

{\color{red}
注意$M$仅仅是左$K[G]$-模,并没有双$K[G]$-模结构呀,
怎么谈论Hochschild上同调?

\color{purple}
(强行规定$G$在$M$上的右作用恒为1,
通过$K$-线性扩张得到$K[G]$在$M$的右作用,
这样就得到$M$的双$K[G]$-模结构了。)}

\begin{proof}
注意到$\Hom(G^n,M)$中的元素可以自然地$K$-线性延拓为
$\Hom(K[G]^n,M)$中的元素,这给出它们之间的同构。
然后注意到$A=K[G]$的Hochschild上链复形的微分算子的显式表达式,
(见定义\ref{Hochschild上同调-def}的下方)
它与群上同调相应的上链复形的微分算子显式表达式
“相同”。细节从略。
\end{proof}

若熟悉李代数同调,我们可以将李代数同调与
其泛包络代数的Hochschild同调联系起来:

\begin{example}(李代数同调)%(Lie algebra (co)homology)
对于李代数$\mfkg$,$M$为李代数$\mfkg$的一个左$K$-模。
令$A:=\mcalU(\mfkg)$为$\mfkg$的泛包络代数,
则$A$自然有左$A$-模结构。
(再通过某种“比较平凡”的方式给出右作用?与上例类似?)
则有同构
%$\mathfrak{g}$:Lie algebra, $M$ is a $\mathfrak{g}$-module.
%consider $A:=\mathcal{U}(\mathfrak{g})$ universal enveloping algebra,
%$M$ is a left $A$-module, with
$$H\downdot(\mathcal{U}(\mathfrak{g}),M)
\cong H^{\Lie}\downdot(\mathfrak{g},M)$$
其中左边是$A$关于$M$的Hochschild同调,右边是李代数同调。
\end{example}

并没有在此叙述\textbf{李代数同调}的定义。
留给感兴趣者。此处从略。

事实上,也可以考虑\textbf{群的同调}、\textbf{李代数上同调},
它们也有对应的Hochschild同调、上同调。








