\chapter{非交换代数}
我们需要\textbf{代数拓扑}、\textbf{同调代数}的预备知识,
并且采用同调代数的标准术语、记号,
诸如链复形、上同调、导出函子等等。
首先介绍基本的记号与概念。

在本课,我们给定一个特征$0$的含幺交换环$K$(例如一个域),
考虑含幺结合$K-$代数$A$(注意$A$未必是交换代数),
则$A$首先为交换环$K$上的模。
$A$的$K-$代数结构给出如下$K-$模同态:
\begin{eqnarray*}
A\otimes_KA        &\rightarrow& A\\
(a_1,a_2)          &\mapsto    & a_1a_2
\end{eqnarray*}
由$A$的结合性,$(a_1a_2)a_3=a_1(a_2a_3)$
对$A$中任意元素$a_1,a_2,a_3$成立.

对于含幺结合$K$-代数$A$,
回顾$A$的\textbf{反代数}(opposite algebra)
\index{opposite algebra\kong 反代数}
$A\op$.反代数$A\op$作为$K$-模与$A$完全相同,记号如下:
\begin{eqnarray*}
\text{id}:A &\rightarrow & A\op\\
x&\mapsto&x\op
\end{eqnarray*}
但是$A\op$具有与$A$“相反”的乘法,
具体地,对于$A\op$中的元素$x\op,y\op$,
成立
$$x\op y\op:=(yx)\op$$

\begin{definition}
对于含幺结合$K$-代数$A$,我们定义$K-$代数$A^e$为
$$A^e:=A\ten_K A\op$$
即$A$与$A\op$的$K-$代数张量积。
\end{definition}

容易验证对于任何两个含幺结合$K-$代数$A,B$,总有
$$(A\ten_K B)\op=A\op\ten_{K}B\op$$
从而容易得到
$$(A\op)^e=(A^e)\op$$

对于$K-$代数$A$,回顾
\textbf{双$A-$模}($A$-bimodule)的概念如下:

\begin{definition}对于$K-$代数$A$,
双$A-$模是指如下资料:

(1)$K$-模$M$;

(2)$A$在$M$上的左、右$K-$线性作用,

并且满足相容性:$(a.m).b=a.(m.b)$对任意$m\in M$以及$a,b\in A$成立。
\end{definition}

例如,$A$本身自然有双$A-$模结构,
$A$在其上的左、右作用即为左乘、右乘。
再比如$K-$模张量积$A\ten_KA$具有如下双$A-$模结构:
$$b.(a_1\ten a_2):=(ba_1)\ten a_2$$
$$(a_1\ten a_2).b:=a_1\ten(a_2b)$$
其中$a_1,a_2,b\in A$.

我们不再回顾左模、右模的概念了,
也不去回顾右模与左模的平衡张量积。

\begin{prop}设$M$为双$A-$模,

(1)$M$可自然地视为左$A^e$-模:
$$(a_1\ten a_2\op).m=a_1.m.a_2$$

(2)$M$可自然地视为右$A^e$-模:
$$m.(a_1\ten a_2\op)=a_2.m.a_1$$
反之,左(右)$A^e$模也可视为双$A-$模。
\end{prop}

\begin{proof}
容易验证。
\end{proof}

特别地如果$M,N$都是双$A-$模,那么考虑平衡张量积
$M\ten_{A^e}N$,它的双$A-$模结构具体如下:
$$a.(m\ten n)=(a.m)\ten n=m\ten(n.a)$$
$$(m\ten n).b=m\ten (n.b)=(b.m)\ten n$$
对于任何$m\in M,n\in N,a,b\in A$成立。

\begin{definition}(余中心 cocenter)
\index{cocenter\kong 余中心}
对于双$A-$模$M$,称双$A-$模
$$M\ten_{A^e}A$$
为$M$的\textbf{余中心}(cocenter)。
\end{definition}

容易看出,对任意的$m\in M,a\in A$,
在余中心$M\ten_{A^e}A$当中,成立
$$(m.a)\ten 1=m\ten(a.1)=m\ten a=m\ten(1.a)=(a.m)\ten 1$$
从而$(m.a-a.m)\ten 1=0$.
事实上,$M$的余中心具有如下结构:

\begin{prop}
对于双$A-$模$M$,则有如下双$A-$模同构
$$M\ten_{A^e}A\cong M/\{(m.a-a.m)|a\in A,m\in M\}$$
\end{prop}

\begin{proof}
考虑如下的双$A-$模链复形
$$\p\downdot:A\ten A\ten A\ra A\ten A\ra A\ra 0$$
其中
\begin{eqnarray*}
\p:a_1\ten a_2\ten a_3 &\mapsto& a_1a_2\ten a_3-a_1\ten a_2a_3\\
a_1\ten a_2&\mapsto&a_1a_2
\end{eqnarray*}
容易验证$\p^2$=0(由$A$的结合性),
从而$\p\downdot$为双$A-$模链复形。
并且显然$\p:A\ten A\ra A$是满同态。

断言链复形$\p\downdot$为正合(exact)的。
\index{exact\kong 正合}
事实上,$\p\downdot$到其自身的恒等链映射与零链映射是链同伦的。
我们构造如下的链同伦$h\downdot$:
\begin{eqnarray*}
h:a_1&\mapsto&1\ten a_1\\
a_1\ten a_2&\mapsto&1\ten a_1\ten a_2
\end{eqnarray*}
容易验证,对于任意的$\varphi=a_1\ten a_2\in A\ten A$,成立
\begin{eqnarray*}
   (\p h+h\p)\varphi
&=&(\p h+h\p)(a_1\ten a_2)\\
&=&\p(1\ten a_1\ten a_2)+h(a_1a_2)\\
&=&a_1\ten a_2-1\ten a_1a_2+1\ten a_1a_2\\
&=&a_1\ten a_2=\varphi\\
\end{eqnarray*}
从而对于$\varphi\in A\ten A$,
如果$\p\varphi=0$,那么
$$\varphi=(\p h+h\p)\varphi=\p(h\varphi)$$
这说明链复形$\p\downdot$在$A\ten A$处正合,
因此$\p\downdot$是正合的。\vs

接下来,将函子$M\ten_{A^e}-$作用于链复形
$\p\downdot$,得到如下的双$A-$模链复形:
$$M\ten_{A^e}\p\downdot:
M\ten A\ra M\ra M\ten_{A^e}A\ra 0$$
由张量函子的右正合性,上述链复形也是正合的。
其中注意到双$A-$模同构
\begin{eqnarray*}
M\ten_{A^e}(A\ten A\ten A)&\cong& M\ten A\\
m\ten(a_1\ten a_2\ten a_3)&\mapsto&(a_3.m.a_1)\ten a_2
\end{eqnarray*}
以及双$A-$模同构
\begin{eqnarray*}
M\ten_{A^e}(A\ten A)&\cong& M\\
m\ten(a_1\ten a_2)&\mapsto& a_2.m.a_1
\end{eqnarray*}

于是正合列$M\ten_{A^e}\p$的边界映射有如下具体表达式:

\begin{eqnarray*}
M\ten_{A^e}\p\downdot:
M\ten A&\to& M\\
m\ten A&\mapsto&m.a-a.m
\end{eqnarray*}

从而由正合性,易知
$$M\ten_{A^e}A\cong M/\{(m.a-a.m)|a\in A,m\in M\}$$
\end{proof}

可见,$M$的余中心无非是$M$当中“非交换的部分”
商掉之后所得到的“交换的部分”,如此望文生义。
例如,如果$A$为交换$K-$代数,
那么$A$本身作为双$A-$模,其余中心为$A$本身.

\begin{definition}(Hochschild 同调)

对于双$A-$模$M$,以及非负整数$n$,记
$$H_n(A,M):=\Tor_n^{A^e}(M,A)$$
称为$M$的第$n$个Hochschild同调。特别地,我们记
$$\HH_n(A):=H_n(A,A)$$
\end{definition}

由定义以及导出函子的基础知识,容易知道
双$A-$模$M$的第$0$个Hochschild同调
$$H_0(A,M)=M\ten_{A^e}A=M/\{(m.a-a.m)|a\in A,m\in M\}$$
正是$M$的余中心。









