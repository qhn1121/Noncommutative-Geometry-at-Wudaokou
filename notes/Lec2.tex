\chapter{循环同调}

与上一章一样,我们仍假设$K$为特征零的含幺交换环,
$A$为$K$-代数,且作为$K$-模是投射的。
不过,在本章我们新增一条假定:
$$\bbQ\inj K$$
也就是说,有理数域能够嵌入到$K$中。
(事实上,任何特征零的域都满足此假定。)

\section{Connes复形$C^{\lmd}\downdot(A)$}
%Cyclic homology%
%$\HH\downdot(A)\rightsquigarrow$noncommutative differential forms.
%$$B\dashleftarrow\text{de rham}$$

回顾对于$K$-代数$A$,若$A$交换,
则其Hochschild同调$\HH\downdot(A)$
可以被理解为“空间”$A$上的“微分形式”。
本节我们进一步研究$\HH\downdot(A)$.

\begin{notation}
对于$K$-代数$A$,双$A$-模$M=A$.
考虑其Hochschild链复形$C\downdot(A):=C\downdot(A,A)$:
$$C_n(A):=C_n(A,A)\cong A^{\ten n+1}$$
(回顾定义\ref{Hochschild链复形-def}).我们考虑群
$\bbZ/(n+1)\bbZ$在$C_n(A)$上的如下左$K$-作用:
记$\bbZ/(n+1)\bbZ$的生成元为$\lmd$,则
\begin{eqnarray*}
\lmd:C_n(A)&\to&C_n(A)\\
a_0\ten a_1\ten\cdots\ten a_n
&\mapsto&
(-1)^na_n\ten a_0\ten\cdots\ten a_{n-1}
\end{eqnarray*}
考虑$C_n(A)$模掉此群作用,所得的商$K$-模记为
$$C^{\lmd}_n(A):=C_n(A)/(1-\lmd)$$
其中的元素称之为\textbf{循环余不变量}
(cyclic co-invariant)。
\index{cyclic co-invariant\kong 循环余不变量}
\end{notation}

%cyclic group action
%$$C\downdot(A,A)=C\downdot(A)=A\ten A$$
%$$C_n(A)=A^{\ten n+1}$$
%we consider the $\bbZ/(n+1)\bbZ$ acting on $C_n A$,
%$$\lmd: C_n(A)\to C_n(A)$$
%$$a_0\ten a_1\ten...\ten a_n\mapsto
%(-1)^na_n
%\ten a_0\ten...\ten a_{n-1}$$
%then $\lmd$ is generator,
%$$\lmd^{n+1}=(-1)^{n(n+1)}=1$$
%Let
%$$C^{\lmd}\downdot(A):=C\downdot(A)/(1-\lmd)$$
%cyclic co-invariant.

容易验证,
$$\lmd^{n+1}a_0\ten a_1\ten\cdots\ten a_n=
(-1)^{n(n+1)}a_0\ten a_1\ten\cdots\ten a_n
=a_0\ten a_1\ten\cdots\ten a_n$$
即$\lmd^{n+1}=\id$.可见这的确是$\bbZ/(n+1)\bbZ$的作用。

回顾Bar-复形,我们可以直观地视为“直线上依次排列质点,相邻两两碰撞”;
而在这里,商掉$\lmd$循环作用后,直观地更像是“圆周上排列质点”。

我们将说明,Hochschild链复形$C\downdot(A)$的边缘算子$b$,
沿商映射$C\downdot(A)\surj C^{\lmd}\downdot(A)$下降,诱导了
$C\downdot^{\lmd}(A)$的链复形结构(称之为Connes复形)。

%\begin{prop}
%对于$K$代数$A$,定义$b‘:C_{n}(A)\to C_{n-1}(A)$诱导了
%$C^{\lmd}\downdot$的链复形结构。
%Hochschild differential $b$(on $C\downdot(A)$) induces a
%differential on $C\downdot(A)/(1-\lmd)$
%\end{prop}

%\begin{proof}

%事实上,我们定义$b‘:C^{\lmd}_n(A)\to C_n^{\lmd}(A)$如下:
%\begin{eqnarray*}
%b’:C^{\lmd}_n(A)&\to& C_n^{\lmd}(A)\\
%\overline{
%   a_0\ten a_1\ten\cdots\ten a_n
%         }
%&\mapsto&
%   \sum_{k=0}^{n-1}(-1)^k
%       \overline{
%         a_0\ten\cdots\ten(a_ka_{k+1})\ten\cdots a_n
%       }\\
%&&
%   +(-1)^n\overline{
%     (a_{n+1}a_0)\ten a_1\ten\cdots\ten a_n
%         }
%\end{eqnarray*}
%即$b:C_n(A)\to C_{n-1}(A)$沿商映射的下降。
%(回顾Hochschild链复形的边缘算子$b$的显式表达,见
%定义\ref{Hochschild链复形-def}的下方,令$M=A$以及$m=a_0$.)

%我们需要验证如此$b'$是良定的,
%即与$\lmd$作用等价类代表元选取无关。
%{\color{purple}(直观感觉比较显然,但还是暴力验证一下吧)}

%Let us define
%$$b':C\downdot(A)\to C_{\bullet-1}(A)$$
%$$b'(a_0\ten...\ten a_n)
%=a_0a_1\ten a_2\ten...\pm...\pm a_0\ten a_1\ten...\ten a_{n-1}a_n$$

%(($C\downdot(A),b')=(B\downdot A\to A)$)

%claim that
%$$(1-\lmd)b'=b(1-\lmd)$$

%$$[b,\lmd]=(1-\lmd)(b-b')$$

%In particular, $b:\im(1-\lmd)\mapsto\im(1-\lmd)$

%"proof":
%%%%%%%%%%画圈圈%%%%%%%%%%%555

%为此只需验证对于任意$n\geq 1$,以及任意$\omg\in C_n(A)$,
%$$(1-\lmd)(\omg)=0\Rightarrow(1-\lmd)\circ b(\omg)=0$$

%\end{proof}

\begin{lemma}
对于$K$-代数$A$,我们定义算子
$b':C\downdot(A)\to C_{\bullet-1}(A)$如下:
\begin{eqnarray*}
b':C_{n}(A)&\to&C_{n-1}(A)\\
a_0\ten a_1\ten\cdots\ten a_n&\mapsto&
\sum_{k=0}^{n-1}(-1)^{k}
a_0\ten\cdots\ten(a_ka_{k+1})\ten\cdots\ten a_n
\end{eqnarray*}
则成立:

(1)$b'\circ b'=0$,

(2)对任意$n\geq 1$,则以下图表交换:
$$\xymatrix{
     C_n(A) \ar[d]^{1-\lmd} \ar[r]^{b'}
   & C_{n-1}(A)\ar[d]^{1-\lmd}
\\
     C_n(A) \ar[r]^b
   & C_{n-1}(A)
}$$
\end{lemma}

\begin{proof}
注意到有同构$C_n(A)\cong B_nA(\cong A^{\ten n+1})$,
其中$B\downdot A$为Bar-复形;容易看出这里定义的$b'$在此同构下,正是
Bar复形当中的边缘算子,从而$b'\circ b'=0$,
也就是说$(C\downdot(A),b')$是一个链复形,
并且同构于Bar-复形$(B\downdot A, b)$.
(这里有轻微的记号混用:Bar-复形$(B\downdot A, b)$当中的“$b$”
并不是本引理当中Hochschild链复形$(C\downdot(A),b)$当中的“$b$”,
前者在此是临时记号。)\vs

我们再来看(2).回顾$b:C_n(A)\to C_{n-1}(A)$的显式表达式
(见定义\ref{Hochschild链复形-def}的下方,并且令其中$M=a$以及$m=a_0$)
(注意此图中的$b$与$b'$并不是同一个映射,它们的具体表达式相差一项),
直接验算之:
\begin{eqnarray*}
& & (1-\lmd)\circ b'
        \big(
             a_0\ten a_1\ten\cdots\ten a_n
        \big)\\
&=&
    (1-\lmd)\left(
               \sum_{k=0}^{n-1}(-1)^k
                   a_0\ten\cdots\ten(a_ka_{k+1})\ten\cdots\ten a_n
            \right)\\
&=&
    \sum_{k=0}^{n-1}(-1)^k
         a_0\ten\cdots\ten(a_ka_{k+1})\ten\cdots\ten a_n\\
& &
    +\sum_{k=0}^{n-2}(-1)^{k+n}
         a_n\ten a_0\ten\cdots\ten(a_ka_{k+1})\ten\cdots\ten a_{n-1}\\
& &
    -(a_{n-1}a_n)\ten a_0\ten\cdots\ten a_{n-2}\\
&=&
    b(a_0\ten a_1\ten\cdots\ten a_n)
    -(-1)^n(a_na_0)\ten a_1\ten\cdots\ten a_{n-1}\\
& &
    +(-1)^n
    \sum_{k=0}^{n-1}(-1)^k
        a_n\ten a_0\ten\cdots\ten(a_ka_{k+1})\ten\cdots\ten a_{n-1}\\
& &
    -(a_{n-1}a_n)\ten a_0\ten\cdots\ten a_{n-2}\\
&=&
    b\big(
       a_0\ten a_1\ten\cdots\ten a_n
       -(-1)^na_n\ten a_0\ten\cdots\ten a_{n-1}
     \big)\\
&=&
    b\circ(1-\lmd)(a_0\ten a_1\ten\cdots\ten a_n)
\end{eqnarray*}

也就是说,
$$(1-\lmd)\circ b'=b\circ(1-\lmd)$$
从而此图表交换,证毕。
\end{proof}

此图表的交换关系也可改写为
$$[b,\lmd]=(1-\lmd)\circ(b-b')$$
其中$[b,\lmd]:=b\circ\lmd-\lmd\circ b$.\vs

此引理给出了链复形$(C\downdot(A),b')$与
$(C\downdot(A),b)$之间的链映射:
$$(1-\lmd)\downdot:(C\downdot(A),b')\to(C\downdot(A),b)$$
然而注意到
$$C_n^{\lmd}(A):=C_n(A)/(1-\lmd)=\coker (1-\lmd)_n$$
于是我们(在由$K$-模链复形构成范畴当中)
考虑链映射$(1-\lmd)\downdot$的余核,
这给出了$C\downdot^{\lmd}(A)$的链复形结构:

\begin{definition}(Connes复形)
\index{Connes' complex\kong Connes复形}
对于$K$-代数$A$,考虑链映射
$$(1-\lmd)\downdot:(C\downdot(A),b')\to(C\downdot(A),b)$$
的余核链复形
$$(C\downdot^{\lmd}(A),b^{\lmd}):=\coker[(1-\lmd)\downdot]$$
称其为\textbf{Connes复形}。并且记
$$H^{\lmd}\downdot(A):=H\downdot(C\downdot^{\lmd}(A))$$
称之为$A$的\textbf{循环同调}(cyclic homology).
\index{cyclic homology\kong 循环同调}
\end{definition}

也就是说,有如下的交换图表:

$$\xymatrix{
     \cdots \ar[r]
   & C_{n+1}\ar[r]^{b'}  \ar[d]^{1-\lmd}
   & C_{n  }\ar[r]^{b'}  \ar[d]^{1-\lmd}
   & C_{n-1}\ar[r]       \ar[d]^{1-\lmd}
   & \cdots
\\
     \cdots \ar[r]
   & C_{n+1}\ar[r]^{b }  \ar@{->>}[d]
   & C_{n  }\ar[r]^{b }  \ar@{->>}[d]
   & C_{n-1}\ar[r]       \ar@{->>}[d]
   & \cdots
\\
     \cdots \ar[r]
   & C_{n+1}^{\lmd}\ar[r]^{b^{\lmd}} \ar[d]
   & C_{n  }^{\lmd}\ar[r]^{b^{\lmd}} \ar[d]
   & C_{n-1}^{\lmd}\ar[r]            \ar[d]
   & \cdots
\\
   &0
   &0
   &0
   &
}$$
此交换图表每一横行都为链复形,其中第三横行为Connes复形;
每一列都是右短正合的。
并且容易知道:Connes复形的边缘算子$b^{\lmd}$
正是Hochschild链复形的边缘算子$b$沿商映射
$C\downdot(A)\surj C^{\lmd}\downdot(A)$的下降。

%so, $b$ acts on $C^{\lmd}\downdot(A)=C\downdot(A)/(1-\lmd)$
%,called \textbf{Connes' complex}
%$$H^{\lmd}\downdot(A)=H\downdot(C^{\lmd}\downdot(A),b)$$

\section{循环双复形$CC\ddowndot(A)$}

\begin{lemma}(平均算子)对于任意$K$-代数$A$,以及$n\geq 0$,
引入平均算子$\mathcal{N}:C_n(A)\to C_n(A)$:
$$\mcalN:=1+\lmd+\lmd^2+\cdots+\lmd^n$$
则此算子满足以下性质:

(1)$b'\mcalN=\mcalN b$

(2)$(1-\lmd)\mcalN=\mcalN(1-\lmd)=0$.
此外,如果有理数域$\bbQ\hookrightarrow K$,
那么对于任意$n\geq 0$,以下链复形是正合的:
$$\cdots\to
C_n(A)\xra{\mcalN}C_n(A)\xra{1-\lmd}
C_n(A)\xra{\mcalN}C_n(A)\xra{1-\lmd}
C_n(A)\surj C_n^{\lmd}(A)\to 0
$$
%Introduce
%$$\mathcal{N}:C_n(A)\to C_n(A)$$
\label{循环双复形的行正合性-lemma}
\end{lemma}
%%%%%%%%%%%平均算子%%%%%%%%%'

%\begin{prop}
%$$b'\mathcal{N}=Nb$$
%\end{prop}

%we have an exact sequence
%%%%%%%平均算子‘%%%%%%'
\begin{proof}(1)任意固定$n\geq 1$,
为了区分算子在不同空间的作用,
我们采用临时记号
$$\left\{\begin{array}{l}
\lmd:C_n(A)\to C_n(A)\\
\overline{\lmd}:C_{n-1}(A)\to C_{n-1}(A)
\end{array}\right.\quad
\left\{\begin{array}{l}
\mcalN:=1+\lmd+\cdots+\lmd^n\\
\overline{\mcalN}:=1+\overline{\lmd}+\cdots+\overline{\lmd}^{n-1}
\end{array}\right.$$
则在此记号下我们需要证$b'\mcalN=\overline{\mcalN}b$.

定义缩并算子
\begin{eqnarray*}
s:C_n(A)&\to& C_{n-1}(A)\\
a_0\ten a_1\ten\cdots\ten a_n
&\mapsto&
(a_0a_1)\ten\cdots\ten a_n
\end{eqnarray*}
则容易验证(稍微注意一下正负号,确实都是正号)
$$
b =\sum\limits_{k=0}^n    \overline{\lmd}^ks\lmd^{-k}\quad\quad
b'=\sum\limits_{k=0}^{n-1}\overline{\lmd}^ks\lmd^{-k}
$$
于是有
$$
b'\mcalN=
  \left(
    \sum_{k=0}^{n-1}
      \overline{\lmd}^ks\lmd^{-k}
  \right)
  \left(
    \sum_{l=0}^{n  }
      \lmd^l
  \right)
=
 \sum_{0\leq k\leq n-1
       \atop
       0\leq l\leq n}
     \overline{\lmd}^ks\lmd^l
$$
同理也有
$$\overline{\mcalN}b
=\sum_{0\leq k\leq n-1
       \atop
       0\leq l\leq n}
     \overline{\lmd}^ks\lmd^l$$
从而$b'\mcalN=\overline{\mcalN}b$.

(2)给定$n\geq 0$,注意到$\lmd^{n+1}=1$,从而
$$(1-\lmd)\mcalN=(1-\lmd)(1+\lmd+\cdots+\lmd^n)=1-\lmd^{n+1}=0$$
同理$\mcalN(1-\lmd)=0$.因此该图表是链复形,只需再验证正合性。

现在假设$\bbQ$是$K$的子环。我们构造如下链同伦:

$$\xymatrix{
   \cdots \ar[r]
  &C_n(A) \ar[r]^{\mcalN} \ar[d]^{\id}
  &C_n(A) \ar[r]^{1-\lmd} \ar[d]^{\id} \ar[ld]_g
  &C_n(A) \ar[r]^{\mcalN} \ar[d]^{\id} \ar[ld]_f
  &C_n(A) \ar[r]^{      } \ar[d]^{\id} \ar[ld]_g
  &\cdots
\\
   \cdots \ar[r]
  &C_n(A) \ar[r]^{\mcalN}
  &C_n(A) \ar[r]^{1-\lmd}
  &C_n(A) \ar[r]^{\mcalN}
  &C_n(A) \ar[r]^{      }
  &\cdots
}$$
其中$f,g:C_n(A)\to C_n(A)$定义为
$$\left\{\begin{array}{l}
f:=\frac{1}{n+1}(\lmd^{n-1}+2\lmd^{n-2}+3\lmd^{n-3}+\cdots+n)\\
g:=\frac{1}{n+1}
\end{array}\right.$$
(利用了$\bbQ\inj K$)则容易验证
$$f(1-\lmd)+\mcalN g=g\mcalN+(1-\lmd)f=1$$
从而证毕。
\end{proof}

特别地,当$K$为域时(注意我们总假定$\ch K=0$)成立正合性。
链同伦$f,g$的构造来自于(关于变元$\lmd$的多项式的)
欧几里得辗转相除法。

%\begin{definition}
%the cyclic bi-complex $CC\downdot(A)$
%%%%%%%双复形%%%%%%
%\end{definition}

由此引理,我们可构造出如下的\textbf{循环双复形}
(cyclic bicomplex),记为$CC\ddowndot(A)$:
\index{cyclic bicomplex\kong 循环双复形}

$$\xymatrix{
   \vdots \ar[d]^b
  &\vdots \ar[d]^{-b'}
  &\vdots \ar[d]^b
  &\vdots \ar[d]^{-b'}
  &
\\
   C_2(A) \ar[d]^b
  &C_2(A) \ar[d]^{-b'} \ar[l]_{1-\lmd}
  &C_2(A) \ar[d]^b     \ar[l]_{\mcalN}
  &C_2(A) \ar[d]^{-b'} \ar[l]_{1-\lmd}
  &\cdots              \ar[l]
\\
   C_1(A) \ar[d]^b
  &C_1(A) \ar[d]^{-b'} \ar[l]_{1-\lmd}
  &C_1(A) \ar[d]^b     \ar[l]_{\mcalN}
  &C_1(A) \ar[d]^{-b'} \ar[l]_{1-\lmd}
  &\cdots              \ar[l]
\\
   C_0(A)
  &C_0(A)   \ar[l]_{1-\lmd}
  &C_0(A)   \ar[l]_{\mcalN}
  &C_0(A)   \ar[l]_{1-\lmd}
  &\cdots   \ar[l]
}$$
其中对于任意$p,q\geq 0$,
$CC_{p,q}(A)=C_p(A)$为该图表的从下往上第$p$行,
从左往右第$q$列的节点;此图表的偶数列与奇数列为
$(C\downdot (A),b)$与$(C\downdot (A),-b')$交替。
并且注意到,此图表不是交换的,而是对于其中每一个方框
都满足\textbf{反交换性}。\vs

我们回顾一些同调代数工具:

\begin{definition}(双复形的全复形)

对于任意的含幺交换环K(这里暂时不必假定$\ch K=0$),
以及$K$-模双复形$(A\ddowndot,\td,\p)$:
$$\xymatrix{
    \vdots  \ar[d]
  & \vdots  \ar[d]
  & \vdots  \ar[d]
  &
\\
    A_{2,0} \ar[d]^{\td_{2,0}}
  & A_{2,1} \ar[d]^{\td_{2,1}}  \ar[l]_{\p_{2,1}}
  & A_{2,2} \ar[d]^{\td_{2,2}}  \ar[l]_{\p_{2,2}}
  & \cdots                      \ar[l]
\\
    A_{1,0} \ar[d]^{\td_{1,0}}
  & A_{1,1} \ar[d]^{\td_{1,1}}  \ar[l]_{\p_{1,1}}
  & A_{1,2} \ar[d]^{\td_{1,2}}  \ar[l]_{\p_{1,2}}
  & \cdots                      \ar[l]
\\
    A_{0,0}
  & A_{0,1}   \ar[l]_{\p_{2,1}}
  & A_{0,2}   \ar[l]_{\p_{2,2}}
  & \cdots    \ar[l]
}$$
即:
$$\left\{\begin{array}{l}
\td_{p,q}:A_{p,q}\to A_{p-1,q}\\
\p_{p,q }:A_{p,q}\to A_{p,q-1}
\end{array}\right.
$$
使得该图表每一行、每一列都是链复形,并且满足反交换关系
$$\p_{p-1,q}\circ\td_{p,q}+\td_{p,q-1}\circ\p_{p,q}=0$$

则我们定义双复形$A\ddowndot$的\textbf{全复形}(total complex)
\index{total complex\kong 全复形}
$(\Tot\downdot(A\ddowndot),d)$如下:
$$\left\{\begin{array}{rcl}
\Tot_n(A\ddowndot)&:=&\bigoplus\limits_{p+q=n}A_{p,q}\\
d_n&:=&\sum\limits_{p+q=n}(\td_{p,q}+\p_{p,q})
\end{array}\right.$$
\end{definition}

对于两个双复形$A\ddowndot$与$A'\ddowndot$,
我们可以去定义双复形之间的态射$f\ddowndot:A\ddowndot\to A'\ddowndot$,
进而考虑双复形范畴。双复形的态射自然诱导了相应的全复形之间的链映射,
也就是说$\Tot$具有函子性。我们还有以下同调代数工具:

\begin{lemma}设$f\ddowndot:A\ddowndot\to A'\ddowndot$为双复形之间的态射。
如果对于任意$n\geq 0$,链映射
$$f_{n,\bullet}:A_{n,\bullet}\to A'_{n,\bullet}$$
为拟同构(quasi-isomorphism),
\index{quasi-isomorphism\kong 拟同构}
(即它诱导的任意阶同调对象之间的态射均为同构),
那么链映射
$$\Tot\downdot(f\ddowndot):
\Tot\downdot(A\ddowndot)\to\Tot\downdot(A'\ddowndot)$$
也为拟同构。
\label{双复形的全复形-同调代数机器-lemma}
\end{lemma}

\begin{proof}
同调代数工具,承认之。
\end{proof}

我们回到循环双复形$CC\ddowndot(A)$.由上述同调代数工具,
我们可以给出循环同调
$H^{\lmd}\downdot(A):=H\downdot(C^{\lmd}\downdot(A))$
的另一种定义:

\begin{thm}对于$K$-代数$A$,假设$\bbQ\inj K$,记
$$HC\downdot(A):=H\downdot(\Tot\downdot(CC\ddowndot(A)))$$
为$A$的循环双复形的全复形的同调,那么有自然的同构
$$HC\downdot(A)\cong H^{\lmd}\downdot(A)$$
\end{thm}

\begin{proof}
对于循环双复形$CC\ddowndot(A)$,
我们再考虑另一个双复形$CC'\ddowndot(A)$如下:
$$\xymatrix{
   \vdots \ar[d]
  &\vdots \ar[d]
  &\vdots \ar[d]
  &\vdots \ar[d]
  &
\\
   C_2^{\lmd}(A)  \ar[d]^{b_2^{\lmd}}
  &0 \ar[d]  \ar[l]
  &0 \ar[d]  \ar[l]
  &0 \ar[d]  \ar[l]
  &\cdots    \ar[l]
\\
   C_1^{\lmd}(A)  \ar[d]^{b_1^{\lmd}}
  &0 \ar[d]  \ar[l]
  &0 \ar[d]  \ar[l]
  &0 \ar[d]  \ar[l]
  &\cdots    \ar[l]
\\
   C_0^{\lmd}(A)
  &0  \ar[l]
  &0  \ar[l]
  &0  \ar[l]
  &\cdots    \ar[l]
}$$
考虑双复形之间的态射
$$f\ddowndot:CC\ddowndot(A)\to CC'\ddowndot(A)$$
其中$f_{n,0}:C_n(A)\to C_n^{\lmd}(A)$为商映射。
由引理\ref{循环双复形的行正合性-lemma}
知$CC\ddowndot(A)$的每一行都是正合的,
从而容易验证$f\ddowndot$满足引理
\ref{双复形的全复形-同调代数机器-lemma}
的使用条件,因此我们有同构
$$
H\downdot(\Tot\downdot(CC\ddowndot(A)))\cong
H\downdot(\Tot\downdot(CC'\ddowndot(A)))
$$
上式左边,由定义,即为$HC\downdot(A)$;
而再注意到$\Tot\downdot(CC'\ddowndot)$正是Connes复形$C^{\lmd}\downdot$,
从而上式右边为循环同调$H^{\lmd}\downdot(A)$.
\end{proof}

也就是说,循环同调(Connes复形的同调)
自然同构于循环双复形的全复形的同调。

%Let $\text{Tot}(CC\downdot(A))$ be the total complex of $CC\downdot(A)$
%with a natural map of complexs
%$$\text{Tot}(CC\downdot(A))\to C^{\lmd}\downdot(A)$$
%is a quasi-isomorphism.
%$$HC\downdot(A):=H\downdot(\text{Tot}(CC\downdot(A)))
%\cong H^{\lmd}\downdot(A)$$

%Intuitively, we can delete "exact sequence" and get rid of 蓝色框框,

%%%%%%%同调代数工具%%%%%%%

\section{Connes算子$\mcalB$}

%%%%%%%%%%%%%%%%2019.3.5 TUE 第二周%%%%%%%%%%%%%%%%%
我们将给出循环同调的更多等价定义方式,
并计算一些具体例子。
本节均假定$\bbQ\inj K$(甚至直接把$K$当成特征零的域)。
我们需要更多的同调代数工具:

\begin{lemma}(杀掉可缩复形)
\label{杀掉可缩复形引理-lemma}

对于$K$-模链复形
$$\cdots\to
A_{n+1}\oplus B_{n+1}\xra{\td}
A_n\oplus B_n\xra{\td}
A_{n-1}\oplus B_{n-1}\to\cdots$$
其中
$$\td=
\begin{pmatrix}
\alpha & \beta \\
\gamma & \delta
\end{pmatrix}$$
并且$(B\downdot,\delta)$是可缩链复形,其同伦逆
$$h:B\downdot\to B_{\bullet+1}$$
使得$h\delta+\delta h=1$.那么下述图表交换:

$$
\xymatrix{
   \cdots   \ar[r]
 & A_{n+1}  \ar[r]^{\alpha-\beta h\gamma} \ar[d]^\fai
 & A_{n  }  \ar[r]^{\alpha-\beta h\gamma} \ar[d]^\fai
 & A_{n-1}  \ar[r]^{                    } \ar[d]^\fai
 & \cdots
\\
   \cdots                 \ar[r]
 & A_{n+1}\oplus B_{n+1}  \ar[r]^{\td}
 & A_{n  }\oplus B_{n  }  \ar[r]^{\td}
 & A_{n-1}\oplus B_{n-1}  \ar[r]
 & \cdots
}
$$
并且此图表的每一行都为链复形,并且链映射
$$\fai:=
\begin{pmatrix}
1\\
-h\gamma
\end{pmatrix}$$
为拟同构。
\end{lemma}

\begin{proof}
注意到$\delta^2=0$,以及
$$0=\td^2
   =\begin{pmatrix}
       \alpha &\beta\\
       \gamma &\delta
    \end{pmatrix}^2
   =
    \begin{pmatrix}
       \alpha^2+\beta\gamma
      &\alpha\beta+\beta\gamma\\
       \gamma\alpha+\delta\gamma
      &\gamma\beta+\delta^2
    \end{pmatrix}
$$
从而我们有
$$
\left\{\begin{array}{rcl}
\alpha^2&=&-\beta\gamma\\
\alpha\beta&=&-\beta\delta\\
\gamma\alpha&=&-\delta\gamma\\
\gamma\beta&=&0
\end{array}\right.
$$
再注意到$h\delta+\delta h=1$,
直接计算验证可知$\fai\downdot$
的确为链复形之间的链映射。细节略。

再注意链映射
$$\fai\downdot:
(A\downdot,\alpha-\beta h\gamma)\to
(A\downdot\oplus B\downdot,\td)$$
为单射,并且其余核
$$\coker\fai\downdot\cong
(B\downdot,\delta)$$
是正合的,因此$\fai\downdot$为拟同构。
\end{proof}

这个引理的功能是,如果给定的链复形
$(A\downdot\oplus B\downdot,\td)$当中
“含有正合的部分”$(B\downdot,\delta)$,
那我们可以把这个“正合的部分”剔除掉,
得到一个“不那么冗余”的链复形$(A\downdot,\alpha-\beta h\delta)$,
并且此复形与原来的复形的各阶同调自然同构。\vs

我们将此引理用于循环双复形$CC\ddowndot(A)$
的全复形$\Tot\downdot(CC\ddowndot(A))$上。
回顾$CC\ddowndot(A)$为如下双复形:

$$\xymatrix{
   \vdots \ar[d]^b
  &\vdots \ar[d]^{-b'}
  &\vdots \ar[d]^b
  &\vdots \ar[d]^{-b'}
  &
\\
   C_2(A) \ar[d]^b
  &C_2(A) \ar[d]^{-b'} \ar[l]_{1-\lmd}
  &C_2(A) \ar[d]^b     \ar[l]_{\mcalN}
  &C_2(A) \ar[d]^{-b'} \ar[l]_{1-\lmd}
  &\cdots              \ar[l]
\\
   C_1(A) \ar[d]^b
  &C_1(A) \ar[d]^{-b'} \ar[l]_{1-\lmd}
  &C_1(A) \ar[d]^b     \ar[l]_{\mcalN}
  &C_1(A) \ar[d]^{-b'} \ar[l]_{1-\lmd}
  &\cdots              \ar[l]
\\
   C_0(A)
  &C_0(A)   \ar[l]_{1-\lmd}
  &C_0(A)   \ar[l]_{\mcalN}
  &C_0(A)   \ar[l]_{1-\lmd}
  &\cdots   \ar[l]
}$$

注意到该双复形的第偶数列为Hochschild链复形(链映射$b$),
第奇数列为Bar-复形(链映射$-b'$).
注意Bar-复形是正合的,并且有同伦逆
\begin{eqnarray}
h:C_n(A)&\to&C_{n+1}(A)\\
a_0\ten a_{1}\ten\cdots\ten a_{n}&\mapsto&
1\ten a_0\ten a_{1}\ten\cdots\ten a_{n}
\end{eqnarray}
使得$b'h+hb'=1$.

现在,注意到
\begin{eqnarray*}
\Tot_n(CC\ddowndot(A))&=&
  \left(
    \bigoplus_{p+q=n\atop\text{$q$为偶数}}
      CC_{p,q}(A)
  \right)
  \oplus
  \left(
    \bigoplus_{p+q=n\atop\text{$q$为奇数}}
      CC_{p,q}(A)
  \right)\\
&=:&
  X_n\oplus Y_n
\end{eqnarray*}
也就是说,我们把循环双复形$CC\ddowndot(A)$
的全复形$(\Tot\downdot(CC\ddowndot(A)),\td)$写为:
$$\cdots\to
X_{n+1}\oplus Y_{n+1}\xra{\td}
X_n\oplus Y_n\xra{\td}
X_{n-1}\oplus Y_{n-1}\to\cdots$$
边缘算子矩阵$\td=
\begin{pmatrix}
\alpha  &\beta\\
\gamma  &\delta
\end{pmatrix}$
留给读者。但是要注意$(Y\downdot,\delta)$的正合性
是由Bar-复形$(C\downdot(A),-b')$的正合性所诱导的;
$\delta$也存在同伦逆,仍记为$h$.

综上,对$\Tot\downdot(CC\ddowndot(A))$
使用引理\ref{杀掉可缩复形引理-lemma},我们得到以下结果:

\begin{prop}
对于$K$-代数$A$,考虑以下双复形$\mcalB\ddowndot(A)$:
$$\xymatrix{
    \vdots \ar[d]
  & \vdots \ar[d]
  & \vdots \ar[d]
\\
    A^{\ten 3}  \ar[d]^b
  & A^{\ten 2}  \ar[d]^b  \ar[l]_{\mcalB}
  & A^{      }            \ar[l]_{\mcalB}
\\
    A^{\ten 2}  \ar[d]^b
  & A^{      }            \ar[l]_{\mcalB}
  &
\\
    A
  &
  &
}$$
此图表的最左下角为第$0$行$0$列,右下角空白处都为$0$,具体地,
$$
\mcalB_{p,q}(A)=
\left\{\begin{array}{ll}
   CC_{p-q,2q}(A) & p\geq q\\
   0              & p<    q
\end{array}\right.
$$
(也就是说,$\mcalB\ddowndot$的结点是由
将循环双复形$CC\ddowndot(A)$的第奇数列(Bar-复形)
都删掉,再将原来第$2l$列整体向左、上各平移$l$格所得)
其中\textbf{Connes算子}$\mcalB:C_n(A)\to C_{n+1}(A)$
\index{Connes' operator\kong Connes算子}
定义为以下的复合:
$$\xymatrix{
    C_{n+1}(A) \ar[d]^b
  & C_{n+1}(A) \ar[d]^{-b'} \ar@/_/[l]_{1-\lmd}
  & C_{n+1}(A) \ar[d]^{b  } \ar[l]_{\mcalN}
\\
    C_{n  }(A)
  & C_{n  }(A) \ar[l]_{1-\lmd} \ar@/^/[u]^{h}
  & C_{n  }(A) \ar@/^/[l]_{\mcalN}
}$$

$$\mcalB:=(1-\lmd)h \mcalN$$
那么,存在自然的双复形单同态
$$\mcalB\ddowndot(A)\inj CC\ddowndot(A)$$
并且其诱导的全复形的链映射
$$\Tot\downdot(\mcalB\ddowndot(A))
\inj\Tot\downdot(CC\ddowndot(A))$$
为拟同构。
\label{Connes算子与循环同调-prop}
\end{prop}

\begin{proof}
只需注意到
$$\Tot_n(\mcalB\ddowndot)
=\left(
    \bigoplus_{p+q=n\atop\text{$q$为偶数}}
      CC_{p,q}(A)
  \right)
\inj
 \Tot_n(CC\ddowndot(A))$$
直接使用引理\ref{杀掉可缩复形引理-lemma},细节从略。
但是要验证$\mcalB\ddowndot(A)$的确是双复形,
即需要验证反交换关系
$$\mcalB\circ b+b\circ\mcalB$$
而这是容易的,验证如下:
\begin{eqnarray*}
\mcalB\circ b &=& (1-\lmd)h\mcalN b
               =  (1-\lmd)hb'\mcalN\\
              &=& (1-\lmd)(1-b'h)\mcalN
               =  (1-\lmd)\mcalN-(1-\lmd)b'h\mcalN\\
              &=& -b(1-\lmd)h\mcalN
               =  -b\circ\mcalB
\end{eqnarray*}
从而证毕。
\end{proof}

于是我们得到循环同调的又一等价定义:
$$H^{\lmd}\downdot(A)\cong
H\downdot(\Tot\downdot(\mcalB\ddowndot))$$

我们可以将链复形$\Tot\downdot(\mcalB\ddowndot)$
适当改写,使得形式更加美观:


%\textbf{Cyclic v.s. de Rham}

%Last time "Cyclic homology"

%$$\lmd:a_0\ten...\ten a_n\mapsto(-1)^n a_n\ten a_1\ten...\ten a_{n+1}$$

%$H^{\lmd}\downdot(A)=H\downdot(C\downdot(A)/(1-\lmd),b)$

%$\mcalN:C_n(A)\to C_n(A)$

%$\mcalN=1+\lmd+...+\lmd^n$

%$HC\downdot(A)=\text{Tot}\text{(cyclic bi-complex)}$
%%%%昨天的双复形%%%%%

%Today:

%\begin{prop}
%$$B\circ b+b\circ B=0$$
%where $B$ is Connes' operator
%\end{prop}

%Easy to check.

\begin{prop}对于$K$-代数$A$,以及形式变元$u$,考虑如下链复形:
$$(CC\downdot(A),b+u\mcalB)$$
其中
$$CC_n(A):=(C\downdot(A)[u^{-1}])_n
:=\bigoplus_{k=0}^{\infty}
   u^{-k}C_{n-2k}(A)$$
(注意这是有限直和)换句话说,我们给定以下分次
$$\deg(b)=-1,\quad\deg(B)=1,\quad\deg(u)=-2$$

那么此链复形的同调自然同构于循环同调:
$$H\downdot(CC\downdot(A),b+uB)\cong H\downdot^{\lmd}(A)$$
%$(b+uB)^2=0$
%where $u$ is a formal variable.
\end{prop}

\begin{proof}
这个几乎显然。注意到
\begin{eqnarray*}
  \Tot_n(\mcalB\ddowndot(A))&=&
  \bigoplus_{k=0}^{\infty}
    \mcalB_{n-k,k}(A)
=
  \bigoplus_{k=0}^{\infty}
    C_{n-2k}(A)
\\
  CC_n(A)&=&
  \bigoplus_{k=0}^{\infty}
    u^{-k}C_{n-2k}(A)
\end{eqnarray*}
于是有自然的链复形同构
\begin{eqnarray*}
  \Tot\downdot(\mcalB\ddowndot(A))&\to& CC\downdot(A)\\
  \mcalB_{n-k,k}(A)&\mapsto& u^{-k}C_{n-2k}(A)
\end{eqnarray*}
容易验证此对应也保持相应的边缘算子。证毕。
\end{proof}

注意,我们还可以考虑$(CC\downdot(A),b)$,
它与$(CC\downdot(A),b+u\mcalB)$具有不同的边缘算子:
前者的同调我们早已知道是Hochschild同调,
而后者的同调为循环同调。

\begin{rem}(复几何的背景)

%\textbf{Hodge theory}

%A complex m.f.d., $X$, complex(or algebraic) geometry,
%has an operator $\pbar$;

%topology: consider de-rham operator $\td=\p+\pbar$.

%Consider $\td_u:=\pbar+u\p$, called "Hodge filtration".
%A bridge between "topology" and "complex geometry".

%"Stability condition".

对于复流形$X$,它作为光滑流形,有外微分算子$\td$;
再注意到它的复结构,有算子$\pbar\,$——
前者代表拓扑,而后代表复几何。它们之间有关系
$$\td=\pbar+\p$$
并且满足
$$\p^2=\pbar^2=0\quad\p\pbar+\pbar\p=0$$

我们考虑以下“拓扑与复几何之间的桥梁”:
$$\td_u:=\pbar+u\p$$
称此算子为霍奇滤链(Hodge filtration)
\index{Hodge filtration\kong 霍奇滤链},
其中$0\leq u\leq 1$.注意$\td_u$满足稳定性条件
$\td_u^2=0$,即$\pbar$与$\td$
的“过渡”的任何一个“中间状态”都仍为外微分算子。
\end{rem}

{\color{gray}
所以,似乎可以如下粗暴地对应?

$$\begin{tabular}{|c|c|}
\hline
  复几何                  &   非交换几何       \\
\hline
  复流形$X$               &   $K$-代数$A$      \\
  $\Omg_X\updot$          &   $CC\downdot(A)$  \\
  $\pbar$                 &   $b$              \\
  $\p$                    &   $u\mcalB$        \\
  $\td$                   &   $b+u\mcalB$      \\
  $H_{\text{DR}}\updot(X)$&   $H^{\lmd}\downdot(A)$\\
  $H\updot_{\pbar}(X)$    &   $\HH\downdot(A)$ \\
  \hline
\end{tabular}$$

这格表格似乎不太对吧,应该是
Hochschild同调$\HH\downdot(A)$对应于“非交换版本的”
微分形式$\Omg\updot$,从之前的例子能看出来。
}

\begin{definition}(周期循环同调与负循环同调)
对于$K$-代数$A$,与$CC\downdot(A)$类似,我们还可以去定义以下:

%\textbf{ homology}

(1)定义\textbf{周期循环复形}(periodic cyclic complex)
\index{periodic cyclic complex\kong 周期循环复形}
$$CC^{\per}\downdot(A):=
(C\downdot(A)(\!(u)\!),b+u\mcalB)$$
该复形的同调
$$\HC^{\per}\downdot(A):=H\downdot(CC^{\per}\downdot(A),b+uB)$$
%is called periodic cyclic homology.
称之为\textbf{周期循环同调}(periodic cyclic homology)。\vs

(2)定义\textbf{负循环复形}(negative cyclic complex)
\index{negative cyclic complex\kong 负循环复形}
$$CC^-\downdot(A):=(C\downdot(A)[\![u]\!],b+uB)$$
该复形的同调
$$\HC^-\downdot(A):=H\downdot(CC^{-}\downdot(A),b+uB)$$
称之为\textbf{负循环同调}(negative cyclic homology)。
\end{definition}

注意上述定义当中的“$[\![u]\!]$”
是指关于形式变元$u$的形式幂级数,
而“$(\!(u)\!)$”为关于$u$的Laurent级数。由定义,显然有
$$CC\downdot(A)\cong CC^{\per}\downdot(A)/CC^-\downdot(A)$$

至此,我们定义出了$CC\downdot(A),
CC\downdot^{\per}(A)$以及$CC\downdot^-(A)$。
事实上,这三者都有深刻的物理背景,见下表:
$$
  \begin{tabular}{|c|l|l|}
  \hline
    非交换几何中的对象    &   几何、物理背景            &  几何、物理背景     \\
  \hline
    $CC\downdot^{per}(A)$ &   open-closed string states &  de-Rham cohomology \\
    $CC\downdot(A)$       &   open string states        &  gauge theory       \\
    $CC^-\downdot(A)$     &   closed string states      &  gravity            \\
  \hline
  \end{tabular}
$$
其中特别注意,周期循环同调是de-Rham上同调的“非交换版本”,
我们将在后文举例说明。
%$$CC\downdot^{per}\rightsquigarrow\text{}
%\text{open-closed string states}$$

%$$CC\downdot(A)\rightsquigarrow
%\text{open string states}\leftrightsquigarrow\text{gauge theory}$$

%$$CC^-\downdot(A)\rightsquigarrow
%\text{closed string states}\leftrightsquigarrow\text{gravity}$$

\section{一些例子}

%Why "$B$" is like de-rham differential?

回顾约化Bar-复形$\overline{B}\downdot(A)$
(见定义\ref{约化Bar复形-def}),
我们可以类似地通过约化Bar-复形来构造类似的“循环双复形”:
在$K\inj A$的条件下,考虑约化Hochschild链复形
$$\overline{C}_n(A):=
\overline{C}_n(A,A)\cong A\ten\overline{A}^{\ten n}$$
{\color{gray}
类似去定义循环算子$\lmd:\overline{C}_n(A)\to\overline{C}_n(A)$,
其显式表达式与非约化情形完全相同;以及平均算子
$$\mcalN:\overline{C}_n(A,A)\cong A\ten\overline{A}^{\ten n}$$
}
{\color{red}
可惜是错的,类似于此前的$\lmd,\mcalN$并不良定。比如
$$0=\lmd(0)=\lmd(a_0\ten \overline{1})=-1\ten\overline{\lmd}\neq 0$$
}

但是,Connes算子$\mcalB:\overline{C}_n(A)\to\overline{C}_{n+1}(A)$
是有意义的,运算规则与非约化情形完全相同,具体地,
\begin{eqnarray*}
    \mcalB(a_0\ten\overline{a_1}\ten\cdots\ten\overline{a_n})
&=& \widetilde{(1-\lmd)}h\widetilde{\mcalN}
    (a_0\ten\overline{a_1}\ten\cdots\ten\overline{a_n})\\
&=& \widetilde{(1-\lmd)}h
      \left(
        \sum_{k=0}^n(-1)^{n(n+1-k)}
          a_k\ten \overline{a_{k+1}}\ten\cdots\ten\overline{a_n}
          \ten\overline{a_0}\ten\cdots\ten\overline{a_{k-1}}
      \right)\\
&=&   \widetilde{(1-\lmd)}
      \left(
        \sum_{k=0}^n(-1)^{n(n+1-k)}
          1\ten\overline{a_k}\ten \overline{a_{k+1}}
          \ten\cdots\ten\overline{a_n}
          \ten\overline{a_0}\ten\cdots\ten\overline{a_{k-1}}
      \right)\\
&=&   \sum_{k=0}^n(-1)^{nk}
        1\ten\overline{a_k}\ten \overline{a_{k+1}}
         \ten\cdots\ten\overline{a_n}
         \ten\overline{a_0}\ten\cdots\ten\overline{a_{k-1}}
\end{eqnarray*}
其中$a_{-1}:=a_n$.
%%%%%%%%%Reduced complex%%%%%%%%%

%$$B:\overline{C}_n(A)\to\overline{C}_{n+1}(A)$$
%$$a_0\ten\overline{a_1}\ten...\ten\overline{a_n}
%\mapsto\sum_i\pm1\ten\overline{a_i}
%\ten\cdots\ten\overline{a_n}\ten...\ten\overline{a_{i-1}}$$

\begin{prop}对于$K$-代数$A$,假设$K\inj A$,
则有如下双复形$\overline{\mcalB}\ddowndot(A)$:
$$\xymatrix{
    \vdots \ar[d]
  & \vdots \ar[d]
  & \vdots \ar[d]
\\
    A\ten\overline{A}^{\ten 2}  \ar[d]^b
  & A\ten\overline{A}^{      }  \ar[d]^b  \ar[l]_{\mcalB}
  & A^{      }                            \ar[l]_{\mcalB}
\\
    A\ten\overline{A}^{      }  \ar[d]^b
  & A^{      }                            \ar[l]_{\mcalB}
  &
\\
    A
  &
  &
}$$
记此双复形的全复形为$\overline{CC}\downdot(A)
:=\Tot\downdot(\overline{\mcalB}\ddowndot(A))$,则有自然同构
$$H^{\lmd}\downdot(A)\cong H\downdot(\overline{CC}\downdot(A))$$
\end{prop}

也就是说,在$K\inj A$的条件下,
我们可以用约化版本的双复形来计算循环同调。

\begin{proof}
考虑商映射$\pi\downdot:C\downdot(A,A)
\surj\overline{C}\downdot(A,A)$自然诱导的双复形同态
$$\pi\ddowndot:
\mcalB\ddowndot(A)\surj\overline{\mcalB}\ddowndot(A)$$

注意$\pi\ddowndot$限制在双复形的每一列上,
都为相应链复形的拟同构(这里使用了引理
\ref{约化Hoschild复形与拟同构-lemma}),
因此根据引理\ref{双复形的全复形-同调代数机器-lemma},
其诱导的全复形之间的同态
$$
\Tot\downdot(\mcalB\ddowndot(A))
\surj\Tot\downdot(\overline{\mcalB}\ddowndot(A))$$
为拟同构。再注意性质
\ref{Connes算子与循环同调-prop},
上式左边的同调即为循环同调,从而证毕。
\end{proof}

与非约化情形类似,我们也可以
$$\Tot\downdot(\overline{\mcalB}\ddowndot(A))\cong
\overline{C}\downdot(A)[u^{-1}],b+u\mcalB)$$
甚至去定义“约化周期循环同调”、
“约化负循环同调”,此处不再赘述。\vs

本节接下来给出循环同调的一些典型的计算实例。

\begin{example}
对于环$K$,设$K$-代数$A=K$,那么其循环同调
$$
H^{\lmd}_n(K)\cong
  \left\{
    \begin{array}{ll}
      K & \text{$n$为偶数}\\
      0 & \text{$n$为奇数}\\
    \end{array}
  \right.
$$
%$\overline{A}=0$.
%So,
%$$\overline{A}\downdot A=A\ten "0"$$
%$$b=0\quad B=0$$
\end{example}
我们早已具体计算出$K[x^1,x^2,...,x^n]$的Hochschild同调,
特别地$\HH\downdot(K)$只有第零个是非平凡的(同构于$K$),
其余都为$0$.不过,$H^{\lmd}\downdot(K)$与$\HH\downdot(K)$
并不相同。

\begin{proof}我们采用最简便的方法去计算,
当然采用约化循环双复形啦。在本例中,
$$\overline{A}=K/K=0$$
从而双复形$\overline{\mcalB}\ddowndot(K)$为以下:
$$
  \xymatrix{
     \vdots \ar[d]
    &\vdots \ar[d]
    &\vdots \ar[d]
    &\vdots \ar[d]
  \\
     0      \ar[d]
    &0      \ar[d]   \ar[l]
    &0      \ar[d]   \ar[l]
    &K               \ar[l]
  \\
     0      \ar[d]
    &0      \ar[d]   \ar[l]
    &K               \ar[l]
    &
  \\
     0      \ar[d]
    &K               \ar[l]
    &
    &
  \\
     K
    &
    &
    &
  }
$$
其全复形$\overline{CC}\downdot(K)$为以下
$$\cdots\to 0\xra{0}K\xra{0}0\xra{0}K$$
从而易求循环同调。
\end{proof}

当然我们也可以按照循环同调最原始的定义去计算,
其实也不难算,如下:

\begin{proof}[另一种计算方式]直接计算。此时,
$$C_n(K)\cong K^{\ten n+1}\cong K$$
我们记其生成元
$$\veps_n:=\underbrace{1\ten 1\ten\cdots\ten 1}
_{n+1\text{个}}\in C_n(K)$$
容易验证算子$b$与算子$\lmd$的作用
$$
  b(\veps_n)=
    \left\{
      \begin{array}{ll}
        \veps_{n-1} & \text{$n$为偶数} \\
        0           & \text{$n$为奇数}
      \end{array}
    \right.
\quad
  \lmd(\veps_n)=
    \left\{
      \begin{array}{ll}
        \veps_{n  } & \text{$n$为偶数} \\
        -\veps_{n  }& \text{$n$为奇数}
      \end{array}
    \right.
$$
因此,易知Connes复形
$C^{\lmd}\downdot(K):=C\downdot{K}/(1-\lmd)$具体如下:
$$\cdots\to 0\xra{0}K\xra{0}0\xra{0}K$$
对它取同调,即得循环同调。
\end{proof}

%$$\overline\downdot(A)[]u^{-1}\rightsquigarrow HC_n(K)=
%\left\{\begin{array}{ll}
%K & \text{$n\geq 0$ even}\\
%0 & \text{otherwise}
%\end{array}\right.$$
%$$K,Ku^{-1},Ku^{-2}...H\downdot(\bbC P^{\infty}=B\cup(1))$$
%$$C^{\lmd}\downdot A=C\downdot(K)/(1-\lmd)$$
%$$C_n(K):1\overbrace{\ten1\ten...\ten1}^{n}=:\veps_n$$
%$$\lmd(\veps_n)=(-1)^n\veps_n$$
%$$\left\{
%\begin{matrix}
%\veps\text{even is cyclic invariant}\\
%\veps\text{odd$=0$ in} C\downdot^{\lmd}(K)
%\end{matrix}
%\right.$$
%$$b(\veps_{2m})=\veps_{2m-1}$$
%$$0=b\text{acts on} C^{\lmd}\downdot(K)$$
%$$H^{\lmd}\downdot(K)=C\downdot^{\lmd}(K)=
%K\veps_0\oplus K\veps_2\oplus...$$

%compute it by 3 ways...

接下来,考虑$A=K[x^1,x^2,...,x^n]$为$n$元多项式环的情形,
我们企图取计算$A$的循环同调。注意在之前
我们已经使用Koszul复形求出了
$A=K[x^1,x^2,...,x^n]$的Hochschild同调。

\begin{lemma}设$A=K[x^1,x^2,...,x^n]$为$n$元多项式环,
考虑微分形式代数$\Omg_A\updot:=K[x^1,...,x^n;\td x^1,...\td x^n]$,
注意$\Omg_A\updot$上有外积运算$\wedge$与外微分运算$\td$.
考虑以下$K$-模同态
\begin{eqnarray*}
\Phi:\overline{C}_p(A)&\to& \Omg^p_A\\
a_0\ten\overline{a_1}\ten\cdots\ten\overline{a_p}
&\mapsto& \frac{1}{p!}
a_0\td a_1\wedge\td a_2\wedge\cdots\wedge\td a_p
\end{eqnarray*}
则$\Phi$是良定的,并且成立:
$$
  \left\{
    \begin{array}{l}
       \Phi\circ b=0\\
       \Phi\circ\mcalB=\td\circ\Phi
    \end{array}
  \right.
$$
其中$b:\overline{C}_p(A)\to\overline{C}_{p-1}(A)$
为约化Hochschild复形的边缘算子,
$\mcalB:\overline{C}_{p-1}(A)\to\overline{C}_{p}(A)$
为约化的Connes算子。
%Explicitly, we define
\label{Hochschild-Konstant-Rosenberg算子-lemma}
\end{lemma}

\begin{proof}
$\Phi$的良定性,即$\overline{A}$中元素与代表元选取无关。
而此代表元选取至多相差“常数项”(即$K$中元素),
它在外微分$\td$的作用下为零。因此$\Phi$良定。

我们来验证$\Phi\circ b=0$.暴力验证如下:
\begin{eqnarray*}
&&   \Phi\circ b
      (a_0\ten\overline{a_1}\ten\cdots\ten\overline{a_n})\\
&=&
    \frac{1}{p!}
    \Bigg(
      a_0a_1\td a_2\wedge\cdots\wedge\td a_p
 +
      \sum_{k=1}^{p-1}(-1)^k
        a_0\td a_1\wedge\cdots\wedge
        \td(a_ka_{k+1})
        \wedge\cdots\wedge\td a_{p}\\
&&+
      (-1)^pa_0a_p
      \td a_1\wedge\cdots\wedge\td a_{p-1}
    \Bigg)\\
&=&
    \frac{1}{p!}
    \Bigg(
      a_0a_1\td a_2\wedge\cdots\wedge\td a_p
 +
      \sum_{k=1}^{p-1}(-1)^k
        a_0a_k\td a_1\wedge\cdots\wedge
        \widehat{\td a_k}
        \wedge\cdots\wedge\td a_{p}\\
&&+
      \sum_{k=1}^{p-1}(-1)^k
        a_0a_{k+1}\td a_1\wedge\cdots\wedge
        \widehat{\td a_{k+1}}
        \wedge\cdots\wedge\td a_{p}
 +
      (-1)^pa_0a_p
      \td a_1\wedge\cdots\wedge\td a_{p-1}
    \Bigg)\\
&=&
    0
\end{eqnarray*}

第二个等式$\Phi\circ\mcalB=\td\circ\Phi$
也容易直接验证:一方面,
\begin{eqnarray*}
&& \Phi\circ\mcalB
     (a_0\ten\overline{a_1}\ten\cdots\ten\overline{a_n})\\
&=&
   \sum_{k=0}^p\frac{(-1)^{pk}}{(p+1)!}
     \Big(
       \td a_k\wedge\td a_{k+1}\wedge\cdots\wedge\td a_p
     \Big)
     \wedge
     \Big(
       \td a_0\wedge\td a_{1}\wedge\cdots\wedge\td a_{k-1}
     \Big)\\
&=&
   \frac{1}{p!}
     \td a_0\wedge\td a_1\wedge\cdots\wedge\td a_{n}
\end{eqnarray*}
而另一方面,
\begin{eqnarray*}
  \td\circ\Phi
      (a_0\ten\overline{a_1}\ten\cdots\ten\overline{a_p})
=
    \frac{1}{p!}
    \td(a_0\td a_{1}\wedge\cdots\wedge\td a_{p})
=
    \frac{1}{p!}
    \td a_0\wedge\td a_{1}\wedge\cdots\wedge\td a_{p}
\end{eqnarray*}
从而得证。
\end{proof}

由此引理,我们即可去计算$A:=K[x^1,x^2,...,x^n]$的循环同调。

\begin{prop}对于$A:=K[x^1,x^2,...,x^n]$,则其循环同调
$$H^{\lmd}_n(A)\cong
\left\{
  \begin{array}{ll}
    ({\Omg^n_A}\big/{\td\Omg^{n-1}_A})\oplus K &  \text{$n$为偶数}\\
    {\Omg^n_A}\big/{\td\Omg^{n-1}_A}           &  \text{$n$为奇数}\\
  \end{array}
\right.
\label{C[x^i]的循环同调}
$$
\end{prop}

\begin{proof}
事实上,刚才的引理
\ref{Hochschild-Konstant-Rosenberg算子-lemma}
表明,$\Phi$诱导以下两个双复形之间的态射:
$$
  \begin{tabular}{cccc}
    \xymatrix{
       \vdots             \ar[d]
      &\vdots             \ar[d]
      &\vdots             \ar[d]
    \\
       \overline{C}_2(A)  \ar[d]^{b}
      &\overline{C}_1(A)  \ar[d]^{b}  \ar[l]_{\mcalB}
      &\overline{C}_0(A)              \ar[l]_{\mcalB}
    \\
       \overline{C}_1(A)  \ar[d]^{b}
      &\overline{C}_0(A)              \ar[l]_{\mcalB}
      &
    \\
       \overline{C}_0(A)
      &
      &
    }
      &&&
    \xymatrix{
       \vdots             \ar[d]
      &\vdots             \ar[d]
      &\vdots             \ar[d]
    \\
       \Omg^2_A  \ar[d]^{0}
      &\Omg^1_A  \ar[d]^{0}  \ar[l]_{\td}
      &\Omg^0_A              \ar[l]_{\td}
    \\
       \Omg^1_A  \ar[d]^{0}
      &\Omg^0_A              \ar[l]_{\td}
      &
    \\
       \Omg^0_A
      &
      &
    }
  \end{tabular}
$$
{\color{red}(按村儿里的规矩,此处应该有立方交换图)}

其中左边为$\overline{\mcalB}\ddowndot(A)$,而
右边的每一行均为de-Rham上链复形,每一列的边缘算子都为零。
注意到$\Phi$是满射,以及我们早已用Koszul复形得到的
$$H_n(\overline{C}\downdot(A))\cong\HH_n(A)
\cong\Omg^n_A\cong H_n(\Omg\updot_A,0)$$
从而双复形同态$\Phi$限制在每一列上都为拟同构,于是由引理
\ref{双复形的全复形-同调代数机器-lemma},立刻知道
$$\Phi:\Tot\downdot(\overline{\mcalB}\ddowndot(A),b,\mcalB)
\to\Tot\downdot(\Omg\updot_A,0,\td)$$
为拟同构。上式左边的同调即为$A$的循环同调,
而右边的同调可以直接计算。只需要注意到(Poincare引理)de-Rham复形
$$\cdots\xra{\td}\Omg\updot(A)
\xra{\td}\Omg\updot(A)\xra{\td}\Omg\updot(A)\to 0$$
的(上)同调满足
$$
  H^n(\Omg\updot_A,\td)=
  \left\{
    \begin{array}{ll}
      K  &  n=0\\
      0  &  n>0
    \end{array}
  \right.
$$
因此容易计算出
$$H^{\lmd}_n(A)\cong
\left\{
  \begin{array}{ll}
    ({\Omg^n_A}\big/{\td\Omg^{n-1}_A})\oplus K &  \text{$n$为偶数}\\
    {\Omg^n_A}\big/{\td\Omg^{n-1}_A}           &  \text{$n$为奇数}\\
  \end{array}
\right.
$$
\end{proof}



%$$\Phi:(\overline{C\downdot(A)},b)\to(\Omg^0(A),0)$$
%si a chain map.
%%%%%%Isildur%%%%%%%%

%So, we have an isomorphism
%$$\HH\downdot(A)\cong \Omg\updot (A)$$
%,then $\HH\downdot(A)=\Omg\updot(A)=\bbC[x^i,\td x^i]$.

%Check:
%$$\Phi\circ B=\td\circ\Phi$$
%%%%%%%这很明显吧%%%%%%%%%%%

%So,

%$$\Phi:(\overline(C)\downdot(A),b,B)\to(\Omg\updot(A),0,\td)$$

%Cyclic homology,
%$$\Phi:(\overline(C)\downdot(A),b+uB)\to
%(\Omg\updot(A)[u^{-1}],u\td )$$

%If $A=K[x^i]$
%$$\cdots\xra{\td}\Omg\updot(A)
%\xra{\td}\Omg\updot(A)\xra{\td}\Omg\updot(A)\to 0$$

%then $$H\downdot(\Omg\updot(A),\td)=H_0=K$$
%"Poincare lemma"

%\begin{prop}
%$$HC\downdot(K[x^i])=\frac{\Omg\updot(A)}
%                    {d\Omg\updot(A)}
%\oplus u^{-1}K[u^{-1}]$$

%\end{prop}
\begin{rem}
容易知道,Connes算子
$$\mcalB:\overline{C}_n(A)\to\overline{C}_{n+1}(A)$$
在$\ker b$上的限制,可以下降为Hochschild同调之间的同态
$$\mcalB:\HH_n(A)\to\HH_{n+1}(A)$$
\end{rem}
Hochschild同调扮演的角色相当于微分形式,而此时Connes算子
扮演的则是外微分。

\begin{rem}
双复形满同态
$$\Phi:(\overline{\mcalB}\ddowndot(A),b,\mcalB)\surj
(\Omg\updot_A,0,\td)
$$
其实是可裂(split)的。具体地,存在双复形同态
\begin{eqnarray*}
\eta:\Omg\updot(A)&\to&\overline{C}\downdot(A)\\
a_0\td a_1\wedge\td a_2\wedge\cdots\wedge\td a_p
&\mapsto&\sum_{\sigma\in S_p}(-1)^{\sgn \sigma}
           a_0\ten a_{\sigma(1)}\ten\cdots\ten a_{\sigma(p)}
\end{eqnarray*}
使得$\Phi\circ\eta=\id$.
\end{rem}

容易验证(简单的组合技巧)$\eta$的确诱导了双复形同态
$$\eta:(\Omg\updot_A,0,\td)\to
(\overline{\mcalB}\ddowndot(A),b,\mcalB)
$$

\section{循环上同调}
本章最后,简单介绍一下循环上同调(Cyclic cohomology)。
%$$C_n(A)\cong A^{\ten n+1}$$
对于双$A$-模$M$,
回顾我们之前已经介绍的Hochschild上链复形
$$C^n(A,M):=\Hom(A^{\ten n},A)$$
特别地,当$M=A$时,我们给出以下记号:

\begin{notation}对于$K$-代数$A$,以及$n\geq0$,
我们记Hochschild上链复形
$$C^n(A):=C^{n}(A,A)\cong\Hom(A^{\ten n},A)
\cong\Hom(A^{\ten n+1},K)$$
并且将该Hochschild上链复形的微分算子记为$b^*$.
\end{notation}

我们此前考虑同构$C_n(A)\cong A^{\ten n+1}$,
而Hochschild上链复形$C^n(A)\cong\Hom(A^{\ten n+1},K)$
恰为其对偶;微分算子$b^*$的作用即为$b$的对偶:
即对任意$f\in C^n(A)\cong\Hom(A^{\ten n+1},K)$以及
$\omg\in A^{\ten n+2}\cong C_{n+1}(A)$,成立
$$(b^*f)(\omg)=f(b(\omg))$$

与循环余不变量对偶,我们可以谈论循环不变量:

\begin{definition}(循环不变量)

对于$f\in C^n(A)$,称$f$为循环不变量
(cyclic invariant),
\index{cyclic invariant\kong 循环不变量}
如果对任意的$a_0,...,a_n\in A$,成立
$$f(a_0,a_1,...,a_n)=(-1)^nf(a_n,a_0,...,a_{n-1})$$
记$C^n(A)$当中的循环不变量之全体为$C^n_{\lmd}(A)$.
\end{definition}

容易验证$b^*(C^{n}_{\lmd}(A))\subseteq C^{n+1}_{\lmd}(A)$,
从而$(C\updot_{\lmd}(A),b^*)$为$(C\updot(A),b^*)$的子复形。
(不必暴力验证了,由循环余不变量对偶过去就行)
看图说话即可:

$$
  \xymatrix{
     \cdots       \ar[r]
    &C_n(A)       \ar[r]^{b'}   \ar[d]^{1-\lmd}
    &C_{n-1}(A)   \ar[r]        \ar[d]^{1-\lmd}
    &\cdots
  \\
     \cdots       \ar[r]
    &C_n(A)       \ar[r]^{b }   \ar@{->>}[d]
    &C_{n-1}(A)   \ar[r]        \ar@{->>}[d]
    &\cdots
  \\
     \cdots       \ar[r]
    &C_n^{\lmd}(A)\ar[r]        \ar[d]
    &C_{n-1}^{\lmd}(A) \ar[r]   \ar[d]
    &\cdots
  \\
    &0&0&
  }
\quad
  \xymatrix{
     \cdots
    &C^n(A)       \ar[l]
    &C^{n-1}(A)   \ar[l]_{b'^*}
    &\cdots       \ar[l]
  \\
     \cdots
    &C^n(A)       \ar[l]        \ar[u]^{1-\lmd^*}
    &C^{n-1}(A)   \ar[l]^{b^*}  \ar[u]^{1-\lmd^*}
    &\cdots       \ar[l]
  \\
     \cdots
    &C^n_{\lmd}(A)\ar[l]        \ar@{^{(}->}[u]
    &C^{n-1}_{\lmd}(A) \ar[l]   \ar@{^{(}->}[u]
    &\cdots            \ar[l]
  \\
    &0\ar[u]
    &0\ar[u]
    &
  }
$$

左图我们早已熟悉,注意它的每一列都是右正合的。
将反变左正合函子$\Hom(-,K)$作用于左图即得到右图,
右图的每一列都是左正合的。


%cyclic cochains form a subcomplex
%$$C_{\lmd}^n(A)\subseteq C^n(A)$$
%which is dual to
%$$C_n(A)\to C_n^{\lmd}(A)$$
%$$b^*:C_{\lmd}^n(A)\to C_{\lmd}^{n+1}(A)$$

\begin{definition}(循环上同调)
对于$K$-代数$A$定义$A$的\textbf{循环上同调}
(cyclic cohomology)
\index{cyclic cohomology\kong 循环上同调}
$$H\updot_{\lmd}(A):=\HC\updot(A)
:=H\updot(C_{\lmd}\updot(A),d^*)$$
\end{definition}

作为例子,我们具体计算一下第零个循环上同调。

\begin{example}对于$K$-代数$A$,则有
$$H^0_{\lmd}(A)=\{f\in\Hom(A,K)|
\forall x,y\in A,f(xy)=f(yx)\}$$
\end{example}

\begin{proof}直接计算即可。
只需考虑Hochschild上链复形
$$0\to C^0_\lmd(A)\xra{b^*}C^1_{\lmd}(A)
\xra{b^*}C^2_{\lmd}(A)\to\cdots$$
易知$C^0_{\lmd}(A)=C^0(A)=\Hom(A,K)$,从而
$$H^0_{\lmd}(A)=\ker(b^*:C^0(A)\to C^1(A))$$

对于$f\in\Hom(A,K)$,若$b^*f=0$,则对于任意$x,y\in A$,有
$$0=(b^*f)(x,y)=f(b(x\ten y))=f(xy-yx)=f(xy)-f(yx)$$
从而可知
$$H^0_{\lmd}(A)=\{f\in\Hom(A,K)|
\forall x,y\in A,f(xy)=f(yx)\}$$
\end{proof}

像$H^0_{\lmd}(A)$当中的线性算子那样,满足
$$f(xy)=f(yx)\quad(\forall x,y\in A)$$
的线性算子称之为\textbf{迹算子}。

高阶的循环上同调可被认为是“导出的”迹算子。

%$$f\in C_{\lmd}^0(A):A\to K$$
%$$b^*f\in C^1_{\lmd}(A)$$
%$$b^*f(a_0,a_1)=f(b(a_0\ten a_1))=f(a_0a_1-a_1a_0)$$
%$$b^*=0\iff f:A/[A,A]\to K$$
%$f$ behaves like "trace".
%$HC^0(A)$="trace operators"


