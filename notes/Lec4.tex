%%%%%%%%%%%%2019.3.12第三周Mon%%%%%%%%%%%%%%%%%%%%%%%
%%%%%%%%%%%%%%%%下周一,周坚代课%%&%%%%%%%%%%%%%%%%%%
%%%%%%%%%%%%%%%%下周二,随堂考试%%%%%%%%%%%%%%%%%%%%%

%\textbf{Coalgebra and homotopy associativity}

\chapter{余代数与同伦结合性}

%\textbf{Co-algebra}
%%%%%结合性%%%%%%%%
%%%%%%分次交换%%%%%%%%%
\section{余代数与分次余代数}
首先简要回顾一下\textbf{余代数}(co-algebra)
\index{co-algebra\kong 余代数}
的概念。对于$K$-代数$A$,$A$上的乘法$m:A\ten A\to A$的结合性
可用如下交换图来描述:

$$
  \xymatrixcolsep{5pc}
  \xymatrixrowsep{5pc}
  \xymatrix{
     A\ten A\ten A  \ar[r]^-{m\ten 1}  \ar[d]^-{1\ten m}
    &A\ten A                           \ar[d]^-m
  \\
     A\ten A        \ar[r]^-m
    &A
  }
$$

将上述图表中的箭头全部反向,即得到\textbf{余结合律}的概念:
对于$K$-模$A$,以及$K$-模同态$\yc:A\to A\ten A$,
若以下图表交换
$$
  \xybigcol
  \xybigrow
  \xymatrix{
    A          \ar[r]^{\yc}   \ar[d]^{\yc}
   &A\ten A                   \ar[d]^{\yc\ten 1}
  \\
    A\ten A    \ar[r]^{1\ten\yc}
   &A\ten A\ten A
  }
$$
则称$\yc$满足\textbf{余结合律},
%此时称$(A,\yc)$为\textbf{余代数}(co-algebra)。
运算“$\yc$”称为\textbf{余乘}(co-product).
\index{co-product\kong 余乘}
类似地我们可以谈论\textbf{余交换律},
乘法$m:A\ten A\to A$与余乘$\yc:A\to A\ten A$的交换律、
余交换律分别由以下交换图表描述:
$$
  \xybigrow
  \xymatrix{
    A\ten A \ar[rr]^{\tau}  \ar[dr]^m
    &&
    A\ten A \ar[dl]_{m}
  \\
    &A&
  }
\quad\quad
  \xybigrow
  \xymatrix{
    A\ten A
    &&
    A\ten A \ar[ll]_{\tau}
  \\
    &A \ar[ul]_{\yc}\ar[ur]^{\yc}
    &
  }
$$
其中$\tau: x\ten y\mapsto y\ten x$为$A\ten A$的对合自同构。

对于$K$-代数$A$,我们总是假定$A$含幺。
事实上,存在唯一的$K$-代数同态
$$i:K\to A$$
而$A$的\textbf{幺元}$1\in A$即为$1\in K$在该同态下的像。
在此意义下,我们不妨重新定义什么是$A$的幺元:
称$K$-模同态$i:K\to A$为$A$的幺元,如果以下图表交换:
$$
  \xybigrow
  \xybigcol
  \xymatrix{
     &
       A\ar@{=}[ld]_{\cong}\ar@{=}[rd]^{\cong}
     &
  \\
       K\ten A  \ar[r]^{i\ten 1}
     & A\ten A  \ar[u]_{m}
     & A\ten K  \ar[l]_{1\ten i}
  }
$$

将以上图表的箭头全部反向,
则得到\textbf{余幺元}(co-unit)
\index{co-unit\kong 余幺元}
的概念:对于配以余乘$\yc$的$K$-模$A$,
称$K$-模同态$\veps:A\to K$为关于$\yc$的
\textbf{余幺元},如果以下图表交换:
$$
  \xybigrow
  \xybigcol
  \xymatrix{
     &
       A\ar@{=}[ld]_{\cong}\ar@{=}[rd]^{\cong}\ar[d]^{\yc}
     &
  \\
       K\ten A
     & A\ten A  \ar[l]_{\veps\ten 1}\ar[r]^{1\ten \veps}
     & A\ten K
  }
$$

对于$K$-模$A$,若$A$配以(满足余结合律的)余乘$\yc$,
以及关于该余乘的余幺元$\veps:A\to K$,则称
$(A,\yc,\veps)$为$K$-\textbf{余代数}。

若$(A,\yc_A)$与$(B,\yc_B)$都为$K$-余代数,称$K$-模同态
$\fai:A\to B$为$K$-\textbf{余代数同态},如果对任意$x\in A$,成立
$$\yc_B(\fai(x))=\fai(\yc_A(x))$$

\begin{rem}%If
若$(A,\triangle)$ 为$K$-余代数,%is a co-algebra, then
考虑对偶映射$\yc^*:(A\ten A)^*\to A^*$,则
$(A^*,\triangle^*)$ %is an algebra
具有如下$K$-代数结构:
$$A^*\ten A^*\to(A\ten A)^*\xra{\triangle^*}A^*$$
\end{rem}
%%%%%见图%%%%%%%5

用反变函子$\Hom(-,K)$翻转余代数图表的箭头而已;
但是要注意,对一个代数取对偶,未必能得到余代数。
也就是说,某种意义下余代数比代数包含更多的信息。

现在我们谈论余代数的分次版本。

\begin{definition}(分次余代数)
%$C\downdot\in Mod_K^{\bbZ}$ is a graded coalgebra over $K$,
%if there is a coproduct ($\deg=0$)
%$$\triangle:C\to C\ten C$$
%%%coproduct%%%%%%%
\index{graded co-algebra\kong 分次余代数}

设$A=\bigoplus\limits_{k\in\bbZ}A_k$为分次$K$-模,
$\yc:A\to A\ten A$为$A$的余乘(满足余结合律),
称$(A,\yc)$为\textbf{分次余代数}(graded co-algebra),
若$\yc$与$A$的分次满足以下相容性:任意$k\in\bbZ$,
$$\yc(A_k)\subseteq (A\ten A)_k$$
\end{definition}

我们自然也可以谈论$\yc$的\textbf{分次余交换性},
见下述交换图(与非分次情形完全一样):
$$
\xybigrow
  \xymatrix{
    A\ten A
    &&
    A\ten A \ar[ll]_{\tau}
  \\
    &A \ar[ul]_{\yc}\ar[ur]^{\yc}
    &
  }
$$
不过要注意,这里的$\tau$为分次对合自同构
(见定义\ref{分次对合自同构}),服从Koszul符号法则:
$$\tau: x\ten y\mapsto (-1)^{xy}y\ten x$$


%counit$\veps:C\to K$ satisfying
%%%%%%%%counit%%%%%%%%%
\begin{definition}(余超导子)%co-derivation

对于分次$K$-余代数$A$,以及$d\in\bbZ$,
称次数为$d$的分次$K$-模同态$\delta:A\to A$
为$d$次\textbf{齐次余超导子},若以下图表交换
$$
  \xybigrow
  \xybigcol
  \xymatrix{
      A  \ar[r]^{\delta}  \ar[d]_{\yc}
    & A  \ar[d]^{\yc}
  \\
      A\ten A \ar[r]^-{\delta\ten 1+1\ten\delta}
    & A\ten A
  }
$$
即满足“\textbf{余莱布尼茨法则}”。
%$\delta: C\to C$ satisfying
%%%%%co-Leibnitz rule%%%%%%%
\end{definition}
注意到上述图表默认Koszul符号法则。
记$A$的$d$次齐次余超导子之全体为$\Coder(A)_d$,
以及
$$\Coder(A):=\bigoplus_{d\in\bbZ}\Coder(A)_d$$
称其中的元素为\textbf{余超导子}。
与超导子类似,余超导子之全体也有李超代数结构:

\begin{prop}对于$K$-余代数$A$,
%$Coder(C)$ is a graded Lie algebra, where
以及齐次余超导子$D_1,D_2\in\Coder(A)$,定义
$$[D_1,D_2]:=D_1\circ D_2-(-1)^{D_1D_2}D_2\circ D_1$$
则$[D_1,D_2]\in\Coder(A)$,进而$(\Coder(A),[,])$
构成李超代数。
\end{prop}
\begin{proof}
与分次代数的超导子完全类似,直接验证即可,从略。
\end{proof}

有了余超导子,可以相应地去定义\textbf{微分分次余代数}
(differential graded co-algebra):
\index{differential graded co-algebra\kong 微分分次余代数}
%A differential graded co-algebra is a co-algebra
%$C$ with a co-derivation $\delta:C\to C$ such that
%$\deg\delta=1$ and $\delta^2=0$.

\begin{definition}(微分分次余代数)
对于$K$-余代数$A$,
以及满足$\delta^2=0$的$1$次齐次余超导子$\delta$,
则称$(A,\delta)$为微分分次余代数。
\end{definition}

类似地,微分分次余代数自然可以视为分次上链复形。

\begin{rem}((co-)augmentation)
\index{augmentation}%我也不知道这屌东西该怎么中(人)文(话)说

(1)对于$K$-代数$A$,我们把从$A$到$K$的$K$-模同态称为augmentation;

(2)对于$K$-余代数$A$,我们把从$K$到$A$的$K$-模同态称为co-augmentation
\end{rem}

与“幺元”的箭头刚好相反。
{\color{purple}笔者建议将augmentation意译为“赋值”。}

%co-augmentation

%(Recall:
%$A$:$K$-algebra. augmentation is an algebra morphism
%$A\to K$.)
%Co-algebra $(C,\triangle)$ is called co-augmentation if
%there is a co-algebra map $K\to C$.

%%%%%%co-algebra的同态怎么定义?自行脑补%%%%%%%%%%

%$(C,\triangle)$ is co-commutative,if
%%%%co-commutative%%%%%%

\begin{example}%$V\in Mod_K^{\bbZ}$,
设$V$为分次$K$-模,%$$T^c(V):=\bigoplus_{n\geq 0}V^{\ten n}$$
考虑张量代数$T(V):\bigoplus\limits_{n\geq 0}V^{\ten n}$,定义
\begin{eqnarray*}
\triangle:T(V)&\to& T(V)\ten T(V)\\
v_1\ten\cdots\ten v_n&\mapsto&
\sum_{i=0}^n(v_1\ten\cdots\ten v_i)\ten(v_{i+1}\ten\cdots\ten v_n)
\end{eqnarray*}
则$(T(V),\yc)$构成余代数。
\end{example}
%Check:$\triangle$ is a co-product, and what is its dual?
容易验证如此$\yc$满足余结合律:
\begin{eqnarray*}
& & (\yc\ten 1)\circ\yc(v_1\ten\cdots\ten v_n)\\
&=&
     \sum_{0\leq i\leq j\leq n}
          (a_1\ten\cdots\ten a_{i})
          \ten(a_{i+1}\ten\cdots\ten a_{j})
          \ten(a_{j+1}\ten\cdots\ten a_n)\\
&=&
     (1\ten\yc)\circ\yc(v_1\ten\cdots\ten v_n)
\end{eqnarray*}
并且配以余幺元$\veps:$
$$
\veps\big|_{V^{\ten n}}=
\left\{
  \begin{array}{ll}
     0      &  \text{如果} n>0\\
     \id_K  &  \text{如果} n=0
  \end{array}
\right.
$$

\begin{rem}%$$\overline{T^c}(V):=\bigoplus_{n\geq 1}V^{\ten n}$$
$T(V)$还有另一个余乘结构$\overline{\yc}$
$$\overline{\triangle}(v_1\ten...\ten v_n)
:=
\sum_{i=1}^{n-1}(v_1\ten\cdots\ten v_i)\ten(v_{i+1}\cdots v_n)$$
%$\overline{T^c}(V)$ is a co-product on $\overline{T^c}(V)$.
\end{rem}
这也是容易验证的。不过这个余乘结构不存在余幺元。

%$Coder(C)$ is all the co-derivation....

\section{结合性}
%\textbf{Associativity.}

$$\bullet:A\ten A\to A$$
$\Longrightarrow$
$$m:A[1]\ten A[1]\to A[1]$$
$$sa_1\ten sa_2\mapsto(-1)^{|a_1|+1}s(a_1a_2)$$

Observation:
Associativity of $\bullet\Longleftrightarrow[m,m]=0$

%%%%%叼~~%%%%%

%%%%%%%%%%2019.3.18第四周 周一%%%%%%%%%%%%%%%%%%%%%%%
%%%%%%%%%%%%%%%%李思出差,别人代课%%%%%%%%%%%%%%%%%%%

\textbf{Poisson structure and quantization}

Motiration

Classical Mechanics

Phase space$(X,\omg)\rightsquigarrow$ Sympletic mfd

observables: $C^{\infty}(X)$: smooth function

Poisson bracket: 
$C^{\infty}(X)\ten X^{\infty}(X)\to C^{\infty}(X)$

Dynamics - Hamiltionian function $H\in C^{\infty}(X)$
s.t. $\frac{\td f}{\td t}=\{H,f\}$.


Quantum Picture:

Hilbert space
$$\frac{\td A_t}{\td t}=\frac{i}{\hbar}[\hat{H},A_t]$$

Associative algebra.

$A_{\infty}$-method and Hochcshild homology

\textbf{Goal: Formality thm(Kontsevich)}

Now, let us begin...

\vs

\textbf{Poisson bracket}

\begin{definition}
let $X$ be a smooth manifold, a Poisson bracket 
on $C^{\infty}(X)$ is  a $\bbR$-linear 
$$\{,\}:C^{\infty}(X)\times C^{\infty}(X)\to C^{\infty}(X)$$
sarisfies

skew symmetry

Leibnitz rule

Jacobi identity
\end{definition}

Poisson tensor:
$P\in\Gamma(X,\wedgeform{2}TX)=\PV_X^2$ s.t.
$$\{f,g\}=\langle P, \td f\wedge \td g\rangle$$
$$=P^{ij}(\p_i f\p_jg-\p_ig\p_jf)$$

Check: Jacobi identity $\iff[P,P]=0$, where 
$[,]$ is Schouten-Nijenhuis bracket.

\begin{definition}
A Poisson manifold is a pair $(X,P)$ such that 
$[P,P]=0$ for some $P\in\PV_X^2$.
\end{definition}
(geometry of classical mechanics)

\begin{example}
if $(X,\omg)$ is a sympletic manifold, 
$$\omg=\frac{1}{2}\omg_{ij}\td x^i\wedge\td x^{j}$$
such that $\omg$ is non-degenerated ,$\omg_{ij}=-\omg_{ji}$,
then
$$\omg^{-1}:=\frac{1}{2}\omg^{ij}\p_i\wedge\p_j$$
is a Poisson bracket.where $(\omg^{ij}):=(\omg_{ij})^{-1}$
\end{example}

\begin{example}
Let $\mfkg$ is a Lie algebra, $X:=\mfkg^*$ its dual, 
$$[,]:\mfkg\times\mfkg\to\mfkg$$
then 
$$P_{\mfkg}\in \mfkg\ten\wedgeform{2}\mfkg$$
(locally)
if $\{e^i\}$ is a basis of $\mfkg$, $\{e_i\}$ its dual,
$$[e^i,e^j]=C^{ij}_ke^k$$
then
$$P_{\mfkg}=X^kC^{ij}_k\p_i\wedge\p_j$$
check: it satisfies Jacobi identity, i.e. 
$(\mfkg^*,P_{\mfkg})$ is a Possion manifold.
\end{example}

\begin{rem}
$P_{\mfkg}$ is not invertible $\iff$ 
$\mfkg^*$ is not sympletic.
\end{rem}

\textbf{Star product:}

\begin{definition}
A star product $*$ on a Possion mfd $(X,P)$ 
is a $\bbR[[\hbar]]$-linear map
$$C^{\infty}(X)[[\hbar]]\times C^{\infty}(X)[[\hbar]]\to C^{\infty}(X)[[\hbar]]$$
$$f*g\mapsto \sum_{k\geq 0}\hbar^kC_k(f,g)$$
such that:

(1) $*$ is associative

(2) $f*g=fg\mod \hbar$

(3) $\frac{1}{2}(f*g-g*f)=\hbar\{f,g\}\mod\hbar^2$

(4) $C_k(f,g)=C_k^{i_1...i_l;j_i,...,j_m}(\p_{i_1}...\p_{i_l}f)
(\p_{j_1}...\p_{j_m}g)$, where $C_k^{...}$ is in $C^{\infty}(X)$.
(bi-differential operator)
\vs


then, $(C^{\infty},*)$ is the \textbf{deformation quantization} on $(X,P)$.
\end{definition}

the first non-commutativity of $*$ is given 
by Poosson bracket.

Fundamental Question:
[Dewilde lemote 1983] for a Poisson manifold, is there $\exists$
a deformation quantization?

[Fedosov 1983,1994]

\begin{example}
$(C^{\infty}(X),\{,\})\rightsquigarrow(C^{\infty(X)[[\hbar]]},*)$

$$X=\bbR^m$$
Poisson bracket
$$P=P^{ij}\p_i\wedge\p_j$$
where $P^{ij}\in\bbR$ are constant.
then we define

%$$f*g:=\exp{\frac{\hbar}{2}P^{ij}\frac{\p}{\p y_i}\frac{\p}{\p z_j}}\big|_{y=z=x}f(y)g(z)$$
%%%%%%%此处有图%%%%%55
\end{example}


%%%%见笔记%%%


\begin{example}
when $\bbR^m=\bbR^{2n}$ even dimension, $\{x^1,...,x^n;p_1,...,p_n\}$,
$$P=\sum\pp{p_i}\wedge\pp{x^i}$$
then Moyal product becomes Weyl algebra and
$$[x^i,p_j]=x^i*p_j-p_j*x^i=\hbar\delta_{ij}$$
\end{example}

\begin{example}
$\mfkg$ Lie algebra, $X=\mfkg^*$,$P_{\mfkg}$,
consider (quantum) universal evenloping algebra
$$\mcalU_{\hbar}(\mfkg)=\bigoplus_{k}\mfkg^{k}[[\hbar]]\big/\sim$$ 

the relation is 
$$a\ten b-b\ten a=\hbar[a,b]$$

$$\mcalU_{\hbar}(\mfkg)\big|_{\hbar=1}$$
is universal evenloping algebra.

$$\mcalU_{\hbar}(\mfkg)\big|_{\hbar=0}=\Sym\updot(\mfkg)$$

\end{example}

PBW theorem:
there is a vector space bijienction:
$$\Phi:\Sym\updot(\mfkg)[[\hbar]]\to\mcalU_{\hbar}(\mfkg)$$
%%%%PBW%%%55

$\mcalU_{\hbar}(\mfkg)$ has a natural associative algebra 
structure by $\ten$, $\Phi^{*}(\ten)$ defines an associative algebra structure on 
$\Sym\updot(\mfkg)[[\hbar]]=:\mcalO(X)[[\hbar]]$

Check: it is a deformation quantization of $(X,P_{\mfkg})$.

\begin{example}
$M$ is a mfd, $X=T^*M$ cotangent bundle, which admits a sympletic structure
$$\omg=\sum\td x^i\wedge\td p_i$$
$$\mcalO(X)=\Gamma(M,\Sym\updot(TM))$$ 
\end{example}

$f\in\mcalO(X)$,
$$f=f(x,p)=\sum_{I}f_I(X)p^{I}(X)$$
and $\omg^{-1}$ Poisson bracket $\{,\}$ is a Poisson algebra.

$D_X$ differential operator on $X$.
$\exists$ filtration by its order 
$$D_X^{(0)}\subseteq D_X^{(1)}\subseteq...\subseteq D_X^{(m)}...$$
(consists of $\sum A^{i_1...i_k}\p_{i_1}...\p_{i_k}$)

Check:
$$[D_X^{(m)},D_X^{(n)}]\subseteq D_X^{(m+n-1)}$$
$$D_X^{(m)}\circ D_X^{(n)}\subseteq D_X^{(m+n)}$$
$$(D_X,\circ)\text{ is associative}$$


\begin{definition}
$$D_X^{\hbar}:=\bigoplus_{m\geq 0}D^{(k)}_X\hbar^m\subseteq D_X[\hbar]$$
is a $\bbR[\hbar]$-module
\end{definition}

HW: what is $k$ in Def above?????

(such that $D_X^{\hbar}$ can be understood as a Deformation Quantization of 
$(\mcalO(X),\{,\})$ over ring $\bbR[\hbar]$.

\begin{example}(Quantum torus)

Let $V=\bbZ^n$ ,define the algebra 
$$e^{V}:=\{e^v|v\in V\}$$
with 
$$e^{v_1}e^{v_2}:=e^{v_1+v_2}$$
then
$e^V$ is algebra function on $(\bbC^*)^n=\bbC[z_1^{\pm 1},...,z_n^{\pm 1}]$.

Let $\omg:V\times V\to \bbZ$ skew symmetric bilinear form
$$e^{v_1}*_{\hbar}e^{v_2}=e^{\hbar\omg(v_1,v_2)}e^{v_1+v_2}$$
check: associativity and it is a D.Q. of Poisson bracket 
$$\{e^{v_1},e^{v_2}\}=\omg(v_1,v_2)e^{v_1+v_2}$$ 



\end{example}






