\chapter{量子场论的背景}
%\textbf{Some basis of QFT}
现在我们开始逐渐去理解Kontsevich的形变量子化的构造;
为此需要一些\textbf{量子场论}
(quantum field theory,简称QFT)背景知识。
\index{quantum field theory\kong 量子场论}

%a physics system always consists of
%$$\mcalS:\mcalE\to\bbR$$
%where $\mcalE$ is the space of field( of infinity dimension)

\section{BV算子}

大致地说(并非严格的数学表述),一个\textbf{物理系统}
包括以下要素:\textbf{场空间}(space of fields)
$\mcalE$与\textbf{作用量}
(action functional)$\mcalS$,
\index{space of fields\kong 场空间}
\index{action functional\kong 作用量}
其中场空间$\mcalE$通常为无穷维空间,作用量
$$\mcalS:\mcalE\to \bbC$$
为场空间$\mcalE$上的函数。
%$\mcalS$ : actional functional%作用量

在经典物理中,态的演化常用变分的临界来描述态的演化:
%Classical physics:
$$\Crit(\mcalS)=\{\delta\mcalS=0\}$$
上式中的$\Crit(\mcalS)$称为$\mcalS$的critical locus,
$\delta$为某个变分导数。
\index{critical locus}

%Quantum physics:
而在量子物理中,态的演化与积分
$$\int_{\mcalE}\mcalO e^{i\mcalS/\hbar}$$
有关,其中$\mcalO$为$\mcalE$上的函数,
称之为\textbf{观测量}(observable);
\index{observable\kong 观测量}
上述积分称之为“\textbf{路径积分}”(path integral)。
\index{path integral\kong 路径积分}

不过要注意,$\mcalE$是无穷维空间,
在$\mcalE$上面积分是说不清道不明的事情;
我们至今还未完全搞明白此积分的严格定义。
我们在本讲义只谈论“数学上的事情”,
数学上暂时没说清楚的东西避而不谈。

%"path integral", $\theta$ is a function on $\mcalE$("observable").

\begin{example}作为场空间$\mcalE$的例子,
以下是近代物理中的常见对象:
$$
  \begin{tabular}{|c|c|}
  \hline
  标量场论    &  $\mcalS=$流形$X$上的全体光滑函数\\
  \hline
  规范理论    &  $\mcalS=$向量丛$E\to X$上的全体联络\\
  \hline
  $\sgm$-模型 &  $\mcalS=$流形$\sgm$与$X$之间的全体光滑映射\\
  \hline
  引力理论    &  $\mcalS=$流形$X$上的全体黎曼度量\\
  \hline
  \end{tabular}
$$
\end{example}

我们再举一些作用量$\mcalS$的例子:

\begin{example}(作用量)

(1)在标量场论$\mcalE=C^{\infty}(X)$中,
对于$X$上的光滑函数$\fai\in\mcalE$,定义
$$\mcalS[\fai]=\int_X|\nabla\fai|^2$$
称之为能量泛函。

(2)在规范理论当中,对于$A\in\mcalE$为向量丛$E\to X$上的联络,
记其曲率张量为
$$F_A:=\td A+\frac{1}{2}[A,A]$$
定义如下的\textbf{杨-米尔斯泛函}(Yang-Mills functional)
$$\YM[A]:=\int_XF_A\wedge*F_A$$

%$$\mcalE=\{\text{connections on $E\to X$}\}$$
%curvature
%is called Yang-Mills functional.
\index{Yang-Mills functional\kong 杨-米尔斯泛函}
\end{example}

%\begin{example}
%$$\mcalE= C^{\infty}(X)$$
%$$\mcalS[\fai]=\int_X|\nabla\fai|^2\quad\fai\in C^{\infty}(X)$$
%(scalar field theory)
%$$\delta\mcalS=0\Longrightarrow\Delta\fai=0$$
%\end{example}
%\begin{example}($\sigma$-module)
%$$\mcalE=map(\Sigma,X)$$
%...
%\end{example}
%\begin{example}(Gravity)
%$$\mcalE=\{\text{metrics on $X$}\}$$
%\end{example}
%以上是四种很经典的例子。
%Problem: How to construct

一个重要的问题是,如何去构造路径积分
$$\int_{\mcalE}\theta e^{i\mcalS/\hbar}$$
我们介绍\textbf{BV方法}(Batalin-Vilkovisky method),
其主要思想是用同调理论来解释测度论。

%We introduce a different method:BV method
%(Batalin-Vilkovisky)
%Philosophy: measure theory$\int_X\mapsto$ Homology theory.....
%Calculus

我们来考察有限维的情形。设$X$为$n$维紧致定向流形,
$\Omg\in\Omg_X^n$为$X$上的一个体积形式,
则$X$上的紧支光滑函数$f$关于该体积形式的积分可以视为如下:
\begin{eqnarray*}
\int_X: C^{\infty}_c(X)&\to&\bbR\\
f&\mapsto&\int_Xf\Omg
\end{eqnarray*}

%$$\int_X:\Omg_X\updot\to\bbR$$
%where $X$ is compact oriented manifold of dimension $n$,

我们考虑
$$\Omg_c(X):=\bigoplus_{p\geq 0}\Omg_c^p(X)$$
为紧支的微分形式,以及$\td:\Omg_c^p(X)\to\Omg_c^{p+1}(X)$为de Rham外微分。
众所周知,
$$H^n(\Omg_c\updot(X),\td)\cong\bbR$$
此式可以给出积分$\int_X$的同调解释:
\begin{eqnarray*}
\int_X:C_c^{\infty}(X)&\to&\bbR\\
f&\mapsto&[f\Omg]\in H^n(\Omg_c\updot(X),\td)\cong\bbR
\end{eqnarray*}

粗俗地说,我们把求$f$关于体积形式$\Omg$的积分
视为取$f\Omg$的同调类;在此意义下,de Rham复形
$(\Omg_C\updot,\td)$扮演了“测度”的角色。

%Observe:
%$$H^n_{DR}(X)=H^n(\Omg\updot,\td)\cong\bbR$$
%$$\int_X:\Omg^n\to H^n_{DR}\cong\bbR$$
%$$\alpha\mapsto[\alpha]$$
%$$\rightsquigarrow\int_X=H_{DR}^n$$
%$$(\Omg\updot(X),\td)\rightsquigarrow\text{measure}$$
%Question:how to $\int_X=H^n\quad n\to\infty$?

%%%%%%%%%%%%%%%%%%%%%%%%%%%%%%%%%%%%%%%%%%%%%%%%%%%%%%%%%
%%%%%%%%%%%%%%%%2019.3.26星期二 第五周%%%%%%%%%%%%%%%%%%%
%%%%%%%%%%%%%%%%%%%%%%%%%%%%%%%%%%%%%%%%%%%%%%%%%%%%%%%%%

%讲一些物理想法
%\textbf{Feynman Diagram}
%recall:
%$$D^n_{DR}=\int_X:\Omg\updot(X)\to\bbR$$
%what if $n\to \infty$?

在物理上我们常要面对无穷维空间,于是在此意义下,
我们需要关心$n\to\infty$时,$H^n(X)$是何物。
这是难以说清楚的,我们不妨换一个角度来看。

%Different philosophy:
%Let $\PV\updot(X):=\Gamma(X,\wedgeform{*}TX)$,
%$\Omg$ be a volume form($n$-form) on $X$,then
%$$f\mapsto\int_Xf\Omg$$
%$$\PV^k(X)\xra{\dashv\Omg}\Omg^{n-k}(X)$$
%is a 1-1 coorespondence.
%%%%缩并%%%%%%%

\begin{definition}
设$X$为$n$维紧致定向流形,$\Omg$为$X$上的一个体积形式,
则有$\Omg$诱导了多重切向量场$\PV\updot(X)$
与微分形式$\Omg_X\updot$之间的$C^{\infty}$-线性同构
\begin{eqnarray*}
\Gma_\Omg:\PV^k(X)&\to&\Omg^{n-k}_X\\
V&\mapsto& V\suobing\Omg
\end{eqnarray*}
其中$V\lrcorner\,\Omg$为$V$关于$\Omg$的缩并.
\end{definition}


在局部坐标下,若$\Omg=\td x^1\wedge\td x^2\cdots\wedge\td x^n$,
$$V=\p_{i_1}\wedge\p_{i_2}\wedge\cdots\wedge\p_{i_k}$$
为多重切向量场,其中指标$i_1<i_2<\cdots<i_k$,则容易知道
$$
  \Gma_\Omg(V)=V\suobing\Omg
= (-1)^{(i_1-1)+(i_2-1)+\cdots+(i_k-1)}
  \cdots\wedge\widehat{\td x^{i_1}}
  \wedge\cdots\wedge\widehat{\td x^{i_k}}\wedge\cdots
$$

\begin{example}
$$(\p_2\wedge\p_3)\suobing(\td x^1\wedge\td x^2\wedge\td x^3\wedge\td x^4)
=-\td x^1\wedge\td x^4$$
\end{example}
以此为例,缩并的运算规则可以理解为:
$\p_i$向右移动与$\td x^i$相遇而湮灭,
其中在$\p_i$移动的过程中穿过几个对象
($\p_j$或者$\td x^j$)就改变几次正负号
(这符合Koszul符号法则的“精神”)。

例如,如果$\Omg=e^{f(x)}\td x^1
\wedge\td x^2\wedge\td x^3\wedge\td x^4$,则
$$\Gma_\Omg(\p_2\wedge\p_3)=e^{f(x)}\td x^1\wedge\td x^4$$
再比如,对于体积形式$\Omg$本身,有
$$\Gma_\Omg^{-1}(\Omg)=1$$
也就是说$1\in\PV^0(X)$对应于$\Omg\in\Omg^n_X$.

当$V\in\PV^1(X)$为切向量场时,
$V\lrcorner\Omg=i_V(\Omg)$就是我们熟悉的内乘运算。

\begin{rem}(多重切向量场的内乘)
类似于关于切向量场$X$的内乘算子
$i_X:\Omg\updot_X\to \Omg^{\bullet-1}_X$,
我们也可以考虑多重切向量场$V\in\PV^p(X)$的内乘
$$i_V:\Omg\updot_X\to\Omg^{\bullet-p}_X$$
使得对任意$\omg\in\Omg^r_X\,(r\geq p)$,以及任意$W\in\PV^{r-p}(X)$,成立
$$\langle i_V(\omg),W\rangle=\langle\omg,V\wedge W\rangle$$
\end{rem}
特别注意,对于多重切向量场$V\in\PV\updot(X)$以及体积形式$\Omg$,
一般来说
$$V\suobing\Omg\neq i_V(\Omg)$$
它们两者之间会相差一些奇怪的正负号。
我们这里的$\PV^k(X)$与$\Omg^{n-k}_X$
的对应是通过缩并实现的,而不是内乘。

\begin{definition}(BV算子)

对于$n$维光滑定向流形$X$,
设$\Omg\in\Omg^n_X$为$X$上的一个体积形式,
定义算子$\yc_\Omg:\PV^{k}(X)\to\PV^{k-1}(X)$,使得下图交换:
$$
  \xymatrix{
     \PV^k(X)        \ar[r]^{\yc_\Omg}  \ar@{=}[d]_{\Gma_\Omg}
    &\PV^{k-1}(X)                       \ar@{=}[d]_{\Gma_\Omg}
  \\
     \Omg_X^{n-k}    \ar[r]^{\td}
    &\Omg_X^{n-k+1}
  }
$$
称$\yc_\Omg$为\textbf{BV算子}(Batalin-Vilkovisky operator)。
\index{Batalin-Vilkovisky operator\kong BV算子}
\end{definition}
%$$\PV^*(X)\leftrightarrow \Omg^{n-\bullet}(X)$$
%$$\yc\leftrightarrow \td$$
无非是将de Rham上链复形$(\Omg\updot_X,\td)$
通过体积形式同构为\textbf{上}链复形
$(\PV\updot(X),\yc_\Omg)$,其实没干什么事情。
特别注意我们规定$\PV^k(X)$的次数为$-k$,
使得$\yc_{\Omg}$是次数为$1$的微分算子(而不被看作边缘算子)。

注意到此时有上同调群的同构
$$H^n(\Omg\updot_X,\td)\cong H^0(\PV\updot(X),\yc_\Omg)$$
回顾我们对积分$\int_Xf\Omg$的同调解释,从而有
$$\int_X:f\mapsto[f]\in H^0(\PV\updot(X),\yc_\Omg)$$
也就是说我们可以把求函数$f$关于体积形式$\Omg$的积分转化成取$f$在
$(\PV\updot(X),\yc_\Omg)$的第零个同调类。这样的好处是,
容易向维数$n\to\infty$的情形推广,毕竟无论维数$n$如何升高,
我们取的总是第零个同调。

不过这样的代价是,问题转化为“如何构造无穷维空间上的BV算子”。

%this "$\yc$" is called "divergence operator w.r.t the volume form".
%when $x\in\PV^1(X)$,check: $\yc(v)=\div_{\Omg}v$.
%$$\td^2=0\Rightarrow\yc^2=0$$
%$$\PV^n\xra{\yc}\PV^{n-1}\xra{\yc}\PV^{n-2}\to\cdots$$
%$\yc$ is also called "BV operator".
%$$\int_{BV}:=H_0(\yc)$$
%good news: $0$ doesn't depend on $\dim(X)$!
%(so, we can $n\to\infty$???)
%\textbf{Difficulty}: when $n\to\infty$, we need to construct $\yc$.

\begin{rem}(广义散度)

事实上,如果$v\in\PV^1(X)$为$X$上的切向量场,则
$$\yc_\Omg(v)=\Div_\Omg(v)$$
正是我们熟悉的关于体积形式$\Omg$的散度。
\end{rem}
于是我们也俗称BV算子为多重切向量场的“广义散度”。

为了书写方便,我们引入一套高效的语言:Grassmann变量。

\begin{notation}(Grassmann变量)

对于$n$维流形$X$,以及$X$的局部坐标卡$U\subseteq X$,
我们考虑分次交换$\bbR$-代数
$$C^{\infty}(U)\ten\Free\{\theta_1,\theta_2,...,\theta_n\}\big/\sim$$
其中生成关系$\sim$为由
$\{\theta_i\theta_j+\theta_j\theta_i|1\leq i,j\leq n\}$生成的理想。
其中分次结构由
$$\deg\theta_i=-1\quad\forall 1\leq i\leq n$$
给出。
\end{notation}
容易发现,无非是将$\PV\updot(U)$当中的$\p_i$重新写为$\theta_i$,从而局部上
$$\PV\updot(U)=C^{\infty}(U)[\theta_1,...,\theta_n]$$
换句话说,$X$上的多重切向量场(局部上)
可以写为关于局部坐标$x^1,...,x^n$以及Grassmann变量的函数
$$\mu=\mu(x^1,...,x^n;\theta_1,...,\theta_n)\in\PV\updot(X)$$

{\color{blue}这里的Grassmann变量$\theta_i$是不是
Doubrovin-Zhang可积系统里面的“超变量”?
}

%\begin{example}
%Let $X$ is a manifold of finite dimension, $\dim X=n$,
%local coordinate $\{x^1,...,x^n\}$ in $\mcalU\subseteq X$,
%volume form
%$$\Omg=e^{f(x)}\td x^1\wedge\cdots\wedge\td x^n$$
%$$\Omg\updot(\mcalU):=C^{\infty}(\mcalU)[\td x^1,...,\td x^n]$$
%where $\td x^i\wedge\td x^j=-\td x^j\wedge\td x^i$.
%$\PV\updot(\mcalU):=C^{\infty}(\mcalU)[\p_1,...,\p_n]$
% where $\p^i\wedge\p^j=-\p^j\wedge\p^i$.
%\end{example}
%introduce grassman variables $\theta_1,...,\theta_n$,
%($\theta _i\cong \p_i$)
%define
%$$\PV\updot(\mcalU):=C^{\infty}(\mcalU)[\theta_1,...,\theta_n]$$
%%%%%%%非交换的微积分%%%%%%%%

\begin{definition}对于流形$X$,局部坐标下我们定义$-1$阶超导子
$$\pp{\theta_i}:\PV\updot(X)\to\PV\updot(X)$$
使得成立

  \begin{eqnarray*}
    \pp{\theta_i}f(x^1,...,x^n)=0,\qquad
    \pp{\theta_i}\theta_j=\delta^i_j
  \end{eqnarray*}

\end{definition}
$\pp{\theta_i}$服从$-1$阶超导子的超莱布尼茨法则,
即对任意$f,g\in\PV\updot(X)$为齐次元,成立
$$\pp{\theta_i}(fg)=\pfrac{f}{\theta_i}g+(-1)^{\deg f}f\pfrac{g}{\theta_i}$$

容易验证,超导子$\pp{\theta_i}$满足关系
$$\pp{\theta_i}\pp{\theta_j}=-\pp{\theta_j}\pp{\theta_i}$$
对任意$1\leq i,j\leq n$成立。特别地,$\left(\pp{\theta_i}\right)^2=0$.

\begin{prop}(BV算子的Grassmann变量表达式)

对于定向流形$X$,设体积形式
$$\Omg=e^{f(x)}\td x^1\wedge\cdots\wedge \td x^n$$
则关于$\Omg$的BV算子$\yc\Omg$在Grassmann变量的意义下具有表达式
$$\yc_{\Omg}=\pp{x^i}\pp{\theta_i}+\pfrac{f}{x^i}\pp{\theta_i}$$

%(Odd Laplacian)
%(这是BV算子的等价定义。。。)
\label{BV算子的超变量表达式-prop}
\end{prop}

\begin{proof}
直接验证之。对于任意
$$V=\mu(x^1,...,x^n)\theta_{i_1}\cdots\theta_{i_k}\in\PV^{k}(X)$$
则有
\begin{eqnarray*}
     \yc_\Omg V
&=&
     \Gma_\Omg^{-1}\circ\td\circ\Gma_\Omg(V)\\
&=&
     \Gma_\Omg^{-1}\circ\td
     \left[
       (-1)^{(i_1-1)+\cdots+(i_k-k)}\mu e^{f}
       \widehat{\td x^{i_1}}\wedge\cdots\wedge\widehat{\td x^{i_k}}
     \right]\\
&=&
     (-1)^{(i_1-1)+\cdots+(i_k-k)}
     \Gma_\Omg^{-1}
     \left[
       (\pfrac{\mu}{x^i}+\mu\pfrac{f}{x^i})e^f
       \sum_{l=1}^k
         (-1)^{i_l-l}
         \widehat{\td x^{i_1}}\wedge\cdots\wedge\td x^{i_l}
         \wedge\cdots\wedge\widehat{\td x^{i_k}}
     \right]\\
&=&
     (\pfrac{\mu}{x^i}+\mu\pfrac{f}{x^i})
     \sum_{l=1}^k
       (-1)^{l-1}
       \theta_{i_1}\cdots\widehat{\theta_{i_l}}\cdots \theta_{i_k}\\
&=&
    \left[
      \pp{x^i}\pp{\theta_i}+\pfrac{f}{x^i}\pp{\theta_i}
    \right](V)
\end{eqnarray*}
从而证毕。
\end{proof}

注意到BV算子的表达式
$$\yc_{\Omg}=\pp{x^i}\pp{\theta_i}+\pfrac{f}{x^i}\pp{\theta_i}$$
长得像二阶微分算子,甚至很像拉普拉斯算子——$\yc_\Omg$因此也被称为
\textbf{奇拉普拉斯算子}(odd Laplacian)。
\index{odd Laplacian\kong 奇拉普拉斯算子}

\begin{prop}设$X$为定向流形,
$\Omg=e^{f(x)}\td x^1\wedge\cdots\wedge\td x^n$为$X$的一个体积形式,
$\yc_\Omg$为关于$\Omg$的BV算子。定义
%Given $\yc_{\Omg}$(BV operator), we define
$$\{,\}:\PV\updot(X)\times\PV\updot(X) \to \PV\updot(X)$$
$$\{\alpha,\beta\}:=
\yc_{\Omg}(\alpha\wedge\beta)-(\yc_{\Omg}\alpha)\wedge\beta-
(-1)^{|\alpha|}\alpha\wedge\yc_{\Omg}\beta$$
即,“$\yc$成为超导子的代价”。
那么$\{,\}$不依赖于体积形式$\Omg$的选取。
%the failure of $\yc_{\Omg}$ being a derivation.
\label{另一种Schouten-Nijenhuis括号-def}
\end{prop}

\begin{proof}
直接验证即可。对任意$\alpha\in\PV^p(X)$以及$\beta\in\PV^q(X)$,成立
\begin{eqnarray*}
     \yc_\Omg(\alpha\wedge\beta)
&=&
     \pp{x^i}\pp{\theta_i}
     (\alpha\wedge\beta)
    +\pfrac{f}{x^j}\pp{\theta_j}
     (\afa\wedge\beta)\\
&=&
     \pp{x^i}
     \left(
       \pfrac{\afa}{\theta_i}
       \wedge\beta
      +(-1)^p\afa\wedge\pfrac{\beta}{\theta_i}
     \right)
    +\pfrac{f}{x^i}
     \left(
       \pfrac{\afa}{\theta_i}\wedge\beta
      +(-1)^p\afa\wedge\pfrac{\beta}{\theta_i}
     \right)\\
&=&
     \pmfrac{\afa}{x^i}{\theta_i}\wedge\beta
    +\pfrac{\afa}{\theta_i}\wedge\pfrac{\beta}{x^i}
    +(-1)^p\pfrac{\afa}{x^i}\wedge\pfrac{\beta}{\theta_i}\\
& &
    +(-1)^p\afa\wedge\pmfrac{\beta}{x^i}{\theta_i}
    +\pfrac{f}{x^i}\pfrac{\afa}{\theta_i}\wedge\beta
    +(-1)^p\pfrac{f}{x^i}\afa\wedge\pfrac{\beta}{\theta_i}\\
&=&
     (\yc_\Omg\afa)\wedge\beta
    +(-1)^p\afa\wedge(\yc_\Omg\beta)\\
& &
    +\pfrac{\afa}{\theta_i}\wedge\pfrac{\beta}{x^i}
    +(-1)^p\pfrac{\afa}{x^i}\pfrac{\beta}{\theta_i}
\end{eqnarray*}
从而得到
$$
  \{\alpha,\beta\}
=
  \pfrac{\afa}{\theta_i}\wedge\pfrac{\beta}{x^i}
 +(-1)^p\pfrac{\afa}{x^i}\wedge\pfrac{\beta}{\theta_i}
$$
从而与$\Omg$的选取无关。
\end{proof}

我们之前也见过类似的运算:Schouten-Nijenhuis括号
(见定义\ref{Schouten-Nijenhuis定义-def});
而这里的$\{,\}$是“另一个版本的Schouten-Nijenhuis括号”:

\begin{lemma}定义$\ref{另一种Schouten-Nijenhuis括号-def}$中的括号
$$\{,\}:\PV^p(X)\times\PV^q(X)\to\PV^{p+q-1}(X)$$
满足性质:对任意$\afa\in\PV^p(X),\beta\in\PV^q(X),\gamma\in\PV^r(X)$,成立:

(1)超反交换性
$$\{\afa,\beta\}=(-1)^{pq}\{\beta,\afa\}$$

(2)超莱布尼茨法则
$$\{\afa,\beta\wedge\gamma\}
=\{\afa,\beta\}\wedge\gamma
+(-1)^{(p-1)q}\beta\wedge\{\afa,\gamma\}$$

(3)若$p=q=1$,则$\{,\}$退化为切向量场李括号:
$$\{\alpha,\beta\}=[\afa,\beta]$$
\end{lemma}

注意超反交换性(1)与性质\ref{Schouten-Nijenhuis公理-prop}
的(2)在正负号上有所出入。

\begin{proof}使用表达式
$$
  \{\alpha,\beta\}
=
  \pfrac{\afa}{\theta_i}\wedge\pfrac{\beta}{x^i}
 +(-1)^p\pfrac{\afa}{x^i}\wedge\pfrac{\beta}{\theta_i}
\eqno{(*)}
$$
直接验证即可,并不困难。对于$\alpha\in\PV^p(X),\,\beta\in\PV^q(X)$
以及$\gamma\in\PV^r(X)$,有
\begin{eqnarray*}
     \{\afa,\beta\}
&=&
     \pfrac{\afa}{\theta_i}\wedge\pfrac{\beta}{x^i}
    +(-1)^p\pfrac{\afa}{x^i}\wedge\pfrac{\beta}{\theta_i}\\
&=&
     (-1)^{(p-1)q}\pfrac{\beta}{x^i}\wedge\pfrac{\afa}{\theta_i}
    +(-1)^{p+p(q-1)}
     \pfrac{\beta}{\theta_i}\wedge\pfrac{\afa}{x^i}\\
&=&
     (-1)^{pq}
     \left(
       \pfrac{\beta}{\theta_i}\wedge\pfrac{\afa}{x^i}
      +(-1)^q\pfrac{\beta}{x^i}\wedge\pfrac{\afa}{\theta_i}
     \right)\\
&=&
     (-1)^{pq}\{\beta,\alpha\}
\end{eqnarray*}
于是超反交换性成立;再看超莱布尼茨法则,
\begin{eqnarray*}
     \{\afa,\beta\wedge\gamma\}
&=&
     \pfrac{\afa}{\theta_i}\wedge
     \pp{x^i}(\beta\wedge\gamma)
    +(-1)^p\pfrac{\afa}{x^i}\wedge
     \pp{\theta_i}(\beta\wedge\gamma)\\
&=&
     \pfrac{\afa}{\theta_i}\wedge
     \left(
       \pfrac{\beta}{x^i}\wedge\gamma
      +\beta\wedge\pfrac{\gamma}{x^i}
     \right)
    +(-1)^p\pfrac{\afa}{x^i}\wedge
     \left(
       \pfrac{\beta}{\theta_i}\wedge\gamma
      +(-1)^q\beta\wedge\pfrac{\gamma}{\theta_i}
     \right)\\
&=&
     \{\alpha,\beta\}\wedge\gamma
    +(-1)^{q(p-1)}
     \beta\wedge\pfrac{\afa}{\theta_i}\wedge\pfrac{\gamma}{x^i}
    +(-1)^{pq+p+q}
     \beta\wedge\pfrac{\afa}{x^i}\wedge\pfrac{\gamma}{\theta_i}\\
&=&
     \{\alpha,\beta\}\wedge\gamma
    +(-1)^{(p-1)q}\beta\wedge\{\alpha,\gamma\}
\end{eqnarray*}
而(3)是更加容易验证的,从略。
\end{proof}
可以体会到Grassmann变量$\theta_i$以及超导子$\pp{\theta_i}$在
张量计算上的优越性:将本该必然面对的数学归纳法、
组合恒等式转化为直接的暴力计算。

{\color{gray}
事实上,我们还可以用$(*)$来暴力验证$\{,\}$的超雅可比恒等式:
$$
  (-1)^{pr}\{\{\afa,\beta\},\gamma\}
 +(-1)^{qp}\{\{\beta,\gamma\},\afa\}
 +(-1)^{rq}\{\{\gamma,\afa\},\beta\}
=0
$$
或者换句话说
$$\{\alpha,\{\beta,\gamma\}\}=(-1)^{p-1}\{\{\alpha,\beta\},\gamma\}
+(-1)^{(p-1)(q-1)}\{\beta,\{\afa,\gamma\}\}$$
(但是这个看起来不像是导子的样子)(此处待仔细验证)
}

%Check: $\{,\}$ is a Schouten-Nijenhuis bracket(up to sign).
%(independent of the choice of $f(x)$)
%In particular,$\{,\}$ is independent of choice of $\Omg$.
%\textbf{Quantization}
%$$\{,\}\xra{?}\yc_{\Omg}$$
%这个过程与量子化的过程是一样的??WTF??

\section{费曼图}
%\textbf{Feymann Diagram}
首先我们考察一个BV算子的例子:
\begin{Example}
考虑一维流形$X=\bbR$,体积形式
$$\Omg:=\frac{1}{\sqrt{2\pi}}e^{-\frac{1}{2}x^2}\td x$$
则BV算子
$$\yc_{\Omg}=\pp{x}\pp{\theta}-x\pp{\theta}$$

特别地,我们得到
$$
  \int_\bbR x^k\Omg
=
  \left\{
    \begin{array}{ll}
      0          &  x=2k+1\\
      (2k-1)!!   &  x=2k
    \end{array}
  \right.
$$
%符号可能不对,小心check
%$\PV=\bbR[x,\theta]$, and
%$$\int_{BV}:\bbR[x,\theta]\mapsto H^0(\yc_{\Omg})$$
\end{Example}

\begin{proof}
注意
$$\Omg=e^{-\frac{1}{2}(x^2-\log 2\pi)}\td x$$
从而由性质\ref{BV算子的超变量表达式-prop},直接写出
$$\yc_{\Omg}=\pp{x}\pp{\theta}-x\pp{\theta}$$

注意到积分的(上)同调解释
\begin{eqnarray*}
  \int_X:C^{\infty}&\to& H^0(\PV\updot(X),\yc_\Omg)\cong\bbR\\
  g&\mapsto& [g]
\end{eqnarray*}
而注意到对任意$x^k\theta\in \PV^1(X)$,
在$H^0(\PV\updot(X),\yc_\Omg)$当中成立
$$0=[\yc\Omg x^k\theta]=(\pp{x}\pp{\theta}-x\pp{\theta})[x^k\theta]
=k[x^{k-1}]-x^{k+1}$$
因此对任意$k\geq 0$,成立
$$[x^{k+2}]=(k+1)[x^k]$$
递推得
$$
  [x^n]=
  \left\{
    \begin{array}{ll}
      (2k-1)!![1]  & n=2k\\
      (2k)!![x]    & n=2k+1
    \end{array}
  \right.
$$
最后注意到
\begin{eqnarray*}
\int_\bbR
  \frac{1}{\sqrt{2\pi}}
  e^{-\frac{1}{2}x^2}\td x=1\qquad
\int_\bbR
  \frac{x}{\sqrt{2\pi}}
  e^{-\frac{1}{2}x^2}\td x=0
\end{eqnarray*}
从而完。
\end{proof}
%$$\int:\bbR[x]\to\bbR$$
%$$g\mapsto\int_{\bbR}g\Omg$$
%if $g=g(x)$, then $\yc_{\Omg}g=0$. Let $[g]$ be the
%$\yc_{\Omg}$- homology class,
%$$\yc_{\Omg}(x^{m-1}\Omg)
%=(m-1)x^{m-2}
%-x^m$$
%$$\Rightarrow [x^m]=(m-1)[x^{m-2}]$$
%so,
%$$
%[x^m]=
%\left\{
%  \begin{array}{cc}
%    0  &  m \text{ is odd}\\
%    (2k-1)!![1]  & m=2k
%  \end{array}
%\right.
%$$
%so,
%$$\int_{\bbR}x^{2k}\Omg=
%(2k-1)!!\int_{\bbR}\Omg=(2k-1)!!$$

\begin{lemma}条件接上,仍考虑体积形式
$$\Omg:=\frac{1}{\sqrt{2\pi}}e^{-\frac{1}{2}x^2}\td x$$
定义算子$\mcalU:\bbR[x,\theta]\to\bbR[x,\theta]$为
$$\mcalU:=
  e^{\frac{1}{2}\pp{x}\pp{x}}$$
则BV算子$\yc_\Omg$满足
$$\yc_{\Omg}=\mcalU^{-1}(-x\pp{\theta})\mcalU$$
\end{lemma}

\begin{proof}注意到众所周知的公式%use the formula
$$e^ABe^{-A}=e^{\ad_A}B$$%then
特别地,在这里%$$\yc_{\Omg}=\pp{x}\pp{\theta}-x\pp{\theta}$$
$$A=-\frac{1}{2}\pp{x}\pp{x},\qquad B=-x\pp{\theta}$$%then

注意到
\begin{eqnarray*}
[A,B]=
     \frac{1}{2}
     [\pp{x}\pp{x},x\pp{\theta}]
=    \pp{x}\pp{\theta}
\end{eqnarray*}
进而
$$[A,[A,B]]=-\frac{1}{2}[\pp{x}\pp{x},\pp{x}\pp{\theta}]=0$$
于是
\begin{eqnarray*}
     \mcalU^{-1}(-x\pp{\theta})\mcalU
&=&
     x^ABe^{-A}
 =
     e^{\ad A}B\\
&=&
     B+[A,B]
 =
     -x\pp{\theta}+\pp{x}\pp{\theta}
 =
     \yc_\Omg
\end{eqnarray*}
从而得证。
%$$e^ABe^{-A}=\pp{x}\pp{\theta}$$
%where $[A,[A,B]]=0$.
\end{proof}

%so,
%$$\mcalU\circ\yc_{\Omg}=(-x\pp{\theta})\circ\mcalU$$
%so,
%$$\mcalU:(\bbR[x,\theta,\yc_{\Omg}])\to
%(\bbR[x,m\theta],-x\pp{\theta})$$
%is a co-chain map.
此引理表明,有如下的交换图表:
$$
  \xymatrix{
     \bbR[x,\theta] \ar[r]^{\yc_\Omg}  \ar[d]^{\mcalU}
    &\bbR[x]                           \ar[d]^{\mcalU}
  \\
     \bbR[x,\theta] \ar[r]^{-x\pp{\theta}}
    &\bbR[x]
  }
$$
以及$\mcalU$诱导上同调群的同构
$$\mcalU:H^0(\bbR[x,\theta],\yc_\Omg)
\xra{\sim}H^0(\bbR[x,\theta],-x\pp{\theta})$$

%$$H\updot(\bbR[x,\theta],-x\pp{\theta})$$
%$$\bbR[x]\theta\xra{-x\pp{\theta}}\bbR[x]$$
%$$g\mapsto -xg(0)$$
%$H^{-1}=0$,$H^0=\bbR$.
%$$\mcalU[g(x)]_{\yc_{\Omg}}=[\mcalU(g)(x)]_{-x\pp{\theta}}
%=[\mcalU(g)(0)]_{-x\pp{\theta}}
%=\mcalU(g)(0)[1]$$

\begin{prop}条件承上,则对于任意的多项式函数$g\in\bbR[x]$,成立
$$
  \int_\bbR g\Omg
=\left.
   e^{-\frac{1}{2}\pp{x}\pp{x}}
 \right|_{x=0}g
$$
\end{prop}
%我们只谈论多项式函数,是为了偷懒,
%不太想讨论收敛性(事实上解析性质很重要)。

\begin{proof}
只需要考虑$[\mcalU(g)]\in H^0(\bbR[x,\theta],-x\pp{\theta})$。
注意到对任意$k\geq 0$,
$$-x\pp{\theta}(x^k\theta)=-x^{k+1}$$
也就是说在$H^0(\bbR[x,\theta],-x\pp{\theta})$当中,
$[x^k]=0$对任意$k\geq 1$成立,从而
$$[\mcalU(g)]=\mcalU(g)(0)
=\left.
   e^{-\frac{1}{2}\pp{x}\pp{x}}
 \right|_{x=0}g
$$
从而易得。
\end{proof}
这个性质将求积分转化为求导,大大简化运算。
(与复变函数的留数定理异曲同工?)
%so,
%$$\int_{\bbR}g(x)\Omg=\mcalU(g)(0)=
%e^{\frac{1}{2}\pp{x}\pp{x}}|_{x=0}g(x)$$
%More generally, check:

更一般地,容易证明对任意$g\in\bbR[x]$
$$\int_{\bbR}g(x+a)\Omg=
\left.
  e^{\frac{1}{2}\pp{x}\pp{x}}
\right|_{x=a}g(x)$$

\begin{Example}现在我们考虑积分%Consider the integral
$$
  \int_{\bbR}
    e^{\big(-\frac{1}{2}x^2+\frac{\lmd}{3!}(x+a)^3\big)\big/\hbar}
    \frac{\td x}{\sqrt{2\pi\hbar}}
$$
其中$\lmd,a\in\bbR$,
在这里体积形式$\Omg=e^{-\frac{1}{2}x^2/\hbar}
\frac{\td x}{\sqrt{2\pi\hbar}}$.
\end{Example}

此式中的“$-\frac{1}{2}x^2$”在物理上可以认为是“自由能”,
三次项$\frac{\lmd}{3!}(x-a)^3$则为“相互作用量”。
相互作用量的存在,使得此积分发散。

处理该积分有两种常见方式:其一是将它视为复平面上的积分,
并且重新规定积分路径(这会出现Airy函数);
或者考察它的($\hbar\to 0$的)渐近展开
$$
  \sum_{n\geq 0}\frac{1}{n!}
    \int_\bbR
      \left(
        \frac{\lmd (x+a)^3}{3!\hbar}
      \right)^n
      e^{-\frac{1}{2}x^2/\hbar}
      \frac{\td x}{\sqrt{2\pi\hbar}}
$$

%this integral is not convergent.... how to deal with it?
%(1)Change the integration Contour inside $\bbC$
%%%%%%%Contour%%%%%%%
%(Airy function)
%(2)We consider only the asymptotic series

在此我们选择后者,将$e^{-\frac{1}{2}x^2\big/\hbar}$展开,
被积函数展开后的每一项
$$
  \frac{1}{n!}
    \int_\bbR
      \left(
        \frac{\lmd (x+a)^3}{3!\hbar}
      \right)^n
      e^{-\frac{1}{2}x^2/\hbar}
      \frac{\td x}{\sqrt{2\pi\hbar}}
$$
都可以使用同调的方法计算(与之前的例子完全类似):
直接套用性质\ref{BV算子的超变量表达式-prop},此时的BV算子为
$\pp{x}\pp{\theta}-\frac{x}{\hbar}\pp{\theta}$,其实不妨相差常数倍,令
$$\yc_{\Omg}:=\hbar\pp{x}\pp{\theta}-x\pp{\theta}$$
并且令
$$\mcalU_{\hbar}:=e^{\frac{\hbar}{2}\pp{x}\pp{x}}$$
则与之前完全类似,有
$$\yc_\Omg
=\mcalU_\hbar^{-1}\circ
\left(-x\pp{\theta}\right)\circ\mcalU_\hbar$$
从而易知
\begin{eqnarray*}
     \int_\bbR
       e^{\big(
            -\frac{1}{2}x^2
            +\frac{\lmd}{3!}(x+a)^3
          \big)
          \big/\hbar}
       \frac{\td x}{\sqrt{2\pi\hbar}}
&\sim&
     \sum_{m\geq 0}\frac{1}{m!}
       \int_\bbR
         \left(
           \frac{\lmd(x+a)^3}{3!\hbar}
         \right)^m
         e^{-\frac{1}{2}x^2\hbar}
       \frac{\td x}{\sqrt{2\pi\hbar}}\\
&=&
     \left.
       \sum_{m\geq 0}\frac{1}{m!}
         e^{\frac{1}{2}\hbar\pp{x}\pp{x}}
         \left(
           \frac{\lmd x^3}{3!\hbar}
         \right)^m
     \right|_{x=a}\\
&=&
     \sum_{k,m\geq 0}
       \frac{1}{k!}
       \frac{1}{m!}
       \left(
         \frac{1}{2}\hbar\p_a^2
       \right)^k
       \left(
         \frac{\lmd a^3}{3!\hbar}
       \right)^m
\end{eqnarray*}

上式最右端具有组合意义,我们接下来详细说明。
记$\mcalP:=\frac{1}{2}\hbar\p_{a}^2$
称之为\textbf{传播子}(propagator),
\index{propagator\kong 传播子}
再记$\mcalI(a):=\frac{\lmd a^3}{3!\hbar}$为“相互作用能”,则上式为
$$
     \int_\bbR
       e^{\big(
            -\frac{1}{2}x^2
            +\frac{\lmd}{3!}(x+a)^3
          \big)
          \big/\hbar}
       \frac{\td x}{\sqrt{2\pi\hbar}}
 \sim
     \sum_{k,m\geq 0}
       \frac{1}{k!}
       \frac{1}{m!}
       \mcalP^k\mcalI^m(a)
$$

我们先来观察$k=1$的情形,
看看$\mcalP\mcalI^m(a)$是什么东西。
记$\mcalP_s=\mcalP_t:=\sqrt{\hbar/2}\p_{a}$,则有
$$\mcalP=\mcalP_t\mcalP_s$$
再令$\mcalI_1=\mcalI_2=\cdots=\mcalI_m:=\mcalI$,则
\begin{eqnarray*}
     \mcalP\mcalI^m(a)
&=&
     \mcalP_t\mcalP_s
     \left(
       \mcalI_1(a)\mcalI_2(a)\cdots\mcalI_m(a)
     \right)\\
&=&
     \mcalP_t
     \left(
       \sum_{u=1}^m
         \mcalI_1(a)\cdots\mcalP_s\mcalI_u(a)\cdots\mcalI_m(a)
     \right)\\
&=&
     \left(
       \sum_{1\leq u<v\leq m}
         \mcalI_1(a)\cdots\mcalP_s\mcalI_u(a)
         \cdots\mcalP_t\mcalI_v(a)\cdots\mcalI_m(a)
     \right.\\
& &
      \quad
      +\sum_{1\leq u\leq m}
         \mcalI_1(a)\cdots\mcalP_t\mcalP_s\mcalI_u(a)
         \cdots\mcalI_m(a)\\
& &
      \quad
      \left.
      +\sum_{1\leq v<u\leq m}
         \mcalI_1(a)\cdots\mcalP_t\mcalI_v(a)
         \cdots\mcalP_s\mcalI_u(a)\cdots\mcalI_m(a)
     \right)
\end{eqnarray*}

我们将$\mcalI_1,...,\mcalI_m$视为$m$个“顶点”,将
$$\mcalI_1(a)\cdots\mcalP_s\mcalI_u(a)
\cdots\mcalP_t\mcalI_v(a)\cdots\mcalI_m(a)$$
视为从“顶点”$u$出发,到“顶点”$v$的“有向边”,
则上式可以粗俗地说成“对所有的$m$个顶点、$1$条边的图求和”。
类似地,考虑
$$\mcalP^k\mcalI^m(a)=\underbrace{\mcalP\cdots\mcalP}_{k}
\underbrace{\mcalI(a)\cdots\mcalI(a)}_m$$
然后类似地展开,得到“对所有$m$个顶点、$k$条边的图求和”。

我们将以上严格表述之,然后得到\textbf{费曼图展开公式}。
我们之前在介绍Kontsevich量子化公式的时候引入了
“带标记的有向图”的概念(见定义\ref{带标记的有向图-def})。
在这里,我们允许出现回路(即,始点与终点相同的有向边),
也允许出现“多箭头”(即从某个点出发到某个点的边可能不止一条)。
但是我们要求顶点集与边集都是有限集。

\begin{notation}(带标记的有向图的有向底图)
对于带标记的有向图$\Gma=(\Gma_0,\Gma_1,\veps)$,
则$\Gma$可以遗忘为图论当中通常的\textbf{多重有向图},
后者称为前者的\textbf{有向底图},记为$\underline{\Gma}$.
\end{notation}

遗忘的方式为“将边的名称去掉”。此操作是显然的,例如
$$
  \xymatrix{
     1 \ar@/^1pc/[r]^{a}  \ar@/_1pc/[r]_{b}
   & 2
  }
\quad
\rightsquigarrow
\quad
  \xymatrix{
     1 \ar@/^1pc/[r]     \ar@/_1pc/[r]
   & 2
  }
$$

我们更习惯将带标记的有向图$\Gma$的顶点集记为$V$,边集记为$E$
(原来使用的记号$\Gma_0,\Gma_1$废止)。
对于有限集$V,E$,定义集合
$$\mcalG_{V,E}:=\{\text{以$V$为顶点集,
以$E$为边集的全体带标记的有向图之全体}\}$$
则容易构造一一对应$\mcalG_{V,E}\cong\{\veps:E\to V\times V\}$.

\begin{definition}
(置换群$S_V\times S_E$在集合$\mcalG_{V,E}$上的作用)

对于有限集合$V,E$,定义群$S_V\times S_E$在集合$\mcalG_{V,E}$上的作用如下:
对$\mcalG_{V,E}$中的任意元素$\veps:E\to V\times V$,
以及置换$\sgm\in S_V,\tau\in S_E$,令
$$(\sgm,\tau).\veps=(\sgm\times\sgm)\circ\veps\circ\tau^{-1}$$
其中
\begin{eqnarray*}
\sgm\times\sgm: V\times V&\to& V\times V\\
(v_1,v_2)&\mapsto&(\sgm(v_1),\sgm(v_2))
\end{eqnarray*}
\end{definition}

讲人话,无非是将带标记的有向图的各顶点、各边的名称重新排列一下。
例如,$V=\{1,2,3\},\,E=\{a,b,c\}$,带标记的有向图
$$
  \Gma=
  \xymatrix{
      1  \ar@/^1pc/[r]^a
    & 2  \ar@/^1pc/[l]^b  \ar[r]^c
    & 3
  }
$$
考虑置换
$$
     \sgm=
  \begin{pmatrix}
    1 & 2 & 3\\
    1 & 3 & 2
  \end{pmatrix}
\qquad
     \tau=
  \begin{pmatrix}
    a & b & c\\
    b & c & a
  \end{pmatrix}
$$
则有
$$
  (\sgm,\tau).\Gma
=\quad
  \xymatrix{
      1  \ar@/^1pc/[r]^b
    & 3  \ar@/^1pc/[l]^c  \ar[r]^a
    & 2
  }
\quad=\quad
  \xymatrix{
      1  \ar@/^1pc/[rr]^b
    & 2  
    & 3  \ar@/^1pc/[ll]^c  \ar[l]_a
  }
$$

以下性质显然成立:
\begin{lemma}
对于有限集合$V,E$,以及$\Gma,\Gma'\in\mcalG_{V,E}$,
则它们的有向底图(作为多重有向图)同构,当且仅当它们位于
群$S_v\times S_E$在$\mcalG_{V,E}$作用的同一个轨道上。
\end{lemma}

也就是说,群$S_V\times S_E$作用的轨道类,无非是底图的同构类。

\begin{example}($|V|=3,|E|=2$的轨道类)

令$V=\{1,2,3\},E=\{a,b\}$,
我们给出$S_V\times S_E$在$\mcalG_{V,E}$作用的轨道类如下:
$$
  \begin{tabular}{cc}
  \hline
  轨道类   &   轨道长度\\
  \hline
    \xymatrix{
       \bullet  \ar[r]
      &\bullet  \ar[r]
      &\bullet
    }
    &$12$
  \\
    \tabularbigrow
    \xymatrix{
       \bullet  \ar[r]
      &\bullet  
      &\bullet  \ar[l]
    }
    &$6$
  \\
    \tabularbigrow
    \xymatrix{
       \bullet  
      &\bullet  \ar[r] \ar[l]
      &\bullet
    }
    &$6$
  \\
    \tabularbigrow
    \xymatrix{
       [\bullet]  \ar[r] 
      &\bullet  
      &\bullet
    }
    &$12$
  \\
    \tabularbigrow
    \xymatrix{
       [\bullet]  
      &\bullet  \ar[l]
      &\bullet
    }
    &$12$
  \\
    \tabularbigrow
    \xymatrix{
       [\bullet]
      &\bullet  \ar[r]
      &\bullet  
    }
    &$12$
  \\
    \tabularbigrow
    \xymatrix{
       \bullet  \ar@/^1pc/[r]
      &\bullet  \ar@/^1pc/[l]
      &\bullet
    }
    &$6$
  \\
    \tabularbigrow
    \xymatrix{
       \bullet  \ar@/^1pc/[r] \ar@/_1pc/[r]
      &\bullet  
      &\bullet
    }
    &$6$
  \\
    \tabularbigrow
    \xymatrix{
       [[\bullet]]  
      &\bullet  
      &\bullet
    }
    &$3$
  \\
    \tabularbigrow
    \xymatrix{
       [\bullet]  
      &[\bullet]  
      &\bullet
    }
    &$6$
  \\
  \hline
  \end{tabular}
$$
这里用有向底图来表示轨道类。
表格中的方括号的含义是,以方括号内的顶点为端点的一条闭路(即从该点出发指向自己的箭头);
嵌套两层方括号就是两条闭路,以此类推。
\end{example}

不要忘记,我们引入这些图论概念,
是为了描述求导运算。

\begin{notation}(费曼规则)(Feynman's rule)

对于带标记的有向图$\Gma=(V,E,\veps)$,定义
$$
     w_{\Gma}(\mcalP,\mcalI(a))
:=
     \prod_{v\in V}
     \mcalP_s^{S(v)}
     \mcalP_t^{T(v)}
     \mcalI(a)
$$
其中$\mcalP:=\frac{1}{2}\hbar\p_a^2$为传播子,
$\mcalP_s=\mcalP_t:=\sqrt{\hbar/2}\p_a$,
$\mcalI(a):=\frac{\lmd a^3}{3!\hbar}$为“相互作用能”。
并且对于顶点$v\in V$,
$$S(v):=|\{e\in E|s(e)=v\}|$$
$$T(v):=|\{e\in E|t(e)=v\}|$$
分别为顶点$v$的\textbf{出度}与\textbf{入度}。
\index{Feynman's rule\kong 费曼规则}
\end{notation}

翻译成人话,对于带标记的有向图
(实际上多重有向图足矣,边的名称没贡献)$\Gma$,
我们按照如下规则给该图赋值:首先对图$\Gma$的每一个顶点赋值,
“有几条边经过此点,就求几次导”;然后将所有顶点的数值相乘。

\begin{example}注意到这里的$\mcalI(a)=\frac{\lmd a^3}{3!\hbar}$为
关于$a$的三次多项式,而$\mcalP_s=\mcalP_t$为一阶微分算子,
从而如果图$\Gma$的某个顶点的度数(入度与出度之和)大于三,那么
$w_\Gma(\mcalP,\mcalI)=0$
\end{example}
因此,我们只需要考虑每个顶点的度数都不超过$3$的图,
这些$\Gma$才能使得$w_\Gma(\mcalP,\mcalI)$取值非平凡。


\begin{lemma}
\begin{eqnarray*}
     \int_\bbR
       e^{\big(
            -\frac{1}{2}x^2
            +\frac{\lmd}{3!}(x+a)^3
          \big)
          \big/\hbar}
       \frac{\td x}{\sqrt{2\pi\hbar}}
&\sim&
     \sum_{k,m\geq 0}
       \frac{1}{k!}
       \frac{1}{m!}
       \left(
         \frac{1}{2}\hbar\p_a^2
       \right)^k
       \left(
         \frac{\lmd a^3}{3!\hbar}
       \right)^m\\
&=&
     \sum_{k,m\geq 0}
       \frac{1}{k!}
       \frac{1}{m!}
       \sum_{\Gma\in\mcalG_{m,k}}
         w_\Gma(\mcalP,\mcalI)
\end{eqnarray*}
其中$\mcalG_{m,k}:=\mcalG_{\{v_1,...,v_m\},\{e_1,...,e_k\}}$,
视为$m$个顶点、$k$条边的带标记的有向图之全体。
\end{lemma}
此式的等号显然成立,仅仅是换了一种说法。

我们将给出因子$\frac{1}{k!}\frac{1}{m!}$的组合解释。
注意到置换群$S_m\times S_k$在$\mcalG_{m,k}$的作用,
其轨道之全体记为$\underline{\mcalG}_{m,k}$,则由之前的论述,
$\underline{\mcalG}_{m,k}$可被视为$m$个顶点、$k$条边的多重有向图之全体。
再注意多重有向图也可按照费曼规则赋值(甚至多重无向图也可以)。

\begin{definition}(带标记的有向图的自同构群)

对于带标记的有向图$\Gma\in\mcalG_{m,k}$,定义其自同构群
$$\Aut(\Gma):=\{\fai\in S_m\times S_k|\fai.\Gma=\Gma\}$$
\end{definition}
其实就是群$S_m\times S_k$在$\Gma$处的稳定子群。

对于多重有向图$\underline{\Gma}\in\underline{\mcalG_{m,k}}$,
则$\underline{\Gma}$可视为$S_m\times_k$在$\mcalG_{m,k}$作用的一条轨道,
该轨道的长度记作$\ell(\underline{\Gma})$;而对于带标记的有向图$\Gma$,
记$\ell(\Gma)$为$\Gma$所在轨道的长度。
则由群论的轨道计数知,
$$
  m^{2k}=|\mcalG_{m,k}|
=\sum_{\underline{\Gma}\in\underline{\mcalG}_{m,k}}
  \ell(\underline{\Gma})
$$
$$
m!k!=|S_m\times S_k|=
\ell(\Gma)|\Aut(\Gma)|
$$

而对于多重有向图$\underline{\Gma}$,我们不去定义它的自同构群,但是注意到
$$|\Aut(\underline{\Gma})|:=|\Aut(\Gma)|$$
是良定的,与代表元的选取无关,因为同一轨道的不同元素的稳定子群共轭。

综上所述,我们得到了如下费曼图公式:
\begin{thm}(费曼图公式)(Feynman diagram formula)

记号同上,则有
\begin{eqnarray*}
     \int_\bbR
       e^{\big(
            -\frac{1}{2}x^2
            +\frac{\lmd}{3!}(x+a)^3
          \big)
          \big/\hbar}
       \frac{\td x}{\sqrt{2\pi\hbar}}
&\sim&
     \sum_{k,m\geq 0}
       \sum_{\Gma\in\underline{\mcalG}_{k,m}}
         \frac{w_\Gma(\mcalP,\mcalI)}
              {|\Aut(\Gma)|}\\
&=&
     \sum_{\Gma\text{为多重有向图}\atop\text{互不同构}}
       \frac{w_\Gma(\mcalP,\mcalI)}
              {|\Aut(\Gma)|}
\end{eqnarray*}
\end{thm}
%%%%%%%%微积分%%%%%%%%%
%introduce graph with cubic vertex,
%each edge: we put $\hbar\pp{a}\pp{a}$
%each point: we put $\frac{\lmd a^3}{3!\hbar}$
%$$\omg_{\gamma}(a)$$
%by the above rule.
%%%%%%这又是什么图%%%%%%%
%\begin{thm}(Feymann graph formula)
%this integration is
%$$
%   \int_{\bbR}
%     e^{(-\frac{1}{2}x^2+\frac{\lmd}{3}(x+a)^3)/\hbar}
%     \frac{\td x}{\sqrt{2\pi\hbar}}
%=
%   \sum_{\Gamma:\text{trivalent graphs}}
%     \frac{w_{\Gamma}(a)}
%          {|\text{Aut}(\gamma)|}
%=
%   \sum
%$$
%\end{thm}
%\begin{proof}
%check.
%\end{proof}

%In general, consider
%%%%%%%%%%%%%%%%%%%2019.4.1 第六周周一%%%%%%%%%%%%%%%%%%%%%%%
\section{传播子与重整化群流算子}

Recall:费曼图——理解积分的渐近展开式。

Asymptotic expansion of
$$\int_\bbR e^{(-\frac{1}{2}x^2+I(x+a))/\hbar}\frac{\td x}{\sqrt{2\pi t_0}}$$
where
$$I(x)=\sum_{n\geq 3}\frac{\lmd_n}{n!}x^n
=e^{\frac{\hbar}{2}\p_x^2}e^{I(a)/\hbar}$$

Feymann Graph formula

$$=\exp\left(\sum_{\Gamma-\text{Commu graphs}}
\frac{\omg_\Gamma(a)}{|Aut(\Gamma)|}\right)$$

Let
$$\mcalP=\frac{1}{2}\p_a^2$$
is called propagator(传播子)
$$e^{\omg(\mcalP,I)/\hbar}=e^{\hbar\mcalP}e^{I/\hbar}$$
$$\omg(\mcalP,I)=\hbar\sum_{\Gamma}\frac{\omg_\Gamma(\mcalP,I)}{|Aut(\Gamma)|}$$

$$I\mapsto\omg(\mcalP,I)$$

Today:

\begin{prop}
Let $\bbR\fps{,\hbar}^+=x^3\bbR\fps{x}\oplus\hbar\bbR\fps{x,\hbar}\subseteq\bbR\fps{x,\hbar}$,
(at least cubic modulo $\hbar$),then
$$\omg(\mcalP,-):I\to\omg(\mcalP,I)$$
is a well-defined transformation on $\bbR\fps{x,\hbar}^+$.
$$\omg(\mcalP,-):\bbR\fps{x,\hbar}^+\mapsto\bbR\fps{x,\hbar}^+$$

and its inverse is $\omg(-\mcalP,-)$.
\end{prop}

\begin{proof}
Let $I\in\bbR\fps{x,\hbar}^+$,
$$I=\sum_{k,g\geq 0}I_{k,g}\hbar^g$$
where $I_{k,g}$ has degree $k$ polynomial on $x$.

If $I_{k,g}\neq 0$, then $2g-2+k\geq 0$
and equality holds only if when $g=1,k=0$.
we need to prove
$$\hbar\sum_{\Gamma}\frac{\omg_\Gamma(\mcalP,I)}{\hbar}$$
is well defined, and in $\bbR\fps{x,\hbar}^+$.

Let $\Gma$ be a connected graph,

…………
%%%%%%%2334%%%%%%%%%%%%%%
%证了一页多。。。见手稿
\end{proof}

$$\bbR\fps{x,\hbar}^+\to \bbR\fps{x,\hbar}^+$$
$$I\mapsto\omg(\Gamma,I)$$
$$e^{\omg(\Gamma,I)/\hbar}=e^{\hbar\mcalP}e^{I/\hbar}$$
$\omg_(\Gamma,-)$ is called \textbf{renormalization group flow operator}.

For $\bbR^n$,consider
$$\int_{\bbR^n}\prod_{i=1}^n\frac{\td x^i}{\sqrt{2\pi\hbar}}
e^{(-\frac{1}{2}Q(x)+I(x+a))/\hbar}$$

where $Q(x)=\sum Q_{ij}x^ix^j$ quadratic ,and $(Q_{ij})>0$ positive.
$$I(x)\in\bbR\fps{x^i,\hbar}^+$$
at least cubic in $x^i$ modulo $\hbar$.

the volume form
$$e^{-\frac{1}{2\hbar}Q(x)}\prod_{i=1}^n\frac{\td x^i}{\sqrt{2\pi\hbar}}$$
$$\yc=\hbar\sum_i\pp{x^i}\pp{\theta_i}-\sum_{i,j}Q_{ij}x^i\pp{\theta_j}
=\mcalU^{-1}(-\sum_{ij}Q_{ij}x^i\pp{\theta_j})\mcalU$$
where
$$\mcalU=e^{\frac{1}{2}\hbar Q^{ij}\pp{x^i}\pp{x^j}}$$
where $(Q^{ij})=(Q_{ij})^{-1}$. Then
$$\int_{\bbR^n}\prod_{i=1}^n\frac{\td x^i}{\sqrt{2\pi\hbar}}
e^{(-\frac{1}{2}Q(x)+I(x+a))/\hbar}
=\frac{1}{\sqrt{\det Q}}
  \left(
    e^{\frac{1}{2}\hbar Q^{ij}\pp{a^i}\pp{a^j}}e^{I(a)/\hbar}
  \right)
=
  \frac{1}{\sqrt{\det Q}}
  \exp\left(
    \sum_{\Gamma-\text{connected graph}}
      \frac{\omg_\Gamma(a)}{Aut(\Gamma)}
  \right)
$$
%%%%%%%%%%Feymann rule%%%%%%%%%%%%%%%%5

Similarly,
$$\mcalP=\frac{1}{2}\sum_{i,j}Q^{ij}\pp{x^i}\pp{x^j}$$
$$\omg(\mcalP,-):\bbR\fps{x^i,\hbar}^+\to\bbR\fps{x^i,\hbar}^+$$
$$I\mapsto \omg(\mcalP,I)$$
$$e^{\omg(\mcalP,I)/\hbar}"="e^{\hbar\mcalP}e^{I/\hbar}$$
it is well defined, invertible...

Now, $\bbR^n$ when $n"\to"\infty$...

Quantum field theory case:

\begin{example}Scalar field theory,$\bbR^D$.
$$\mcalE=C^{\infty}(\bbR^D)$$
smooth functions ,
$$\bbR^n\rightsquigarrow\mcalE$$

$$\mcalS[phi]=\frac{1}{2}
\int_{\bbR^D}
  |\td\phi|^2
 +\frac{\lmd}{4!}
  \int_{\bbR^D}\phi^4
$$
for $\phi\in\mcalE$. we want
$$\int_{\mcalE}e^{-\mcalS[\phi]/\hbar}[D\phi]$$
\end{example}

finite dimension ,
$$\bbR^n=map(n \text{points},\bbR)$$
so,
$$\mcalE=C^{\infty}(\bbR^D)"="\lim_{N\to\infty}C^{\infty}(N \text{points})$$
(取密密麻麻的点?格点场论)

$$i \text{index}\rightsquigarrow x\in\bbR^D$$
$$\sum_i\rightsquigarrow\int_{\bbR^D}\td x$$

free part:
$$\frac{1}{2}\int_X|\td\phi|^2
=\frac{1}{2}\int_X\phi D\phi
$$
where $D=-\sum_i\pp{x^i}\pp{x^i}$ be laplacian:
$$D:\mcalE\to\mcalE$$

propagator
$$\mcalP=D^{-1}=D^{-1}_{x,y}$$
(is called Green's function integral kernel)

$$\Phi:\mcalE\to\mcalE$$
operator, has kernel $\Phi(x,y)$ if
$$\Phi(f)(x)=\int\td y\Phi(x,y)f(y)$$

eg. $\Phi=\id\iff\Phi(x,y)=\delta(x-y)$ delta function.
$$D^{-1}\to D^{-1}(x,y)=\frac{1}{|x-y|^{D-2}}$$
singularity comes from infinite dimensional nature.
(Ultra-Violet singularity紫外发散)

%%%%%%%%%2019.4.2第六周周二%%%%%%%%%%%%%%%%%%%%
%这周要布置作业,要交的作业,不然跟不上了%

%我们要花两节课来讲一个重要概念:重整化
%量子场论的例子,与Hochschild理论的联系

\textbf{Renormalization(I)}

Last time:
$$\int_{\bbR^n}e^{(-\frac{1}{2}\sum x^iQ^{ij}x^j+I(x+a))/\hbar}
\prod_{i=1}^n\frac{\td x^i}{\sqrt{2\pi\hbar}}$$
$$=\frac{1}{\sqrt{\det Q}}e^{\frac{1}{2}\hbar Q^{ij}\pp{a^i}\pp{a^j}}e^{I(a)/\hbar}$$
$$=\frac{1}{\sqrt{\det Q}}\exp
\left(
  \sum_{\Gamma}
   \frac{\omg_\Gamma(P,a)}{Aut\Gamma}
\right)$$


Field theory example:

$\phi^4$-theory on $\bbR^4$
$$\mcalE=C^{\infty}_c(\bbR^4)$$
$$\mcalS[\phi]=\int_{\bbR^4}\frac{1}{2}\phi D\phi+\frac{\lmd}{4!}\int_{\bbR^4}\phi^4$$
$$:=Q(\phi)+I(\phi)$$

$$\int_{\mcalE}[D\phi]e^{-\mcalS[\phi]/\hbar}
"="\frac{1}{\sqrt{\det Q}}\exp
\left(
  \sum_{\Gamma}
    \frac{\omg_{\Gamma}(\mcalP,I)}
         {|Aut(\Gamma)|}
\right)$$

Feymann rule...$G(x,y)$ satisfy
$$D_xG(x,y)=\delta(x,y)$$
(analogue $Q_{ij}Q^{jk}=\delta^{k}_i$)
(Green's function)

where $$D=-\sum_i\pp{x^i}\pp{x^i}$$
is the Laplacian...

in $\bbR^4$,
$$G(x,y)=\frac{1}{|x-y|^2}$$

(in general on $\bbR^d$, $G(x,y)\sim\frac{1}{|x-y|^{d-2}}$ when $d\geq 3$)

Feyman graph formula:

tree level ($\hbar^0$)
%%%%cy%%%%%%

$$\mcalE=C_c^\infty(\bbR^4)$$
(这个无穷维空间是有拓扑的)(这个空间上的广义函数?)

这个例子讲完了,我们稍微再复杂一点,下面呢。。。

in general, for a tree diagram (loop = 0)
%%%%%%%%another example%%%%%%
\textbf{HW:}
the Feymann integral is well-defined.

One loop($\hbar^1$)
%%%%%%%%loop%%%%%%%
(ill defined..)
(Ultra-Violent divergent...)
发散的原因是“两个点离得太近,能量太高”

处理这种发散,引入“重整化”

idea of renormalization:

observe: Green function $G$ is the "inverse of Laplacian".
$$D^{-1}=\int_0^\infty e^{-tD}\td t$$
$e^{-tD}$ is "Heat operator"..

$e^{-tD}$ is represented by an integral kernel
$h_t(x,y)$ such that
$$(e^{-tD}\phi)(x)=\int_{\bbR^d}h_t(x,y)\phi(y)\td y$$

on $\bbR^d$,
$$h_t(x,y)=\frac{1}{(4\pi t)^{d/2}}e^{-\frac{|x-y|^2}{4t}}$$

check:
$$(\p_t+D_x)h_t(x,y)=0$$

\begin{prop}

(1) $h_t$ is smooth if $t>0$, and $h_t\to\delta(x,y)$ when $t\to 0$.

(2) semi-group prop:
$$\int_{\td y}h_{t_1}(x_1,y)h_{t_2}(y,x_2)=h_{t_1+t_2}(x,y)$$
i.e.
$$e^{-t_1 D}e^{-t_2D}=e^{-(t_1+t_2)D}$$
\end{prop}

and we have
$$G(x,y)=\int_0^\infty \td t h_t(x,y)$$

introduce cut-off paramaters
$$0<\veps<L<+\infty$$

Define
$$P^L_{\veps}(x,y)=\int_\veps^L\td h_t(x,y)$$
is smooth(called regularized propagator).
$\veps$ is called uv(紫外) cut-off,
$L$ is  called IR(红外) cut-off.

idea: Replace the propagator $G$ by $P^L_\veps$
and analyze the behavior of the graph as $\veps\to 0$ and $L\to\infty$.

Eg:

%%%%%%%%cnm%%%%%%%%%%%%%

\begin{eqnarray*}
     (1)
&=&
     \lmd^2\int_{\bbR^4}
       \td x\phi^4(x)
         \int_\veps^L
           \frac{\td t_1}{(4\pi t_1)^2}
           \frac{\td t_2}{(4\pi t_2)^2}
             \int_{\bbR^4}
               e^{-\frac{|y|^2}{4t_1}-\frac{|y|^2}{4t_2}}\\
&=&
     \lmd^2\int_{\bbR^4}
       \td x\phi^4(x)
         \int_\veps^L
           \frac{\td t_1}{(4\pi t_1)^2}
           \frac{\td t_2}{(4\pi t_2)^2}
           \frac{(2\pi)^2}{(\frac{1}{2t_1}+\frac{1}{2t_2})^2}\\
&=&
     \frac{\lmd^2}{(4\pi)^2}
     \int_{\bbR^4}
       \td x\phi^4(x)
       \int_\veps^L
         \frac{\td t_1\td t_2}{(t_1+t_2)^2}\\
&=&
     -\frac{\lmd^2\log\veps}{(4\pi)^2}
     \int_{\bbR^4}\td x \phi(x)^4
     +\text{terms smooth when $\veps\to 0$}
\end{eqnarray*}
%%%%%%%%gou%%%%%%%%%%%%%%

idea: $\lmd\to\lmd(\veps)$ depend on $\veps$.
Consider add the following to $\mcalS$:
$$I^{CT}_1(\veps)=\frac{\hbar\lmd^2\log\veps}{(4\pi)^2}\int\td x\phi^4$$
(CT: counter term...用来抵消发散)

$$\mcalS\mapsto\mcalS+I^{CT}(\veps)$$
$$=\frac{1}{2}\int\phi D\phi+\frac{\lmd}{4!}\td x\phi^4+
\frac{\hbar\lmd^2\log\veps}{(4\pi)^2}\int\td x\phi^4$$

Feymann rule
%%%%%%%%量子修正%%%%%%%%%%

\begin{thm}[Physics](物理中少数几个正儿八经的定理)

there exists
$$\lmd(\veps)=\lmd+\hbar\frac{\lmd^2\log\veps}{(4\pi)^2}+\hbar^2+\cdots$$
$$=\lmd+\sum_{g\geq 1}\hbar^gG_g(\lmd,\veps)$$
(dependents on $\lmd,\veps$, singular as $\veps\to 0$.)

such that Let
$$I^{\veps}=\frac{\lmd(\veps)}{4!}\int_{\bbR^4}\td x\phi(x)^4$$
then
$$
  \lim_{\veps\to 0}\sum_{\Gamma}
  \frac{\omg_P(P_\veps^L,I^\veps)}
       {|Aut(\Gamma)|}
$$
exists.
\end{thm}

这个定理挺难证。。。不证了。。。

\begin{example}Quantum mechanics (QFT in dimension 1)

field:
$$\gamma:\bbR\to\text{a space}$$
$$t\mapsto \gamma(t)$$

$$\mcalS[\gma]=\frac{1}{2}\int\bbR\td t|\gma'(t)|^2$$
is called energy...

\end{example}

"Physics fact": For $x,y\in\bbR^d$,consider
$$\int_{\gma:[0,t]\to\bbR^d\atop\gma(0)=x,\quad\gma(t)=y}
[D\gma]e^{-\mcalS[\gma]/\hbar}
=h_t(x,y)$$

"the second story":first order formula,
$$\mcalS\to\mcalS[\cdots]$$

还是下一节课再讲吧。。。





