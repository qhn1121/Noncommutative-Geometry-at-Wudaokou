\chapter{量子场论的背景}
%\textbf{Some basis of QFT}
现在我们开始逐渐去理解Kontsevich的形变量子化的构造;
为此需要一些\textbf{量子场论}
(quantum field theory,简称QFT)背景知识。
\index{quantum field theory\kong 量子场论}

%a physics system always consists of
%$$\mcalS:\mcalE\to\bbR$$
%where $\mcalE$ is the space of field( of infinity dimension)

\section{Grassmann变量与BV算子}

大致地说(并非严格的数学表述),一个\textbf{物理系统}
包括以下要素:\textbf{场空间}(space of fields)
$\mcalE$与\textbf{作用量}
(action functional)$\mcalS$,
\index{space of fields\kong 场空间}
\index{action functional\kong 作用量}
其中场空间$\mcalE$通常为无穷维空间,作用量
$$\mcalS:\mcalE\to \bbC$$
为场空间$\mcalE$上的函数。
%$\mcalS$ : actional functional%作用量

在经典物理中,态的演化常用变分的临界来描述态的演化:
%Classical physics:
$$\Crit(\mcalS)=\{\delta\mcalS=0\}$$
上式中的$\Crit(\mcalS)$称为$\mcalS$的critical locus,
$\delta$为某个变分导数。
\index{critical locus}

%Quantum physics:
而在量子物理中,态的演化与积分
$$\int_{\mcalE}\mcalO e^{i\mcalS/\hbar}$$
有关,其中$\mcalO$为$\mcalE$上的函数,
称之为\textbf{观测量}(observable);
\index{observable\kong 观测量}
上述积分称之为“\textbf{路径积分}”(path integral)。
\index{path integral\kong 路径积分}

不过要注意,$\mcalE$是无穷维空间,
在$\mcalE$上面积分是说不清道不明的事情;
我们至今还未完全搞明白此积分的严格定义。
我们在本讲义只谈论“数学上的事情”,
数学上暂时没说清楚的东西避而不谈。

%"path integral", $\theta$ is a function on $\mcalE$("observable").

\begin{example}作为场空间$\mcalE$的例子,
以下是近代物理中的常见对象:
$$
  \begin{tabular}{|c|c|}
  \hline
  标量场论    &  $\mcalS=$流形$X$上的全体光滑函数\\
  \hline
  规范理论    &  $\mcalS=$向量丛$E\to X$上的全体联络\\
  \hline
  $\sgm$-模型 &  $\mcalS=$流形$\sgm$与$X$之间的全体光滑映射\\
  \hline
  引力理论    &  $\mcalS=$流形$X$上的全体黎曼度量\\
  \hline
  \end{tabular}
$$
\end{example}

我们再举一些作用量$\mcalS$的例子:

\begin{example}(作用量)

(1)在标量场论$\mcalE=C^{\infty}(X)$中,
对于$X$上的光滑函数$\fai\in\mcalE$,定义
$$\mcalS[\fai]=\int_X|\nabla\fai|^2$$
称之为能量泛函。

(2)在规范理论当中,对于$A\in\mcalE$为向量丛$E\to X$上的联络,
记其曲率张量为
$$F_A:=\td A+\frac{1}{2}[A,A]$$
定义如下的\textbf{杨-米尔斯泛函}(Yang-Mills functional)
$$\YM[A]:=\int_XF_A\wedge*F_A$$

%$$\mcalE=\{\text{connections on $E\to X$}\}$$
%curvature
%is called Yang-Mills functional.
\index{Yang-Mills functional\kong 杨-米尔斯泛函}
\end{example}

%\begin{example}
%$$\mcalE= C^{\infty}(X)$$
%$$\mcalS[\fai]=\int_X|\nabla\fai|^2\quad\fai\in C^{\infty}(X)$$
%(scalar field theory)
%$$\delta\mcalS=0\Longrightarrow\Delta\fai=0$$
%\end{example}
%\begin{example}($\sigma$-module)
%$$\mcalE=map(\Sigma,X)$$
%...
%\end{example}
%\begin{example}(Gravity)
%$$\mcalE=\{\text{metrics on $X$}\}$$
%\end{example}
%以上是四种很经典的例子。
%Problem: How to construct

一个重要的问题是,如何去构造路径积分
$$\int_{\mcalE}\theta e^{i\mcalS/\hbar}$$
我们介绍\textbf{BV方法}(Batalin-Vilkovisky method),
其主要思想是用同调理论来解释测度论。

%We introduce a different method:BV method
%(Batalin-Vilkovisky)
%Philosophy: measure theory$\int_X\mapsto$ Homology theory.....
%Calculus

我们来考察有限维的情形。设$X$为$n$维紧致定向流形,
$\Omg\in\Omg_X^n$为$X$上的一个体积形式,
则$X$上的紧支光滑函数$f$关于该体积形式的积分可以视为如下:
\begin{eqnarray*}
\int_X: C^{\infty}_c(X)&\to&\bbR\\
f&\mapsto&\int_Xf\Omg
\end{eqnarray*}

%$$\int_X:\Omg_X\updot\to\bbR$$
%where $X$ is compact oriented manifold of dimension $n$,

我们考虑
$$\Omg_c(X):=\bigoplus_{p\geq 0}\Omg_c^p(X)$$
为紧支的微分形式,以及$\td:\Omg_c^p(X)\to\Omg_c^{p+1}(X)$为de Rham外微分。
众所周知,
$$H^n(\Omg_c\updot(X),\td)\cong\bbR$$
此式可以给出积分$\int_X$的同调解释:
\begin{eqnarray*}
\int_X:C_c^{\infty}(X)&\to&\bbR\\
f&\mapsto&[f\Omg]\in H^n(\Omg_c\updot(X),\td)\cong\bbR
\end{eqnarray*}

粗俗地说,我们把求$f$关于体积形式$\Omg$的积分
视为取$f\Omg$的同调类;在此意义下,de Rham复形
$(\Omg_C\updot,\td)$扮演了“测度”的角色。

%Observe:
%$$H^n_{DR}(X)=H^n(\Omg\updot,\td)\cong\bbR$$
%$$\int_X:\Omg^n\to H^n_{DR}\cong\bbR$$
%$$\alpha\mapsto[\alpha]$$
%$$\rightsquigarrow\int_X=H_{DR}^n$$
%$$(\Omg\updot(X),\td)\rightsquigarrow\text{measure}$$
%Question:how to $\int_X=H^n\quad n\to\infty$?

%%%%%%%%%%%%%%%%%%%%%%%%%%%%%%%%%%%%%%%%%%%%%%%%%%%%%%%%%
%%%%%%%%%%%%%%%%2019.3.26星期二 第五周%%%%%%%%%%%%%%%%%%%
%%%%%%%%%%%%%%%%%%%%%%%%%%%%%%%%%%%%%%%%%%%%%%%%%%%%%%%%%

%讲一些物理想法
%\textbf{Feynman Diagram}
%recall:
%$$D^n_{DR}=\int_X:\Omg\updot(X)\to\bbR$$
%what if $n\to \infty$?

在物理上我们常要面对无穷维空间,于是在此意义下,
我们需要关心$n\to\infty$时,$H^n(X)$是何物。
这是难以说清楚的,我们不妨换一个角度来看。

%Different philosophy:
%Let $\PV\updot(X):=\Gamma(X,\wedgeform{*}TX)$,
%$\Omg$ be a volume form($n$-form) on $X$,then
%$$f\mapsto\int_Xf\Omg$$
%$$\PV^k(X)\xra{\dashv\Omg}\Omg^{n-k}(X)$$
%is a 1-1 coorespondence.
%%%%缩并%%%%%%%

\begin{definition}
设$X$为$n$维紧致定向流形,$\Omg$为$X$上的一个体积形式,
则有$\Omg$诱导了多重切向量场$\PV\updot(X)$
与微分形式$\Omg_X\updot$之间的$C^{\infty}$-线性同构
\begin{eqnarray*}
\Gma_\Omg:\PV^k(X)&\to&\Omg^{n-k}_X\\
V&\mapsto& V\suobing\Omg
\end{eqnarray*}
其中$V\lrcorner\,\Omg$为$V$关于$\Omg$的缩并.
\end{definition}


在局部坐标下,若$\Omg=\td x^1\wedge\td x^2\cdots\wedge\td x^n$,
$$V=\p_{i_1}\wedge\p_{i_2}\wedge\cdots\wedge\p_{i_k}$$
为多重切向量场,其中指标$i_1<i_2<\cdots<i_k$,则容易知道
$$
  \Gma_\Omg(V)=V\suobing\Omg
= (-1)^{(i_1-1)+(i_2-1)+\cdots+(i_k-1)}
  \cdots\wedge\widehat{\td x^{i_1}}
  \wedge\cdots\wedge\widehat{\td x^{i_k}}\wedge\cdots
$$

\begin{example}
$$(\p_2\wedge\p_3)\suobing(\td x^1\wedge\td x^2\wedge\td x^3\wedge\td x^4)
=-\td x^1\wedge\td x^4$$
\end{example}
以此为例,缩并的运算规则可以理解为:
$\p_i$向右移动与$\td x^i$相遇而湮灭,
其中在$\p_i$移动的过程中穿过几个对象
($\p_j$或者$\td x^j$)就改变几次正负号
(这符合Koszul符号法则的“精神”)。

例如,如果$\Omg=e^{f(x)}\td x^1
\wedge\td x^2\wedge\td x^3\wedge\td x^4$,则
$$\Gma_\Omg(\p_2\wedge\p_3)=e^{f(x)}\td x^1\wedge\td x^4$$
再比如,对于体积形式$\Omg$本身,有
$$\Gma_\Omg^{-1}(\Omg)=1$$
也就是说$1\in\PV^0(X)$对应于$\Omg\in\Omg^n_X$.

当$V\in\PV^1(X)$为切向量场时,
$V\lrcorner\Omg=i_V(\Omg)$就是我们熟悉的内乘运算。

\begin{rem}(多重切向量场的内乘)
类似于关于切向量场$X$的内乘算子
$i_X:\Omg\updot_X\to \Omg^{\bullet-1}_X$,
我们也可以考虑多重切向量场$V\in\PV^p(X)$的内乘
$$i_V:\Omg\updot_X\to\Omg^{\bullet-p}_X$$
使得对任意$\omg\in\Omg^r_X\,(r\geq p)$,以及任意$W\in\PV^{r-p}(X)$,成立
$$\langle i_V(\omg),W\rangle=\langle\omg,V\wedge W\rangle$$
\end{rem}
特别注意,对于多重切向量场$V\in\PV\updot(X)$以及体积形式$\Omg$,
一般来说
$$V\suobing\Omg\neq i_V(\Omg)$$
它们两者之间会相差一些奇怪的正负号。
我们这里的$\PV^k(X)$与$\Omg^{n-k}_X$
的对应是通过缩并实现的,而不是内乘。

\begin{definition}(BV算子)

对于$n$维光滑定向流形$X$,
设$\Omg\in\Omg^n_X$为$X$上的一个体积形式,
定义算子$\yc_\Omg:\PV^{k}(X)\to\PV^{k-1}(X)$,使得下图交换:
$$
  \xymatrix{
     \PV^k(X)        \ar[r]^{\yc_\Omg}  \ar@{=}[d]_{\Gma_\Omg}
    &\PV^{k-1}(X)                       \ar@{=}[d]_{\Gma_\Omg}
  \\
     \Omg_X^{n-k}    \ar[r]^{\td}
    &\Omg_X^{n-k+1}
  }
$$
称$\yc_\Omg$为\textbf{BV算子}(Batalin-Vilkovisky operator)。
\index{BV算子}
\label{BV算子-def}
\end{definition}
%$$\PV^*(X)\leftrightarrow \Omg^{n-\bullet}(X)$$
%$$\yc\leftrightarrow \td$$
无非是将de Rham上链复形$(\Omg\updot_X,\td)$
通过体积形式同构为\textbf{上}链复形
$(\PV\updot(X),\yc_\Omg)$,其实没干什么事情。
特别注意我们规定$\PV^k(X)$的次数为$-k$,
使得$\yc_{\Omg}$是次数为$1$的微分算子(而不被看作边缘算子)。

注意到此时有上同调群的同构
$$H^n(\Omg\updot_X,\td)\cong H^0(\PV\updot(X),\yc_\Omg)$$
回顾我们对积分$\int_Xf\Omg$的同调解释,从而有
$$\int_X:f\mapsto[f]\in H^0(\PV\updot(X),\yc_\Omg)$$
也就是说我们可以把求函数$f$关于体积形式$\Omg$的积分转化成取$f$在
$(\PV\updot(X),\yc_\Omg)$的第零个同调类。这样的好处是,
容易向维数$n\to\infty$的情形推广,毕竟无论维数$n$如何升高,
我们取的总是第零个同调。

不过这样的代价是,问题转化为“如何构造无穷维空间上的BV算子”。

%this "$\yc$" is called "divergence operator w.r.t the volume form".
%when $x\in\PV^1(X)$,check: $\yc(v)=\div_{\Omg}v$.
%$$\td^2=0\Rightarrow\yc^2=0$$
%$$\PV^n\xra{\yc}\PV^{n-1}\xra{\yc}\PV^{n-2}\to\cdots$$
%$\yc$ is also called "BV operator".
%$$\int_{BV}:=H_0(\yc)$$
%good news: $0$ doesn't depend on $\dim(X)$!
%(so, we can $n\to\infty$???)
%\textbf{Difficulty}: when $n\to\infty$, we need to construct $\yc$.

\begin{rem}(广义散度)

事实上,如果$v\in\PV^1(X)$为$X$上的切向量场,则
$$\yc_\Omg(v)=\Div_\Omg(v)$$
正是我们熟悉的关于体积形式$\Omg$的散度。
\end{rem}
于是我们也俗称BV算子为多重切向量场的“广义散度”。

为了书写方便,我们引入一套高效的语言:Grassmann变量。

\begin{notation}(Grassmann变量)

对于$n$维流形$X$,以及$X$的局部坐标卡$U\subseteq X$,
我们考虑分次交换$\bbR$-代数
$$C^{\infty}(U)\ten\Free\{\theta_1,\theta_2,...,\theta_n\}\big/\sim$$
其中生成关系$\sim$为由
$\{\theta_i\theta_j+\theta_j\theta_i|1\leq i,j\leq n\}$生成的理想。
其中分次结构由
$$\deg\theta_i=-1\quad\forall 1\leq i\leq n$$
给出。
\end{notation}
容易发现,无非是将$\PV\updot(U)$当中的$\p_i$重新写为$\theta_i$,从而局部上
$$\PV\updot(U)=C^{\infty}(U)[\theta_1,...,\theta_n]$$
换句话说,$X$上的多重切向量场(局部上)
可以写为关于局部坐标$x^1,...,x^n$以及Grassmann变量的函数
$$\mu=\mu(x^1,...,x^n;\theta_1,...,\theta_n)\in\PV\updot(X)$$

{\color{blue}这里的Grassmann变量$\theta_i$是不是
Doubrovin-Zhang可积系统里面的“超变量”?
}

%\begin{example}
%Let $X$ is a manifold of finite dimension, $\dim X=n$,
%local coordinate $\{x^1,...,x^n\}$ in $\mcalU\subseteq X$,
%volume form
%$$\Omg=e^{f(x)}\td x^1\wedge\cdots\wedge\td x^n$$
%$$\Omg\updot(\mcalU):=C^{\infty}(\mcalU)[\td x^1,...,\td x^n]$$
%where $\td x^i\wedge\td x^j=-\td x^j\wedge\td x^i$.
%$\PV\updot(\mcalU):=C^{\infty}(\mcalU)[\p_1,...,\p_n]$
% where $\p^i\wedge\p^j=-\p^j\wedge\p^i$.
%\end{example}
%introduce grassman variables $\theta_1,...,\theta_n$,
%($\theta _i\cong \p_i$)
%define
%$$\PV\updot(\mcalU):=C^{\infty}(\mcalU)[\theta_1,...,\theta_n]$$
%%%%%%%非交换的微积分%%%%%%%%

\begin{definition}对于流形$X$,局部坐标下我们定义$-1$阶超导子
$$\pp{\theta_i}:\PV\updot(X)\to\PV\updot(X)$$
使得成立

  \begin{eqnarray*}
    \pp{\theta_i}f(x^1,...,x^n)=0,\qquad
    \pp{\theta_i}\theta_j=\delta^i_j
  \end{eqnarray*}

\end{definition}
$\pp{\theta_i}$服从$-1$阶超导子的超莱布尼茨法则,
即对任意$f,g\in\PV\updot(X)$为齐次元,成立
$$\pp{\theta_i}(fg)=\pfrac{f}{\theta_i}g+(-1)^{\deg f}f\pfrac{g}{\theta_i}$$

容易验证,超导子$\pp{\theta_i}$满足关系
$$\pp{\theta_i}\pp{\theta_j}=-\pp{\theta_j}\pp{\theta_i}$$
对任意$1\leq i,j\leq n$成立。特别地,$\left(\pp{\theta_i}\right)^2=0$.

\begin{prop}(BV算子的Grassmann变量表达式)

对于定向流形$X$,设体积形式
$$\Omg=e^{f(x)}\td x^1\wedge\cdots\wedge \td x^n$$
则关于$\Omg$的BV算子$\yc\Omg$在Grassmann变量的意义下具有表达式
$$\yc_{\Omg}=\pp{x^i}\pp{\theta_i}+\pfrac{f}{x^i}\pp{\theta_i}$$

%(Odd Laplacian)
%(这是BV算子的等价定义。。。)
\label{BV算子的超变量表达式-prop}
\end{prop}

\begin{proof}
直接验证之。对于任意
$$V=\mu(x^1,...,x^n)\theta_{i_1}\cdots\theta_{i_k}\in\PV^{k}(X)$$
则有
\begin{eqnarray*}
     \yc_\Omg V
&=&
     \Gma_\Omg^{-1}\circ\td\circ\Gma_\Omg(V)\\
&=&
     \Gma_\Omg^{-1}\circ\td
     \left[
       (-1)^{(i_1-1)+\cdots+(i_k-k)}\mu e^{f}
       \widehat{\td x^{i_1}}\wedge\cdots\wedge\widehat{\td x^{i_k}}
     \right]\\
&=&
     (-1)^{(i_1-1)+\cdots+(i_k-k)}
     \Gma_\Omg^{-1}
     \left[
       (\pfrac{\mu}{x^i}+\mu\pfrac{f}{x^i})e^f
       \sum_{l=1}^k
         (-1)^{i_l-l}
         \widehat{\td x^{i_1}}\wedge\cdots\wedge\td x^{i_l}
         \wedge\cdots\wedge\widehat{\td x^{i_k}}
     \right]\\
&=&
     (\pfrac{\mu}{x^i}+\mu\pfrac{f}{x^i})
     \sum_{l=1}^k
       (-1)^{l-1}
       \theta_{i_1}\cdots\widehat{\theta_{i_l}}\cdots \theta_{i_k}\\
&=&
    \left[
      \pp{x^i}\pp{\theta_i}+\pfrac{f}{x^i}\pp{\theta_i}
    \right](V)
\end{eqnarray*}
从而证毕。
\end{proof}

注意到BV算子的表达式
$$\yc_{\Omg}=\pp{x^i}\pp{\theta_i}+\pfrac{f}{x^i}\pp{\theta_i}$$
长得像二阶微分算子,甚至很像拉普拉斯算子——$\yc_\Omg$因此也被称为
\textbf{奇拉普拉斯算子}(odd Laplacian)。
\index{odd Laplacian\kong 奇拉普拉斯算子}

\begin{prop}设$X$为定向流形,
$\Omg=e^{f(x)}\td x^1\wedge\cdots\wedge\td x^n$为$X$的一个体积形式,
$\yc_\Omg$为关于$\Omg$的BV算子。定义
%Given $\yc_{\Omg}$(BV operator), we define
$$\{,\}:\PV\updot(X)\times\PV\updot(X) \to \PV\updot(X)$$
$$\{\alpha,\beta\}:=
\yc_{\Omg}(\alpha\wedge\beta)-(\yc_{\Omg}\alpha)\wedge\beta-
(-1)^{|\alpha|}\alpha\wedge\yc_{\Omg}\beta$$
即,“$\yc$成为超导子的代价”。
那么$\{,\}$不依赖于体积形式$\Omg$的选取。
%the failure of $\yc_{\Omg}$ being a derivation.
\label{另一种Schouten-Nijenhuis括号-def}
\end{prop}

\begin{proof}
直接验证即可。对任意$\alpha\in\PV^p(X)$以及$\beta\in\PV^q(X)$,成立
\begin{eqnarray*}
     \yc_\Omg(\alpha\wedge\beta)
&=&
     \pp{x^i}\pp{\theta_i}
     (\alpha\wedge\beta)
    +\pfrac{f}{x^j}\pp{\theta_j}
     (\afa\wedge\beta)\\
&=&
     \pp{x^i}
     \left(
       \pfrac{\afa}{\theta_i}
       \wedge\beta
      +(-1)^p\afa\wedge\pfrac{\beta}{\theta_i}
     \right)
    +\pfrac{f}{x^i}
     \left(
       \pfrac{\afa}{\theta_i}\wedge\beta
      +(-1)^p\afa\wedge\pfrac{\beta}{\theta_i}
     \right)\\
&=&
     \pmfrac{\afa}{x^i}{\theta_i}\wedge\beta
    +\pfrac{\afa}{\theta_i}\wedge\pfrac{\beta}{x^i}
    +(-1)^p\pfrac{\afa}{x^i}\wedge\pfrac{\beta}{\theta_i}\\
& &
    +(-1)^p\afa\wedge\pmfrac{\beta}{x^i}{\theta_i}
    +\pfrac{f}{x^i}\pfrac{\afa}{\theta_i}\wedge\beta
    +(-1)^p\pfrac{f}{x^i}\afa\wedge\pfrac{\beta}{\theta_i}\\
&=&
     (\yc_\Omg\afa)\wedge\beta
    +(-1)^p\afa\wedge(\yc_\Omg\beta)\\
& &
    +\pfrac{\afa}{\theta_i}\wedge\pfrac{\beta}{x^i}
    +(-1)^p\pfrac{\afa}{x^i}\pfrac{\beta}{\theta_i}
\end{eqnarray*}
从而得到
$$
  \{\alpha,\beta\}
=
  \pfrac{\afa}{\theta_i}\wedge\pfrac{\beta}{x^i}
 +(-1)^p\pfrac{\afa}{x^i}\wedge\pfrac{\beta}{\theta_i}
$$
从而与$\Omg$的选取无关。
\end{proof}

我们之前也见过类似的运算:Schouten-Nijenhuis括号
(见定义\ref{Schouten-Nijenhuis定义-def});
而这里的$\{,\}$是“另一个版本的Schouten-Nijenhuis括号”:

\begin{lemma}定义$\ref{另一种Schouten-Nijenhuis括号-def}$中的括号
$$\{,\}:\PV^p(X)\times\PV^q(X)\to\PV^{p+q-1}(X)$$
满足性质:对任意$\afa\in\PV^p(X),\beta\in\PV^q(X),\gamma\in\PV^r(X)$,成立:

(1)超反交换性
$$\{\afa,\beta\}=(-1)^{pq}\{\beta,\afa\}$$

(2)超莱布尼茨法则
$$\{\afa,\beta\wedge\gamma\}
=\{\afa,\beta\}\wedge\gamma
+(-1)^{(p-1)q}\beta\wedge\{\afa,\gamma\}$$

(3)若$p=q=1$,则$\{,\}$退化为切向量场李括号:
$$\{\alpha,\beta\}=[\afa,\beta]$$
\end{lemma}

注意超反交换性(1)与性质\ref{Schouten-Nijenhuis公理-prop}
的(2)在正负号上有所出入。

\begin{proof}使用表达式
$$
  \{\alpha,\beta\}
=
  \pfrac{\afa}{\theta_i}\wedge\pfrac{\beta}{x^i}
 +(-1)^p\pfrac{\afa}{x^i}\wedge\pfrac{\beta}{\theta_i}
\eqno{(*)}
$$
直接验证即可,并不困难。对于$\alpha\in\PV^p(X),\,\beta\in\PV^q(X)$
以及$\gamma\in\PV^r(X)$,有
\begin{eqnarray*}
     \{\afa,\beta\}
&=&
     \pfrac{\afa}{\theta_i}\wedge\pfrac{\beta}{x^i}
    +(-1)^p\pfrac{\afa}{x^i}\wedge\pfrac{\beta}{\theta_i}\\
&=&
     (-1)^{(p-1)q}\pfrac{\beta}{x^i}\wedge\pfrac{\afa}{\theta_i}
    +(-1)^{p+p(q-1)}
     \pfrac{\beta}{\theta_i}\wedge\pfrac{\afa}{x^i}\\
&=&
     (-1)^{pq}
     \left(
       \pfrac{\beta}{\theta_i}\wedge\pfrac{\afa}{x^i}
      +(-1)^q\pfrac{\beta}{x^i}\wedge\pfrac{\afa}{\theta_i}
     \right)\\
&=&
     (-1)^{pq}\{\beta,\alpha\}
\end{eqnarray*}
于是超反交换性成立;再看超莱布尼茨法则,
\begin{eqnarray*}
     \{\afa,\beta\wedge\gamma\}
&=&
     \pfrac{\afa}{\theta_i}\wedge
     \pp{x^i}(\beta\wedge\gamma)
    +(-1)^p\pfrac{\afa}{x^i}\wedge
     \pp{\theta_i}(\beta\wedge\gamma)\\
&=&
     \pfrac{\afa}{\theta_i}\wedge
     \left(
       \pfrac{\beta}{x^i}\wedge\gamma
      +\beta\wedge\pfrac{\gamma}{x^i}
     \right)
    +(-1)^p\pfrac{\afa}{x^i}\wedge
     \left(
       \pfrac{\beta}{\theta_i}\wedge\gamma
      +(-1)^q\beta\wedge\pfrac{\gamma}{\theta_i}
     \right)\\
&=&
     \{\alpha,\beta\}\wedge\gamma
    +(-1)^{q(p-1)}
     \beta\wedge\pfrac{\afa}{\theta_i}\wedge\pfrac{\gamma}{x^i}
    +(-1)^{pq+p+q}
     \beta\wedge\pfrac{\afa}{x^i}\wedge\pfrac{\gamma}{\theta_i}\\
&=&
     \{\alpha,\beta\}\wedge\gamma
    +(-1)^{(p-1)q}\beta\wedge\{\alpha,\gamma\}
\end{eqnarray*}
而(3)是更加容易验证的,从略。
\end{proof}
可以体会到Grassmann变量$\theta_i$以及超导子$\pp{\theta_i}$在
张量计算上的优越性:将本该必然面对的数学归纳法、
组合恒等式转化为直接的暴力计算。

{\color{blue}
\begin{rem}
回顾之前定义的Schouten-Nijenhuis括号
(见性质\ref{Schouten-Nijenhuis公理-prop})。
而这里的“$\{,\}$”与之相差一些奇怪的正负号。
我们只需适当调整之:

$$\{\alpha,\beta\}:=
-(-1)^{|\afa|}
\left(
\yc_{\Omg}(\alpha\wedge\beta)-(\yc_{\Omg}\alpha)\wedge\beta-
(-1)^{|\alpha|}\alpha\wedge\yc_{\Omg}\beta
\right)
$$
则容易验证如此的括号满足性质\ref{Schouten-Nijenhuis公理-prop}。
\end{rem}
}
%Check: $\{,\}$ is a Schouten-Nijenhuis bracket(up to sign).
%(independent of the choice of $f(x)$)
%In particular,$\{,\}$ is independent of choice of $\Omg$.
%\textbf{Quantization}
%$$\{,\}\xra{?}\yc_{\Omg}$$
%这个过程与量子化的过程是一样的??WTF??

\section{从一维Gauss积分到费曼图}
%\textbf{Feymann Diagram}
首先我们考察一个BV算子的例子:
\begin{Example}
考虑一维流形$X=\bbR$,体积形式
$$\Omg:=\frac{1}{\sqrt{2\pi}}e^{-\frac{1}{2}x^2}\td x$$
则BV算子
$$\yc_{\Omg}=\pp{x}\pp{\theta}-x\pp{\theta}$$

特别地,我们得到
$$
  \int_\bbR x^k\Omg
=
  \left\{
    \begin{array}{ll}
      0          &  x=2k+1\\
      (2k-1)!!   &  x=2k
    \end{array}
  \right.
$$
%符号可能不对,小心check
%$\PV=\bbR[x,\theta]$, and
%$$\int_{BV}:\bbR[x,\theta]\mapsto H^0(\yc_{\Omg})$$
\end{Example}

\begin{proof}
注意
$$\Omg=e^{-\frac{1}{2}(x^2-\log 2\pi)}\td x$$
从而由性质\ref{BV算子的超变量表达式-prop},直接写出
$$\yc_{\Omg}=\pp{x}\pp{\theta}-x\pp{\theta}$$

注意到积分的(上)同调解释
\begin{eqnarray*}
  \int_X:C^{\infty}&\to& H^0(\PV\updot(X),\yc_\Omg)\cong\bbR\\
  g&\mapsto& [g]
\end{eqnarray*}
而注意到对任意$x^k\theta\in \PV^1(X)$,
在$H^0(\PV\updot(X),\yc_\Omg)$当中成立
$$0=[\yc\Omg x^k\theta]=(\pp{x}\pp{\theta}-x\pp{\theta})[x^k\theta]
=k[x^{k-1}]-x^{k+1}$$
因此对任意$k\geq 0$,成立
$$[x^{k+2}]=(k+1)[x^k]$$
递推得
$$
  [x^n]=
  \left\{
    \begin{array}{ll}
      (2k-1)!![1]  & n=2k\\
      (2k)!![x]    & n=2k+1
    \end{array}
  \right.
$$
最后注意到
\begin{eqnarray*}
\int_\bbR
  \frac{1}{\sqrt{2\pi}}
  e^{-\frac{1}{2}x^2}\td x=1\qquad
\int_\bbR
  \frac{x}{\sqrt{2\pi}}
  e^{-\frac{1}{2}x^2}\td x=0
\end{eqnarray*}
从而完。
\end{proof}
%$$\int:\bbR[x]\to\bbR$$
%$$g\mapsto\int_{\bbR}g\Omg$$
%if $g=g(x)$, then $\yc_{\Omg}g=0$. Let $[g]$ be the
%$\yc_{\Omg}$- homology class,
%$$\yc_{\Omg}(x^{m-1}\Omg)
%=(m-1)x^{m-2}
%-x^m$$
%$$\Rightarrow [x^m]=(m-1)[x^{m-2}]$$
%so,
%$$
%[x^m]=
%\left\{
%  \begin{array}{cc}
%    0  &  m \text{ is odd}\\
%    (2k-1)!![1]  & m=2k
%  \end{array}
%\right.
%$$
%so,
%$$\int_{\bbR}x^{2k}\Omg=
%(2k-1)!!\int_{\bbR}\Omg=(2k-1)!!$$

\begin{lemma}条件接上,仍考虑体积形式
$$\Omg:=\frac{1}{\sqrt{2\pi}}e^{-\frac{1}{2}x^2}\td x$$
定义算子$\mcalU:\bbR[x,\theta]\to\bbR[x,\theta]$为
$$\mcalU:=
  e^{\frac{1}{2}\pp{x}\pp{x}}$$
则BV算子$\yc_\Omg$满足
$$\yc_{\Omg}=\mcalU^{-1}(-x\pp{\theta})\mcalU$$
\end{lemma}

\begin{proof}注意到众所周知的公式%use the formula
$$e^ABe^{-A}=e^{\ad_A}B$$%then
特别地,在这里%$$\yc_{\Omg}=\pp{x}\pp{\theta}-x\pp{\theta}$$
$$A=-\frac{1}{2}\pp{x}\pp{x},\qquad B=-x\pp{\theta}$$%then

注意到
\begin{eqnarray*}
[A,B]=
     \frac{1}{2}
     [\pp{x}\pp{x},x\pp{\theta}]
=    \pp{x}\pp{\theta}
\end{eqnarray*}
进而
$$[A,[A,B]]=-\frac{1}{2}[\pp{x}\pp{x},\pp{x}\pp{\theta}]=0$$
于是
\begin{eqnarray*}
     \mcalU^{-1}(-x\pp{\theta})\mcalU
&=&
     x^ABe^{-A}
 =
     e^{\ad A}B\\
&=&
     B+[A,B]
 =
     -x\pp{\theta}+\pp{x}\pp{\theta}
 =
     \yc_\Omg
\end{eqnarray*}
从而得证。
%$$e^ABe^{-A}=\pp{x}\pp{\theta}$$
%where $[A,[A,B]]=0$.
\end{proof}

%so,
%$$\mcalU\circ\yc_{\Omg}=(-x\pp{\theta})\circ\mcalU$$
%so,
%$$\mcalU:(\bbR[x,\theta,\yc_{\Omg}])\to
%(\bbR[x,m\theta],-x\pp{\theta})$$
%is a co-chain map.
此引理表明,有如下的交换图表:
$$
  \xymatrix{
     \bbR[x,\theta] \ar[r]^{\yc_\Omg}  \ar[d]^{\mcalU}
    &\bbR[x]                           \ar[d]^{\mcalU}
  \\
     \bbR[x,\theta] \ar[r]^{-x\pp{\theta}}
    &\bbR[x]
  }
$$
以及$\mcalU$诱导上同调群的同构
$$\mcalU:H^0(\bbR[x,\theta],\yc_\Omg)
\xra{\sim}H^0(\bbR[x,\theta],-x\pp{\theta})$$

%$$H\updot(\bbR[x,\theta],-x\pp{\theta})$$
%$$\bbR[x]\theta\xra{-x\pp{\theta}}\bbR[x]$$
%$$g\mapsto -xg(0)$$
%$H^{-1}=0$,$H^0=\bbR$.
%$$\mcalU[g(x)]_{\yc_{\Omg}}=[\mcalU(g)(x)]_{-x\pp{\theta}}
%=[\mcalU(g)(0)]_{-x\pp{\theta}}
%=\mcalU(g)(0)[1]$$

\begin{prop}条件承上,则对于任意的多项式函数$g\in\bbR[x]$,成立
$$
  \int_\bbR g\Omg
=\left.
   e^{-\frac{1}{2}\pp{x}\pp{x}}
 \right|_{x=0}g
$$
\end{prop}
%我们只谈论多项式函数,是为了偷懒,
%不太想讨论收敛性(事实上解析性质很重要)。

\begin{proof}
只需要考虑$[\mcalU(g)]\in H^0(\bbR[x,\theta],-x\pp{\theta})$。
注意到对任意$k\geq 0$,
$$-x\pp{\theta}(x^k\theta)=-x^{k+1}$$
也就是说在$H^0(\bbR[x,\theta],-x\pp{\theta})$当中,
$[x^k]=0$对任意$k\geq 1$成立,从而
$$[\mcalU(g)]=\mcalU(g)(0)
=\left.
   e^{-\frac{1}{2}\pp{x}\pp{x}}
 \right|_{x=0}g
$$
从而易得。
\end{proof}
这个性质将求积分转化为求导,大大简化运算。
(与复变函数的留数定理异曲同工?)
%so,
%$$\int_{\bbR}g(x)\Omg=\mcalU(g)(0)=
%e^{\frac{1}{2}\pp{x}\pp{x}}|_{x=0}g(x)$$
%More generally, check:

更一般地,容易证明对任意$g\in\bbR[x]$
$$\int_{\bbR}g(x+a)\Omg=
\left.
  e^{\frac{1}{2}\pp{x}\pp{x}}
\right|_{x=a}g(x)$$

\begin{Example}现在我们考虑积分%Consider the integral
$$
  \int_{\bbR}
    e^{\big(-\frac{1}{2}x^2+\frac{\lmd}{3!}(x+a)^3\big)\big/\hbar}
    \frac{\td x}{\sqrt{2\pi\hbar}}
$$
其中$\lmd,a\in\bbR$,
在这里体积形式$\Omg=e^{-\frac{1}{2}x^2/\hbar}
\frac{\td x}{\sqrt{2\pi\hbar}}$.
\end{Example}

此式中的“$-\frac{1}{2}x^2$”在物理上可以认为是“自由能”,
三次项$\frac{\lmd}{3!}(x+a)^3$则为“相互作用能”。
相互作用能的存在,使得此积分发散。

处理该积分有两种常见方式:其一是将它视为复平面上的积分,
并且重新规定积分路径(这会出现Airy函数);
或者考察它的($\hbar\to 0$的)渐近展开
$$
  \sum_{n\geq 0}\frac{1}{n!}
    \int_\bbR
      \left(
        \frac{\lmd (x+a)^3}{3!\hbar}
      \right)^n
      e^{-\frac{1}{2}x^2/\hbar}
      \frac{\td x}{\sqrt{2\pi\hbar}}
$$

%this integral is not convergent.... how to deal with it?
%(1)Change the integration Contour inside $\bbC$
%%%%%%%Contour%%%%%%%
%(Airy function)
%(2)We consider only the asymptotic series

在此我们选择后者,将$e^{-\frac{1}{2}x^2\big/\hbar}$展开,
被积函数展开后的每一项
$$
  \frac{1}{n!}
    \int_\bbR
      \left(
        \frac{\lmd (x+a)^3}{3!\hbar}
      \right)^n
      e^{-\frac{1}{2}x^2/\hbar}
      \frac{\td x}{\sqrt{2\pi\hbar}}
$$
都可以使用同调的方法计算(与之前的例子完全类似):
直接套用性质\ref{BV算子的超变量表达式-prop},此时的BV算子为
$\pp{x}\pp{\theta}-\frac{x}{\hbar}\pp{\theta}$,其实不妨相差常数倍,令
$$\yc_{\Omg}:=\hbar\pp{x}\pp{\theta}-x\pp{\theta}$$
并且令
$$\mcalU_{\hbar}:=e^{\frac{\hbar}{2}\pp{x}\pp{x}}$$
则与之前完全类似,有
$$\yc_\Omg
=\mcalU_\hbar^{-1}\circ
\left(-x\pp{\theta}\right)\circ\mcalU_\hbar$$
从而易知
\begin{eqnarray*}
     \int_\bbR
       e^{\big(
            -\frac{1}{2}x^2
            +\frac{\lmd}{3!}(x+a)^3
          \big)
          \big/\hbar}
       \frac{\td x}{\sqrt{2\pi\hbar}}
&\sim&
     \sum_{m\geq 0}\frac{1}{m!}
       \int_\bbR
         \left(
           \frac{\lmd(x+a)^3}{3!\hbar}
         \right)^m
         e^{-\frac{1}{2}x^2\big/\hbar}
       \frac{\td x}{\sqrt{2\pi\hbar}}\\
&=&
     \left.
       \sum_{m\geq 0}\frac{1}{m!}
         e^{\frac{1}{2}\hbar\pp{x}\pp{x}}
         \left(
           \frac{\lmd x^3}{3!\hbar}
         \right)^m
     \right|_{x=a}\\
&=&
     \sum_{k,m\geq 0}
       \frac{1}{k!}
       \frac{1}{m!}
       \left(
         \frac{1}{2}\hbar\p_a^2
       \right)^k
       \left(
         \frac{\lmd a^3}{3!\hbar}
       \right)^m
\end{eqnarray*}

上式最右端具有组合意义,我们接下来详细说明。
记$\mcalP:=\frac{1}{2}\hbar\p_{a}^2$
称之为\textbf{传播子}(propagator),
\index{propagator\kong 传播子}
再记$\mcalI(a):=\frac{\lmd a^3}{3!}$为“相互作用能”,则上式为
$$
     \int_\bbR
       e^{\big(
            -\frac{1}{2}x^2
            +\frac{\lmd}{3!}(x+a)^3
          \big)
          \big/\hbar}
       \frac{\td x}{\sqrt{2\pi\hbar}}
 \sim
     \sum_{k,m\geq 0}
       \frac{1}{k!}
       \frac{1}{m!}
       \mcalP^k
       \left(
         \frac{\mcalI(a)}{\hbar}
       \right)^m
$$

我们先来观察$k=1$的情形,
看看$\mcalP\mcalI^m(a)$是什么东西。
记$\mcalP_s=\mcalP_t:=\sqrt{\hbar/2}\p_{a}$,则有
$$\mcalP=\mcalP_t\mcalP_s$$
再令$\mcalI_1=\mcalI_2=\cdots=\mcalI_m:=\mcalI$,则
\begin{eqnarray*}
     \mcalP\mcalI^m(a)
&=&
     \mcalP_t\mcalP_s
     \left(
       \mcalI_1(a)\mcalI_2(a)\cdots\mcalI_m(a)
     \right)\\
&=&
     \mcalP_t
     \left(
       \sum_{u=1}^m
         \mcalI_1(a)\cdots\mcalP_s\mcalI_u(a)\cdots\mcalI_m(a)
     \right)\\
&=&
     \left(
       \sum_{1\leq u<v\leq m}
         \mcalI_1(a)\cdots\mcalP_s\mcalI_u(a)
         \cdots\mcalP_t\mcalI_v(a)\cdots\mcalI_m(a)
     \right.\\
& &
      \quad
      +\sum_{1\leq u\leq m}
         \mcalI_1(a)\cdots\mcalP_t\mcalP_s\mcalI_u(a)
         \cdots\mcalI_m(a)\\
& &
      \quad
      \left.
      +\sum_{1\leq v<u\leq m}
         \mcalI_1(a)\cdots\mcalP_t\mcalI_v(a)
         \cdots\mcalP_s\mcalI_u(a)\cdots\mcalI_m(a)
     \right)
\end{eqnarray*}

我们将$\mcalI_1,...,\mcalI_m$视为$m$个“顶点”,将
$$\mcalI_1(a)\cdots\mcalP_s\mcalI_u(a)
\cdots\mcalP_t\mcalI_v(a)\cdots\mcalI_m(a)$$
视为从“顶点”$u$出发,到“顶点”$v$的“有向边”,
则上式可以粗俗地说成“对所有的$m$个顶点、$1$条边的图求和”。
类似地,考虑
$$\mcalP^k\mcalI^m(a)=\underbrace{\mcalP\cdots\mcalP}_{k}
\underbrace{\mcalI(a)\cdots\mcalI(a)}_m$$
然后类似地展开,得到“对所有$m$个顶点、$k$条边的图求和”。

我们将以上严格表述之,然后得到\textbf{费曼图展开公式}。
我们之前在介绍Kontsevich量子化公式的时候引入了
“带标记的有向图”的概念(见定义\ref{带标记的有向图-def})。
在这里,我们允许出现回路(即,始点与终点相同的有向边),
也允许出现“多箭头”(即从某个点出发到某个点的边可能不止一条)。
但是我们要求顶点集与边集都是有限集。

\begin{notation}(带标记的有向图的有向底图)
对于带标记的有向图$\Gma=(\Gma_0,\Gma_1,\veps)$,
则$\Gma$可以遗忘为图论当中通常的\textbf{多重有向图},
后者称为前者的\textbf{有向底图},记为$\underline{\Gma}$.
\end{notation}

遗忘的方式为“将边的名称去掉”。此操作是显然的,例如
$$
  \xymatrix{
     1 \ar@/^1pc/[r]^{a}  \ar@/_1pc/[r]_{b}
   & 2
  }
\quad
\rightsquigarrow
\quad
  \xymatrix{
     1 \ar@/^1pc/[r]     \ar@/_1pc/[r]
   & 2
  }
$$

我们更习惯将带标记的有向图$\Gma$的顶点集记为$V$,边集记为$E$
(原来使用的记号$\Gma_0,\Gma_1$废止)。
对于有限集$V,E$,定义集合
$$\mcalG_{V,E}:=\{\text{以$V$为顶点集,
以$E$为边集的全体带标记的有向图之全体}\}$$
则容易构造一一对应$\mcalG_{V,E}\cong\{\veps:E\to V\times V\}$.

\begin{definition}
(置换群$S_V\times S_E$在集合$\mcalG_{V,E}$上的作用)

对于有限集合$V,E$,定义群$S_V\times S_E$在集合$\mcalG_{V,E}$上的作用如下:
对$\mcalG_{V,E}$中的任意元素$\veps:E\to V\times V$,
以及置换$\sgm\in S_V,\tau\in S_E$,令
$$(\sgm,\tau).\veps=(\sgm\times\sgm)\circ\veps\circ\tau^{-1}$$
其中
\begin{eqnarray*}
\sgm\times\sgm: V\times V&\to& V\times V\\
(v_1,v_2)&\mapsto&(\sgm(v_1),\sgm(v_2))
\end{eqnarray*}
\end{definition}

讲人话,无非是将带标记的有向图的各顶点、各边的名称重新排列一下。
例如,$V=\{1,2,3\},\,E=\{a,b,c\}$,带标记的有向图
$$
  \Gma=
  \xymatrix{
      1  \ar@/^1pc/[r]^a
    & 2  \ar@/^1pc/[l]^b  \ar[r]^c
    & 3
  }
$$
考虑置换
$$
     \sgm=
  \begin{pmatrix}
    1 & 2 & 3\\
    1 & 3 & 2
  \end{pmatrix}
\qquad
     \tau=
  \begin{pmatrix}
    a & b & c\\
    b & c & a
  \end{pmatrix}
$$
则有
$$
  (\sgm,\tau).\Gma
=\quad
  \xymatrix{
      1  \ar@/^1pc/[r]^b
    & 3  \ar@/^1pc/[l]^c  \ar[r]^a
    & 2
  }
\quad=\quad
  \xymatrix{
      1  \ar@/^1pc/[rr]^b
    & 2
    & 3  \ar@/^1pc/[ll]^c  \ar[l]_a
  }
$$

以下性质显然成立:
\begin{lemma}
对于有限集合$V,E$,以及$\Gma,\Gma'\in\mcalG_{V,E}$,
则它们的有向底图(作为多重有向图)同构,当且仅当它们位于
群$S_v\times S_E$在$\mcalG_{V,E}$作用的同一个轨道上。
\end{lemma}

也就是说,群$S_V\times S_E$作用的轨道类,
无非是有向底图的同构类。

\begin{example}
考虑如下两个带标记的有向图
\begin{eqnarray*}
     \Gma
&:=&
     \xymatrix{
        1  \ar@/^1pc/[r]^a  \ar@/_1pc/[r]_b
       &2  \ar[r]^c
       &3
     }
\\
     \Gma'
&:=&
     \xymatrix{
        1
       &2  \ar[l]_b
       &3  \ar@/^1pc/[l]^a  \ar@/_1pc/[l]_c
     }
\end{eqnarray*}
则显然$\Gma\neq\Gma'$.
\end{example}
考虑它们的有向底图
\begin{eqnarray*}
     \underline{\Gma}
:=
     \xymatrix{
        1  \ar@/^1pc/[r]  \ar@/_1pc/[r]
       &2  \ar[r]
       &3
     }
\qquad
     \underline{\Gma'}
:=
     \xymatrix{
        1
       &2  \ar[l]
       &3  \ar@/^1pc/[l]  \ar@/_1pc/[l]
     }
\end{eqnarray*}
则$\underline{\Gma}\neq\underline{\Gma'}$;
但是它们作为多重有向图是同构的。

\begin{example}($|V|=3,|E|=2$的轨道类)

令$V=\{1,2,3\},E=\{a,b\}$,
我们给出$S_V\times S_E$在$\mcalG_{V,E}$作用的轨道类如下:
$$
  \begin{tabular}{cc}
  \hline
  轨道类   &   轨道长度\\
  \hline
    \xymatrix{
       \bullet  \ar[r]
      &\bullet  \ar[r]
      &\bullet
    }
    &$12$
  \\
    \tabularbigrow
    \xymatrix{
       \bullet  \ar[r]
      &\bullet
      &\bullet  \ar[l]
    }
    &$6$
  \\
    \tabularbigrow
    \xymatrix{
       \bullet
      &\bullet  \ar[r] \ar[l]
      &\bullet
    }
    &$6$
  \\
    \tabularbigrow
    \xymatrix{
       [\bullet]  \ar[r]
      &\bullet
      &\bullet
    }
    &$12$
  \\
    \tabularbigrow
    \xymatrix{
       [\bullet]
      &\bullet  \ar[l]
      &\bullet
    }
    &$12$
  \\
    \tabularbigrow
    \xymatrix{
       [\bullet]
      &\bullet  \ar[r]
      &\bullet
    }
    &$12$
  \\
    \tabularbigrow
    \xymatrix{
       \bullet  \ar@/^0.6pc/[r]
      &\bullet  \ar@/^0.6pc/[l]
      &\bullet
    }
    &$6$
  \\
    \tabularbigrow
    \xymatrix{
       \bullet  \ar@/^0.6pc/[r] \ar@/_0.6pc/[r]
      &\bullet
      &\bullet
    }
    &$6$
  \\
    \tabularbigrow
    \xymatrix{
       [[\bullet]]
      &\bullet
      &\bullet
    }
    &$3$
  \\
    \tabularbigrow
    \xymatrix{
       [\bullet]
      &[\bullet]
      &\bullet
    }
    &$6$
  \\
  \hline
  \end{tabular}
$$
这里用有向底图的同构类来表示轨道类。
表格中的方括号的含义是,以方括号内的顶点为端点的一条闭路(即从该点出发指向自己的箭头);
嵌套两层方括号就是两条闭路,以此类推。
\end{example}

不要忘记,我们引入这些图论概念,
是为了描述求导运算。

\begin{notation}(费曼规则)(Feynman's rule)

对于带标记的有向图$\Gma=(V,E,\veps)$,定义
$$
     w_{\Gma}(\mcalP,\mcalI(a))
:=
     \prod_{v\in V}
     \frac{
           \mcalP_s^{S(v)}
           \mcalP_t^{T(v)}
           \mcalI(a)
          }
          {
           \hbar
          }
=
     \hbar^{-|V|}
     \prod_{v\in V}
       \mcalP_s^{S(v)}
       \mcalP_t^{T(v)}
       \mcalI(a)
$$
其中$\mcalP:=\frac{1}{2}\hbar\p_a^2$为传播子,
$\mcalP_s=\mcalP_t:=\sqrt{\hbar/2}\p_a$,
$\mcalI(a):=\frac{\lmd a^3}{3!}$为“相互作用能”。
并且对于顶点$v\in V$,
$$S(v):=\#\Bigset{e\in E}{s(e)=v}$$
$$T(v):=\#\Bigset{e\in E}{t(e)=v}$$
分别为顶点$v$的\textbf{出度}与\textbf{入度}。
\index{Feynman's rule\kong 费曼规则}
\label{费曼规则}
\end{notation}

翻译成人话,对于带标记的有向图
(实际上多重有向图足矣,边的名称没贡献)$\Gma$,
我们按照如下规则给该图赋值:首先对图$\Gma$的每一个顶点赋值,
“有几条边经过此点,就求几次导”;然后将所有顶点的数值相乘。

\begin{example}注意到这里的$\mcalI(a)=\frac{\lmd a^3}{3!}$为
关于$a$的三次多项式,而$\mcalP_s=\mcalP_t$为一阶微分算子,
从而如果图$\Gma$的某个顶点的度数(入度与出度之和)大于三,那么
$w_\Gma(\mcalP,\mcalI)=0$
\end{example}
因此,我们只需要考虑每个顶点的度数都不超过$3$的图,
这些$\Gma$才能使得$w_\Gma(\mcalP,\mcalI)$取值非平凡。


\begin{lemma}
\begin{eqnarray*}
     \int_\bbR
       e^{\big(
            -\frac{1}{2}x^2
            +\frac{\lmd}{3!}(x+a)^3
          \big)
          \big/\hbar}
       \frac{\td x}{\sqrt{2\pi\hbar}}
&\sim&
     \sum_{k,m\geq 0}
       \frac{1}{k!}
       \frac{1}{m!}
       \left(
         \frac{1}{2}\hbar\p_a^2
       \right)^k
       \left(
         \frac{\lmd a^3}{3!\hbar}
       \right)^m\\
&=&
     \sum_{k,m\geq 0}
       \frac{1}{k!}
       \frac{1}{m!}
       \sum_{\Gma\in\mcalG_{m,k}}
         w_\Gma(\mcalP,\mcalI)
\end{eqnarray*}
其中$\mcalG_{m,k}:=\mcalG_{\{v_1,...,v_m\},\{e_1,...,e_k\}}$,
视为$m$个顶点、$k$条边的带标记的有向图之全体。
\end{lemma}
此式的等号显然成立,仅仅是换了一种说法。

我们将给出因子$\frac{1}{k!}\frac{1}{m!}$的组合解释。
注意到置换群$S_m\times S_k$在$\mcalG_{m,k}$的作用,
其轨道之全体记为$\underline{\mcalG}_{m,k}$,则由之前的论述,
$\underline{\mcalG}_{m,k}$可被视为$m$个顶点、$k$条边的多重有向图之全体。
再注意多重有向图也可按照费曼规则赋值(甚至多重无向图也可以)。

\begin{definition}(带标记的有向图的自同构群)

对于带标记的有向图$\Gma\in\mcalG_{m,k}$,定义其自同构群
$$\Aut(\Gma):=
\Bigset{\fai\in S_m\times S_k}{\fai.\Gma=\Gma}$$
\end{definition}
其实就是群$S_m\times S_k$在$\Gma$处的稳定子群。

对于多重有向图$\underline{\Gma}\in\underline{\mcalG_{m,k}}$,
则$\underline{\Gma}$可视为$S_m\times_k$在$\mcalG_{m,k}$作用的一条轨道,
该轨道的长度记作$\ell(\underline{\Gma})$;而对于带标记的有向图$\Gma$,
记$\ell(\Gma)$为$\Gma$所在轨道的长度。
则由群论的轨道计数知,
\begin{eqnarray*}
  m^{2k}=|\mcalG_{m,k}|
&=&\sum_{\underline{\Gma}\in\underline{\mcalG}_{m,k}}
  \ell(\underline{\Gma})
\\
m!k!=|S_m\times S_k|&=&
\ell(\Gma)|\Aut(\Gma)|\quad (\forall\Gma\in\mcalG_{m,k})
\end{eqnarray*}

而对于多重有向图$\underline{\Gma}$,我们不去定义它的自同构群,但是注意到
$$|\Aut(\underline{\Gma})|:=|\Aut(\Gma)|$$
是良定的,与代表元的选取无关,因为同一轨道的不同元素的稳定子群共轭。

综上所述,我们得到了如下费曼图公式:
\begin{thm}(费曼图公式)(Feynman diagram formula)

记号同上,则有
\begin{eqnarray*}
     \int_\bbR
       e^{\big(
            -\frac{1}{2}x^2
            +\frac{\lmd}{3!}(x+a)^3
          \big)
          \big/\hbar}
       \frac{\td x}{\sqrt{2\pi\hbar}}
&\sim&
     \sum_{k,m\geq 0}
       \sum_{\Gma\in\underline{\mcalG}_{k,m}}
         \frac{w_\Gma(\mcalP,\mcalI)}
              {|\Aut(\Gma)|}\\
&=&
     \sum_{\Gma\text{取遍}\atop\text{多重有向图}}
       \frac{w_\Gma(\mcalP,\mcalI)}
              {|\Aut(\Gma)|}
\end{eqnarray*}
\end{thm}
\begin{proof}
这是显然的,只需要注意到
\begin{eqnarray*}
     \frac{1}{k!}
     \frac{1}{m!}
     \sum_{\Gma\in\mcalG_{m,k}}
       w_\Gma(\mcalP,\mcalI)
&=&
     \frac{1}{|S_k\times S_m|}
     \sum_{\Gma\in\underline{\mcalG}_{m,k}}
       \ell(\Gma)
       w_\Gma(\mcalP,\mcalI)\\
&=&
     \sum_{\Gma\in\underline{\mcalG}_{m,k}}
       \frac{\ell(\Gma)w_\Gma(\mcalP,\mcalI)}
            {\ell(\Gma)|\Aut(\Gma)|}
 =
     \sum_{\Gma\in\underline{\mcalG}_{m,k}}
       \frac{w_\Gma(\mcalP,\mcalI)}
            {|\Aut(\Gma)|}
\end{eqnarray*}
\end{proof}
%%%%%%%%微积分%%%%%%%%%
%introduce graph with cubic vertex,
%each edge: we put $\hbar\pp{a}\pp{a}$
%each point: we put $\frac{\lmd a^3}{3!\hbar}$
%$$\omg_{\gamma}(a)$$
%by the above rule.
%%%%%%这又是什么图%%%%%%%
%\begin{thm}(Feymann graph formula)
%this integration is
%$$
%   \int_{\bbR}
%     e^{(-\frac{1}{2}x^2+\frac{\lmd}{3}(x+a)^3)/\hbar}
%     \frac{\td x}{\sqrt{2\pi\hbar}}
%=
%   \sum_{\Gamma:\text{trivalent graphs}}
%     \frac{w_{\Gamma}(a)}
%          {|\text{Aut}(\gamma)|}
%=
%   \sum
%$$
%\end{thm}
%\begin{proof}
%check.
%\end{proof}

%In general, consider
%%%%%%%%%%%%%%%%%%%2019.4.1 第六周周一%%%%%%%%%%%%%%%%%%%%%%%

注意到我们公式中的图$\Gma$可以有很多“连通分支”,
若进一步将连通分支“分解”之,费曼图公式可以写得更加紧凑。

\begin{notation}

(1)多重有向图$\Gma$可自然地被遗忘为多重(无向)图$\widehat{\Gma}$,
后者称为前者的\textbf{无向底图}。对于带标记的有向图,
类似定义其无向底图(先遗忘为多重有向图)。

(2)称带标记的有向图(或者多重有向图)是连通的,如果其无向底图连通。

(3)对于多重有向图$\Gma$,自行定义其\textbf{连通分支}。
注意连通分支依然是多重有向图。
\end{notation}

对于两个多重有向图$\Gma$与$\Gma'$,
自行定义它们的\textbf{无交并}$\Gma\sqcup\Gma'$
(常简记为$\Gma\Gma'$,这仍然是一个多重有向图)。容易知道,
对任何多重有向图$\Gma$,$\Gma$可被唯一分解为其连通分支的无交并:
$$
  \Gma\cong\Gma_1^{d_1}\Gma_2^{d_2}\cdots\Gma_l^{d_l}
$$
其中$\Gma_1,...,\Gma_l$为$\Gma$的互不同构的连通分支,$d_i\geq 1$,
$$\Gma_i^{d_i}:=\underbrace{\Gma_i\Gma_i\cdots\Gma_i}_{d_i\text{个}}$$

\begin{lemma}记号、条件承上,如果
$$
  \Gma\cong\Gma_1^{d_1}\Gma_2^{d_2}\cdots\Gma_l^{d_l}
$$
则多重有向图$\Gma$的自同构群阶数满足:
$$
  |\Aut(\Gma)|
=
  \left(
    \prod_{i=1}^l
      |\Aut(\Gma_i)|^{d_i}
  \right)
  \left(
    \prod_{i=1}^l
      d_i!
  \right)
$$
\end{lemma}
\begin{proof}[证明大意]
不妨将$\Gma$视为带标记的有向图(任取一个代表元即可),
且$\Gma\in\mcalG_{m,k}$.
一方面,易证上式的“$\geq$”,构造即可。另一方面,
我们要说明$|\Aut(\Gma)|$中的元素“只有这些”,细节从略。
{\color{blue}
真的仅仅是限于篇幅...
}
\end{proof}

\begin{thm}(费曼图公式:指数形式)

记号承上,则成立
\begin{eqnarray*}
     \int_\bbR
       e^{\big(
            -\frac{1}{2}x^2
            +\frac{\lmd}{3!}(x+a)^3
          \big)
          \big/\hbar}
       \frac{\td x}{\sqrt{2\pi\hbar}}
 \sim
     \sum_{\Gma\text{取遍}\atop\text{多重有向图}}
       \frac{w_\Gma(\mcalP,\mcalI)}
              {|\Aut(\Gma)|}
 =
     \exp
     \left(
       \sum_{\Gma\text{取遍}
             \atop\text{\textbf{连通的}多重有向图}}
         \frac{w_\Gma(\mcalP,\mcalI)}
              {|\Aut(\Gma)|}
     \right)
\end{eqnarray*}
其中等号右边的“$\exp$”为按照指数运算规则形式地展开。
\label{费曼图展开:指数形式}
\end{thm}

\begin{proof}
只需验证上式中的等号。
\begin{eqnarray*}
     \sum_{\Gma\text{取遍}\atop\text{多重有向图}}
       \frac{w_\Gma(\mcalP,\mcalI)}
              {|\Aut(\Gma)|}
&=&
     \sum_{r=0}^\infty
       \sum_{\gma=0}^\infty
         \left(
         \sum_{d_1,...,d_r\geq 1\atop d_1+\cdots+d_r=\gma}
           \sum_{\Gma_1,...,\Gma_r\text{连通}\atop\text{互不同构}}
             \frac{w_{\Gma_1^{d_1}\cdots\Gma_r^{d_r}}(\mcalP,\mcalI)}
                  {|\Aut(\Gma_1^{d_1}\cdots\Gma_r^{d_r})|}
         \right)\\
&=&
     \sum_{\gma=0}^\infty
       \sum_{r=0}^\infty
       \left(
         \sum_{d_1,...,d_r\geq 1\atop d_1+\cdots+d_r=\gma}
           \sum_{\Gma_1,...,\Gma_r\text{连通}\atop\text{互不同构}}
             \frac{1}{d_1!}\cdots
             \frac{1}{d_r!}
             \prod_{j=1}^r
               \left(
               \frac{w_{\Gma_j}(\mcalP,\mcalI)}
                    {|\Aut(\Gma_j)|}
               \right)^{d_j}
       \right)\\
&=&
     \sum_{\gma=0}^\infty
       \frac{1}{\gma!}
         \left(
           \sum_{\Gma\text{取遍}\atop
                 \text{\textbf{连通的}多重有向图}}
             \frac{w_\Gma(\mcalP,\mcalI)}
                  {|\Aut(\Gma)|}
         \right)^\gma\\
&=&
     \exp
     \left(
       \sum_{\Gma\text{取遍}
             \atop\text{\textbf{连通的}多重有向图}}
         \frac{w_\Gma(\mcalP,\mcalI)}
              {|\Aut(\Gma)|}
     \right)
\end{eqnarray*}
\end{proof}

\section{重整化群流算子}

%Recall:费曼图——理解积分的渐近展开式。
%Asymptotic expansion of
%$$\int_\bbR e^{(-\frac{1}{2}x^2+I(x+a))/\hbar}\frac{\td x}{\sqrt{2\pi t_0}}$$
%where
%$$I(x)=\sum_{n\geq 3}\frac{\lmd_n}{n!}x^n
%=e^{\frac{\hbar}{2}\p_x^2}e^{I(a)/\hbar}$$
%Feymann Graph formula
%$$=\exp\left(\sum_{\Gamma-\text{Commu graphs}}
%\frac{\omg_\Gamma(a)}{|Aut(\Gamma)|}\right)$$
%Let
%$$\mcalP=\frac{1}{2}\p_a^2$$
%is called propagator(传播子)
%$$I\mapsto\omg(\mcalP,I)$$
%Today:

我们已通过BV上同调给出了渐近展开
\begin{eqnarray*}
     \int_{\bbR}
       e^{(\frac{1}{2}x^2+\frac{\lmd(x+a)^3}{3!})\big/\hbar}
       \frac{\td x}{\sqrt{2\pi\hbar}}
&\sim&
     e^{\frac{\hbar}{2}\p_a^2}
     e^{\frac{\lmd(x+a)^3}{3!\hbar}}
\end{eqnarray*}
并给出了组合解释——费曼图展开。若令
$$w(\mcalP,\mcalI)
:=\hbar\sum_{\Gamma\text{取遍}\atop\text{\textbf{连通的}多重有向图}}
    \frac{w_\Gamma(\mcalP,I)}{|\Aut(\Gamma)|}$$
则有指数型费曼图公式
$$e^{w(\mcalP,\mcalI)/\hbar}=e^{\mcalP}e^{\mcalI/\hbar}$$
其中$\mcalP=\frac{1}{2}\hbar\p^2$为传播子。

上式中的“相互作用项”$\mcalI(x)=\frac{\lmd x^3}{3!}$可以如下推广之:

\begin{notation}记空间
$$\bbR[x]\fps{\hbar}^+:=
x^3\bbR[x]\oplus\hbar\bbR[x]\fps{\hbar}
\subseteq\bbR[x]\fps{\hbar}$$
对于$\mcalI\in\bbR[x]\fps{\hbar}$,称$\mcalI$为相互作用能。
\end{notation}
任取$\mcalI\in\bbR[x]\fps{\hbar}^+$,
以及带标记的有向图(或者多重有向图)$\Gma=(V,E,\veps)$,
我们可类似地按照费曼规则(详见记号\ref{费曼规则})给图$\Gma$赋值:
$$w_\Gma(\mcalP,\mcalI):=
     \hbar^{-|V|}
     \prod_{v\in V}
       \mcalP_s^{S(v)}
       \mcalP_t^{T(v)}
       \mcalI(a)
\in
     \bbR[x]\fps{\hbar}^+
$$
其中$\mcalP=\frac{1}{2}\hbar\p^2_x$为传播子。

接下来,我们要将费曼图推广。
对于相互作用能$\mcalI\in\bbR[x]\fps{\hbar}^+$,令
$$\mcalI=\sum_{k=3}^\infty\mcalI_{k}'x^k$$
其中$\mcalI_{k}'\in\bbR\fps{\hbar}$为形式幂级数。
记$\mcalI_k:=\mcalI_{k}'x^k$,
于是$\mcalI$被分解为$\mcalI=\sum\limits_{k=3}^\infty\mcalI_k$.

一方面我们早已知道
$$
     \int_{\bbR}
       e^{(\frac{1}{2}x^2+\mcalI(x+a))\big/\hbar}
       \frac{\td x}{\sqrt{2\pi\hbar}}
\sim
     e^{\frac{\hbar}{2}\p_a^2}
     e^{\mcalI(a)/\hbar}
$$
之后直接将右边展开,得到费曼图解释。
但另一方面我们还可以如此展开:
\begin{eqnarray*}
     e^{\frac{\hbar}{2}\p_a^2}
     e^{\mcalI(a)/\hbar}
&=&
     e^{\frac{\hbar}{2}\p_a^2}
     e^{\frac{1}{\hbar}
        \sum\limits_{k=3}^\infty
          \mcalI_{k}(a)
       }\\
&=&
     e^{\frac{\hbar}{2}\p_a^2}
     \sum_{q=0}^{\infty}
       \frac{1}{q!}
       \left(
         \frac{1}{\hbar}
         \sum_{k=3}^\infty
           \mcalI_{k}(a)
       \right)^q\\
&=&
     e^{\frac{\hbar}{2}\p_a^2}
     \left(
       \sum_{r=0}^\infty
         \sum_{
               3\leq k_1<...<k_r
               \atop
               d_1,...,d_r\geq 0
              }
           \frac{\hbar^{-(d_1+\cdots+d_r)}}
                {d_1!\cdots d_r!}
           \prod_{j=1}^r
             \mcalI_{k_j}^{d_j}(a)
     \right)
\end{eqnarray*}

观察上式,仍将微分算子级数$e^{\frac{\hbar}{2}\p_a^2}$理解为“添加有向边”;
而括号里的一长串也有组合解释——\textbf{带权的顶点集}。
给“带权的顶点集”顺次添加“有向边”,就得到比
“带标记的有向图的顶点集$V$”更为复杂的组合结构。
大致地说,以前我们给图的所有顶点都赋以相同的$\mcalI(a)$,
但是现在给图的不同顶点赋以不同的$\mcalI_{k}(a)$.

\begin{definition}(顶点带权的带标记的有向图)

\textbf{顶点带权的带标记的有向图}是指如下资料:
$$(V,E,\veps,W,\fai)$$
并且满足:

(1)$(V,E,\veps)$为带标记的有向图;

(2)$W$为集合,映射$\fai:V\to W$为满射。
\end{definition}
我们称$W$为权集,在这里我们只考虑
$W\subseteq\{\mcalI_3,\mcalI_4,...\}
\subseteq\bbR[x]\fps{\hbar}^+$的情形。

我们可以按照加权的类型对顶点带权的带标记的有向图分类。
比如,如果图$\Gma$有$d_1$个顶点赋以权$\mcalI_{k_1}$,
$d_2$个顶点赋以权$\mcalI_{k_2}$,……,
$d_r$个顶点赋以权$\mcalI_{k_r}$,则称该图为
“$\mcalI_{k_1}^{d_1}\cdots\mcalI_{k_r}^{d_r}$型”的。

考虑顶点集为$V$,边集为$E$,且类型为
$\mcalI_{k_1}^{d_1}\cdots\mcalI_{k_r}^{d_r}$的图之全体,
姑且暂时记作$\mcalG_{V,E;\mcalI_{k_1}^{d_1}\cdots\mcalI_{k_r}^{d_r}}$.
注意到总是成立
$$|V|=\sum_{i=1}^rd_i$$

考虑置换群$S_V\times S_E$的\textbf{子群}
$$S_{V,E;\mcalI_{k_1}^{d_1}\cdots\mcalI_{k_r}^{d_r}}
:=\left(S_{d_1}\times\cdots\times S_{d_r}\right)\times S_E
$$
在$\mcalG_{V,E;\mcalI_{k_1}^{d_1}\cdots\mcalI_{k_r}^{d_r}}$
上的作用。其规则与不加权的情形类似,
但是要注意只有所带的权相同的顶点之间才可以置换。

于是对于$\Gma=(V,E,\veps,W,\fai)$,类似去定义其自同构群$\Aut(\Gma)$,
该自同构群是$S_{V,E;\mcalI_{k_1}^{d_1}\cdots\mcalI_{k_r}^{d_r}}$
的子群(该群在该图上作用的稳定子群)。

同样可以考虑群$S_{V,E;\mcalI_{k_1}^{d_1}\cdots\mcalI_{k_r}^{d_r}}$
在$\mcalG_{V,E;\mcalI_{k_1}^{d_1}\cdots\mcalI_{k_r}^{d_r}}$上的作用
的\textbf{轨道};另一方面考虑遗忘
$$
\begin{pmatrix}
  \text{顶点带权的}\\
  \text{带标记的有向图}
\end{pmatrix}
\xra{\text{遗忘}}
\begin{pmatrix}
  \text{顶点带权的}\\
  \text{多重有向图}
\end{pmatrix}
\xra{\text{遗忘}}
\begin{pmatrix}
  \text{顶点带权的}\\
  \text{多重图}
\end{pmatrix}
$$

设$\Gma$为顶点带权的多重图,自行按照费曼规则去定义
$w_\Gma(\mcalP)$.

\begin{example}考虑如下的顶点带权的带标记的有向图$\Gma$如下:
$$
  \xybigcol
  \xymatrix{
    &&&\mcalI_4^{(2)}
      \ar@/_1.2pc/[dd]_{e_1}
      \ar@/^1.2pc/[dd]_{e_3}
      \ar[dd]_{e_2}
  \\
     \mcalI_3^{(1)} \ar@/^2pc/[r]^{a_1}  \ar@/_2pc/[r]_{a_2}
    &\mcalI_4^{(1)} \ar[r]^b
    &\mcalI_3^{(2)} \ar@/^1.2pc/[ur]^c   \ar@/_1.2pc/[dr]_d
    &
  \\
    &&&\mcalI_6^{(1)}
  }
$$
其顶点集、边集分别为
$$V=\{\mcalI_3^{(1)},\mcalI_3^{(2)};
\mcalI_4^{(1)},\mcalI_4^{(2)};\mcalI_6^{(1)}\}$$
$$E=\{a_1,a_2,b,c,d,e_1,e_2,e_3\}$$
并且顶点$\mcalI_k^{(l)}$赋权$\mcalI_k$.
\end{example}
对于传播子$\mcalP=\frac{1}{2}\hbar\p_x^2$,
此图按照费曼规则的赋值为
$$
     w_\Gma(\mcalP)
=
     \left(
     \frac{1}{2}\hbar
     \right)^{|E|}
     \hbar^{-|V|}
     (\p_x^2\mcalI_3^{(1)})
     (\p_x^3\mcalI_3^{(2)})
     (\p_x^3\mcalI_4^{(1)})
     (\p_x^4\mcalI_4^{(2)})
     (\p_x^4\mcalI_6^{(1)})
$$
其中$\mcalI_k^{(l)}:=\mcalI_k$.
还可以考察其自同构群,容易知道
$$|\Aut(\Gma)|=2!\times 3!=12$$
$\Aut(\Gma)$由边$a_1,a_2$之间的置换,
以及$e_1,e_2,e_3$之间的置换生成,
不含有顶点之间的置换。\vs

我们不难得到顶点加权版本的费曼图展开公式:
\begin{prop}
设$\mcalI=\sum\limits_{k=3}^\infty\mcalI_k\in\bbR[x]\fps{\hbar}^+$,
以及传播子$\mcalP=\frac{1}{2}\hbar\p_x^2$,记
$$\mcalW=\{\mcalI_3,\mcalI_4,...\}$$
则成立渐进展开式
\begin{eqnarray*}
     \int_\bbR
       e^{\big(
            -\frac{1}{2}x^2
            +\mcalI(x+a)
          \big)
          \big/\hbar}
       \frac{\td x}{\sqrt{2\pi\hbar}}
&\sim&
     \sum_{\Gma\text{取遍}\atop\text{多重有向图}}
       \frac{w_\Gma(\mcalP,\mcalI)}
              {|\Aut(\Gma)|}
 =
     \exp
     \left(
       \sum_{\Gma\text{取遍}
             \atop\text{\textbf{连通的}多重有向图}}
         \frac{w_\Gma(\mcalP,\mcalI)}
              {|\Aut(\Gma)|}
     \right)\\
&=&
     \sum_{\Gma\text{:顶点带权}
           \atop\text{权集$\subseteq\mcalW$}}
       \frac{w_\Gma(\mcalP)}
              {|\Aut(\Gma)|}
 =
     \exp
     \left(
       \sum_{\Gma\text{:\textbf{连通},顶点带权}
             \atop\text{权集$\subseteq\mcalW$}}
         \frac{w_\Gma(\mcalP)}
              {|\Aut(\Gma)|}
     \right)
\end{eqnarray*}
\label{顶点带权的费曼图展开公式}
\end{prop}

\begin{proof}
与“顶点不带权”(严格地说应该是,所有顶点带相同的权)的情形完全类似,不再赘述。
但是要注意上式最右边两项,求和号下是\textbf{对轨道等价类求和},
每条轨道出一个代表即可。
\end{proof}

有了以上准备工作,我们即可引入\textbf{重整化群流算子}。
以下是本节主要结果:

\begin{thm}
对于传播子$\mcalP=\frac{1}{2}\hbar\p_x^2$,以及
$\mcalI\in\bbR[x]\fps{h}^+$,定义
$$w(\mcalP,\mcalI)
:=\hbar\sum_{\Gamma\text{取遍}\atop\text{\textbf{连通的}多重有向图}}
    \frac{w_\Gamma(\mcalP,I)}{|\Aut(\Gamma)|}$$
那么算子
\begin{eqnarray*}
w(\mcalP,\text{-}):\bbR[x]\fps{h}^+ &\to& \bbR[x]\fps{h}^+\\
\mcalI&\mapsto& w(\mcalP,\mcalI)
\end{eqnarray*}
是良定的,并且具有逆算子$w(-\mcalP,\text{-})$.
该算子称为\textbf{重整化群流算子}(renormalization group flow operator)。
\index{renormalization group flow operator\kong 重整化群流算子}
\label{重整化群流算子-thm}
%$\omg_(\Gamma,-)$ is called \textbf{renormalization group flow operator}.
%is a well-defined transformation on $\bbR\fps{x,\hbar}^+$.
%and its inverse is $\omg(-\mcalP,-)$.
\end{thm}

\begin{proof}%Let
任取$\mcalI\in\bbR[x]\fps{\hbar}^+$,则有
$$w(\mcalP,\mcalI)=
\hbar\sum_{\Gamma\text{取遍}\atop\text{\textbf{连通的}多重有向图}}
    \frac{w_\Gamma(\mcalP,I)}{|\Aut(\Gamma)|}
=\sum_{G\in\bbZ\atop D\geq 0}
   \lmd_{G,D}\hbar^Gx^D
$$
其中$\lmd_{G,D}\in\bbR$.为证明良定性,
我们需要说明$\lmd_{G,D}$满足以下三点:

(1)若$G<0$,则$\lmd_{G,D}=0$;

(2)$\lmd_{0,0}=\lmd_{0,1}=\lmd_{0,2}=0$;

(3)对任意$G\geq 0$,使得$\lmd_{G,D}$非零的$D$至多有限个。

\vs%where $I_{k,g}$ has degree $k$ polynomial on $x$.
\textbf{Step1:}考虑$\mcalI$的分解
$$\mcalI=\sum_{k,g\geq 0}I_{k,g}\hbar^g$$
其中$\mcalI_{k,g}\in\bbR[x]$并且次数为$k$(或者$\mcalI_{k,g}=0$).
则容易验证,如果$\mcalI_{k,g}\neq 0$,那么必有
$$2g-2+k\geq 0$$
并且等号成立当且仅当$g=1,k=0$.

记集合$\mcalW:=\{\mcalI_{k,g}|k,g\geq 0\}$,
则由顶点带权版本的费曼图展开
(性质\ref{顶点带权的费曼图展开公式}),可知
\begin{eqnarray*}
     w(\mcalP,\mcalI)
 =
     \hbar
     \sum_{\Gamma\text{取遍}
           \atop\text{\textbf{连通的}多重有向图}}
       \frac{w_\Gamma(\mcalP,I)}{|\Aut(\Gamma)|}
 =
     \hbar
     \sum_{\Gma\text{:\textbf{连通},顶点带权}
           \atop\text{权集$\subseteq\mcalW$}}
       \frac{w_\Gma(\mcalP)}
            {|\Aut(\Gma)|}
\end{eqnarray*}\vs

%we need to prove
%$$\hbar\sum_{\Gamma}\frac{\omg_\Gamma(\mcalP,I)}{\hbar}$$
%is well defined, and in $\bbR\fps{x,\hbar}^+$.
%Let $\Gma$ be a connected graph,
%%%%%%%2334%%%%%%%%%%%%%%
%证了一页多。。。见手稿

\textbf{Step2:}于是,我们有
$$
  \sum_{G\in\bbZ\atop D\geq 0}
    \lmd_{G,D}\hbar^Gx^D
=
  \hbar
  \sum_{\Gma\text{:\textbf{连通},顶点带权}
        \atop\text{权集$\subseteq\mcalW$}}
    \frac{w_\Gma(\mcalP)}
         {|\Aut(\Gma)|}
$$
我们使用此式来比较系数,研究$\lmd_{G,D}$的性质,
从而给出良定性的证明。

任意固定一个顶点带权的带标记的有向图(的轨道等价类)
$\Gma$,以及$G\in\bbZ,D\in\bbN$,使得
$$\hbar w_{\Gma}(\mcalP)=C\hbar^Gx^D\neq 0$$
其中$C\in\bbR$为常数(“吸收”了$|\Aut(\Gma)|^{-1}$以及“顶点处的系数”)。
记$\Gma$的顶点、边的个数分别为$V,E$.

记$n_{k,g}$为图$\Gma$的赋以权$\mcalI_{k,g}$的顶点的个数,则有等式
$$V=\sum_{k,g\geq 0}n_{k,g}\eqno{(1)}$$
此外,直接用费曼规则展开$w_\Gma(\mcalP)$来计数$\hbar$与$x$的次数,有
$$G-1=E-V+\sum_{k,g\geq 0}gn_{k,g}\eqno{(2)}$$
$$D=\sum_{k,g\geq 0}kn_{k,g}-2E\eqno{(3)}$$

结合上述$(1)(2)(3)$式,我们有
\begin{eqnarray*}
     2G+D-2
&=&
     2E-2V+2\sum_{k,g\geq 0}gn_{k,g}+D\\
&=&
     \sum_{k,g\geq 0}kn_{k,g}
    -2\sum_{k,g\geq 0}n_{k,g}
    +2\sum_{k,g\geq 0}gn_{k,g}\\
&=&
     \sum_{k,g\geq 0}
       (2g-2+k)n_{k,g}\\
&\geq&0
\end{eqnarray*}
等号成立当且仅当$G=1,D=0$,并且只有$n_{0,1}\neq 0$.
此时图$\Gma$只有一个点,没有边,
并且该点的权形如$\mcalI_{0,1}\hbar$,
其中$\mcalI_{0,1}\in\bbR^\times$.\vs

\textbf{step3:}我们再利用一下$\Gma$的连通性。
由连通性可知顶点个数$V$与边的个数$E$满足
$$E\geq V-1$$
再结合$(2)$式,可知$G\geq 0$.也就是说,不会出现$\hbar$的负幂次;
并且如果$G=0$,则由Step2的结论$2G+D-2\geq0$(以及取等条件),知$D\geq 3$.

综上所述,对任意的连通的顶点带权的带标记的有向图(的轨道等价类)$\Gma$,
$$w_\Gma(\mcalP)\in\bbR[x]\fps{\hbar}^+$$

现在,令$\Gma$取遍所有的连通的顶点带权的带标记的有向图(的轨道等价类),
对于给定的$G,D\geq 0$,考虑
$$\sum_{\Gma\text{\textbf{连通},顶点带权}\atop\text{权集}\subseteq\mcalW}
\frac{w_\Gma(\mcalP)}{|\Aut(\Gma)|}$$
对系数$\lmd_{G,D}$的贡献。

对于\textbf{连通}图$\Gma$,断言:如果$w_{\Gma}(\mcalP)\neq 0$,
并且$\Gma$的顶点个数$\geq 2$,那么$n_{0,1}=0$,
也就是说$\Gma$的任何顶点都不可能赋以权$\mcalI_{0,1}$。
原因很简单,这是因为如果某个顶点赋以权$\mcalI_{0,1}$,
注意$\Gma$的连通性,该顶点必有边连接。
然而$\mcalI_{0,1}$关于$x$的次数为$0$,于是$\p_x\mcalI_{0,1}=0$。
因此按照费曼规则计算之,$w_\Gma(\mcalP)=0$,矛盾。

因此,若连通图$\Gma$(不妨顶点个数$\geq2$,
顶点个数为$1$的稍后单独讨论)使得
$$\hbar w_{\Gma}(\mcalP)=C\hbar^Gx^D\neq 0$$
则有
$$
2G+D-2=\sum_{k,g\geq 0\atop n_{k,g}\neq0}(2g+2-k)n_{k,g}
\geq\sum_{k,g\geq 0\atop n_{k,g}\neq0}n_{k,g}=V
$$
(注意$2g-2+k\geq 0$但是等号取不到)从而顶点个数$V$有上界
$$V\leq 2G+D-2$$
因此对系数$\lmd_{G,D}$有贡献的连通图$\Gma$至多有限多个
(单独讨论一下顶点个数$V=1$的情形,直接费曼规则计算之,从略)
因此$\lmd_{G,D}\in\bbR$.良定性证毕。\vs

\textbf{Step4:}至此,我们已有算子
\begin{eqnarray*}
w(\mcalP,\text{-}):\bbR[x]\fps{h}^+ &\to& \bbR[x]\fps{h}^+\\
\mcalI&\mapsto& w(\mcalP,\mcalI)
\end{eqnarray*}
断言$w(\mcalP,\text{-})$可逆,且逆算子为$w(-\mcalP,\text{-})$.
这个几乎显然,只需注意费曼图展开公式
$$e^{w(\mcalP,\mcalI)/\hbar}=e^\mcalP e^{\mcalI/\hbar}$$
从而有
\begin{eqnarray*}
     e^{w(-\mcalP,w(\mcalP,\mcalI))/\hbar}
 =
     e^{-\mcalP}
     e^{w(\mcalP,\mcalI)/\hbar}
 =
     e^{-\mcalP}e^{\mcalP}
     e^{\mcalI/\hbar}
 =
     e^{\mcalI/\hbar}
\end{eqnarray*}
因此$w(-\mcalP,w(\mcalP,\mcalI))=\mcalI$.另一边同理,从而证毕。
\end{proof}



\section{$n$维Gauss积分}

回顾之前若干节。我们已经熟悉了一维情形,即通过BV上同调给出积分
$$
  \int_\bbR
    e^{(-\frac{1}{2}x^2+\mcalI(x+a))/\hbar}
    \frac{\td x}
         {\sqrt{2\pi\hbar}}
\eqno{(*)}
$$
的渐近展开,其中相互作用能
$$\mcalI(x)=\sum\limits_{k,g}\hbar^g\in\bbR[x]\fps{\hbar}^+$$
我们引入了一种有向图,每个顶点都赋以$\mcalI(x)$,
每条有向边都赋以微分算子$\frac{1}{2}\hbar\p_x\ten\p_x$;
依照费曼规则给这样的图赋值,
进而给出了积分$(*)$的渐近展开——费曼图公式。
此外,我们通过更精细的组合技巧,
即图的顶点可以赋予不同的$\mcalI_{k,g}$,
引入了重整化群流算子。

本节探讨更加一般的有限维情形,类似于“多元正态分布”。

\begin{Example}
我们考察积分
$$
  \int_{\bbR^n}
    e^{
        (-\frac{1}{2}Q(x)
       +\mcalI(x+a))/\hbar
      }
    \bigwedge_{i=1}^n
      \frac{\td x^i}
           {\sqrt{2\pi\hbar}}
$$
其中$x=(x^1,...,x^n)\in\bbR^n$,$Q(x):=Q_{ij}x^ix^j$为二次函数,
二次型矩阵$(Q_{ij})$对称、严格正定、常值;相互作用能
$\mcalI(x)\in\bbR[x^1,...,x^n]\fps{\hbar}^+$,其中
$$\bbR[x^1,...,x^n]\fps{\hbar}^+:=
\left(
\bigoplus_{1\leq i\leq j\leq k\leq n}
  x^ix^jx^k\bbR[x^1,...,x^n]\fps{\hbar}
\right)
\oplus\hbar\bbR[x^1,...,x^n]\fps{\hbar}
$$
\end{Example}
%For $\bbR^n$,consider
%at least cubic in $x^i$ modulo $\hbar$.

与之前完全类似,我们先考察BV上同调。
在这里,体积形式为%the volume form
$$
  \Omg=e^{-\frac{1}{2\hbar}Q(x)}
       \bigwedge_{i=1}^n
         \frac{\td x^i}
              {\sqrt{2\pi\hbar}}
$$
需要注意在这里
$$\int_{\bbR^n}\Omg=\frac{1}{\sqrt{\det Q}}$$

考察BV算子,由性质\ref{BV算子的超变量表达式-prop}直接写出
\begin{eqnarray*}
     \yc_\Omg'
&=&
     \pp{x^i}\pp{\theta_i}
    -\frac{1}{2\hbar}
     \pp{x^i}
     \left(
       Q_{kl}x^kx^l
     \right)
     \pp{\theta_i}\\
&=&
     \pp{x^i}\pp{\theta_i}
    -\frac{1}{\hbar}
     Q_{ik}x^k\pp{\theta_i}
\end{eqnarray*}
不妨相差常数倍(不会影响同调),令BV算子
$$\yc_{\Omg}:=
     \hbar\pp{x^i}\pp{\theta_i}
    -Q_{ik}x^k\pp{\theta_i}
$$

\begin{lemma}
记号承上,令传播子
$$\mcalP=\frac{1}{2}\hbar Q^{ij}\pp{x^i}\pp{x^j}$$
其中$(Q^{ij})$为$(Q_{ij})$的逆矩阵。记$\mcalU=e^\mcalP$,则成立
$$\yc_\Omg=
\mcalU^{-1}
\left(
  -Q_{ij}x^i\pp{\theta_j}
\right)
\mcalU$$
\end{lemma}

\begin{proof}
与一维情形完全类似。容易验证
$$[\mcalP,-Q_{ij}x^i\pp{\theta_j}]=
  -\frac{1}{2}\hbar Q^{ij}Q_{kl}
  [\pp{x^i}\pp{x^j},x^l\pp{\theta_k}]
=
  -\hbar\pp{x^i}\pp{\theta_i}
$$
$$[\mcalP,[\mcalP,-Q_{ij}x^i\pp{\theta_j}]]
=[\mcalP,-\hbar\pp{x^i}\pp{\theta_i}]=0$$
从而有
\begin{eqnarray*}
     \mcalU^{-1}
     \left(
       -Q_{ij}x^i\pp{\theta_j}
     \right)\mcalU
&=&
     \exp(\ad_{-\mcalP})
     \left(
       -Q_{ij}x^i\pp{\theta_j}
     \right)\\
&=&
     -Q_{ij}x^i\pp{\theta_j}
     -[\mcalP,-Q_{ij}x^i\pp{\theta_j}]\\
&=&
     \hbar\pp{x^i}\pp{\theta_i}
    -Q_{ij}x^i\pp{\theta_j}=\yc_\Omg
\end{eqnarray*}
\end{proof}

注意$\mcalU$显然可逆,从而$\mcalU$诱导了上链复形的同构:
$$
  \mcalU:(\PV_{\bbR^n}\updot,\yc_\Omg)
  \xra{\sim}
  (\PV_{\bbR^n}\updot,-Q_{ij}x^i\pp{\theta_j})
$$
与一维情形类似,
容易验证在$H^0(\PV_{\bbR^n}\updot,-Q_{ij}x^i\pp{\theta_j})$当中,
$[(x^1)^{d_1}(x^2)^{d_2}\cdots(x^n)d^n]=0$
(如果$d_1,...,d_n$不全为零)——这只需注意到
$$-Q_{ij}x^i\pp{\theta_j}(-Q^{kl}\theta_k)=x^l$$

从而易知成立如下的渐近展开:
\begin{prop}记号承上,则有
\begin{eqnarray*}
     \int_{\bbR^n}
       e^{
           (-\frac{1}{2}Q(x)
          +\mcalI(x+a))/\hbar
         }
       \bigwedge_{i=1}^n
         \frac{\td x^i}
              {\sqrt{2\pi\hbar}}
&\sim&
     \left.
       e^{\frac{1}{2}\hbar Q^{ij}\pp{x^i}\pp{x^j}}
       e^{\mcalI(x)/\hbar}
     \right|_{x=a}
     \int_{\bbR^n}\Omg\\
&=&
     \frac{1}{\sqrt{\det Q}}
     e^{\frac{1}{2}\hbar Q^{ij}\pp{a^i}\pp{a^j}}
     e^{\mcalI(a)/\hbar}
\end{eqnarray*}
\end{prop}

我们仿照一维情形,给出上式的费曼图解释。
与一维情形不同的是,传播子
$\mcalP=\frac{1}{2}\hbar Q^{ij}\pp{x^i}\pp{x^j}$
更加复杂,为此我们需要稍微推广一下费曼规则。

给定相互作用能$\mcalI\in\bbR[x^1,...,x^n]\fps{\hbar}^+$,
以及传播子$\mcalP=\frac{1}{2}\hbar Q^{ij}\pp{x^i}\pp{x^j}$,
我们企图给多重图(带标记的有向图的遗忘)赋值。
我们通过例子来说明费曼规则。

\begin{example}若多重图$\Gma$如下:
$$
  \xymatrix{
     \bullet  \ar@{-}@/^1pc/[r]
              \ar@{-}@/_1pc/[r]
    &\bullet  \ar@{-}[r]
    &\bullet
  }
$$
则首先给每一条边的两端添加标记(文字),得到:

$$
  \xymatrix{
     \bullet  \ar@{-}@/^1pc/[r]^<<{a_1}^>>{b_1}
              \ar@{-}@/_1pc/[r]_<<{a_2}_>>{b_2}
    &\bullet  \ar@{-}[r]^<<{c}^>>{d}
    &\bullet
  }
$$
然后直接写出
$$
  w_\Gma(\mcalP,\mcalI)
:=
  \left(\frac{1}{2}\hbar Q^{i_1j_1}\right)
  \left(\frac{1}{2}\hbar Q^{i_2j_2}\right)
  \left(\frac{1}{2}\hbar Q^{cd}\right)
  \left(\p_{a_1}\p_{a_2}\frac{\mcalI}{\hbar}\right)
  \left(\p_{b_1}\p_{b_2}\p_c\frac{\mcalI}{\hbar}\right)
  \left(\p_{d}\frac{\mcalI}{\hbar}\right)
$$
这里的文字$a_1,a_2,b_1,b_2,c,d$视为从$1$到$n$的求和指标,
服从爱因斯坦求和约定。
\end{example}

显然如此定义的$w_\Gma(\mcalP,\mcalI)$与边两端的标记文字选取无关
(只要保证这些文字两两不同)。
对于带标记的有向图$\Gma=(V,E,\veps)$,
我们当然也能给它赋值$w_\Gma(\mcalP,\mcalI)$,
只需要将$\Gma$遗忘为多重图。
我们再看一个例子:

\begin{example}考虑如下多重图$\Gma$:
$$
  \xymatrix{
     &&\bullet   \ar@{-}@/_1pc/[d]
                 \ar@{-}@/^1pc/[d]
     &
  \\
     &&\bullet   \ar@{-}[dl]
                 \ar@{-}[dr]
     &
  \\
       \bullet   \ar@{-}[r]
     & [\bullet]
     &&\bullet
  }
$$
该图的顶点数$V=5$,边数$E=6$,
特别注意方括号代表一条闭路(即起始点与终点相同)。
我们来考察$w_\Gma(\mcalP,\mcalI)$
\end{example}

\begin{proof}[解]
一样地,首先给每条边的两端标记文字:
$$
  \xymatrix{
     &&\bullet   \ar@{-}@/_1pc/[d] _<<<{a}_>>>{a'}
                 \ar@{-}@/^1pc/[d] ^<<<{b}^>>>{b'}
     &
  \\
     &&\bullet   \ar@{-}[dl] _<<<{d}_>>>{d'}
                 \ar@{-}[dr] ^<<<{e}^>>>{e'}
     &
  \\
       \bullet   \ar@{-}[r] ^<<{c}^>>{c'}
     & [\bullet]_{f,f'}
     &&\bullet
  }
$$
从而从图中直接读出
\begin{eqnarray*}
     w_\Gma(\mcalP,\mcalI)
&=&
     (\frac{1}{2}\hbar)^E\hbar^{-V}
     Q^{aa'}Q^{bb'}Q^{cc'}Q^{dd'}Q^{ee'}Q^{ff'}\\
& &
     \left(\p_{a}\p_{b}\mcalI\right)
     \left(\p_{a'}\p_{b'}\p_{d}\p_{e}\mcalI\right)
     \left(\p_{c}\mcalI\right)
     \left(\p_{c'}\p_{d'}\p_{f}\p_{f'}\mcalI\right)
     \left(\p_{e'}\mcalI\right)
\end{eqnarray*}
\end{proof}

通过以上两例,不难总结出\textbf{费曼规则}。
与一维情形完全类似,也有费曼图展开公式:

\begin{thm}(费曼图公式:$n$维情形)
记号承上,则有渐近展开式
\begin{eqnarray*}
     \int_{\bbR^n}
       e^{(-\frac{1}{2}Q(x)+\mcalI(x+a))/\hbar}
       \bigwedge_{i=1}^n
         \frac{\td x^i}{\sqrt{2\pi\hbar}}
&\sim&
     \frac{1}{\sqrt{\det Q}}
     \left(
       e^{\frac{1}{2}\hbar Q^{ij}
          \pp{a^i}\pp{a^j}}
       e^{\mcalI(a)/\hbar}
     \right)\\
&=&
     \frac{1}{\sqrt{\det Q}}
     \sum_{\Gma\text{取遍}\atop\text{多重有向图}}
       \frac{w_\Gma(\mcalP,\mcalI)}
              {|\Aut(\Gma)|}\\
&=&
     \frac{1}{\sqrt{\det Q}}
     \exp
     \left(
       \sum_{\Gma\text{取遍}
             \atop\text{\textbf{连通的}多重有向图}}
         \frac{w_\Gma(\mcalP,\mcalI)}
              {|\Aut(\Gma)|}
     \right)
\end{eqnarray*}
\end{thm}

\begin{proof}
与定理\ref{费曼图展开:指数形式}的证明完全相同,从略。
\end{proof}

与一维情形类似,我们也可以引入重整化群流算子:

\begin{thm}(重整化群流算子:$n$维情形)

对于传播子$\mcalP=\frac{1}{2}\hbar Q^{ij}\pp{x^i}\pp{x^j}$,
以及相互作用能$\mcalI\in\bbR[x^1,...,x^n]\fps{\hbar}^+$,令
$$w(\mcalP,\mcalI):=
         \sum_{\Gma\text{取遍}
             \atop\text{\textbf{连通的}多重有向图}}
           \frac{w_\Gma(\mcalP,\mcalI)}
                {|\Aut(\Gma)|}
$$
则算子
$$w(\mcalP,\text{-}):
\bbR[x^1,...,x^n]\fps{\hbar}^+\to
\bbR[x^1,...,x^n]\fps{\hbar}^+$$
是良定的,并且具有逆算子$w(-\mcalP,\text{-})$.
\end{thm}

\begin{proof}
与一维情形(定理\ref{重整化群流算子-thm})完全类似,
引入顶点带权的带标记的有向图。从略。
\end{proof}

%where
%$$\mcalU=e^{\frac{1}{2}\hbar Q^{ij}\pp{x^i}\pp{x^j}}$$
%where $(Q^{ij})=(Q_{ij})^{-1}$. Then
%%%%%%%%%%Feymann rule%%%%%%%%%%%%%%%%5
%Similarly,
%$$\mcalP=\frac{1}{2}\sum_{i,j}Q^{ij}\pp{x^i}\pp{x^j}$$
%$$\omg(\mcalP,-):\bbR\fps{x^i,\hbar}^+\to\bbR\fps{x^i,\hbar}^+$$
%$$I\mapsto \omg(\mcalP,I)$$
%$$e^{\omg(\mcalP,I)/\hbar}"="e^{\hbar\mcalP}e^{I/\hbar}$$
%it is well defined, invertible...

\section{无穷维情形初探:$\phi^4$-场论}
%Now, $\bbR^n$ when $n"\to"\infty$...
%Quantum field theory case:

现在,我们试图类比$n$维Gauss积分,
讨论无穷维的量子场论,定义路径积分
$\int_\mcalE e^{\mcalS[\phi]/\hbar}[D\fai]$.
本节谈论$\bbR^d$标量场论的一种情形:

\begin{Example}%Scalar field theory,$\bbR^D$.
对于$d\geq 3$,考虑场空间
$$\mcalE=C_c^{\infty}(\bbR^d)$$%smooth functions ,
即$\bbR^d$上的紧支、光滑函数之全体。定义$\mcalE$上泛函
$\mcalS:\mcalE\to\bbR$如下:
$$\mcalS[\phi]=\frac{1}{2}
\int_{\bbR^d}
  |\nabla\phi|^2
 +\frac{\lmd}{4!}
  \int_{\bbR^d}\phi^4
$$
\end{Example}%for

对于$\phi\in\mcalE$,我们希望定义出路径积分%we want
$$“\int_{\mcalE}e^{-\mcalS[\phi]/\hbar}[D\phi]”$$

首先注意到
$$
\int_{\bbR^d}|\nabla \phi|^2
=\int_{\bbR^d}\phi D\phi
$$
其中$D=-\sum\limits_{i=1}^d\pp{x^i}\pp{x^i}$为Laplace算子(注意负号)。

不难将此无穷维情形与之前的$n$维情形类比如下:
$$
  \begin{tabular}{|c|c|}
    \hline
      $n$维                &  无限维\\
    \hline
      $\bbR^n$             &  $\mcalE=C_c^\infty(\bbR^d)$\\
      $x_i,\,
      1\leq i\leq n$       &  $\fai(x),x\in\bbR^d$\\
      $\sum\limits_{i=1}^n$&  $\int_{\bbR^d}\td x$\\
      $x^iQ_{ij}x^j$       &  $\int_{\bbR^d}\phi D\phi$\\
      $\mcalI(x)$          &  $\frac{\lmd}{4!}\int_{\bbR^d}\phi^4$\\
    \hline
  \end{tabular}
$$

有一种想法是,我们将$\bbR^n$视为
$$\bbR^n=C^\infty(\bbR^d\text{当中的}n\text{个点})$$
之后令“$n\to\infty$”,取$\bbR^d$的一个“均匀”、
稠密的点集,从而达到无限维。这在物理上叫做“格点场论”。

%$$\bbR^n\rightsquigarrow\mcalE$$
%finite dimension ,
%$$\bbR^n=map(n \text{points},\bbR)$$
%so,$$\mcalE=C^{\infty}(\bbR^D)"="\lim_{N\to\infty}C^{\infty}(N \text{points})$$
%(取密密麻麻的点?格点场论)
%$$i \text{index}\rightsquigarrow x\in\bbR^D$$
%$$\sum_i\rightsquigarrow\int_{\bbR^D}\td x$$
%free part:
%$$\frac{1}{2}\int_X|\td\phi|^2
%=\frac{1}{2}\int_X\phi D\phi$$
%where $D=-\sum_i\pp{x^i}\pp{x^i}$ be laplacian:
%$$D:\mcalE\to\mcalE$$
%propagator

在有限维情形,我们有传播子
$\mcalP:=\frac{1}{2}\hbar Q^{ij}\pp{x^i}\pp{x^j}$,
我们希望得到它的无限维版本。
注意$(Q_{ij})$的无限维版本是Laplace算子$D$;
于是为了得到$(Q^{ij})$的无限维版本,
自然需要考虑Lapalce算子$D$的“逆”。
幸运的是,分析学在这方面给我们足够支持。

\begin{lemma}(Lapalce算子的格林函数)

对于$d\geq 3$,记$S_d$为$d$维单位球体的表面积,
考虑$\bbR^d\setminus\{0\}$上的函数
$$G(x):=\frac{1}{(d-2)S_d}
        \frac{1}{|x|^{d-2}}
$$
则成立
$$DG(x)=\delta(x)$$
其中$\delta(x)$为Dirac delta-广义函数,Laplace算子$D$视为对广义函数求导。
\end{lemma}

这是偏微分方程当中众所周知的结果。若我们再令
$$G(x,y):=G(x-y)$$
则对任意$f\in C^\infty_c(\bbR^d)$以及任意$x\in\bbR^d$,成立
$$
  f(x)=\int_{\bbR^d}
         G(x,y)(Df)(y)
         \td y
$$

容易看出,算子
\begin{eqnarray*}
G:C^\infty_c(\bbR^d) &\to& C^\infty_c(\bbR^d)\\
f&\mapsto&\left(
            x\mapsto\int_{\bbR^d}G(x,y)f(y)\td y
          \right)
\end{eqnarray*}
是Laplace算子$D$的“逆算子”,
“矩阵元”$G(x,y)$类比于有限维情形的$Q^{ij}$.

%$$\mcalP=D^{-1}=D^{-1}_{x,y}$$
%(is called Green's function integral kernel)
%$$\Phi:\mcalE\to\mcalE$$
%operator, has kernel $\Phi(x,y)$ if
%$$\Phi(f)(x)=\int\td y\Phi(x,y)f(y)$$
%eg. $\Phi=\id\iff\Phi(x,y)=\delta(x-y)$ delta function.
%$$D^{-1}\to D^{-1}(x,y)=\frac{1}{|x-y|^{D-2}}$$
%singularity comes from infinite dimensional nature.
%(Ultra-Violet singularity紫外发散)

%%%%%%%%%2019.4.2第六周周二%%%%%%%%%%%%%%%%%%%%
%这周要布置作业,要交的作业,不然跟不上了%

%我们要花两节课来讲一个重要概念:重整化
%量子场论的例子,与Hochschild理论的联系
%\textbf{Renormalization(I)}
%Last time:
%$$\int_{\bbR^n}e^{(-\frac{1}{2}\sum x^iQ^{ij}x^j+I(x+a))/\hbar}
%\prod_{i=1}^n\frac{\td x^i}{\sqrt{2\pi\hbar}}$$
%$$=\frac{1}{\sqrt{\det Q}}e^{\frac{1}{2}\hbar Q^{ij}\pp{a^i}\pp{a^j}}e^{I(a)/\hbar}$$
%$$=\frac{1}{\sqrt{\det Q}}\exp
%\left(
%  \sum_{\Gamma}
%   \frac{\omg_\Gamma(P,a)}{Aut\Gamma}
%\right)$$
%Field theory example:
\vs

现在,我们考虑$d=4$的情形,
即$\mcalE=C^{\infty}_c(\bbR^4)$,作用量$\mcalS$为
%$\phi^4$-theory on $\bbR^4$
\begin{eqnarray*}
     \mcalS[\phi]
&=&
     \int_{\bbR^4}
       \frac{1}{2}\phi D\phi
      +\frac{\lmd}{4!}\int_{\bbR^4}\phi^4\\
&:=&
     Q(\phi)+\mcalI(\phi)
\end{eqnarray*}

此时,Green函数
$$G(x,y)=\frac{1}{2S_4}\frac{1}{|x-y|^2}
=\frac{15}{64\pi^2}\frac{1}{|x-y|^2}$$

我们定义“传播子”
$$\mcalP\,":="\,\frac{1}{2}\hbar D(x,y)\p_\phi^2$$
并且模仿有限维情形取定义路径积分:

\begin{definition}对于$\mcalE=C_c^\infty(\bbR^4)$,定义
$$
  \int_{\mcalE}
    e^{-\mcalS[\phi]/\hbar}
    [D\phi]
\,"="\,
  \frac{1}{\sqrt{\det Q}}
  \exp
  \left(
    \sum_{\Gamma\text{取遍}
          \atop\text{\textbf{连通的}多重有向图}}
      \frac{\omg_{\Gamma}(\mcalP,\mcalI)}
           {|\Aut(\Gamma)|}
  \right)
$$
其中“$\frac{1}{\sqrt{\det Q}}$”为某个非零常数,
$w_\Gma(\mcalP,\mcalI)$的含义我们将稍后举例说明。
\end{definition}

这里的$w_\Gma(\mcalP,\mcalI)$服从如下的\textbf{费曼规则}:

\begin{notation}对于$\phi\in C_c^\infty(\bbR^4)$,
以及连通的多重有向图$\Gma$,
记$\Gma$的顶点集$V=\{v_1,...,v_m\}$,边集为$E$,
顶点$v_i$的度数(入度$+$出度)为$d_i$,则定义
\begin{eqnarray*}
     w_\Gma(\mcalP,\mcalI)
=
     (\frac{1}{2}\hbar)^{|E|}
     \hbar^{-|V|}
     \prod_{i=1}^m
       \left(
         \frac{\lmd}{(4-d_i)!}
         \int_{\bbR^4}
           \phi^{4-d_i}(x_i)
           \td x_i
       \right)
     \prod_{e\in E}
       G(x_{s(e)},x_{t(e)})
\end{eqnarray*}
特别规定,如果图$\Gma$的某个顶点度数大于$4$,则$w_\Gma(\mcalP,\mcalI)=0$.
其中映射
\begin{eqnarray*}
s,t:E &\to& \{1,2,...,m\}
\end{eqnarray*}
将边$e\in E$分别映为其始点、终点的下指标。
\end{notation}
其实对于多重图$\Gma$也完全可以定义$w_\Gma(\mcalP,\mcalI)$,
边的定向对$w_\Gma(\mcalP,\mcalI)$的结果没有贡献。
不过我们需要注意,$G(x,y)\propto \frac{1}{|x-y|^2}$,
当$x,y$非常接近时$G$趋近于无穷,
正是所谓\textbf{紫外发散}(ultra-violet divergence)现象。
\index{ultra-violet divergence\kong 紫外发散}
于是对于积分$w_\Gma(\mcalP,\mcalI)$,我们需要考虑收敛性。

%Feymann rule...$G(x,y)$ satisfy
%$$D_xG(x,y)=\delta(x,y)$$
%(analogue $Q_{ij}Q^{jk}=\delta^{k}_i$)
%(Green's function)
%where $$D=-\sum_i\pp{x^i}\pp{x^i}$$
%is the Laplacian...
%in $\bbR^4$,
%$$G(x,y)=\frac{1}{|x-y|^2}$$
%(in general on $\bbR^d$, $G(x,y)\sim\frac{1}{|x-y|^{d-2}}$ when $d\geq 3$)
%Feyman graph formula:

\begin{example}
考虑多重图$\Gma$为如下:
$$
  \xymatrix{
     \phi          \ar@{--}[rd]
    &&&\phi        \ar@{--}[ld]
  \\
     \phi          \ar@{--}[r]
    &\bullet       \ar@{-}[r]
    &\bullet
    &\phi          \ar@{--}[l]
  \\
    \phi           \ar@{--}[ru]
    &&&\phi        \ar@{--}[lu]
  }
$$
该图有两个顶点、一条边(虚线无视之),则在相差常数倍意义下,
\begin{eqnarray*}
     w_\Gma(\mcalP,\mcalI)
&\propto&
     \lmd^2
     \int_{\bbR^4}\td x
     \int_{\bbR^4}\td y
       \phi^3(x)
       \phi^3(y)
       G(x,y)\\
&\propto&
     \lmd^2
     \int_{\bbR^4}\td x
     \int_{\bbR^4}
       \phi^3(x)
       \phi^3(x+y)
       \frac{1}{|y|^2}
       \td y
\end{eqnarray*}

容易看出积分$w_\Gma(\mcalP,\mcalI)$此时是收敛的。
\label{phi^4费曼规则-example}
\end{example}

\begin{example}
再考察另一个多重给有向图$\Gma$如下:
$$
  \xymatrix{
     \phi          \ar@{--}[rd]
    &&&\phi        \ar@{--}[ld]
  \\
    &\bullet       \ar@{-}@/^1.3pc/[r]
                   \ar@{-}@/_1.3pc/[r]
    &\bullet
    &
  \\
    \phi           \ar@{--}[ru]
    &&&\phi        \ar@{--}[lu]
  }
$$
该图$\Gma$有两个顶点、两条边。此时,在相差常数倍意义下,
\begin{eqnarray*}
     w_\Gma(\mcalP,\mcalI)
&\propto&
     \lmd^2
     \int_{\bbR^4\times\bbR^4}
       \phi^2(x)\phi^2(y)G^2(x,y)
       \td x\td y\\
&\propto&
     \lmd^2
     \int_{\bbR^4\times\bbR^4}
       \phi^2(x)\phi^2(y)
       \frac{1}{|x-y|^4}
       \td x\td y\\
&=&
     \lmd^2
     \int_{\bbR^4}
       \phi^2(x)
       \td x
     \int_{\bbR^4}
       \phi^2(x+y)
       \frac{1}{|y|^4}
       \td y
\end{eqnarray*}
容易看出此积分发散。
\label{phi^4费曼规则-发散example}
\end{example}

此外,如果图$\Gma$当中有闭路,即某条边的始点与终点相同,
则$w_\Gma(\mcalP,\mcalI)$也不是良好定义,这是更坏的情形。
产生如此\textbf{紫外发散}的原因是边太多(“能量太高”)。

对于一种特殊的情形——树状图,$w_\Gma$一定是良好定义的:

\begin{prop}记号承上,如果连通图$\Gma$是\textbf{树形图}
(即,顶点数$=$边数$+1$),则积分$w_\Gma(\mcalP,\mcalI)$收敛。
\end{prop}

\begin{proof}
对$\Gma$的顶点个数$|V|$归纳。
$|V|=2$的情形已经得证,见例子\ref{phi^4费曼规则-example}.

现在,给定$m\geq 2$,
若对于任何顶点个数为$m$的树状图$\Gma'$,
$w_{\Gma'}(\mcalP,\mcalI)$都收敛,
则任意给定树状图$\Gma$使得顶点个数为$m+1$.
由图论知识,必存在$\Gma$的度为$1$的顶点(树叶),
且从$\Gma$当中去掉该顶点及其连接的边之后所得的图仍为树形图。

记顶点$v_0$为树$\Gma$的一片树叶,
$\Gma$的其余$m$个顶点记为$v_1,v_2,...,v_m$,不妨$v_1$与$v_0$相邻,
记顶点$v_1$的度数为$d_1\geq 1$.
记$\Gma$去掉$v_0$及其连接的边所得到的图为$\Gma'$,则$\Gma'$
是以$v_1,...,v_n$为顶点的树。
由归纳假设,积分
$$
  w_{\Gma'}(\mcalP,\mcalI)
\propto
  \lmd^m
  \int_{(\bbR^4)^m}
    \phi^{d_1}(x_1)F(x_1,...,x_m)
    \td x_1\cdots\td x_m
$$
收敛(读者自行按费曼规则补全$F(x_1,...,x_m)$的定义)。从而
\begin{eqnarray*}
     w_\Gma(\mcalP,\mcalI)
&\propto&
     \lmd^{m+1}
     \int_{\bbR^4}
       \phi^3(x_0)
       \td x_0
     \int_{(\bbR^4)^{m}}
       G(x_0,x_1)
       \phi^{d_1-1}(x_1)
       F(x_1,...,x_n)
       \td x_1\cdots\td x_m\\
&\propto&
     \lmd^{m+1}
     \int_{(\bbR^4)^{m}}
       \phi^{d_1-1}(x_1)
       F(x_1,...,x_n)
       \td x_1\cdots\td x_m
     \int_{\bbR^4}
       \phi^3(x_1+x_0)
       \frac{1}{|x_0|^2}
       \td x_0
\end{eqnarray*}
从而易知收敛。(要利用$\phi$的紧支性)。
\end{proof}

%tree level ($\hbar^0$)
%%%%cy%%%%%%
%$$\mcalE=C_c^\infty(\bbR^4)$$
%(这个无穷维空间是有拓扑的)(这个空间上的广义函数?)
%这个例子讲完了,我们稍微再复杂一点,下面呢。。。
%in general, for a tree diagram (loop = 0)
%%%%%%%%another example%%%%%%
%\textbf{HW:}
%the Feymann integral is well-defined.
%One loop($\hbar^1$)
%%%%%%%%loop%%%%%%%
%(ill defined..)
%(Ultra-Violent divergent...)
%发散的原因是“两个点离得太近,能量太高”

一般来说,$w_\Gma(\mcalP,\mcalI)$是发散的。
在量子场论中采用\textbf{重整化}(renormalization)
\index{renormalization\kong 重整化}
的手段处理这种紫外发散。下面介绍重整化的想法。
首先从Laplace算子$D$的“逆算子”说起。注意到
%idea of renormalization:
%observe: Green function $G$ is the "inverse of Laplacian".
$$D^{-1}=\int_0^\infty e^{-tD}\td t$$
这里的$e^{-tD}$ 为\textbf{热算子}(heat operator),
\index{heat operator\kong 热算子}
我们有偏微分方程当中众所周知的如下结果:

\begin{lemma}(热核与热方程)

记$D$为$\bbR^d$中的Laplace算子。对于$t>0$,
定义$\bbR^d\times\bbR_+$上的函数
$$h_t(x)=\frac{1}{(4\pi t)^{d/2}}
           e^{-\frac{|x|^2}{4t}}$$
称该函数为$\bbR^d$上的\textbf{热核}(heat kernel),
\index{heat kernel\kong 热核}
则以下性质成立:

(1)热算子$e^{-tD}$可由热核表示:
对任意$\phi\in C_c^\infty(\bbR^d)$,
$$(e^{-tD}\phi)(x)
=\int_{\bbR^d}h_t(x-y)\phi(y)\td y$$

(2)热核满足如下\textbf{热方程}:
$$(\p_t+D)h_t(x)=0$$

(3)恒等逼近:
$$\lim_{t\to 0^+}h_t(x)=\delta(x)$$

(4)半群性质:对任意$t_1,t_2>0$,以及$x,y\in\bbR^4$,
$$\int_{\bbR^d}
    h_{t_1}(x-z)h_{t_2}(z-y)\td z
=   h_{t_1+t_2}(x-y)$$
\end{lemma}
%is represented by an integral kernel
%$h_t(x,y)$ such that%check:
%and we have
此外容易知道Green函数$G$满足
$$G(x,y)=\int_0^\infty h_t(x-y)\td t$$

\begin{notation}(正则化传播子)

引入\textbf{截断参数}(cut-off parameters)
$$0<\veps<L<+\infty$$%Define
定义$\bbR^d\times\bbR^d$上的光滑函数
$$P^L_{\veps}(x,y):=\int_\veps^L h_t(x,y)\td t$$
称之为\textbf{正则化传播子}(regularized propagator)。
\index{regularized propagator\kong 正则化传播子}
\end{notation}

这里的参数$\veps$称为\textbf{紫外截断}(ultra-violet cut-off),
$L$称为\textbf{红外截断}(infrared cut-off)。
\index{ultra-violet cut-off\kong 紫外截断}
\index{infrared cut-off\kong 红外截断}

重整化的基本想法是,将原先的传播子当中的$G(x,y)$替换为
$P^L_\veps(x,y)$,然后分析$\veps\to 0$,$L\to\infty$
时$w_{\Gma}(\mcalP_\veps^L,\mcalI)$的行为。
%idea: Replace the propagator $G$ by $P^L_\veps$
%and analyze the behavior of the graph as $\veps\to 0$ and $L\to\infty$.
%Eg:
%%%%%%%%cnm%%%%%%%%%%%%%

特别地,当$d=4$时,
$$
  P^L_\veps(x,y)
=
  \int_\veps^L
    h_t(x-y)\td t
=
  \int_\veps^L
    \frac{1}{(4\pi t)^2}
    e^{-\frac{|x-y|^2}{4t}}
    \td t
$$\vs

\begin{example}再看例子\ref{phi^4费曼规则-发散example},
即图$\Gma$为如下:
$$
  \xymatrix{
     \phi          \ar@{--}[rd]
    &&&\phi        \ar@{--}[ld]
  \\
    &\bullet       \ar@{-}@/^1.2pc/[r] ^{P^L_\veps}
                   \ar@{-}@/_1.2pc/[r] _{P^L_\veps}
    &\bullet
    &
  \\
    \phi           \ar@{--}[ru]
    &&&\phi        \ar@{--}[lu]
  }
$$
对于截断参数$\veps<L$,试计算$w_\Gma(\mcalP^L_\veps,\mcalI)$.
\end{example}

\begin{proof}[解]
直接计算之,
\begin{eqnarray*}
     w_\Gma(\mcalP^L_\veps,\mcalI)
&\propto&
     \lmd^2
     \int_{\bbR^4\times\bbR^4}
       \phi^2(x)\phi^2(y)
       P^L_\veps(x,y)^2
       \td x\td y\\
&=&
     \lmd^2
     \int_{\bbR^4}\td x
       \int_{\bbR^4}
         \phi^2(x)\phi^2(x+y)\td y
         \int_{\veps}^L
           \int_\veps^L
             \frac{\td t_1}{(4\pi t_1)^2}
             \frac{\td t_2}{(4\pi t_2)^2}
             e^{-\frac{|y|^2}{4t_1}}
             e^{-\frac{|y|^2}{4t_2}}
\end{eqnarray*}

使用Taylor展开,令
$$
  \phi^2(x)\phi^2(x+y)
=:\phi^4(x)+R(x,y)
$$
由于函数$\Phi$是紧支的,从而存在常数$C>0$,
使得对任意$x,y\in\bbR^4$,成立
$$|R(x,y)|\leq C|y|$$
从而对每个$x\in\bbR^4$,
\begin{eqnarray*}
     \limsup_{\veps\to 0^+}
       \int_{\bbR^4}
         |R(x,y)|\td y
         \int_{\veps}^\infty
           \int_\veps^\infty
             \frac{\td t_1}{(4\pi t_1)^2}
             \frac{\td t_2}{(4\pi t_2)^2}
             e^{-\frac{|y|^2}{4t_1}}
             e^{-\frac{|y|^2}{4t_2}}
 \leq
     C'\int_{\bbR^4}
         \frac{|y|}{|y|^4}\td y<+\infty
\end{eqnarray*}
因此有
\begin{eqnarray*}
     w_\Gma(\mcalP^L_\veps,\mcalI)
&\propto&
     \lmd^2
     \int_{\bbR^4}\phi^4(x)\td x
       \int_{\bbR^4}\td y
         \int_{\veps}^L
           \int_\veps^L
             \frac{\td t_1}{(4\pi t_1)^2}
             \frac{\td t_2}{(4\pi t_2)^2}
             e^{-\frac{|y|^2}{4t_1}}
             e^{-\frac{|y|^2}{4t_2}}\\
& &
    +(\text{$\veps\to 0$时的收敛项})\\
&=&
     \lmd^2\int_{\bbR^4}
       \phi^4(x)\td x
         \int_\veps^L
           \int_\veps^L
             \frac{\td t_1}{(4\pi t_1)^2}
             \frac{\td t_2}{(4\pi t_2)^2}
               \int_{\bbR^4}
                 e^{-\frac{|y|^2}{4t_1}-\frac{|y|^2}{4t_2}}
                 \td y\\
& &
    +(\text{$\veps\to 0$时的收敛项})\\
&=&
     \lmd^2\int_{\bbR^4}
       \phi^4(x)\td x
         \int_\veps^L
           \int_\veps^L
             \frac{\td t_1}{(4\pi t_1)^2}
             \frac{\td t_2}{(4\pi t_2)^2}
             \frac{(2\pi)^2}{(\frac{1}{2t_1}+\frac{1}{2t_2})^2}\\
& &
    +(\text{$\veps\to 0$时的收敛项})\\
&=&
     \frac{\lmd^2}{(4\pi)^2}
     \int_{\bbR^4}
       \phi^4(x)\td x
       \int_\veps^L
         \int_\veps^L
           \frac{\td t_1\td t_2}{(t_1+t_2)^2}\\
& &
    +(\text{$\veps\to 0$时的收敛项})\\
&=&
     -\frac{\lmd^2\log\veps}{(4\pi)^2}
     \int_{\bbR^4}\phi(x)^4\td x\\
& &
    +(\text{$\veps\to 0$时的收敛项})\\
\end{eqnarray*}
\end{proof}

不过要注意$L\to \infty$的时候积分
$\int_\veps^L
         \int_\veps^L
           \frac{\td t_1\td t_2}{(t_1+t_2)^2}$
也是发散的,即“\textbf{红外发散}”。
为处理$\veps\to 0$的发散,我们对相互作用能进行\textbf{量子修正}:
%idea: $\lmd\to\lmd(\veps)$ depend on $\veps$.
%Consider add the following to $\mcalS$:
令
$$\mcalI^{CT}(\veps):=
\frac{\hbar\lmd^2\log\veps}{(4\pi)^2}\int_{\bbR^4}\phi^4\td x$$
%(CT: counter term...用来抵消发散)

\begin{eqnarray*}
\mcalS&\mapsto&\mcalS+\mcalI^{CT}(\veps)\\
&=&\frac{1}{2}\int\phi D\phi+\frac{\lmd}{4!}\phi^4\td x+
\frac{\hbar\lmd^2\log\veps}{(4\pi)^2}\int\phi^4\td x
\end{eqnarray*}

%Feymann rule
%%%%%%%%量子修正%%%%%%%%%%

一般地,在量子物理中,关于$\veps\to 0$的量子修正,有如下定理:

\begin{thm}记号承上,则存在%there exists
关于$\veps$的函数
$$\lmd(\veps)=\lmd+\hbar\frac{\lmd^2\log\veps}{(4\pi)^2}+\cdots
=\lmd+\sum_{g\geq 1}\hbar^gG_g(\lmd,\veps)$$
%(dependents on $\lmd,\veps$, singular as $\veps\to 0$.)
%such that Let
使得极限
$$
  \lim_{\veps\to 0}\sum_{\Gamma\text{取遍}\atop\text{多重有向图}}
  \frac{\omg_P(P_\veps^L,\mcalI^\veps)}
       {|\Aut(\Gamma)|}
$$
存在,其中
$$\mcalI^{\veps}:=\frac{\lmd(\veps)}{4!}\int_{\bbR^4}\phi(x)^4\td x$$
\end{thm}
\begin{proof}
比较困难,从略。
\end{proof}
%这个定理挺难证。。。不证了。。。

\section{dGBV代数与Master方程}
%%%%%%%%%%%2019.4.8第七周周一%%%%%%%%%%%%%%%%%
%%%%%%%%%%%%%%这周继续量子场论%%%%%%%%%%%%%%%%
%\textbf{Homotopic Renormalization}
%Recall:
%$$\int\rightsquigarrow\textbf{BV homology}$$
%scalar field theory in $\bbR^4$.
%$\infty$-dimension$\rightsquigarrow$ UV-divergence.

我们讨论一些代数结构。

\begin{definition}(dGBV代数)
(differential Gerstenhaber-Batalin-Vilkoviskty algebra)
%A differential BV (or DGBV) is a triple $(\mcalA,Q,\yc)$,s.t.

\textbf{dGBV代数}是指满足如下性质的三元组$(\mcalA,Q,\yc)$:

(1)$\mcalA=\bigoplus\limits_{k\in\bbZ}\mcalA^k$
为分次交换代数,配以乘法$\cdot$;

(2)$Q:\mcalA\to\mcalA$ 为$1$阶超导子,且$Q^2=0$;

(3)$\yc:\mcalA\to\mcalA$满足$\yc(\mcalA^k)\subseteq\mcalA^{k+1}$,
$\yc^2=0$,并且对任意$\afa\in\mcalA$为齐次元,算子
\begin{eqnarray*}
\delta_\afa:\mcalA&\to&\mcalA\\
\beta&\mapsto&
  (-1)^{|\afa|}
  \left(
    \yc(\afa\beta)-\yc(\afa)\beta-(-1)^{|\afa|}\afa\yc(\beta)
  \right)
\end{eqnarray*}
为$|\afa|+1$阶超导子。
%is a "2-nd differential operator" (BV operator),such that
%$\deg\yc=1,\yc^2=0$.

(4)$[Q,\yc]:=Q\circ\yc+(-1)^{1+1}\yc\circ Q=0$
\index{dGBV代数}
\label{dGBV代数-def}
\end{definition}

这里的$Q$称为微分算子;而$\yc$为\textbf{BV算子},
\index{BV算子}
是之前(定义\ref{BV算子-def})的代数抽象。

\begin{rem}(GBV代数)

先不看$Q$,若二元组$(\mcalA,\yc)$满足上述定义中的(1)(3),
则称$(\mcalA,\yc)$为\textbf{GBV代数}。
\index{GBV algebra\kong Gerstenhaber-Batalin-Vilkoviskty代数}
\end{rem}

\begin{thm}(GBV代数诱导Gerstenhaber代数)

设$(\mcalA,\yc)$为GBV代数,定义BV括号
\begin{eqnarray*}
\{,\}:\mcalA\times\mcalA &\to& \mcalA\\
\{\afa,\beta\}&:=&
(-1)^{|\afa|}
\left(
\yc(\afa\beta)-(\yc\afa)\beta-(-1)^\afa\afa\yc\beta
\right)
\end{eqnarray*}
则$\{\mcalA,\{,\}\}$满足定义\ref{Gerstenhaber代数-def}
的(1)(2)(3)条。
\label{GBV诱导Gerstenhaber-thm}
\end{thm}

这样的代数$(\mcalA,\{,\})$称为\textbf{BV代数},
\index{BV代数}
它与Gerstenhaber代数非常相似,区别在于这里是
$$\{,\}:\mcalA^p\times\mcalA^q\to\mcalA^{p+q+1}$$
(而不是映到$\mcalA^{p+q-1}$)

%This bracket ${,}$ has degree $1$, satisfies:
%(1)graded Jacobi identity
%(2)Leibniz rule...
%(BV-algebra...)(similar as Gerstenhaber algebra)

\begin{proof}
$\{,\}$的双线性性显然,超反交换性直接验证。
超莱布尼茨法则就是定义\ref{dGBV代数-def}的(3);
而超雅可比恒等式
$$
  \{f,\{g,h\}\}=\{\{f,g\},h\}+(-1)^{(f+1)(g+1)}\{g,\{f,h\}\}
$$
比较复杂,我们稍后证明。
\end{proof}

\begin{lemma}\label{BV算子引理-lem}
设$(\mcalA,\yc)$为GBV代数,
则对$\mcalA$中的齐次元$f,g$,成立
$$\yc\{f,g\}=\{\yc f,g\}+(-1)^{f+1}\{f,\yc g\}$$
\end{lemma}

\begin{proof}利用$\yc^2=0$,以及
$\yc(fg)=(-1)^f\left(\{f,g\}+f\yc(g)+(-1)^f\yc(f)g\right)$直接验证之:
\begin{eqnarray*}
     \yc\{f,g\}
&=&
     (-1)^f\yc
     \left(
       \yc(fg)-\yc(f)g-(-1)^ff\yc(g)
     \right)\\
&=&
     (-1)^{f+1}
     \left(
       \yc(f)g+(-1)^ff\yc(g)
     \right)\\
&=&
     (-1)^{f+1}
     \left(
       (-1)^{f+1}
       \left(
         \{\yc f,g\}+
         {\color{blue}
           \yc(f)\yc(g)
         }
       \right)
      +(-1)^{f+f}
       \left(
         \{f,\yc g\}+(-1)^f
         {\color{blue}
           \yc(f)\yc(g)
         }
       \right)
     \right)\\
&=&
     \{\yc f,g\}+(-1)^{f+1}\{f,\yc g\}
\end{eqnarray*}
\end{proof}

现在我们给出定理\ref{GBV诱导Gerstenhaber-thm}剩余部分的证明。

\begin{proof}[超雅可比恒等式的证明]
需要反复使用公式
\begin{eqnarray*}
\yc(fg)&=&(-1)^f\left(\{f,g\}+f\yc(g)+(-1)^f\yc(f)g\right)\\
\yc\{f,g\}&=&\{\yc f,g\}+(-1)^{f+1}\{f,\yc g\}
\end{eqnarray*}
以及超莱布尼茨法则。

首先考察
\begin{eqnarray}
     (-1)^g\{f,\{g,h\}\}\nonumber
&=&
     \{f,\yc(gh)-\yc(g)h+(-1)^{g+1}g\yc(h)\}\nonumber\\
&=&
     {\color{blue}
       \{f,\yc(gh)\}
     }
    -\{f,\yc(g)h\}
    +(-1)^{g+1}\{f,g\yc(h)\}
\end{eqnarray}

而由引理\ref{BV算子引理-lem}可知
$$\{f,\yc(gh)\}=(-1)^{f+1}\yc\{f,gh\}+(-1)^f\{\yc(f),gh\}$$
代入(4.1)式蓝色部分,得到
\begin{eqnarray}
     (-1)^g\{f,\{g,h\}\}\nonumber
&=&
     (-1)^{f+1}
     {\color{red}
      \yc\{f,gh\}
     }
    +(-1)^f\{\yc(f),gh\}\nonumber\\
& &
    -\{f,\yc(g)h\}
    +(-1)^{g+1}\{f,g\yc(h)\}
\end{eqnarray}

再看上式中红色的项,先用超莱布尼茨法则展开,
\begin{eqnarray*}
     \yc\{f,gh\}
&=&
     \yc
     \left(
       \{f,g\}k+(-1)^{(f+1)g}g\{f,h\}
     \right)\\
&=&
     (-1)^{f+g+1}
     \left(
       \{\{f,g\},h\}
      +\{f,g\}\yc(h)
      +(-1)^{f+g+1}\yc\{f,g\}h
     \right)\\
& &
    +(-1)^{(f+1)g+g}
     \left(
       \{g,\{f,h\}\}
      +g\yc\{f,h\}
      +(-1)^g\yc(g)\{f,h\}
     \right)\\
&=&
     {\color{purple}
       (-1)^{f+g+1}\{\{f,g\},h\}
      +(-1)^{fg}\{g,\{f,h\}\}
     }\\
& &
    +(-1)^{f+g+1}\{f,g\}\yc h
    +\left(
       \{\yc f,g\}+(-1)^{f+1}\{f,\yc g\}
     \right)h\\
& &
    +(-1)^{fg}g
     \left(
       \{\yc f,h\}+(-1)^{f+1}\{f,\yc h\}
     \right)
    +(-1)^{(f+1)g}\yc(g)\{f,\yc h\}
\end{eqnarray*}
代入(4.2)即可。
注意深红色部分出现了想要的$\{\{f,g\},h\}$与$\{g,\{f,h\}\}$;
接下来将(4.2)等号右边最后两项用超莱布尼茨法则展开,
抵消其余的项。细节从略。
\end{proof}

dGBV代数的典型例子为如下:

\begin{example}%$M$ is a manifold with volume form $\Omg$,
设$M$为光滑定向流形,$\Omg$为$M$的一个体积形式,则令
\
$$
\left\{
  \begin{array}{rcl}
    \mcalA&:=&(\PV\updot(M),\wedge)\\
    \yc   &:=&\text{定义\ref{BV算子-def}中的BV算子}\\
    Q     &:=&0
  \end{array}
\right.
$$
那么$(\mcalA,\yc,Q)$为dGBV代数。
\end{example}
我们早已熟知,不必再证。
回顾在局部坐标下,若$\Omg=e^{f(x)}\td^nx$,则
$$\yc=\pp{x^i}\pp{\theta_i}-\pfrac{f}{x^i}\pp{\theta_i}$$
这个dGBV代数比较平凡,毕竟$Q=0$.

对于一般的dGBV,有以下关于$Q$的等式:

\begin{lemma}
设$(\mcalA,\yc,Q)$为dGBV,则对任意齐次元$f,g\in\mcalA$,成立
$$Q\{f,g\}=\{Qf,g\}+(-1)^{f+1}\{f,Qg\}$$
\label{dGBV的Q引理-lemma}
\end{lemma}

\begin{proof}
要利用$Q$的超导子性,以及相容性$Q\circ\yc+\yc\circ Q=0$.直接验证如下:
\begin{eqnarray*}
     Q\{f,g\}
&=&
     (-1)^fQ
     \left(
       \yc(fg)-\yc(f)g-(-1)^ff\yc(g)
     \right)\\
&=&
     (-1)^{f+1}
     \yc
     \left(
       Q(f)g+(-1)^ffQ(g)
     \right)
    -(-1)^fQ\left(\yc(f)g\right)
    -Q\left(f\yc g\right)\\
&=&
     (-1)^{f+1}\yc\left(Q(f)g\right)
    -\yc\left(fQ(g)\right)\\
& &
    -(-1)^f
     \left(
       Q(\yc(f)g)
      +(-1)^{f+1}\yc(f)Q(g)
     \right)
    -\left(
       Q(f)\yc(g)+(-1)^ffQ(\yc g)
     \right)\\
&=&
     \left(
       \{Qf,g\}
      +{\color{blue}Q(f)\yc(g)}
      +(-1)^{f+1}{\color{red}\yc(Qf)g}
     \right)\\
& &
    -(-1)^f
     \left(
       \{f,Qg\}
      +{\color{green}f\yc(Qg)}
      +(-1)^f{\color{gray}\yc(f)Q(g)}
     \right)\\
& &
    +(-1)^f{\color{red}\yc(Qf)g}
    +{\color{gray}\yc(f)Q(g)}\\
& &
    -{\color{blue}Q(f)\yc(g)}
    +(-1)^f{\color{green}f\yc(Qg)}\\
&=&
     \{Qf,g\}+(-1)^{f+1}\{f,Qg\}
\end{eqnarray*}
\end{proof}

\begin{definition}(经典Master方程)
%Let be a DGBV. $I_0\in\mcalA_0$(i.e. degree$=0$) is said to%satisfy (CME), if
\index{classical master equation\kong 经典Master方程}

设$(\mcalA,Q,\yc)$为dGBV代数,对于$\mcalA$中的$0$次元$I_0$,
称$I_0$满足\textbf{经典Msater方程}
(classical master equation,简称\textbf{CME}),若

$$QI_0+\frac{1}{2}\{I_0,I_0\}=0$$
\end{definition}

在微分几何中,此方程很像联络的平坦性。
以后会知道,它与\textbf{形变理论}联系密切。

\begin{lemma}
设$(\mcalA,\yc,Q)$为dGBV代数,
$I_0\in\mcalA^0$满足经典Master方程,则算子
\begin{eqnarray*}
Q_{I_0}:\mcalA&\to&\mcalA\\
f&\mapsto&Qf+\{I_0,f\}
\end{eqnarray*}
为次数为$1$的超导子,并且
$$Q_{I_0}^2=0$$
\end{lemma}

\begin{proof}
注意$Q$与$\{I_0,-\}$都是次数为$1$的超导子
(后者等价于超雅可比恒等式),从而$Q_{I_0}=Q+\{I_0,-\}$
也是次数为$1$的超导子。再看$Q_{I_0}^2$,
对任意$f\in\mcalA$,首先由超雅可比,有
\begin{eqnarray*}
\{I_0,\{I_0,f\}\}=
\{\{I_0,I_0\},f\}+(-1)^{1\times 1}\{I_0,\{I_0,f\}\}
\end{eqnarray*}
从而
$$\{I_0,\{I_0,f\}\}=\frac{1}{2}\{\{I_0,I_0\},f\}$$
接下来,注意使用引理\ref{dGBV的Q引理-lemma},以及
$I_0$满足经典Master方程,从而
\begin{eqnarray*}
     Q_{I_0}^2(f)
&=&
     Q\left(Qf+\{I_0,f\}\right)
    +\{I_0,Qf+\{I_0,f\}\}\\
&=&
     Q\{I_0,f\}+\{I_0,Qf\}+\{I_0,\{I_0,f\}\}\\
&=&
     \{QI_0,f\}+\frac{1}{2}\{\{I_0,I_0\},f\}\\
&=&
     \{QI_0+\frac{1}{2}\{I_0,I_0\},f\}\\
&=&
     0
\end{eqnarray*}
\end{proof}

%$$Q+\{I_0,-\}\curvearrowright\mcalA$$
%this operator acts as a differential, i.e.
%$$(Q+\{I_0,-\})^2=0$$
%(check,need to use Jacobi identity, HW)

\begin{rem}
(1)与辛几何的泊松括号类比,注意$\{I_0,I_0\}$未必等于零。
这里的BV括号$\{,\}$,也称作“\textbf{奇泊松结构}”(odd Poisson structure)。
\index{odd Poisson structure\kong 奇泊松结构}

(2)在物理学中,任意“规范对称”的作用量$\mcalI_0$
都给出一个经典Master方程的解。
\end{rem}

%This bracket is also called "odd Poisson structure"...
%\textbf{Fact}:
%any action functional with gauge symmetry leads to a solution of CME.
%(What the fuck is "guage symmetry"?)

对于满足经典Master方程的$I_0\in\mcalA^0$,由于$Q_{I_0}^2=0$,
从而可以考虑其上同调:

\begin{definition}(经典BRST)

对于dGBV代数$(\mcalA,\yc,Q)$,
以及满足经典Master方程的$I_0\in\mcalA^0$,记
$$\Obs^{\cl}_{I_0}(\mcalA):=H^0(\mcalA,Q_{I_0})$$
称为\textbf{经典BRST}(Becchi-Rouet-Stora-Tyutin)。
\end{definition}
%\textbf{classical observable}
%(is called classical BRST)

\begin{example}(来自奇点理论的例子)
%(from singularity theory)

考虑dGBV代数$(\PV_{\bbC^n}\updot,\yc,0)$,即
$$
  \left\{
    \begin{array}{rcl}
      \mcalA&=&\PV\updot_{\bbC^n}
               \cong\bbC[z^1,...,z^n;\theta_1,...,\theta_n]\\
      \yc   &=&\sum\limits_{i=1}^n\pp{z^i}\pp{\theta_i}\\
      Q     &=&0
    \end{array}
  \right.
$$
其中$\mcalA$的分次由$\deg\theta_i=-1$给出。
设多项式$f\in\bbC[z^1,...,z^n]=\mcalA^0$满足
$$\Crit(f):=\{z\in\bbC^n|\p_if(z)=0\,,\forall 1\leq i\leq n\}=\{0\}$$
即$0\in\bbC^n$为$f$(唯一的)孤立奇点,则$f$满足经典Master方程,
并且其经典BRST为
$$\Obs^{\cl}_f=\bbC[z^1,...,z^n]\big/(\p_1f,...,\p_nf)$$
%where $\deg\theta_i=-1$(super variable).
%$f(z)$ has a isolated singularity  at $z=0$,
%then $\Crit(f)=\{0\}$.
%and $f$ solves CME: $\{f,f\}=0$(trivial...),and also satisfies QMF:
%$\yc e^{f/\hbar}=0$.
\end{example}
\begin{proof}
注意$Q=0$,此时的经典Master方程即为$\{I_0,I_0\}=0$.
在本例中,BV括号$\{,\}$就是(负的)Schouten-Nijenhuis括号,
从而$\{f,f\}=0$是平凡的。我们看它的BRST,
$$
  \Obs^{\cl}_f=
    \frac{\bbC[z^1,...,z^n]}
         {\im\left(
               \{f,-\}:\PV^1_{\bbC^n}\to\PV^0_{\bbC^n}
             \right)
         }
$$
而对于切向量场$X\in\mcalA^{-1}$,若
$$X=X^i\theta_i$$
则直接计算得
$$\{f,X\}=\yc(fX)-f\yc(X)=\pfrac{f}{z^i}X^i$$
从而证毕。
\end{proof}

在奇点理论中,环$\bbC[z^1,...,z^n]$的理想
$J_f:=(\pfrac{f}{z^1},...,\pfrac{f}{z^n})$称为\textbf{Jacobi理想},
$$\mcalA_f:=\bbC[z^1,...,z^n]\big/J_f\quad(=\Obs^{cl}_f)$$
称为$f$的\textbf{Milnor环}。作为复线性空间,称
$$\mu:=\dim_{\bbC}\mcalA_f$$
为\textbf{Milnor数}。由交换代数(Hilbert零点定理)可知,
当$0$为$f$的孤立奇点时,$\mu<+\infty$.

%then , the classical observable
%$$Obs^{cl}=H^0(\mcalA,\{f,-\})=
%\text{Koszul resolution of $\Crit(f)$}
%=\bbC[z^i]\big/(\p_if)$$
%is called "Milnor ring",parametrizing deformations of $f$.
%This Milnor ring is an Artinian ring(so, of finite length),
%$$\mu:=\dim_\bbC\bbC[z^i]\big/(\p_if)$$
%is called Milnor ring..


%(quantum version)

现在考虑量子版本。

\begin{definition}(量子Master方程)

设$(\mcalA,\yc,Q)$为dGBV代数,以及$I\in\mcalA^0\fps{\hbar}$,
称$I$满足\textbf{量子Master方程}(Quantum Master Equation,简称\textbf{QME}),
\index{Quantum Master Equation\kong 量子Master方程}若
$$QI+\hbar\yc I+\frac{1}{2}\{I,I\}=0$$
\end{definition}
量子Master方程其实是无穷多个方程。
记$I=\sum\limits_{k=0}^\infty I_k\hbar^k$,比较$\hbar^k$的系数,
$$
  \left\{
    \begin{array}{l}
      QI_0+\frac{1}{2}\{I_0,I_0\}=0\\
      QI_1+\yc I_0+\frac{1}{2}
        \left(
          \{I_0,I_1\}+\{I_1,I_0\}
        \right)=0\\
      QI_2+\yc I_1+\frac{1}{2}
        \left(
          \{I_0,I_2\}+\{I_1,I_1\}+\{I_2,I_0\}
        \right)=0\\
      \cdots\cdots
    \end{array}
  \right.
$$
其中第一个方程即为经典Master方程。
容易看出,经典Master方程是量子Master方程“$\hbar\to 0$”的“极限”。

\begin{lemma}
设$(\mcalA,\yc,Q)$为dGBV代数,则对于任意$I\in\mcalA^0\fps{\hbar}$,
以及任意$k\geq 0$,成立
$$\yc(I^k)=k\yc(I)I^{k-1}+\frac{k(k-1)}{2}\{I,I\}I^{k-2}$$
\end{lemma}

\begin{proof}
其实不妨$I\in\mcalA^0$(并没有用到$\hbar$的信息).
注意$I$与$\mcalA$中任何元素交换。首先由超莱布尼茨法则易知
$$\{I,I^{k-1}\}=(k-1)\{I,I\}I^{k-2}$$
于是有
\begin{eqnarray*}
     \yc(I^k)&=&\yc(II^{k-1})
 =
     \{I,I^{k-1}\}+\yc(I)I^{k-1}
    +I\yc(I^{k-1})\\
&=&
     (k-1)\{I,I\}I^{k-2}
    +I^{k-1}\yc(I)
    +I\yc(I^{k-1})
\end{eqnarray*}
于是
$$\yc(I^k)-I\yc(I^{k-1})=(k-1)\{I,I\}I^{k-2}+I^{k-1}\yc(I)$$
于是易知对任意$1\leq s\leq k$,都有
$$I^{s-1}\yc(I^{k-s+1})-I^s\yc(I^{k-s})
=(k-s)\{I,I\}I^{k-2}+I^{k-1}\yc(I)$$
将上式的$s$从$1$到$k$求和即可。
\end{proof}

\begin{prop}(量子Master方程的等价形式)

对于dGBV代数$(\mcalA,\yc,Q)$,则量子Master方程等价于
$$(Q+\hbar\yc)e^{I/\hbar}=0$$
\end{prop}

\begin{proof}直接验证,注意\textbf{不必}
将$I$展开为$I=\sum\nolimits_{k\geq 0}I_k\hbar^k$.只需注意
\begin{eqnarray*}
     Qe^{I/\hbar}
 =
     Q\sum_{k=0}^\infty
        \frac{I^k}
             {k!\hbar^k}
 =
     \sum_{k=0}^\infty
       \frac{kQ(I)I^{k-1}}
            {k!\hbar^k}
 =
     \frac{1}{\hbar}
     Q(I)e^{I/\hbar}
\end{eqnarray*}
以及由之前引理,
\begin{eqnarray*}
     \yc e^{I/\hbar}
&=&
     \sum_{k=0}^\infty
       \frac{\yc(I^k)}{\hbar^kk!}\\
&=&
     \sum_{k=0}^\infty
       \frac{
              k\yc(I)I^{k-1}
             +\frac{1}{2}k(k-1)\{I,I\}I^{k-2}
            }
            {\hbar^kk!}\\
&=&
     \frac{\yc(I)}{\hbar}
     \sum_{k=1}^\infty
       \frac{I^{k-1}}{(k-1)!\hbar^{k-1}}
    +\frac{\{I,I\}}{2\hbar^2}
     \sum_{k=2}^\infty
       \frac{I^{k-2}}{(k-2)!\hbar^{k-2}}\\
&=&
     \left(
       \frac{\yc(I)}{\hbar}
      +\frac{\{I,I\}}{2\hbar^2}
     \right)
     e^{I/\hbar}
\end{eqnarray*}

综上,有
$$
  (Q+\hbar\yc)e^{I/\hbar}
=\frac{1}{\hbar}
 \left(
   Q(I)+\hbar\yc(I)+\frac{1}{2}\{I,I\}
 \right)
 e^{I/\hbar}
$$
可见上式为$0$当且仅当$I$满足量子Master方程。
\end{proof}

注意$(Q+\hbar\yc)^2=0$,于是“$e^{I/\hbar}$”可以视为
关于“微分”$(Q+\hbar\yc)$的“闭微分形式”,
如果$I$满足量子Master方程。

%i.e. , $e^{I/\hbar}$ is the analogue of "closed differential form".
%$$\rightsquigarrow\int e^{I/\hbar}\quad
%\text{in Physis: Quantum gauge symmetry}$$
%(某种意义下的不变的测度)
%$$\xra{?}\int \afa e^{I/\hbar}=:\langle\afa\rangle$$
%co-relator function
%\textbf{quantum observable}

\begin{prop}
对于dGBV代数$(\mcalA,\yc,Q)$,
以及满足量子Master方程的$I\in\mcalA^0\fps{\hbar}$,
则算子
\begin{eqnarray*}
Q_I:\mcalA\fps{\hbar}&\to&\mcalA\fps{\hbar}\\
f&\mapsto&(Q+\hbar\yc)f+\{I,f\}
\end{eqnarray*}
为一阶超导子,且$Q_I^2=0$;此外,对于$\afa\in\mcalA^0\fps{\hbar}$,
$Q_I(\afa)=0$当且仅当
$$(Q+\hbar\yc)(\afa e^{I/\hbar})=0$$
\end{prop}

\begin{proof}显然$Q_I$为一阶超导子。
注意$(Q+\hbar\yc)^2=0$,以及对任意$f\in\mcalA\fps{\hbar}$成立
$$\{I,\{I,f\}\}=\frac{1}{2}\{\{I,I\},f\}$$
因此
\begin{eqnarray*}
     Q_I^2(f)
&=&
     \left(
       Q+\yc\hbar+\{I,-\}
     \right)^2f\\
&=&
     (Q+\hbar\yc)^2f+\{I,\{I,f\}\}
    +(Q+\hbar\yc)\{I,f\}
    +\{I,(Q+\hbar\yc)f\}\\
&=&
     \{I,\{I,f\}\}
    +\{(Q+\hbar\yc)I,f\}\\
&=&
     \{(Q+\hbar\yc)I+\frac{1}{2}\{I,I\},f\}\\
&=&
     0
\end{eqnarray*}
再注意
\begin{eqnarray*}
     (Q+\hbar\yc)(\afa e^{I/\hbar})
&=&
     ((Q+\hbar\yc)\afa)e^{I/\hbar}
    +\afa(Q+\hbar\yc)e^{I/\hbar}
    +\hbar\{\afa,e^{I/\hbar}\}\\
&=&
     \left(
       (Q+\hbar\yc)\afa+\{I,\afa\}
     \right)e^{I/\hbar}
\end{eqnarray*}
因此得证。
\end{proof}

与经典情形类似,对于满足量子Master方程的$I$,定义\textbf{量子BRST}
$$\Obs^q_I:=H^0(\mcalA\fps{\hbar},Q_I)$$

%if $\afa\in Obs^q$, then
%$$(Q+\hbar\yc)\afa+\{I,\afa\}=0
%\iff\left
%(Q+\hbar\yc)(\afa e^{I/\hbar})=0
%\right)$$
%(i.e. $\afa e^{I/\hbar}$ is a "closed form")

\section{$(-1)$-辛几何(待补)}

\textbf{$(-1)$-symplectic geometry(BV-formalism)}

Toy model:
Consider $(V,Q,\omg)$,where $(V,Q)$ is a finite dimensional complex,
$$Q:V\to V,Q^2=0$$
$$\omg:\wedgeform{2}V\to\bbC$$
symplectic pair(non-degenerate) of degree $-1$.
And $\omg$ is compactible with $Q$:
$$Q(\omg)=0\iff
\omg(Q(-),-)+\omg(-,Q(-))=0$$

$(V,Q,\omg)$ is called degree $(-1)$-symplectic space.

We can construct a DGBV algebra
$$\mcalA=\mcalO(v)=\prod_{n\geq 0}\Sym^n(V^*)$$
$Q$ is the induced derivation on $\mcalA$.
(Eg:
$$(QI)(\afa_1\ten\cdots\ten\afa_n)=
\sum_i\pm I(\afa_1,...,\hat{\afa_i},...,\afa_n)$$
)

$\omg:V^*\cong V[1]$, example:
$\omg=\td x^i\wedge\td\theta_i$.
$$\omg\in\wedgeform{2}(V^*)\longleftarrow \wedgeform{2}(V[1])=\Sym^2(V)[2]$$
$K=\omg^{-1}\in\Sym^2(V),\deg(K)=1$ is the Poisson kernel of $\omg$.

locally,
$$\omg=\td x^i\wedge\td\theta_i$$
$$K=\pp{x^i}\ten\pp{\theta_i}+\pp{\theta_i}\ten\pp{x^i}$$

We define the BV operator
$$\yc_K:\mcalO(V)\to\mcalO(V)$$
$$\Sym^n(V^*)\to\Sym^{n-2}(V^*)$$
by contracting with $K\in\Sym^2(V)$.

Expicitly
$$\yc_K(\afa_1\ten...\ten\afa_n)
=\sum_{i,j}\pm\langle K,\afa_i,\afa_j\rangle\afa_1\ten...\hat{\afa_i}...\hat{\afa_j}...\ten\afa_n$$

\begin{prop}(HW)
$$(\mcalA=\mcalO(V),Q,\yc_K)$$
is a DGBV algebra.
\end{prop}

eg: $\mcalA=\bbC[x^i,\theta_i]$,$\omg=\td x^i\wedge\td\theta_i$, and
$$\yc_K=\pp{x^i}\pp{\theta_i}$$

\textbf{QFT case}
$$V,Q\rightsquigarrow
\mcalE\updot=\Gma(X,E\updot)$$
smooth section of some graded bundle $E\updot$.
$$\cdots\xra{Q}\mcalE^{-1}\xra{Q}\mcalE^0\xra{Q}\mcalE^1\to\cdots$$
$Q$ is a differential operator, $Q^2=0$. ($\infty$-dimension)

$$\omg\rightsquigarrow\text{local $(-1)$-symplectic pair}$$
$$\omg(\afa,\beta)=\int_X\langle\afa,\beta\rangle,\quad\afa,\beta\in\mcalE$$

$K=\omg^{-1}=\delta\text{-function supported on} X\subseteq X\times X$.
(UV-divergence)

$V^*\to\mcalE^*$ is distribution...

$\yc_K\curvearrowright$ Distribution is ill-defined!

%%%%%%%%%%%2019.4.9第七周周二%%%%%%%%%%%%%%%%%%%%%
%还要将讲量子场论,今天是正儿八经的干货

\section{同伦重整化(待补)}
\textbf{Homotopic renormailzation II}
$(-1)$-symplectic geometry.

Toy model$(V,Q,\omg)$,$(V,Q)$ finite dimensional complex.
$\omg:\wedgeform{2}V\to\bbC$
non-degenerated, of degree $-1$.

$K=\omg^{-1}$ Poisson kernel, $\deg K=1$,
$$K\in\Sym^2(V)\quad (!!!)$$
so,
$$\yc_K\curvearrowright\mcalO(V)=\prod_{n}\Sym^n(V^*)$$
$$\Sym^n(V^*)\to\Sym^{n-2}(V^*)$$
$\yc_K$: BV operator, $\yc_K^2=0$.

$\Rightarrow$ DGBV:$(\mcalO(V),Q,\yc_K)$.

Example: $\mcalO(V)\rightsquigarrow\PV(X)$.

CME:
$$QI_0+\frac{1}{2}\{I_0,I_0\}=0$$

QME:
$$QI_0+\hbar\yc I+\frac{1}{2}\{I_0,I_0\}=0$$

Today: QFT case.
Vector space $V\rightsquigarrow\mcalE\updot=\Gma(X,E\updot)$,
smooth of graded vector bundles on $X$.

Differential $Q\rightsquigarrow$
$$\xra{Q}\mcalE^{-1}\xra{Q}\mcalE^{0}\xra{Q}\mcalE^{1}\xra{Q}\cdots$$
where $Q$ is a differential operator$(Q^2=0)$.

Assume $(\mcalE\updot,Q)$ is an \textbf{elliptic} complex.
(我们不讲什么叫“椭圆”的,也不讲椭圆算子、PDE的结果).
For example, $Q=\td$ be de-Rham differential.

Fact:$H\updot(\mcalE,Q)$ is of finite dimension.

$\omg\rightsquigarrow\omg$: local $(-1)$-symplectic,
$$\omg(\afa,\beta)=\int_X\langle\afa,\beta\rangle$$
where
$$\langle,\rangle: E\updot\ten E\updot\to Dens_X$$
where "Dens" is the "density line bundle"(if $X$ is orientable. $Dens_X=\wedgeform{n}T_X^*$)

such that

$\omg(\afa,\beta)=-(-1)^{|\afa||\beta|}$

non-degenerate:$\omg(\afa,-)=0\iff\afa=0$

Q-compactible:
$\omg(Q(-),-)+\omg(-,Q(-))=0$.

Then, we get a structure
$$(\mcalE,Q,\omg)$$
$\infty$-dimensional $(-1)$-symplectic space.

\begin{example}(Chern-Simons Theory)

Let $X$ be a $3$-dim manifold. $\mfkg$ is a Lie algebra
with Killing pair $Tr$. $\mcalE=\Omg\updot(X,\mfkg)[1],Q=\td$.
$$\Omg(\afa,\beta)=\int_XTr(\afa,\beta)$$
for $\afa,\beta\in\mcalE$.$\deg(\omg)=-1$.
\end{example}
(The degree $\deg\Omg^k:=k-1$ is called "ghost number"...鬼数?!)

DGBV:

(1)
$$V^*\rightsquigarrow\mcalE^*=\text{distribution}$$
(有拓扑,要取连续的对偶)。

(2)
$$V^*\ten V^*\rightsquigarrow\mcalE^*\hat{\ten}\mcalE^*=\text{complete tensor product}$$
(EG:$$C^\infty(X)\hat{\ten}C^\infty(X)=C^{\infty}(X\times X)$$
)
这些泛函分析里都有。。。

distributions on $X\times X\iff$ dual of $\mcalE\hat{\ten}\mcalE$.

so,
$$\Sym^2(\mcalE^*)$$
is defined.
($S_2$-invariant distribution on $X\times X$)

$\rightsquigarrow\Sym^n(\mcalE^*)$: $S_n$-invariant distributions on
$\underbrace{X\times\cdots\times X}_n$.

$\rightsquigarrow\mcalO(\mcalE)=\prod_n\Sym^n(\mcalE^*)$ is defined.
(graded commutative algebra)

$Q:\mcalE\to\mcalE$ induces $Q:\mcalE^*\to\mcalE^*$, and then
induces derivation $Q\curvearrowright\mcalO(\mcalE)$.

$$K\in\omg^{-1}\not\in\Sym^2(\mcalE)$$
but $K$ is a distribution...

$$s(x)=\int\langle K(x,y),s(y)\rangle\td y$$
$K(x,y)$ is $\delta$-function, supported on $X\subseteq X\times X$.

$$\yc_K:\mcalO(\mcalE)\to\mcalO(\mcalE)$$

but
$$\yc:\Sym^n(\mcalE*)\to\Sym^{n-2}(\mcalE*)$$
is ill-defined(两个广义函数无法配对)
这个不良定义,即“紫外发散”.(UV-problem)

处理此紫外发散的方式:重整化。

So, we introduce "homotopic renormalization"(Costello)!

Assume $(\mcalE,Q)$ is elliptic complex,(e.g. $Q=\td,\pbar$...)
By the regularity of elliptic operator,
$$H\updot(\text{distribution},Q)=H\updot(\text{smooth},Q)$$

$K_0:=\omg^{-1}$ is a destribution,
$$Q(K):=(Q\ten 1+ 1\ten Q)K=0$$
(by compatibility of $\omg$ with $Q$.)

by regularity, we have
$$K_0=K_r+Q(P_r)$$
(in PDE,it is called "Paramatrix")

where $K_r\in\Sym^2(\mcalE)$ is smooth.
and $P_r$ is a (singular) distributional section.

$K_r$ is smooth, $Q(K_r)=0$.

{\color{blue}
(PDE与同伦论联系密切,就好比微积分和线性代数联系密切)
}

$$\yc_{K_r}:\Sym^n(\mcalE^*)\to\Sym^{n-2}(\mcalE^*)$$
contracting with the smooth $K_r\in\Sym^2(\mcalE)$.
$$[\yc_{K_r},Q]=0$$

so,
$$(\mcalO(\mcalE),Q,\yc_{K_r})$$
is a DGBV algebra.($r$-regularized DGBV)
%%%%%%%%%%%%%画饼%%%%%%%%%%%%%%55

Now, if
$$K_0=K_r+QP_r=K_{r'}+QP_{r'}$$
2 ways to ...
So,
$$K_{r'}-K_r=QP^{r'}_r$$
($P_r^{r'}:=P_{r'}=P_r$)Notices that
$QP^{r'}_r$is smooth, then by regularity of elliptic operator,
$P^{r'}_r$ is smooth.

Denote
$$\yc_{K_r}=\pp{K_r}$$
contracting with $K_r$, $2^{nd}$ order.

$$\p_{P_r^{r'}}=\pp{P_r^{r'}}\curvearrowright\mcalO(\mcalE)$$
contracting with $P_r^{r'}\in\Sym^2(\mcalE)$.$\deg(P_r^{r'})=0$.

\begin{lemma}
$$[Q,\p_{P_r^{r'}}]=\yc_{K}-\yc_{K'}$$
\end{lemma}
\begin{proof}
Homework.
\end{proof}

\begin{cor}
$$(Q+\hbar\yc_{r'})e^{\hbar\p_{P_r^{r'}}}
=e^{\hbar\p_{P^{r'}_r}}(Q+\hbar\yc_r)
$$
as operators acting on $\mcalO(\mcalE)$.
\end{cor}
\begin{proof}["proof"]
Using the formula
$$e^ABe^{-A}=e^{\ad_A}B$$
%%%%%%%476%%%%%%%%%%%
\end{proof}

%%%%%%%%%意义何在?%%%%%%%%%%%%%%%

\begin{definition}
$\mcalO(\mcalE),Q,\yc_r=\yc_{K_i}$ regularized DGBV,
$\mcalI[r]\in\mcalO(\mcalE)\fps{\hbar}$ is said to satisfy
$r$-effective QME, if
$$(Q+\hbar\yc_r)e^{\mcalI[r]/\hbar}=0$$
\end{definition}
\begin{prop}
Let $\mcalI[r]$ be a $r$-effective solution of QME, defined
$\mcalI[r']$ by
$$e^{\mcalI[r']/\hbar}=e^{\hbar\p_{P_r^{r'}}}e^{\mcalI[r]/\hbar}$$
Then $\mcalI[r']$ also satisfies QME.
\end{prop}

\begin{proof}
%%%%%%%%%%%also QME%%%%%%%%%%%
\end{proof}

\begin{definition}
An effective solution of QME is defined by a family
$$\{\mcalI[r]\in\mcalO(\mcalE)\fps{\hbar}\}$$
satisfying the following properties:

(1) $(Q+\hbar\yc_r)e^{\mcalI[r]/\hbar}=0$ for all $r$(QME)

(2) $e^{\mcalI[r']/\hbar}=e^{\hbar\p_{P_r^{r'}}}e^{\mcalI[r]/\hbar}$
(called Homotopic RG flow equation)
\end{definition}

In practice, Introduce metric on $E$. $Q^*$ is the adjoint of $Q$,
$$\Box=[Q,Q^*]=QQ^*+Q^*Q$$
is the Laplacian.

for $t\geq 0$, $K_t$ the heat kernel for $e^{t\Box}$.
$K_0=$delta-function, $K_t$ smooth for $t>0$.
$$K_t-K_0=
\int_0^t[Q,Q^*]e^{-t\Box}\td t
=[Q,\int_0^tQ^*e^{-u\Box}\td u]$$
$P_0^t$ be the kernel of $\int_0^tQ^*e^{-u\Box}\td u$
$$P_0^t=(Q^*\ten 1)\int_0^tK_u\td u$$
$$K_t-K_0=Q(P_0^t)$$

Define the regularized propagator
$$P_\veps^L=\int_\veps^L(Q^*\ten 1)K_u\td u$$

$$K_\veps-K_L=Q(P_{\veps}^L)$$








