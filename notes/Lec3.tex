%%%%%%%%%%%%2019.3.12第三周Mon%%%%%%%%%%%%%%%%%%%%%%%
%%%%%%%%%%%%%%%%下周一,周坚代课%%&%%%%%%%%%%%%%%%%%%
%%%%%%%%%%%%%%%%下周二,随堂考试%%%%%%%%%%%%%%%%%%%%%

%\textbf{Coalgebra and homotopy associativity}



\chapter{形变量子化}

%%%%%%%%%%2019.3.18第四周 周一%%%%%%%%%%%%%%%%%%%%%%%
%%%%%%%%%%%%%%%%李思出差,别人代课%%%%%%%%%%%%%%%%%%%
本章开始,正式搞一些事情。
经典力学与量子力学的框架众所周知,大致如下:

\index{phase space\kong 相空间}
\index{symplectic manifold\kong 辛流形}
\index{observable\kong 观测量}
\index{Hamiltonian\kong 哈密顿量}

$$
  \begin{tabular}{|c|c|c|}
    \hline
             &经典力学 &量子力学\\
    \hline
    相空间   &辛流形$(X,\omg)$   &希尔伯特空间$\mcalH$\\
    观测量   &光滑函数           &厄密特算子  \\
    演化方程 &$\frac{\td f}{\td t}=\{H,f\}$
    &$\frac{\td A_t}{\td t}=\frac{i}{\hbar}[\hat{H},A_t]$\\
    \hline
  \end{tabular}
$$

我们将利用结合代数的Hochschild(上)同调,以及$A_{\infty}$方法,
来证明Kontsevich的Formality theorem.

{\color{red}(待完善)}

%\textbf{Poisson structure and quantization}

%Motiration
%Classical Mechanics
%Phase space$(X,\omg)\rightsquigarrow$ Sympletic mfd
%observables: $C^{\infty}(X)$: smooth function
%Poisson bracket:
%$C^{\infty}(X)\ten X^{\infty}(X)\to C^{\infty}(X)$
%Dynamics - Hamiltionian function $H\in C^{\infty}(X)$
%s.t. $\frac{\td f}{\td t}=\{H,f\}$.

%Quantum Picture:
%Hilbert space
%$$\frac{\td A_t}{\td t}=\frac{i}{\hbar}[\hat{H},A_t]$$
%Associative algebra.
%$A_{\infty}$-method and Hochcshild homology
%\textbf{Goal: Formality thm(Kontsevich)}
%Now, let us begin...
%\vs

\section{泊松几何与辛几何}
%\textbf{Poisson bracket}

本节简要回顾一下泊松几何。

\begin{definition}(泊松括号)
\index{Poisson bracket\kong 泊松括号}
%let $X$ be a smooth manifold, a Poisson bracket
%on $C^{\infty}(X)$ is  a $\bbR$-linear
%$$\{,\}:C^{\infty}(X)\times C^{\infty}(X)\to C^{\infty}(X)$$
%sarisfies
%skew symmetry
%Leibnitz rule
%Jacobi identity

设$X$为光滑流形,$\{,\}:C^{\infty}(X)\times C^{\infty}(X)\to C^{\infty}(X)$
为$\bbR$-双线性映射。称$\{,\}$为$X$上的\textbf{泊松括号}(Poisson bracket),
如果$\{,\}$满足:
对任意$f,g,h\in C^{\infty}(X)$,成立

(1)反对称性:$$\{f,g\}=-\{g,f\}$$

(2)Jacobi恒等式:
$$\{f,\{g,h\}\}+\{g,\{h,f\}\}+\{h,\{f,g\}\}=0$$

(3)Leibnitz法则:
$$\{f,gh\}=\{f,g\}h+g\{f,h\}$$
\end{definition}


泊松括号的定义的(1)(2)表明$(C^{\infty}(X),\{,\})$为李代数,
而(3)表明对任意$f\in C^{\infty}(X)$,映射
\begin{eqnarray*}
X_f:C^{\infty}(X)&\to& C^{\infty}(X)\\
g &\mapsto& \{f,g\}
\end{eqnarray*}
为导子,从而$X_f$为$X$上的光滑切向量场,
在局部坐标下形如$X_f=X_f^i\pp{u^i}$.于是有
$$\{f,g\}=X_f^i\pfrac{g}{u^i}$$
但又注意到$\{f,g\}=-\{g,f\}$以及切向量场$X_g$,
从而易知泊松括号$\{,\}$在局部坐标$(u^i)$下的表达式必形如
$$\{f,g\}=P^{ij}\pfrac{f}{u^i}\pfrac{g}{u^j}$$
并且容易验证:
\begin{lemma}设$\{,\}$为光滑流形$X$上的泊松括号,
并且在局部坐标$(u^i)$下的表达式为
$$\{f,g\}=P^{ij}\pfrac{f}{u^i}\pfrac{g}{u^j}$$
那么对任意指标$i,j,k$,成立
$$P^{ij}=-P^{ji}$$
$$P^{is}\pfrac{P^{jk}}{u^s}+
P^{js}\pfrac{P^{ki}}{u^s}+
P^{ks}\pfrac{P^{ij}}{u^s}=0$$
\end{lemma}

\begin{proof}
容易验证$P^{ij}=-P^{ji}$等价于泊松括号的反对称性$\{f,g\}=-\{g,f\}$,
这是因为对任意光滑函数$f,g$,局部上有
$$P^{ij}\pfrac{f}{u^i}\pfrac{g}{u^j}=\{f,g\}=-\{g,f\}
=-P^{ij}\pfrac{g}{u^i}\pfrac{f}{u^j}=
-P^{ji}\pfrac{f}{u^i}\pfrac{g}{u^j}$$
从而$(P^{ij}+P^{ji})\pfrac{f}{u^i}\pfrac{g}{u^j}=0$,
因此由$f,g$的任意性,有$P^{ij}=-P^{ji}$.\vs

再看第二个式子。事实上它等价于泊松括号的雅可比恒等式。
对任意$f,g,h\in C^{\infty}(X)$,局部坐标下有
\begin{eqnarray*}
     \{f,\{g,h\}\}
&=&
     \{f,P^{ij}\pfrac{g}{u^i}\pfrac{h}{u^j}\}\\
&=&
     P^{kl}\pfrac{f}{u^k}
     \pfrac{P^{ij}}{u^l}
     \pfrac{g}{u^i}
     \pfrac{h}{u^j}
    +P^{ij}P^{kl}\pfrac{f}{u^k}
     \left(
       \pmfrac{g}{u^i}{u^l}\pfrac{h}{u^j}
      +\pmfrac{h}{u^j}{u^l}\pfrac{g}{u^i}
     \right)
\end{eqnarray*}
将$f,g,h$轮换再相加,适当更改求和指标,合并整理得
\begin{eqnarray*}
& &
     \{f,\{g,h\}\}+\{g,\{h,f\}\}+\{h,\{f,g\}\}\\
&=&
     \pfrac{f}{u^i}
     \pfrac{g}{u^j}
     \pfrac{h}{u^k}
     \left(
       P^{is}\pfrac{P^{jk}}{u^s}
      +P^{js}\pfrac{P^{ki}}{u^s}
      +P^{ks}\pfrac{P^{ij}}{u^s}
     \right)
    +P^{kl}(P^{ij}+P^{ji})
     \pfrac{f}{u^k}
     \pmfrac{g}{u^i}{u^l}
     \pfrac{h}{u^j}\\
& &
    +P^{kl}(P^{ij}+P^{ji})
     \pfrac{g}{u^k}
     \pmfrac{h}{u^i}{u^l}
     \pfrac{f}{u^j}
    +P^{kl}(P^{ij}+P^{ji})
     \pfrac{h}{u^k}
     \pmfrac{f}{u^i}{u^l}
     \pfrac{g}{u^j}
\end{eqnarray*}
注意到$P^{ij}=-P^{ji}$,以及$f,g,h$的任意性,从而有
$$
 P^{is}\pfrac{P^{jk}}{u^s}
+P^{js}\pfrac{P^{ki}}{u^s}
+P^{ks}\pfrac{P^{ij}}{u^s}
=0
$$
得证。
\end{proof}

我们可以使用张量的语言来描述泊松括号结构:

\begin{definition}(泊松张量)
对于光滑流形$X$,以及$P\in\PV^2_X$为$2$-切向量场,
在局部坐标下表达式为
$$P=P^{ij}\pp{u^i}\wedge\pp{u^j}$$
其中$P^{ij}=-P^{ji}$.称$P$为\textbf{泊松张量}(Poisson tensor),
\index{Poisson tensor\kong 泊松张量}
如果$P$在局部坐标下满足如下雅可比恒等式:
$$
 P^{is}\pfrac{P^{jk}}{u^s}
+P^{js}\pfrac{P^{ki}}{u^s}
+P^{ks}\pfrac{P^{ij}}{u^s}
=0
$$
\end{definition}
容易看出泊松括号与泊松张量的一一对应关系:
对于泊松括号$\{,\}$,若局部上有
$\{f,g\}=P^{ij}\pfrac{f}{u^i}\pfrac{g}{u^j}$,
则考虑泊松张量
$$P:=\frac{1}{2}P^{ij}\pp{u^i}\wedge\pp{u^j}$$
反过来,由泊松张量也能得到泊松括号。并且容易知道
$$\{f,g\}=\langle P,\td f\wedge\td g\rangle$$

%Poisson tensor:
%$P\in\Gamma(X,\wedgeform{2}TX)=\PV_X^2$ s.t.
%$$\{f,g\}=\langle P, \td f\wedge \td g\rangle$$
%$$=P^{ij}(\p_i f\p_jg-\p_ig\p_jf)$$

%Check: Jacobi identity $\iff[P,P]=0$, where
%$[,]$ is Schouten-Nijenhuis bracket.

事实上泊松张量的雅可比恒等式可以用Schouten-Nijenhuis括号等价刻画:

\begin{prop}对于光滑流形$X$,以及$P\in\PV^2_X$,
则$P$为泊松张量当且仅当
$$[P,P]=0$$
其中$[,]$为Schouten-Nijenhuis括号
(见定义\ref{Schouten-Nijenhuis定义-def})。
\end{prop}

\begin{proof}
局部坐标下验证。取局部坐标$(u^i)$,
令$P=P^{ij}\pp{u^i}\wedge\pp{u^j}$,则有
\begin{eqnarray*}
     [P,P]
&=&
     \big[(P^{ij}\pp{u^i})\wedge\pp{u^j},
     (P^{kl}\pp{u^k})\wedge\pp{u^l}\big]\\
&=&
     [P^{ij}\pp{u^i},P^{kl}\pp{u^k}]\wedge\pp{u^j}\wedge\pp{u^l}
    -[P^{ij}\pp{u^i},\pp{u^l}]\wedge\pp{u^j}\wedge P^{kl}\pp{u^k}\\
& &
    -[\pp{u^j},P^{kl}\pp{u^k}]\wedge P^{ij}\pp{u^i}\wedge\pp{u^l}
    +[\pp{u^j},\pp{u^l}]\wedge P^{ij}\pp{u^i}\wedge P^{kl}\pp{u^k}
\end{eqnarray*}
上式右端共有四项,首先注意最后一项
$$[\pp{u^j},\pp{u^l}]\wedge P^{ij}\pp{u^i}\wedge P^{kl}\pp{u^k}
=P^{ij}P^{kl}\delta_{jl}\pp{u^j}\wedge\pp{u^i}\wedge\pp{u^k}
=\sum_{j=1}^nP^{ij}P^{kj}
 \pp{u^j}\wedge\pp{u^i}\wedge\pp{u^k}=0$$
最后一个等号是因为指标$i,k$的(反)对称性;
从而暴力展开,注意利用$P^{ij}=-P^{ji}$以及适当更改求和指标,有
\begin{eqnarray*}
     [P,P]
&=&
     [P^{ij}\pp{u^i},P^{kl}\pp{u^k}]\wedge\pp{u^j}\wedge\pp{u^l}\\
& &
    -[P^{ij}\pp{u^i},\pp{u^l}]\wedge\pp{u^j}\wedge P^{kl}\pp{u^k}
    -[\pp{u^j},P^{kl}\pp{u^k}]\wedge P^{ij}\pp{u^i}\wedge\pp{u^l}\\
&=&
     P^{ij}\pfrac{P^{kl}}{u^i}
     \pp{u^k}\wedge\pp{u^j}\wedge\pp{u^l}
    -P^{kl}\pfrac{P^{ij}}{u^k}
     \pp{u^i}\wedge\pp{u^j}\wedge\pp{u^l}\\
& &
    +P^{kl}\pfrac{P^{ij}}{u^l}
     \pp{u^i}\wedge\pp{u^j}\wedge\pp{u^k}
    -P^{ij}\pfrac{P^{kl}}{u^j}
     \pp{u^k}\wedge\pp{u^i}\wedge\pp{u^l}\\
&=&
     \left(
       -P^{sj}\pfrac{P^{ki}}{u^s}
       -P^{sk}\pfrac{P^{ij}}{u^s}
       +P^{ks}\pfrac{P^{ij}}{u^s}
       -P^{js}\pfrac{P^{ik}}{u^s}
     \right)
     \pp{u^i}\wedge\pp{u^j}\wedge\pp{u^k}\\
&=&
     -4P^{sj}\pfrac{P^{ki}}{u^s}
     \pp{u^i}\wedge\pp{u^j}\wedge\pp{u^k}\\
&=&
     -8\sum_{i<j<k}
       \left(
          P^{sj}\pfrac{P^{ki}}{u^s}
         +P^{sk}\pfrac{P^{ij}}{u^s}
         +P^{si}\pfrac{P^{jk}}{u^s}
       \right)
     \pp{u^i}\wedge\pp{u^j}\wedge\pp{u^k}
\end{eqnarray*}
可见$[P,P]=0$当且仅当雅可比恒等式成立,证毕。
\end{proof}

于是我们自然地引入泊松流形的概念:

\begin{definition}(泊松流形)
\index{Poisson manifold\kong 泊松流形}

\textbf{泊松流形}(Poisson manifold)
是指二元组$(X,P)$,其中$X$为光滑流形,
$P\in\PV^2_X$满足$[P,P]=0$.
%A Poisson manifold is a pair $(X,P)$ such that
%$[P,P]=0$ for some $P\in\PV_X^2$.
\end{definition}
众所周知,这是经典力学的几何模型。
%(geometry of classical mechanics)
接下来看一些泊松流形的例子:

\begin{example}
if $(X,\omg)$ is a sympletic manifold,
$$\omg=\frac{1}{2}\omg_{ij}\td x^i\wedge\td x^{j}$$
such that $\omg$ is non-degenerated ,$\omg_{ij}=-\omg_{ji}$,
then
$$\omg^{-1}:=\frac{1}{2}\omg^{ij}\p_i\wedge\p_j$$
is a Poisson bracket.where $(\omg^{ij}):=(\omg_{ij})^{-1}$
\end{example}

\begin{example}
Let $\mfkg$ is a Lie algebra, $X:=\mfkg^*$ its dual,
$$[,]:\mfkg\times\mfkg\to\mfkg$$
then
$$P_{\mfkg}\in \mfkg\ten\wedgeform{2}\mfkg$$
(locally)
if $\{e^i\}$ is a basis of $\mfkg$, $\{e_i\}$ its dual,
$$[e^i,e^j]=C^{ij}_ke^k$$
then
$$P_{\mfkg}=X^kC^{ij}_k\p_i\wedge\p_j$$
check: it satisfies Jacobi identity, i.e.
$(\mfkg^*,P_{\mfkg})$ is a Possion manifold.
\end{example}

\begin{rem}
$P_{\mfkg}$ is not invertible $\iff$
$\mfkg^*$ is not sympletic.
\end{rem}

\section{星积}
\textbf{Star product:}

\begin{definition}
A star product $*$ on a Possion mfd $(X,P)$
is a $\bbR[[\hbar]]$-linear map
$$C^{\infty}(X)[[\hbar]]\times C^{\infty}(X)[[\hbar]]\to C^{\infty}(X)[[\hbar]]$$
$$f*g\mapsto \sum_{k\geq 0}\hbar^kC_k(f,g)$$
such that:

(1) $*$ is associative

(2) $f*g=fg\mod \hbar$

(3) $\frac{1}{2}(f*g-g*f)=\hbar\{f,g\}\mod\hbar^2$

(4) $C_k(f,g)=C_k^{i_1...i_l;j_i,...,j_m}(\p_{i_1}...\p_{i_l}f)
(\p_{j_1}...\p_{j_m}g)$, where $C_k^{...}$ is in $C^{\infty}(X)$.
(bi-differential operator)
\vs


then, $(C^{\infty},*)$ is the \textbf{deformation quantization} on $(X,P)$.
\end{definition}

the first non-commutativity of $*$ is given
by Poosson bracket.

Fundamental Question:
[Dewilde lemote 1983] for a Poisson manifold, is there $\exists$
a deformation quantization?

[Fedosov 1983,1994]

\begin{example}
$(C^{\infty}(X),\{,\})\rightsquigarrow(C^{\infty(X)[[\hbar]]},*)$

$$X=\bbR^m$$
Poisson bracket
$$P=P^{ij}\p_i\wedge\p_j$$
where $P^{ij}\in\bbR$ are constant.
then we define

%$$f*g:=\exp{\frac{\hbar}{2}P^{ij}\frac{\p}{\p y_i}\frac{\p}{\p z_j}}\big|_{y=z=x}f(y)g(z)$$
%%%%%%%此处有图%%%%%55
\end{example}


%%%%见笔记%%%


\begin{example}
when $\bbR^m=\bbR^{2n}$ even dimension, $\{x^1,...,x^n;p_1,...,p_n\}$,
$$P=\sum\pp{p_i}\wedge\pp{x^i}$$
then Moyal product becomes Weyl algebra and
$$[x^i,p_j]=x^i*p_j-p_j*x^i=\hbar\delta_{ij}$$
\end{example}

\begin{example}
$\mfkg$ Lie algebra, $X=\mfkg^*$,$P_{\mfkg}$,
consider (quantum) universal evenloping algebra
$$\mcalU_{\hbar}(\mfkg)=\bigoplus_{k}\mfkg^{k}[[\hbar]]\big/\sim$$

the relation is
$$a\ten b-b\ten a=\hbar[a,b]$$

$$\mcalU_{\hbar}(\mfkg)\big|_{\hbar=1}$$
is universal evenloping algebra.

$$\mcalU_{\hbar}(\mfkg)\big|_{\hbar=0}=\Sym\updot(\mfkg)$$

\end{example}

PBW theorem:
there is a vector space bijienction:
$$\Phi:\Sym\updot(\mfkg)[[\hbar]]\to\mcalU_{\hbar}(\mfkg)$$
%%%%PBW%%%55

$\mcalU_{\hbar}(\mfkg)$ has a natural associative algebra
structure by $\ten$, $\Phi^{*}(\ten)$ defines an associative algebra structure on
$\Sym\updot(\mfkg)[[\hbar]]=:\mcalO(X)[[\hbar]]$

Check: it is a deformation quantization of $(X,P_{\mfkg})$.

\begin{example}
$M$ is a mfd, $X=T^*M$ cotangent bundle, which admits a sympletic structure
$$\omg=\sum\td x^i\wedge\td p_i$$
$$\mcalO(X)=\Gamma(M,\Sym\updot(TM))$$
\end{example}

$f\in\mcalO(X)$,
$$f=f(x,p)=\sum_{I}f_I(X)p^{I}(X)$$
and $\omg^{-1}$ Poisson bracket $\{,\}$ is a Poisson algebra.

$D_X$ differential operator on $X$.
$\exists$ filtration by its order
$$D_X^{(0)}\subseteq D_X^{(1)}\subseteq...\subseteq D_X^{(m)}...$$
(consists of $\sum A^{i_1...i_k}\p_{i_1}...\p_{i_k}$)

Check:
$$[D_X^{(m)},D_X^{(n)}]\subseteq D_X^{(m+n-1)}$$
$$D_X^{(m)}\circ D_X^{(n)}\subseteq D_X^{(m+n)}$$
$$(D_X,\circ)\text{ is associative}$$


\begin{definition}
$$D_X^{\hbar}:=\bigoplus_{m\geq 0}D^{(k)}_X\hbar^m\subseteq D_X[\hbar]$$
is a $\bbR[\hbar]$-module
\end{definition}

HW: what is $k$ in Def above?????

(such that $D_X^{\hbar}$ can be understood as a Deformation Quantization of
$(\mcalO(X),\{,\})$ over ring $\bbR[\hbar]$.

\begin{example}(Quantum torus)

Let $V=\bbZ^n$ ,define the algebra
$$e^{V}:=\{e^v|v\in V\}$$
with
$$e^{v_1}e^{v_2}:=e^{v_1+v_2}$$
then
$e^V$ is algebra function on $(\bbC^*)^n=\bbC[z_1^{\pm 1},...,z_n^{\pm 1}]$.

Let $\omg:V\times V\to \bbZ$ skew symmetric bilinear form
$$e^{v_1}*_{\hbar}e^{v_2}=e^{\hbar\omg(v_1,v_2)}e^{v_1+v_2}$$
check: associativity and it is a D.Q. of Poisson bracket
$$\{e^{v_1},e^{v_2}\}=\omg(v_1,v_2)e^{v_1+v_2}$$



\end{example}




