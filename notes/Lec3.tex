%%%%%%%%%%2019.3.11 Mon 第三周%%%%%%%%
\chapter{乘积}

\textbf{Product}

Recall:
Commutative                 non-Commutative

polyvectorfield             $(C\updot(A,A),\p)$

differential form           $(C\downdot(A,A),b)$

$(\Omg_X\updot)$            $(C\downdot(A),b,\mcalB)$
%%%%%%%%%%%

cyclic homology  $H\downdot(C\downdot(A)[u^{-1}],b+u\mcalB)$

negative homology  $H\downdot(C\downdot(A)[u],b+u\mcalB)$

periodic homology  (analogue of de Rham cohomology)
$H\downdot(C\downdot(A)(u),b+u\mcalB)$

Today:

\section{多重切向量场与Schouten-Nijenhuis括号}

\begin{definition}
$A$ is a $\bbZ$-graded algebra,$A=\bigoplus\limits_{k\in\bbZ}A_k$,
such that
$$A_k\cdot A_l\subseteq A_{k+l}$$
and associative.

$A$ is graded commutative if
$$a_k\cdot a_l=(-1)^{kl}a_la_k$$
\end{definition}

\begin{definition}
$\mfkg$ is a graded Lie algebra(super Lie algebra)
if

(1) $\mfkg=\bigoplus\limits_{k\in\bbZ}\mfkg_k$

(2) Lie bracket $[,] \mfkg_k\times\mfkg_l\mapsto\mfkg_{k+l}$
which is graded skew-symmetric
$$[a,b]=-(-1)^{\deg a\deg b}[b,a]$$

(3)Graded Jacobi identity
$$[[a,b],c]=[a,[b,c]]-(-1)^{\deg a\deg b}[b,[a,c]]$$
\end{definition}


\begin{example}
(1)$(\Omg_X\updot,\wedge)$ is a graded commutative algebra.

(2) $\text{PV}_X:=\Gamma(X,\wedge^*TX)$poly vector field,
is graded commutative alg.
\end{example}

Schousten-Nijenhuis bracket$\{,\}$:

$$\{,\}:PV^p\times PV^q\to PV^{p+q-1}$$

$$\xi=\xi_1\wedge...\wedge\xi_p$$
$$\eta=\eta_1\wedge...\wedge\eta_q$$

then

$$\{\xi,\eta\}=\sum_{i,j}(-1)^{i+j}[\xi_1,\eta_j]
\xi_1\wedge...\widehat{\xi_i}...\xi_p\wedge\eta_1...\widehat{\eta_j}...\eta_q$$

Check:

(1) $\{,\}$ is well-defined and coordinate independent.(HW)

(2) $\{\xi,\eta\}=-(-1)^{(\deg\xi-1)(\deg\eta-1)}\{\eta,\xi\}$
%%%%%%%%%Details%%%%%%

\begin{notation}
$A$ is graded, then $(A[1])_n:=A_{n-1}$ shifted gradation...
\end{notation}

So, $PV_X[1]$ is a graded Lie algebra, and
$$(PV_X[1])_0=T_X$$
is a Lie algebra.

(3) graded  Leibnitz rule
$$\{\alpha,\beta\wedge\gamma\}
=\{\alpha,\beta\}\gamma+(-1)^{(\deg\alpha-1)\deg\beta}
\beta\{\alpha,\gamma\}$$
(挪之前的分次)

\begin{prop}

(1)$(PV,\wedge)$ graded algebra

(2)$(PV[1],\{,\})$ graded Lie alg

(3) (1)(2)is compactible(Leibnitz rule)

\end{prop}

Gerstenhaber algebra(Wiki, HW)
(in physics, Classical BV algebra)


\section{Shuffle乘积}

\begin{definition}
Let $S_n$ be the symmetric group, A (p,q)-Shuffle is
 a permutation $\sigma\in S_{p+q}$ such that
 $$\sigma(1)<\sigma(2)<...<\sigma(p)$$
  $$\sigma(p+1)<\sigma(p+2)<...<\sigma(q)$$
Let
$$Sh_{p,q}:=\text{ all the $p,q$-Shuffle}$$
\end{definition}

Let $A,A'$ be to $K$-algebras,
$M,M'$ are $A,A'$-bimodule. We define the Shuffle product $\times$

$$C_p(A,M)\times C_q(A',M')\to C_{p+q}(A\ten A',M\ten M')$$

$$(m,a_1,...,a_p)\times(m',a_1',...,a_q')\mapsto
\sum_{\sigma\in Sh_{p,q}}
  (-1)^{|\sigma|}
  (m\ten m',\sigma(a_1,...,a_p,a_1',...,a_q'))
$$

Here $\sigma(a_1,...,a_p,a_1',...,a_q')=(a_{\sigma^{-1}(1)})$
%要取逆。。。因为是左作用%

\begin{prop}
The Shuffle product $\times$ is compatible with
Hochschild differential $b$: i.e.

$$b(x\times y=b(x)\times y+(-1)^{\deg x}x\times b(y)$$
\end{prop}

%%%%%%%又画图了%%%%%%
%读论文就是将非人话翻译成人话的过程,
%写论文就是将人话写成非人话的过程。
%                               ——李思

%“我写的公式也不一定对。。。但基本上是对的,up to sign”

\begin{cor}
we get a chain complex map
$$C\downdot(A,M)\ten C\downdot(A',M')
\to
C\downdot(A\ten A',M\ten M')$$

pass to homology, we get

$$H\downdot(A,M)\ten H\downdot(A',M')
\to
H\downdot(A\ten A',M\ten M')$$

In particular,
$$C\downdot(A)\ten C\downdot(A')\to C\downdot(A\ten A')$$
\end{cor}

\begin{thm}(K\"{u}nneth Forumla)

Assume $a,A'$ are flat over $K$,
then Shuffle product gives an isomorphism
$$\HH\downdot(A)\ten\HH\downdot(A')\cong\HH\downdot(A\ten A')$$
\end{thm}

%不证了,标准的证明

Functoriality:

$$\fai:A\to B$$
is a map of $K$-algebra, then

$$\fai\downdot:C\downdot(A)\to C\downdot(B)$$

induces

$$\fai:\HH\downdot(A)\to \HH\downdot(B)$$
%%%%%%%代数映射%%%%%%%5

\begin{cor}
Let $A$ be a commutative associative algebra, then
$(\HH\downdot(A),\times)$ is a graded commutative algebra.
\end{cor}

HW: If $A=K[x^i]$, then
$$(\HH\downdot(A),\times)\cong(\Omg_A\updot,\wedge)$$

\section{Cup乘积}

$$C\updot(A,A)=\bigoplus_p(C^p(A,A))$$

%%%%%%%%%%%%%树%%%%%%%%%

\begin{definition}
For $f\in C^p(A,A)$,$g\in C^q(A,A)$. Define cup product
$$f\cup g\in C^{p+q}(A,A)$$
%%%%%%%shadiao%%%%
\end{definition}

\begin{prop}
Cup product is compatible with Hochschild differential $\p$:

$$\p(f\cup g)=(\p f)\cup g+(-1)^{\deg f}f\cup\p g$$
\end{prop}

%%%%%%%这岂不是吃屎%%%%%%%

\begin{cor}
There is a well-defined cup product
$$\cup: H^p(A,A)\times H^{q}(A,A)\to H^{p+q}(A,A)$$
\end{cor}

HW: If $A=\bbC[x^i]$,then
$$(H\updot(A,A),\cup)\cong (PV_A,\wedge)$$

\section{Gerstenhaber 乘积}
\textbf{Gerstenhaber algebra(自己查定义)}

\begin{definition}Gershenhaber product
$$C^{p}(A,A)\times C^q(A,A)\to C^{p+q-1}(A,A)$$
$$(f,g)\to f\circ g$$

%%%%%%%Gerstenhaber%%%%%%%%%5
\end{definition}

\begin{prop}
$$\p(f\circ g)-(\p f)\circ g-(-1)^{\deg g-1}f\circ\p g
=\pm(f\cup g-(-1)^{\deg f\deg g}g\cup f)$$
$$=\text{the failure of $\circ$ being a chain map
is measured bu the commutativity of cup product}$$
\end{prop}

\begin{proof}
%%%%%%%暴力计算,需要画很多图,从略%%%%%%55
HW
\end{proof}

\begin{cor}
$(\H\updot(A,A),\cup)$ is a graded commutative algebra.
\end{cor}

\begin{proof}
Omit.
\end{proof}

%%注记:很深刻的结论。。。。%%%
%%

\begin{definition}(Cerstenhaber bracket)
$$\{f,g\}=f\circ g-(-1)^{(f-1)(g-1)}g\circ f$$
%这是非交换版本的NR括号
\end{definition}

\begin{prop}
$$\p\{f,g\}=\{\p f,g\}\pm\{f,\p g\}$$
and induces
$\{,\}$ defines on $H\updot(A,A)$.
this is  the analogue of Schouten-Nijenhuis bracket.
\end{prop}

%%%%%%%%%%%%2019.3.12第三周Mon%%%%%%%%%%%%%%%%%%%%%%%
%%%%%%%%%%%%%%%%下周一,周坚代课%%&%%%%%%%%%%%%%%%%%%
%%%%%%%%%%%%%%%%下周二,随堂考试%%%%%%%%%%%%%%%%%%%%%

\textbf{Coalgebra and homotopy associativity}

We will talk with the category
$\bbZ$-graded $K$-modules
$$Mod_k^{\bbZ}$$

object: $C=\bigoplus\limits_{k\in\bbZ}C_k$

morphism:$f:C\to D$ is said to have degree $n$ if
$$f:C_k\to C_{k+n}$$

$$\Hom(C,D):=\bigoplus_n(C,D)_n$$
(i.e. degree$=n$)
where
$$\Hom(C,D)_n=\bigoplus_m\Hom(C_m,D_{m+n})$$

now,$C,D\in Mod_{K}^{\bbZ}$
$$(C\ten D)_n:=\sum_{m+l=n}C_m\ten D_l$$
we define the flip
$$\tau:C\downdot\ten D\downdot\to D\downdot\ten C\downdot$$
$$c_m\ten d_l\mapsto(-1)^{ml}d_l\ten c_m$$

$\rightsquigarrow$ Koszul sign rule:

Eg. $f\in\Hom(C,D)$,$g\in\Hom(C',D')$,then
$$(f\ten g)(x\ten y)=(-1)^{|g||x|}f(x)g(y)$$

$C\downdot$ graded $K$-module. Graded duality??
$$C^*_n:=\Hom(C_{-n},K)$$
$$C^*\downdot:=\bigoplus_n C_n^*$$

A complex is $\bbZ$- graded $K$-module
$$\cdots\to C_{-1}\xra{\td}C_0\xra{\td}C_1\xra{\td}C_2\to\cdots$$
$\deg\td=1$,$\td^2=0$.

Complex $(C\downdot,\td)\rightsquigarrow$dual complex
$(C\downdot^*,\td^*)$is the dual of
$C_{-(n+1)}\xra{\td}C_{-n}$.

Shift

$(C\downdot,\td)\rightsquigarrow(C\downdot[1])$
$$(C\downdot[1])_n:=C_{n+1}$$
differential$=-\td$

$(C\downdot,\td_C)$ and $(D\downdot,\td_D)$,then
$(C\downdot \ten D\downdot,\td)$
$$\td:C_p\ten D_q\to C_{p+1}\ten D_q\oplus C_p\ten D_{q+1}$$
$$\td=\td_C\ten 1+(-1)^p1\ten \td_D$$

$A$ graded $K$-algebra,
$$A=\bigoplus_nA_n\in Mod_K^{\bbZ}$$
$$m:A\ten A\to A$$
degree of $m$ is $0$.

satisfying associativity
%%%%%结合性%%%%%%%%

Category of associative graded $K$-algebra
$$Ass-alg_K^{\bbZ}$$

$A\in Ass-alg_K^{\bbZ}$ is called (graded) "commutative"
if

%%%%%%分次交换%%%%%%%%%

A differential graded $K$-algebra $(A\downdot,\td)$
$\td(\alpha\cdot\beta)=(\td\alpha)\beta+(-1)^{|\alpha|}\alpha\td\beta$

%%%%分次微分%%%%%%%

$V$ is a graded $K$-module, then define
$$Sym^m(V)=V^{\ten m}\big/\sim$$
where
$$\alpha\ten\beta\sim(-1)^{|\alpha||\beta|}\beta\ten\alpha$$

and define
$$\wedgeform{m}(V):=V^{\ten m}\big/\sim$$
where
$$\alpha\ten\beta\sim-(-1)^{|\alpha||\beta|}\beta\ten\alpha$$

If $V=V_0$, then $Sym^n(V_0)$ is the usual symmetric tensor,
and $\wedgeform{n}(V_0)$ exterior.

HW:%%必考!!

$$Sym^n(V\downdot[1])\cong\wedgeform{n}(V)[n]$$
here $[n]=([1])^{n\text{times}}$

%%非常重要!!!%

We have natural forgetful functor
$$Ass-alg_K^{\bbZ}\to Mod_K^{\bbZ}$$
$$Commu-alg_K^{\bbZ}\to Mod_K^{\bbZ}$$

whose left adjoints are called "free objects".

\begin{definition}
 $V\in Mod_K^{\bbZ}$, define the tensor algebra
$$T(V):=K\oplus V\oplus V^{\ten a}\oplus\cdots=\bigoplus_{n\geq 0}V^{\ten n}$$
with algebra structure given by $\ten$:
$$(v_1\ten...\ten v_p)\cdot((v_{p+1}\ten...\ten v_{p+q}))
=v_1\ten...\ten v_{p+q}$$
\end{definition}

$$T:Mod_K^{\bbZ}\to Ass-alg_K^{\bbZ}$$
$$V\mapsto T(V)$$

\begin{prop}
$T$ is the left adjoint of
$$Ass-alg_K^{\bbZ}\to Mod_K^{\bbZ}$$
\end{prop}
这个显然,不证了。

\begin{definition}
$V\in Mod_K^{\bbZ}$, we define $Sym(V)$ by
$$Sym(V):=\bigoplus_{m\geq 0}^n(V)$$
$$Sym:Mod_K^{\bbZ}\to Commu-alg_K^{\bbZ}$$
\end{definition}

\begin{prop}
$Sym$ is the left adjoint of
$$Commu-alg_K^{\bbZ}\to Mod_K^{\bbZ}$$
\end{prop}

Let $A\in Ass-alg_K^{\bbZ}$,$M$ is a (graded) bi-module,
a derivation
$$D:A\to M$$
is a $K$-linear map ,and satisfies
$$D(ab)=D(a)b\pm aD(b)$$
%也可以考虑导子的degre. 不过一般都是degree0 的
%涉及正负号时,Koszul rule

%%%%导子图%%%%%%%%

\begin{prop}(if $V$ is a graded $K$-module)
$$\Der(T(V),M)\cong\Hom(V,M)$$
\end{prop}

In particular,
$$\Der(T(V),T(V))\cong\Hom(V,T(V))$$

Check:If $D_1,D_2\in\Der(A,A)$,then
$$[D_1,D_2]:=D_1\circ D_2-(-1)^{|D_1||D_2|}D_2\circ D_1$$
is also a (graded) derivation.

i.e. $(\Der(A,A),[-,-])$ is a graded Lie algebra.

\textbf{Co-algebra}

\begin{definition}
$C\downdot\in Mod_K^{\bbZ}$ is a graded coalgebra over $K$,
if there is a coproduct ($\deg=0$)
$$\triangle:C\to C\ten C$$
%%%coproduct%%%%%%%
\end{definition}

counit$\veps:C\to K$ satisfying
%%%%%%%%counit%%%%%%%%%

co-derivation

$\delta: C\to C$ satisfying
%%%%%co-Leibnitz rule%%%%%%%

A differential graded co-algebra is a co-algebra
$C$ with a co-derivation $\delta:C\to C$ such that
$\deg\delta=1$ and $\delta^2=0$.

co-augmentation

 (Recall:
 $A$:$K$-algebra. augmentation is an algebra morphism
$A\to K$.)

Co-algebra $(C,\triangle)$ is called co-augmentation if
there is a co-algebra map $K\to C$.
%%%%%%co-algebra的同态怎么定义?自行脑补%%%%%%%%%%

$(C,\triangle)$ is co-commutative,if
%%%%co-commutative%%%%%%

\begin{rem}
If $(C,\triangle)$ is a co-algebra, then
$(C^*,\triangle^*)$ is an algebra
$$A\ten A=C^*\ten C^*\to(C\ten C)^*\xra{\triangle^*}C^*=A$$
\end{rem}
%%%%%见图%%%%%%%5

\begin{example}
$V\in Mod_K^{\bbZ}$,
$$T^c(V):=\bigoplus_{n\geq 0}V^{\ten n}$$
$$\triangle:T(V)\to T(V)\ten T(V)$$
$$v_1\ten...\ten v_n\mapsto
\sum_{i=0}^n(v_1\ten...\ten v_i)\ten(v_{i+1}\ten...\ten v_n)$$
\end{example}

Check:$\triangle$ is a co-product, and what is its dual?

\begin{example}
$$\overline{T^c}(V):=\bigoplus_{n\geq 1}V^{\ten n}$$
$$\overline{\triangle}(v_1\ten...\ten v_n)
=
\sum_{i=1}^{n-1}(v_1\ten...\ten v_i)\ten(v_{i+1}\ten v_n)$$

$\overline{T^c}(V)$ is a co-product on $\overline{T^c}(V)$.
\end{example}

$Coder(C)$ is all the co-derivation....

\begin{prop}
$Coder(C)$ is a graded Lie algebra, where
$$[D_1,D_2]=D_1\circ D_2-(-1)^{|D_1||D_2|}D_2\circ D_1$$
\end{prop}

\textbf{Associativity.}

$$\bullet:A\ten A\to A$$
$\Longrightarrow$
$$m:A[1]\ten A[1]\to A[1]$$
$$sa_1\ten sa_2\mapsto(-1)^{|a_1|+1}s(a_1a_2)$$

Observation:
Associativity of $\bullet\Longleftrightarrow[m,m]=0$

%%%%%叼~~%%%%%







