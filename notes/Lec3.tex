%%%%%%%%%%2019.3.11 Mon 第三周%%%%%%%%
\chapter{乘积}

%\textbf{Product}

%Recall:
%Commutative                 non-Commutative

%polyvectorfield             $(C\updot(A,A),\p)$

%differential form           $(C\downdot(A,A),b)$

%$(\Omg_X\updot)$            $(C\downdot(A),b,\mcalB)$
%%%%%%%%%%%

%cyclic homology  $H\downdot(C\downdot(A)[u^{-1}],b+u\mcalB)$

%negative homology  $H\downdot(C\downdot(A)[u],b+u\mcalB)$

%periodic homology  (analogue of de Rham cohomology)
%5$H\downdot(C\downdot(A)(u),b+u\mcalB)$

%Today:

\section{分次模与Koszul符号法则}

本节我们集中起来澄清一些关于分次模、分次代数的概念,
并且力图阐明分次代数中出现的正负号。
这里的“分次”如不加说明,指的都是$\bbZ$-分次。

首先我们考虑分次$K$-模。

\begin{definition}(分次$K$-模范畴)

(1)称$K$-模$M$为($\bbZ$-)\textbf{分次$K$-模}
(graded $K$-module),
\index{graded $K$-module\kong 分次$K$-模}
若$M$具有如下分次结构:
$$M=\bigoplus_{k\in\bbZ}M_k$$

(2)若$M,N$为分次$K$-模,称$K$-模同态$f:M\to N$
为次数为$d$的\textbf{齐次$K$-模同态},
若对于任意$n\in \bbZ$,成立
$$f(M_n)\subseteq N_{n+d}$$
\end{definition}

对于分次代数,我们可以定义\textbf{齐次元},
以及齐次元的\textbf{次数},不再赘述。
对于齐次元$a\in A$,将$a$的次数记为$\deg a$,
或者简记为$|a|$.

平凡的例子:通常的$K$-模自然有分次$K$-模结构——
只需将该模中的任何元素都认为是$0$次齐次元。

我们还可以谈论以分次$K$-模为对象的范畴:

\begin{notation}(分次$K$-模范畴)
对于分次$K$-模$M,N$,对任意$d\in\bbZ$,记
$$\Hom(M,N)_d:=\bigoplus_{n\in\bbZ}\Hom(M_n,N_{n+d})$$
即次数为$d$的分次$K$模同态之全体。再记
$$\Hom(M,N):=\bigoplus_{d\in\bbZ}\Hom(M,N)_d$$
称这里面的元素为分次$K$-模同态。

我们考虑如下分次$K$-模范畴,记为$\Modcat{\bbZ}{K}$:

(1)$\Obj=$全体分次$K$-模;

(2)$\Mor(M,N)=\Hom(M,N)$为分次$K$-模同态。
\end{notation}

注意对任何分次$K$-模$M,N$,
$\Hom(M,N)$自然有分次$K$-模结构,其中的$d$次齐次元
即为$M$到$N$的次数为$d$的齐次同态。

%We will talk with the category
%$\bbZ$-graded $K$-modules
%$$Mod_k^{\bbZ}$$

%object: $C=\bigoplus\limits_{k\in\bbZ}C_k$

%morphism:$f:C\to D$ is said to have degree $n$ if
%$$f:C_k\to C_{k+n}$$

%$$\Hom(C,D):=\bigoplus_n(C,D)_n$$
%(i.e. degree$=n$)
%where
%$$\Hom(C,D)_n=\bigoplus_m\Hom(C_m,D_{m+n})$$
对于两个分次$K$-模,它们作为$K$-模的张量积,
也有自然的分次结构:

\begin{definition}(分次$K$-模的张量积)
对于分次$K$-模$M,N$,则张量积$M\ten N$自然有如下分次结构:
$$M\ten N=\bigoplus_{k\in\bbZ}(M\ten N)_k$$
其中
$$(M\ten N)_k:=\bigoplus_{p,q\in\bbZ\atop p+q=k}
                  M_p\ten N_q$$
\end{definition}
容易验证这给出了$M\ten N$的分次$K$-模结构。

\begin{definition}
对于分次$K$-模$M,N$,定义如下分次$K$-模同态:
\begin{eqnarray*}
\tau:M\ten N&\to& N\ten M\\
x\ten y&\mapsto&(-1)^{\deg x\deg y}y\ten x
\end{eqnarray*}
其中$x,y$分别为$M,N$中的任意的齐次元。
\end{definition}

这是一个次数为$0$的齐次$K$-模同构。
注意这里的正负号。

%we define the flip
%$$\tau:C\downdot\ten D\downdot\to D\downdot\ten C\downdot$$
%$$c_m\ten d_l\mapsto(-1)^{ml}d_l\ten c_m$$

%$\rightsquigarrow$ Koszul sign rule:
\begin{notation}(Kuszul符号法则)

设$M,M',N,N'$均为分次$K$-模,
则自然有如下的分次$K$-模同态:
\begin{eqnarray*}
\Hom(M,N)\ten\Hom(M',N')&\to&\Hom(M\ten M',N\ten N')\\
(f\ten g)(m\ten m')&:=&(-1)^{\deg g\deg m}f(m)\ten g(m')
\end{eqnarray*}
其中$f,g,m,m'$分别为$\Hom(M,N),\Hom(M',N'),M,M'$当中的任意齐次元。
\label{Koszul符号法则-notation}
\end{notation}

依然注意正负号。以后我们总是
默认$f\ten g$在$m\ten m'$上如此作用。

我们还可以定义分次$K$-模的对偶模
(与通常的对偶模仍然在正负号上有些区别):

%Eg. $f\in\Hom(C,D)$,$g\in\Hom(C',D')$,then
%$$(f\ten g)(x\ten y)=(-1)^{|g||x|}f(x)g(y)$$
\begin{definition}(分次对偶模)
对于分次$K$-模$M$,定义
$$M^*:=\bigoplus_{n\in\bbZ}M^{*}_n$$
其中
$$M^*_n:=\Hom(M_{-n},\bbZ)$$
\end{definition}
易知$M^*$具有分次$K$-模结构,
并且有自然的配对
$$M^*_n\times M_{-n}\to K$$

%$C\downdot$ graded $K$-module. Graded duality??
%$$C^*_n:=\Hom(C_{-n},K)$$
%$$C^*\downdot:=\bigoplus_n C_n^*$$

\begin{rem}(分次K-模上链复形)
%A complex is $\bbZ$- graded $K$-module
对于分次$K$-模$C$,以及$\td\in\Hom(C,C)_1$,
即次数为$1$的齐次同态。如果$\td\circ\td=0$,则自然有$K$-模上链复形:
$$\cdots\to C_{-1}\xra{\td}C_0\xra{\td}C_1\xra{\td}C_2\to\cdots$$
%$\deg\td=1$,$\td^2=0$.
\end{rem}
这是我们在同调代数当中早已熟知的。
我们以后就将上链复形与带有$\td$的分次$K$-模等同。
本节我们采用上链复形的语言(即$\deg\td=1$),
链复形($\deg\p=-1$)的情形完全类似。

这里讲到的“上链复形”,与通常同调代数当中的
上链复形在各种操作上都会可能相差正负号;
为了区分,我们称这里的“上链复形”为“\textbf{分次上链复形}”。

我们还可以考虑分次上链复形
$(C\downdot,\td)$的分次对偶,仍为分次上链复形:
$$\cdots\to C_{-1}^*\xra{\td^*}C_0^*
\xra{\td^*}C_1^*\xra{\td^*}C_2^*\to\cdots$$

%Complex $(C\downdot,\td)\rightsquigarrow$dual complex
%$(C\downdot^*,\td^*)$is the dual of
%$C_{-(n+1)}\xra{\td}C_{-n}$.
%Shift

\begin{definition}(分次上链复形的平移)
对于分次$K$-模上链复形$(C\downdot,\td)$,
定义分次上链复形
$(C\downdot[1],\td_{[1]})$如下:
$$(C\downdot[1])_n:=C_{n+1}$$
并且微分算子$\td_{[1]}$使得下图交换:
$$\xymatrix{
   (C[1])_n \ar[r]^{\td_{[1]}}  \ar@{=}[d]
  &(C[1])_{n+1}               \ar@{=}[d]
\\
   C_{n+1}  \ar[r]^{-\td}
  &C_{n+2}
}$$
%differential$=-\td$
\end{definition}
注意$\td_{[1]}$当中的负号。
类似地,对任意$l\in\bbZ$,可以去定义$l$-平移
$(C[l]\downdot,\td_{[l]})$,特别注意符号
$$\td_{[l]}=(-1)^l\td$$

对于一般的分次$K$-模,我们也可以考虑其平移,
这无非是重新规定齐次元的次数。

\begin{definition}(分次上链复形的张量积)

对于分次上链复形$(C\downdot,\td_C)$与$(D\downdot,\td_D)$,
定义$(C\ten D)\downdot$的分次上链复形结构$\td$如下:
%$(C\downdot \ten D\downdot,\td)$
\begin{eqnarray*}
\td:C_p\ten D_q &\to& C_{p+1}\ten D_q\oplus C_p\ten D_{q+1}\\
\td &=&\td_C\ten 1+(-1)^p1\ten \td_D
\end{eqnarray*}
\label{分次上链复形的张量积-def}
\end{definition}

仍然要注意正负号。
容易验证$\td\circ\td=0$,从而
$((C\ten D)\downdot,\td)$确实为分次上链复形。

对于分次$K$-模,我们仍可以谈论对称张量、反对称张量:

\begin{definition}设$V$为分次$K$-模,对任意$m\geq0$,

(1)定义$m$阶\textbf{超对称张量空间}如下:
%$V$ is a graded $K$-module, then define
$$\Sym^m(V)=V^{\ten m}\big/\sim$$
其中等价关系$\sim$由以下生成:
对任意齐次元$\alpha,\beta\in V$,
$$\alpha\ten\beta\sim(-1)^{\deg\alpha\deg\beta}\beta\ten\alpha$$

(2)定义$m$阶\textbf{超反称张量空间}如下:
$$\wedgeform{m}(V):=V^{\ten m}\big/\sim$$
其中等价关系$\sim$由以下生成:
对任意齐次元$\alpha,\beta\in V$,
$$\alpha\ten\beta\sim-(-1)^{\deg\alpha\deg\beta}\beta\ten\alpha$$
\end{definition}

若$V=V_0$为通常的$K$-模,则$Sym^n(V_0)$
与$\wedgeform{n}(V_0)$即为通常的对称张量、外张量。

%我们记
%$$\Sym(V):=\bigoplus_{m\geq 0}\Sym^m(V)$$
%$$\wedgeform{*}(V):=\bigoplus_{m\geq 0}\wedgeform{m}(V)$$
%则它们有自然的分次$K$-模结构。
对于分次$K$-模$V$,以及任意的$m\geq 0$,
$\Sym^m(V)$有以下自然的分次$K$-模结构:
$$\Sym^m(V)=\bigoplus_{d\in\bbZ}
\big[\Sym^m(V)\big]_d$$
$$\big[\Sym^m(V)\big]_d
:=\Span_K\left\{
           v_1\odot v_2\odot\cdots\odot v_m
           \big|\sum\limits_{i=1}^n\deg v_i=d
         \right\}
$$
超反称张量空间$\wedgeform{m}(V)$
也有完全类似的分次$K$-模结构。

回顾分次$K$-模的平移,以下结果十分重要:

%If $V=V_0$, then $Sym^n(V_0)$ is the usual symmetric tensor,
%and $\wedgeform{n}(V_0)$ exterior.

%HW:%%下周二必考!!
\begin{prop}对于分次$K$-模$V$,
以及任意$n\geq0$,则有分次$K$-模同构:
$$\Sym^n(V[1])\cong(\wedgeform{n}(V))[n]$$
%here $[n]=([1])^{n\text{times}}$
\end{prop}
%%非常重要!!!%

\begin{proof}对于任意$d\in\bbZ$,
首先看看它们的齐次分量$(\Sym^n(V[1]))_d$
与$\big((\wedgeform{n}(V))[n]\big)_d$中的元素具有何种形式。
我们用$v_1,...,v_n$表示$V$中的$d_1,...,d_n$
次齐次元,根据定义容易验证
$$(\Sym^n(V[1]))_d=
\Span_K \{v_1\odot v_2\odot\cdots\odot v_n|
d_1+d_2+\cdots d_n=n+d\}$$
$$\big((\wedgeform{n}(V))[n]\big)_d=
\Span_K \{v_1\wedge v_2\wedge\cdots\wedge v_n|
d_1+d_2+\cdots d_n=n+d\}$$
从而它们都为$(V^{\ten n})_{n+d}$的商模。

考虑$K$-模自同构
\begin{eqnarray*}
\Phi_{n,d}:(V^{\ten n})_{n+d}&\to&(V^{\ten n})_{n+d}\\
v_1\ten v_2\ten\cdots\ten v_n
&\mapsto&(-1)^{d_1+2d_2+\cdots+nd_n}
v_1\ten v_2\ten\cdots\ten v_n
\end{eqnarray*}
断言该自同构$\Phi_{n,d}$诱导了模同构
\begin{eqnarray*}
\fai_{n,d}:(\Sym^n(V[1]))_d&\to&
\big((\wedgeform{n}(V))[n]\big)_d\\
v_1\odot v_2\odot\cdots\odot v_n
&\mapsto&(-1)^{d_1+2d_2+\cdots+nd_n}
v_1\wedge v_2\wedge\cdots\wedge v_n
\end{eqnarray*}

为此,只需要验证$\fai_{n,d}$的良定性
(与代表元选取无关)。若$\fai_{n,d}$良定,
则容易构造其逆映射,进而命题得证。

特别注意,$\Sym^n(V[1])$作为$V^{\ten n}$的商模,
商掉的等价关系由
$$x\ten y\sim(-1)^{(\deg x-1)(\deg y-1)}y\ten x$$
生成,这直接由定义验证({\color{red}要特别小心});
而$(\wedgeform{n}(V))[n]$作为$V^{\ten n}$的商模,
商掉的等价关系由
$$x\ten y\sim-(-1)^{\deg x\deg y}y\ten x$$
生成。于是只需验证对任意$1\leq l\leq n_1$,成立
\begin{eqnarray*}
&&
    \Phi_{n,d}\Big(
                 \cdots( v_{l}\ten v_{l+1}
                   -(-1)^{(d_l-1)(d_{l+1}-1)}v_{l+1}\ten v_{l}
                 )\cdots
              \Big)\\
&=&
    (-1)^{\sum\limits_{i=1}^nid_i}
    \Big(
      \cdots(
        v_l\ten v_{l+1}-(-1)^{(d_l-1)(d_{l+1}-1)-d_{l+1}+d_l}
        v_{l+1}\ten v_{l}
      )\cdots
    \Big)\\
&=&
    (-1)^{\sum\limits_{i=1}^nid_i}
    \Big(
      \cdots(
        v_l\ten v_{l+1}+(-1)^{d_ld_{l+1}}
        v_{l+1}\ten v_{l}
      )\cdots
    \Big)\\
&\equiv& 0\in ((\wedgeform{n}(V))[n])_d
\end{eqnarray*}


\end{proof}


\section{分次代数与分次李代数}

\begin{definition}(分次结合代数)
对于结合$K$-代数$A$:
\label{分次结合代数-def}
%$A$ is a $\bbZ$-graded algebra,$A=\bigoplus\limits_{k\in\bbZ}A_k$,
%such that
%$$A_k\cdot A_l\subseteq A_{k+l}$$
%and associative.

(1)称$A$为($\bbZ$-)\textbf{分次结合代数}(associative graded algebra),
\index{graded algebra\kong 分次代数}
若$A$具有分次$K$-模结构:
$$A=\bigoplus_{n\in\bbZ}A_n$$
并且与乘法相容:对任意$k,l\in\bbZ$,有
$$A_k\cdot A_l\subseteq A_{k+l}$$

(2)若$A$为分次结合代数,
称$A$为\textbf{分次交换代数},
若$A$还满足以下\textbf{分次交换性}:
对任意$a_k\in A_k,a_l\in A_l$,
$$a_k\cdot a_l=(-1)^{kl}a_l\cdot a_k$$

%$A$ is graded commutative if
%$$a_k\cdot a_l=(-1)^{kl}a_la_k$$
\end{definition}

特别注意分次交换性的正负号。
分次交换代数的典型例子是,
光滑流形$X$上的微分形式$\Omg\updot_X$,
配以外积运算$\wedge$.

不过注意,多项式代数$K[x^1,...,x^n]$自然有分次结构,
是分次代数,但它不满足分次交换性。
{\color{blue}(仅仅是“交换的分次代数”23333)}

\begin{definition}(\textbf{分次李代数})
\index{graded Lie algebra\kong 分次李代数}
\index{Lie super algebra\kong 李超代数}

$K$-代数$(\mfkg,[,])$ 称为\textbf{分次李代数}
(graded Lie algebra),
或者\textbf{李超代数}(Lie super algebra),如果以下满足:

(1) $\mfkg$具有分次
$\mfkg=\bigoplus\limits_{k\in\bbZ}\mfkg_k$,
使得对任意$k,l\in\bbZ$,成立
$$[\mfkg_k,\mfkg_l]\subseteq\mfkg_{k+l}$$

(2) 乘法$[,]:\mfkg\times\mfkg\to\mfkg$满足如下\textbf{分次反交换性}:
%which is graded skew-symmetric
对$A$中任意齐次元$a,b$,成立
$$[a,b]=-(-1)^{\deg a\deg b}[b,a]$$

(3)对于$A$中任何齐次元$a,b,c$,成立如下\textbf{分次雅可比恒等式}:
$$(-1)^{\deg b\deg c}[c,[a,b]]
 +(-1)^{\deg c\deg a}[a,[b,c]]
 +(-1)^{\deg a\deg b}[b,[c,a]]=0$$
\end{definition}
%Graded Jacobi identity
%$$[[a,b],c]=[a,[b,c]]-(-1)^{\deg a\deg b}[b,[a,c]]$$

我们可以将“分次”(graded)与“超”(super)进行同义词替换,
比如“分次雅可比恒等式”也可以称为“超雅可比恒等式”,
“分次交换性”可以称为“超交换性”等等,
甚至将“分次线性空间”称为“超空间”。

容易验证,超雅可比恒等式也可以改写为:
$$[c,[a,b]]=
[c,[a,b]]+(-1)^{\deg a\deg c}[a,[c,b]]$$

也容易验证,对于李超代数$(A,[,])$,
则$[,]$在$A$的零次分量$A_0$的限制,
给出了$A_0$的李代数结构。
回顾李代数的情形,
李括号的雅可比恒等式反映了某种导子性质;
而李超代数完全类似,
上述超雅可比恒等式其实表明某种“超导子”性质。

\begin{notation}为了省事,我们引入一个记号约定:
对于分次代数或者分次李代数(以及后文将介绍的分次模),
若$a$为其次元,我们简记
$$(-1)^a:=(-1)^{\deg a}$$
也就是说,$(-1)$的幂次当中出现齐次元的次数时,省略“$\deg$”。
\end{notation}

例如,李超代数的超雅可比恒等式可简记为
$$(-1)^{bc}[c,[a,b]]
 +(-1)^{ca}[a,[b,c]]
 +(-1)^{ab}[b,[c,a]]=0$$
或者
$$[c,[a,b]]=
[c,[a,b]]+(-1)^{ac}[a,[c,b]]$$

\begin{lemma}(分次结合代数诱导分次李代数)

设$(A,\cdot)$为分次结合代数,
则其乘法自然诱导出$A$的分次李代数结构如下:
定义
\begin{eqnarray*}
\quad[,]:A\times A &\to& A\\
    \quad     [a,b]&:= & a\cdot b-(-1)^{ab}b\cdot a
\end{eqnarray*}
其中任意$a,b\in A$为齐次元。
则$(A,[,])$构成分次李代数,
并且与$(A,\cdot)$具有相同的分次。
\label{分次结合代数诱导李超代数-lemma}
\end{lemma}
这与由通常的结合代数通过“对易子”得到李代数的方式类似,
不过要稍微注意正负号。
\begin{proof}
直接暴力验证即可,从略。注意这里的
$$(-1)^{ab}:=(-1)^{\deg a\deg b}=(-1)^{ba}$$
为偷懒的记号。
\end{proof}

我们可以考虑以分次$K$-代数的范畴:

\begin{definition}(分次结合代数范畴)
%$A$ graded $K$-algebra,
%$$A=\bigoplus_nA_n\in Mod_K^{\bbZ}$$
%$$m:A\ten A\to A$$
%degree of $m$ is $0$.
%satisfying associativity
%Category of associative graded $K$-algebra
%$$Ass-alg_K^{\bbZ}$$

我们定义如下的\textbf{分次结合$K$-代数范畴},
记为$\Assalgcat{\bbZ}{K}$:

(1)$\Obj=$全体分次结合$K$-代数;

(2)$\Mor$:对任意两个分次结合$K$-代数$A,B$,
$$\Hom(A,B):=\bigoplus_{n\in\bbZ}(\Hom(A,B))_n$$
其中
$$(\Hom(A,B))_n:=\{f\text{为$K$-代数同态}|
f(A_d)\subseteq B_{d+n}\,\forall d\in\bbZ\}$$
$(\Hom(A,B))_n$当中的元素称之为$n$次齐次$K$-代数同态。
\end{definition}

%$A\in Ass-alg_K^{\bbZ}$ is called (graded) "commutative" if
类似地,考虑\textbf{分次交换代数范畴},
它是分次结合代数范畴的全子范畴,记为
$$\Commualgcat{\bbZ}{K}$$

\begin{definition}(分次双$A$-模)

设$A=\bigoplus\limits_{n\in\bbZ}A_n$
为分次结合$K$-代数,$M$为双$A$-模,
称$M$为\textbf{分次双$A$-模},
若$M$配以分次$K$-模结构
$$M=\bigoplus_{n\in\bbZ}M_n$$
并且与$A$的模作用相容:对任意$p,q\in\bbZ$,
$$A_p.M_q\subseteq M_{p+q}$$
$$M_p.A_q\subseteq M_{p+q}$$
\end{definition}

然后对于两个分次双$A$-模$M,N$,
也可以定义何为“分次双$A$-模同态”,
并且从$M$到$N$的分次双$A$-模同态之全体,
亦有自然的分次双$A$-模结构。

特别地,对于分次$K$-代数$A$,
$A$自身有自然的分次双$A$-模结构。

%Let $A\in Ass-alg_K^{\bbZ}$,$M$ is a (graded) bi-module,
%a derivation
%$$D:A\to M$$
%is a $K$-linear map ,and satisfies
%$$D(ab)=D(a)b\pm aD(b)$$
%也可以考虑导子的degre. 不过一般都是degree0 的
%涉及正负号时,Koszul rule

\begin{definition}(导子)
对于分次$K$-代数$A$,以及分次双$A$-模$M$,
称$K$-线性同态
$$D:A\to M$$
为$A$的一个取值于$M$的\textbf{导子},
若对$A$中的任何齐次元$a,b$,成立
$$D(ab)=D(a).b+(-1)^a a.D(b)$$

我们将$A$的取值于$M$的导子之全体记为$\Der_0(A,M)$.
\end{definition}
这个定义当中并没有用到$M$的分次。
事实上对于一般的双$A$-模$M$,
我们都可以如此谈论$\Der_0(A,M)$.

导子的作用可以用如下交换图描述:

$$
  \xymatrixcolsep{5pc}
  \xymatrixrowsep{5pc}
  \xymatrix{
      A\ten A   \ar[r]^m    \ar[d]^{1\ten D+ D\ten 1}
    & A                     \ar[d]^D
  \\
      A\ten M\oplus M\ten A \ar[r]^-m
    & M
  }
$$
其中$m$表示$A$中的乘法,
然后特别注意$1\ten D$以及$D\ten 1$
在$A\ten A$上的作用服从\textbf{Koszul符号法则}
(回顾记号\ref{Koszul符号法则-notation}).


%%%%导子图%%%%%%%%搞定

\begin{definition}(超导子)

对于分次$K$代数$A$,以及$d\in\bbZ$,
称次数为$d$的分次$K$-模同态
$$D:A\to A$$
为次数为$d$的\textbf{超导子},
若满足如下的\textbf{超莱布尼茨法则}:
对$A$中任何齐次元$a,b$,成立
$$D(ab)=D(a)b+(-1)^{d\cdot\deg a}aD(b)$$

记次数为$d$的超导子之全体为$\Der(A,A)_d$,并且记
$$\Der(A,A):=\bigoplus_{d\in\bbZ}\Der(A,A)_d$$
\end{definition}

易知$\Der(A,A)$有自然的分次$K$-模结构。
超导子$D$的作用可由如下交换图来描述:

$$
  \xymatrixcolsep{5pc}
  \xymatrix{
       A\ten A  \ar[r]^-{D\ten 1+1\ten\td}  \ar[d]^-m
     & A\ten A                              \ar[d]^-m
  \\
       A        \ar[r]^-{D}
     & A
  }
$$
%%%%分次微分%%%%%%%搞定

\begin{rem}(微分分次代数)对于$K$-代数$A$,
以及次数为$1$的超导子$\td\in\Der(A,A)_1$,
如果$\td^2=0$,则
$(A,\td)$正是我们在之前
(见定义\ref{微分分次代数-def})
定义的\textbf{微分分次代数}。
\end{rem}
分次微分代数$(A,\td)$自然可视为分次上链复形。
当然我们也可以考虑次数为$-1$的超导子,
亦可定义出类似版本的分次微分代数
(不过我们更推荐使用上链复形的语言)。

\begin{lemma}(由超导子构成的李超代数)

对于分次$K$-代数$A$,
若$D_1,D_2\in\Der(A,A)$为齐次的超导子,定义
$$[D_1,D_2]:=D_1\circ D_2-(-1)^{D_1D_2}D_2\circ D_1$$
则$[D_1,D_2]$是次数为$\deg D_1+\deg D_2$的超导子。
从而我们定义了
$$[,]:\Der(A,A)\times\Der(A,A)\to\Der(A,A)$$
使得$(\Der(A,A),[,])$为李超代数。
%i.e. $(\Der(A,A),[-,-])$ is a graded Lie algebra.
\end{lemma}
%A differential graded $K$-algebra $(A\downdot,\td)$
%$\td(\alpha\cdot\beta)=(\td\alpha)\beta+(-1)^{|\alpha|}\alpha\td\beta$

\begin{proof}对于齐次超导子$D_1,D_2$,
只需要验证$[D_1,D_2]$仍然是超导子,然后由
引理\ref{分次结合代数诱导李超代数-lemma}即可知
$(\Der(A,A),[,])$为李超代数。

暴力验证之(还是写一下过程吧),
对$A$中任意齐次元$a_1,a_2$,有
\begin{eqnarray*}
& &
     [D_1,D_2](a_1a_2)
 =
     \big(
       D_1D_2-(-1)^{D_1D_2}
     \big)
     (a_1a_2)\\
&=&
     D_1\big(
          D_2(a_1)a_2+(-1)^{a_1D_2}a_1D_2(a_2)
        \big)\\
& &
    -(-1)^{D_1D_2}D_2
    \big(
      D_1(a_1)a_2+(-1)^{a_1D_1}a_1D_1(a_2)
    \big)\\
&=&
    D_1D_2(a_1)a_2
    +(-1)^{D_1(a_1+D_2)}
    {\color{red}
      D_2(a_1)D_1(a_2)
    }\\
& &
    +(-1)^{a_1D_2}
    \big[
      {\color{blue}
        D_1(a_1)D_2(a_2)
      }
      +(-1)^{a_1D_1}
      a_1D_1D_2(a_2)
    \big]\\
& &
    -(-1)^{D_1D_2}
    \Big[
      D_2D_1(a_1)a_2
      +(-1)^{D_2(D_1+a_1)}
      {\color{blue}
        D_1(a_1)D_2(a_2)
      }\\
& &
    +(-1)^{a_1D_1}
    \big(
      {\color{red}
        D_2(a_1)D_1(a_2)
      }
      +(-1)^{D_2a_1}a_1D_2D_1(a_2)
    \big)
    \Big]\\
&=&
    \big[
      D_1D_2-(-1)^{D_1D_2}D_2D_1
    \big](a_1)a_2\\
& &
    +(-1)^{a_1(D_1+D_2)}a_2
    \big[
      D_1D_2-(-1)^{D_1D_2}D_2D_1
    \big](a_2)\\
&=&
    [D_1,D_2](a_1)a_2
    +(-1)^{a_1(D_1+D_2)}a_1[D_1,D_2](a_2)
\end{eqnarray*}
可见$[D_1,D_2]$确实是次数为
$(\deg D_1+\deg D_2)$的超导子,证毕。
\end{proof}\vs

%We have natural forgetful functor
最后简要介绍一下函子性:
我们有遗忘函子
\begin{eqnarray*}
\Assalgcat{\bbZ}{K}   &\to& \Modcat{\bbZ}{K}\\
\Commualgcat{\bbZ}{K} &\to& \Modcat{\bbZ}{K}
\end{eqnarray*}
%whose left adjoints are called "free objects".

我们考虑该函子的左伴随“自由分次结合代数”
以及“自由分次交换代数”,
即范畴论当中的
{\color{gray}“普遍真理”(呵呵呵呵呵呵)}:

{\begin{center}
\Large 自由是遗忘的左伴随
\end{center}}

\begin{definition}(张量代数or自由分次结合代数)

设$V$为分次$K$-模,定义分次$K$-模
 %, define the tensor algebra
$$T(V):=%K\oplus V\oplus V^{\ten a}\oplus\cdots=
\bigoplus_{n\geq 0}V^{\ten n}$$
%with algebra structure given by $\ten$:
其中$V^{\ten 0}:=K$;
并且张量积“$\ten$”给出了$T(V)$的乘法结构:
$$(v_1\ten\cdots\ten v_p)\ten(v_{p+1}\ten\cdots\ten v_{p+q})
=v_1\ten\cdots\ten v_{p+q}$$
从而使得$(T(V),\ten)$为分次结合代数,
称之为由$V$生成的\textbf{自由分次结合代数}。
\end{definition}

这的确是一种非常“自由”的构造方式。
并且容易验证$T$的函子性:
\begin{eqnarray*}
T:\Modcat{\bbZ}{K}&\to& \Assalgcat{\bbZ}{K}\\
V&\mapsto& T(V)
\end{eqnarray*}
%\begin{prop}
%$T$ is the left adjoint of
%$$Ass-alg_K^{\bbZ}\to Mod_K^{\bbZ}$$
%\end{prop}
%这个显然,不证了。

同样,我们可以考虑自由生成的分次交换代数:

\begin{definition}(自由分次交换代数)

设$V$为分次$K$-模,定义分次$K$-模
 %, define the tensor algebra
$$\Sym(V):=%K\oplus V\oplus V^{\ten a}\oplus\cdots=
\bigoplus_{n\geq 0}\Sym^n(V)$$
%with algebra structure given by $\ten$:
其中$\Sym^0(V):=K$;
并且对称张量积“$\odot$”给出了$\Sym(V)$的乘法结构:
$$(v_1\odot\cdots\odot v_p)\odot
(v_{p+1}\odot\cdots\odot v_{p+q})
=v_1\odot\cdots\odot v_{p+q}$$
从而使得$(\Sym(V),\odot)$为分次结合代数,
称之为由$V$生成的\textbf{自由分次交换代数}。
\end{definition}

也容易验证$\Sym$的函子性:
\begin{eqnarray*}
\Sym:\Modcat{\bbZ}{K} &\to& \Commualgcat{\bbZ}{K}\\
V &\mapsto& \Sym(V)
\end{eqnarray*}

%\begin{prop}
%$Sym$ is the left adjoint of
%$$Commu-alg_K^{\bbZ}\to Mod_K^{\bbZ}$$
%\end{prop}
\begin{prop}(伴随对)
对于任意分次$K$-模$V$,以及分次结合$K$-代数$A$、
分次交换$K$-代数$B$,注意$A,B$首先是分次$K$-模:

(1)存在(关于$V$、$A$)自然的一一对应
$$\Hom_{\Modcat{\bbZ}{K}}(V,A)\cong
\Hom_{\Assalgcat{\bbZ}{K}}(T(V),A)$$

(2)存在(关于$V$、$B$)自然的一一对应
$$\Hom_{\Modcat{\bbZ}{K}}(V,B)\cong
\Hom_{\Commualgcat{\bbZ}{K}}(\Sym(V),B)$$
\end{prop}
\begin{proof}
易证,从略。
\end{proof}
用范畴论的语言,此性质表明,
函子$T$与$\Sym$分别为相应的遗忘函子的左伴随。
或者还可以表述为如下泛性质,看图即可:
$$
  \xymatrix{
      V \ar@{^{(}->}[r]  \ar[rd]_{\forall f}
    & T(V)\ar@{.>}[d]^{\exists !}
  \\
    & A
  }
\quad\quad
  \xymatrix{
      V \ar@{^{(}->}[r]  \ar[rd]_{\forall f}
    & \Sym(V)\ar@{.>}[d]^{\exists !}
  \\
    & B
  }
$$

\begin{prop}%(if $V$ is a graded $K$-module)
设$V$为分次$K$-模,$M$为$K$-模,则有一一对应
$$\Der_0(T(V),M)\cong\Hom_K(V,M)$$
\end{prop}
\begin{proof}
这个也几乎显然,从略。
\end{proof}
%In particular,
%$$\Der(T(V),T(V))\cong\Hom(V,T(V))$$

\section{多重切向量场与Schouten-Nijenhuis括号}

众所周知,对于光滑流形$X$,
$X$上的微分形式$\Omg\updot_X$配以外积$\wedge$
构成分次交换代数(若再考虑外微分$\td$,还有微分分次代数结构)。
本节我们介绍另一重要的经典例子:光滑流形上的多重切向量场,
并给出其上的李超代数结构:Schouten-Nijenhuis括号。

%\begin{example}
%(1)$(\Omg_X\updot,\wedge)$ is a graded commutative algebra.
%(2) $\text{PV}_X:=\Gamma(X,\wedge^*TX)$poly vector field,
%is graded commutative alg.
%\end{example}

\begin{definition}(多重切向量场)
对于光滑流形$X$,称$X$的切丛的外积丛
$\wedgeform{*}(TX)$的截面为\textbf{多重切向量场}
(polyvector field)。
\index{polyvector field\kong 多重切向量场}
并且记
$$\PV_X:=\Gamma(X,\wedgeform{*}(TX))$$
为多重切向量场之全体。
\end{definition}

$\PV_X$有显然的$C^{\infty}(X)$-模结构。
与微分形式类似,容易定义$\PV_X$上的外积$\wedge$,
使得$(\PV_X,\wedge)$为分次交换$C^{\infty}(X)$-代数,
其分次由以下给出:
$$\PV_X=\bigoplus_{k\geq 0}\PV^k_X$$
其中$\PV^k_X$中的元素形如
$$\xi_1\wedge\xi_2\wedge\cdots\wedge\xi_k$$
的$C^{\infty}(X)$-线性组合,
其中$\xi_i$为$X$上的光滑切向量场。
称$\PV^k_X$中的元素为\textbf{$k$-向量}。

回顾$X$的切向量场的李括号$[,]$运算,
这给出了切向量场的李代数结构;
接下来我们企图将李括号运算延拓到多重切向量场上,
从而得到$\PV_X[1]$的李超代数结构。
(注意这里要平移一下分次,使得把切向量场视为零次元。)

\begin{definition}(Schouten-Nijenhuis 括号)
\index{Schouten-Nijenhuis 括号}

对于光滑流形$X$,定义$\PV_X$上的$\bbR$-双线性映射
\begin{eqnarray*}
\{,\}:\PV^p_X\times \PV^q_X &\to& \PV^{p+q-1}_X\\
\{f,g\}     &=&0\\
\{f,\xi\}   &=&   (-1)^{p}\{\xi,f\}=
                  \sum_{k=1}^p(-1)^{k}
                  \xi_k(f)(\cdots\wedge\widehat{\xi_k}
                  \wedge\cdots)\\
\{\xi,\eta\}&=& \sum_{1\leq i\leq p\atop 1\leq j\leq q}(-1)^{i+j}
                  [\xi_i,\eta_j]\wedge
                  (\cdots\wedge\widehat{\xi_i}
                  \wedge\cdots)
                  \wedge(\cdots\wedge\widehat{\eta_j}
                  \wedge\cdots)
\end{eqnarray*}
其中任意$f,g\in C^{\infty}(X)=\PV^0_X$以及
\begin{eqnarray*}
\xi=\xi_1\wedge\cdots\wedge\xi_p\quad\quad
\eta=\eta_1\wedge\cdots\wedge\eta_q
\end{eqnarray*}
\end{definition}
%Check:
我们需要验证$\{,\}$的良定性:
与$\xi,\eta$的代表元的选取无关。
这只需暴力验证即可。从略。

\begin{prop}(Schouten-Nijenhuis括号的性质)

对于光滑流形$X$,则$(\PV_X,\{,\})$满足如下性质:

(1)若$\xi,\eta\in\PV^1_X$为通常的切向量场,则
$$\{\xi,\eta\}=[\xi,\eta]$$

(2)对任意$p,q\geq 0$,
任意$\xi\in\PV^p_X$以及$\eta\in\PV_X^q$,成立
$$\{\xi,\eta\}=-(-1)^{(p-1)(q-1)}\{\eta,\xi\}$$

(3)对任意$p,q,r\geq 0$,
任意$\xi\in\PV^p_X$以及$\eta\in\PV_X^q,\phi\in\PV_X^r$,成立
$$\{\xi,\eta\wedge\phi\}=\{\xi,\eta\}\wedge\phi
+(-1)^{(p-1)q}\eta\wedge\{\xi,\phi\}$$
\label{Schouten-Nijenhuis公理-prop}
\end{prop}

%(1) $\{,\}$ is well-defined and coordinate independent.(HW)

%(2) $\{\xi,\eta\}=-(-1)^{(\deg\xi-1)(\deg\eta-1)}\{\eta,\xi\}$
%%%%%%%%%Details%%%%%%

%\begin{notation}
%$A$ is graded, then $(A[1])_n:=A_{n-1}$ shifted gradation...
%\end{notation}

%So, $PV_X[1]$ is a graded Lie algebra, and
%$$(PV_X[1])_0=T_X$$
%is a Lie algebra.

%(3) graded  Leibnitz rule
%$$\{\alpha,\beta\wedge\gamma\}
%=\{\alpha,\beta\}\gamma+(-1)^{(\deg\alpha-1)\deg\beta}
%\beta\{\alpha,\gamma\}$$
%(挪之前的分次)

%\begin{prop}

%(1)$(PV,\wedge)$ graded algebra

%(2)$(PV[1],\{,\})$ graded Lie alg

%(3) (1)(2)is compactible(Leibnitz rule)

%\end{prop}

\begin{proof}
(1)(2)容易验证,从略;
(3)暴力验证,建议使用数学归纳法。从略。
\end{proof}

事实上,这三条性质可作为Schouten-Nijenhuis括号的公理:
满足此性质的括号如果存在,只能如此定义。
例如,对任意的$f\in\PV^0_X=C^{\infty}(X)$,
以及$\xi,\eta\in\PV^1_X$,如果$\{,\}$满足上述三条性质,那么
$$\{\eta,f\xi\}=\{\eta,f\wedge\xi\}=
\{\eta,f\}\xi+(-1)^{(1-1)\times 0}f\{\eta,\xi\}
=\{\eta,f\}\xi+f[\eta,\xi]$$
但另一方面,又有
$$\{\eta,f\xi\}=[\eta,f\xi]=\eta(f)\xi+f[\eta,\xi]$$
比较两式,从而必有
$$\{\eta,f\}=\eta(f)$$
再反复使用超莱布尼茨法则(3)
以及超反称性(2),
即可得到$\{,\}$的完整定义。

\begin{prop}对于光滑流形$X$,
Schouten-Nijenhuis括号$\{,\}$满足如下超雅可比恒等式:
对任意$p,q,r\geq0$以及任意
$\xi\in\PV^p_X,\,\eta\in\PV_X^q,\,\phi\in\PV_X^r$,成立
$$
  \{\xi,\{\eta,\phi\}\}=
  \{\{\xi,\eta\},\phi\}+(-1)^{(p-1)(q-1)}
  \{\eta,\{\xi,\phi\}\}
$$
从而$(\PV_X[1],\{,\})$构成李超代数。
\end{prop}

\begin{proof}我们打算详细写出过程。
在证明的过程中,我们将反复使用性质
\ref{Schouten-Nijenhuis公理-prop}.
对任意的$p,q,r>0$,任取
$$\xi=\xi_1\wedge\cdots\wedge\xi_p\in\PV_X^p$$
$$\eta=\eta_1\wedge\cdots\wedge\eta_q\in\PV_X^q$$
$$\phi=\phi_1\wedge\cdots\wedge\phi_r\in\PV^r_X$$
$$f,g,h\in\PV^0_X=C^{\infty}(X)$$
为方便书写,我们引入如下记号:
$$\begin{array}{rclr}
   \overline{\xi_i} &:=&
   \xi_1\wedge\cdots\wedge\widehat{\xi_i}\wedge\cdots\wedge\xi_p
   &\forall \,1\leq i\leq p
\\
   \overline{\eta_j} &:=&
   \eta_1\wedge\cdots\wedge\widehat{\eta_j}\wedge\cdots\wedge\eta_q
   &\forall \,1\leq j\leq q
\\
   \overline{\phi_k} &:=&
   \phi_1\wedge\cdots\wedge\widehat{\phi_k}\wedge\cdots\wedge\phi_r
   &\forall \,1\leq k\leq r
\end{array}$$
我们将对$p$归纳。

\textbf{预备情形}:若$p=q=r=0$,则结论平凡。
若$p,q,r$当中恰有两个为$0$,不妨$p=q=0$,此时$r>0$,从而任取
$f,g\in\PV^0_X$以及$\phi\in\PV^r_X$,此时的超雅可比恒等式为
$$\{f,\{g,\phi\}\}+\{g,\{f,\phi\}\}=0$$
注意到
$$\{f,\phi\}=\sum_{k=1}^r(-1)^k\phi_k(f)\overline{\phi_k}$$
从而有
\begin{eqnarray*}
     \{f,\{g,\phi\}\}
&=&  \{f,
         \sum_{k=1}^r
           (-1)^k\phi_k(g)\overline{\phi_k}
     \}\\
&=&
     \sum_{k=1}^r\phi_k(g)
       \left(
         \sum_{j<k}(-1)^j
           \phi_j(f)\overline{\phi_{jk}}
        +\sum_{j>k}(-1)^{j+1}
           \phi_j(f)\overline{\phi_{kj}}
       \right)\\
&=&
     \sum_{1\leq i<j\leq r}(-1)^{i+j}
       \big[
         \phi_i(g)\phi_j(f)-\phi_j(g)\phi_i(f)
       \big]
       \overline{\phi_{ij}}
\end{eqnarray*}
其中对于$j<k$,简写记号
$$\overline{\phi_{jk}}
:=\cdots\wedge\widehat{\phi_j}
\wedge\cdots\wedge\widehat{\phi_k}\wedge\cdots$$
注意观察上式关于$f,g$的反对称性,
容易发现$\{f,\{g,\phi\}\}=-\{g,\{f,\phi\}\}$,从而证毕。

于是,我们在接下来的证明中,
不妨$p,q,r$当中为零的至多只有一个。\vs


\textbf{$p$的起始步:}现在开始对$p$归纳,
首先考虑起始步$p=0$.由之前讨论,不妨$q,r>0$.
任取$f\in\PV^0_X$,只需证
$$\{f,\{\eta,\phi\}\}=\{\{f,\eta\},\phi\}
+(-1)^{q-1}\{\eta,\{f,\phi\}\}\eqno{(*)}$$

暴力展开验证之,注意到
\begin{eqnarray*}
     \{f,\{\eta,\phi\}\}
&=&
     \sum_{j=1}^q
       \sum_{k=1}^r
         (-1)^{j+k}
         \{
           f
          ,
           [\eta_j,\phi_k]
           \wedge\overline{\eta_j}
           \wedge\overline{\phi_k}
         \}\\
&=&
    -\sum_{j=1}^q
       \sum_{k=1}^r
         (-1)^{j+k}
         [\eta_j,\phi_k](f)
               \overline{\eta_j}
         \wedge\overline{\phi_k}\\
& &
    -\sum_{j=1}^q
       \sum_{k=1}^r
         (-1)^{j+k}
         [\eta_j,\phi_k]\wedge
         \{
           f
         ,
           \overline{\eta_j}\wedge\overline{\phi_k}
         \}
\end{eqnarray*}
再打开$(*)$的右边:
\begin{eqnarray*}
     \{\{f,\eta\},\phi\}
&=&
     \sum_{j=1}^q(-1)^j
       \{\eta_j(f)\overline{\eta_j},\phi\}\\
&=&
     \sum_{j=1}^q(-1)^j
       \Big[
         (-1)^{(q-1)(r-1)}
         \{\eta_j(f),\phi\}\wedge
         \overline{\eta_j}
        +\eta_j(f)\{\overline{\eta_j},\phi\}
       \Big]\\
&=&
     \sum_{j=1}^q
       \sum_{k=1}^r
         (-1)^{j+k}
         \phi_k(\eta_j(f))
         \overline{\eta_j}\wedge\overline{\phi_k}\\
& &
    +\sum_{j=1}^q(-1)^j
       \eta_j(f)\{\overline{\eta_j},\phi\}
\end{eqnarray*}
\begin{eqnarray*}
     (-1)^{q-1}\{\eta,\{f,\phi\}\}
&=&
     (-1)^{q-1}
     \sum_{k=1}^r(-1)^k
       \{\eta,\phi_k(f)\overline{\phi_k}\}\\
&=&
     (-1)^{q-1}
     \sum_{k=1}^r(-1)^k
       \Big[
         \{\eta,\phi_k(f)\}\wedge
         \overline{\phi_k}
        +(-1)^{q-1}
         \phi_k(f)\{\overline{\eta,\overline{\phi_k}}\}
       \Big]\\
&=&
    -\sum_{j=1}^q
       \sum_{k=1}^r
         (-1)^{j+k}
         \eta_j(\phi_k(f))
         \overline{\eta_j}\wedge\overline{\phi_k}\\
& &
    +\sum_{k=1}^r(-1)^k
       \phi_k(f)\{\eta,\overline{\phi_k}\}
\end{eqnarray*}
因此有
\begin{eqnarray*}
     \{\{f,\eta\},\phi\}+(-1)^{q-1}\{\eta,\{f,\phi\}\}
&=&
     \sum_{j=1}^q
       \sum_{k=1}^r
         (-1)^{j+k}
         (\phi_k(\eta_j(f))-\eta_j(\phi_k(f)))
         \overline{\eta_j}\wedge\overline{\phi_k}\\
& &
    +\sum_{j=1}^q(-1)^j
       \eta_j(f)\{\overline{\eta_j},\phi\}
    +\sum_{k=1}^r(-1)^k
       \phi_k(f)\{\eta,\overline{\phi_k}\}\\
&=&
    -\sum_{j=1}^q
       \sum_{k=1}^r
         (-1)^{j+k}
         [\eta_j,\phi_k](f)
               \overline{\eta_j}
         \wedge\overline{\phi_k}\\
& &
    +\sum_{j=1}^q(-1)^j
       \eta_j(f)\{\overline{\eta_j},\phi\}
    +\sum_{k=1}^r(-1)^k
       \phi_k(f)\{\eta,\overline{\phi_k}\}\\
\end{eqnarray*}

与$(*)$式比较,只需要再验证恒等式
$$
    -\sum_{j=1}^q
       \sum_{k=1}^r
         (-1)^{j+k}
         [\eta_j,\phi_k]\wedge
         \{
           f
         ,
           \overline{\eta_j}\wedge\overline{\phi_k}
         \}
 =
     \sum_{j=1}^q(-1)^j
       \eta_j(f)\{\overline{\eta_j},\phi\}
    +\sum_{k=1}^r(-1)^k
       \phi_k(f)\{\eta,\overline{\phi_k}\}
$$
即可。而这只需将式子中的Schouten-Nijenhuis括号暴力展开,
并且适当更改求和指标即可,从略。
{\color{red}(不太想写了,打字好累2333)}\vs


\textbf{$p$的归纳步:}
如果该命题对$p$成立,则考虑
$$\xi':=\xi_0\wedge\xi\in\PV_X^{p+1}$$
其中任意$\xi_0\in\PV^0_X$. 我们只需证
$$\{\xi',\{\eta,\phi\}\}=
\{\{\xi',\eta\},\phi\}
+(-1)^{p(q-1)}\{\eta,\{\xi',\phi\}\}\eqno{(**)}$$
注意反复使用$\{,\}$的超反对称性、超莱布尼茨法则,
以及关于$p$的归纳假设,我们简单(但暴力)验证如下:
\begin{eqnarray*}
     \text{$(**)$左边}
&=&
     \{\xi_0\wedge\xi,\{\eta,\phi\}\}\\
&=&
     (-1)^p\xi\wedge
     \{
       \xi_0
      ,
       \{
          \eta
        ,
          \phi
       \}
     \}+\xi_0\wedge
     \{
       \xi
      ,
       \{\eta,\phi\}
     \}\\
&=&
     (-1)^p\xi\wedge
     \Big(
       \{\{\xi_0,\eta\},\phi\}
      +\{\eta,\{\xi_0,\phi\}\}
     \Big)\\
& &
    +\xi_0\wedge
     \Big(
       \{\{\xi,\eta\},\phi\}
      +(-1)^{(p-1)(q-1)}
       \{\eta,\{\xi,\phi\}\}
     \Big)
\end{eqnarray*}
其中最后一步等号用到了归纳假设。
再看$(**)$右边,需要格外小心正负号:
\begin{eqnarray*}
     \text{$(**)$右边}
&=&
     (-1)^{p(q-1)+(p+q-1)(r-1)}
     \{\phi,\{\eta,\xi_0\wedge\xi\}\}
    -(-1)^{p(q-1)+p(r-1)}
     \{\eta,\{\phi,\xi_0\wedge\xi\}\}\\
&=&
     (-1)^{p(q-1)+(p+q-1)(r-1)}
     \Big\{
       \phi
     ,
       {\eta,\xi_0}\wedge\xi
      +(-1)^{q-1}\xi_0\wedge\{\eta,\xi\}
     \Big\}\\
& &
    -(-1)^{p(q+r)}
     \Big\{
            \eta
     ,
            \{\phi,\xi_0\}\wedge\xi
           +(-1)^{r-1}
            \xi_0\wedge\{\phi,\xi\}
     \Big\} \\
&=&
     (-1)^{p(q-1)+(p+q-1)(r-1)}\\
& &
     \Big[
       \{\phi,\{\eta,\xi_0\}\}\wedge\xi
      +(-1)^{(r-1)q}
        {\color{red}
         \{\eta,\xi_0\}\wedge\{\phi,\xi\}
        }\\
& &
       \quad
      +(-1)^{q-1}
       \big(
         {\color{blue}
           \{\phi,\xi_0\}\wedge\{\eta,\xi\}
         }
        +(-1)^{r-1}
         \xi_0\wedge\{\phi,\{\eta,\xi\}\}
       \big)
     \Big]\\
& &
     -(-1)^{p(q+r)}
     \Big[
       \{\eta,\{\phi,\xi_0\}\}\wedge\xi
      +(-1)^{(q-1)r}
       {\color{blue}
         \{\phi,\xi_0\}\wedge\{\eta,\xi\}
       }\\
& &
       \quad
      +(-1)^{r-1}
       \big(
         {\color{red}
           \{\eta,\xi_0\}\wedge\{\phi,\xi\}
         }
        +(-1)^{q-1}
         \xi_0\wedge\{\eta,\{\phi,\xi\}\}
       \big)
     \Big]\\
&=&
     (-1)^p\xi\wedge
     \Big(
       \{\{\xi_0,\eta\},\phi\}
      +\{\eta,\{\xi_0,\phi\}\}
     \Big)\\
& &
    +\xi_0\wedge
     \Big(
       \{\{\xi,\eta\},\phi\}
      +(-1)^{(p-1)(q-1)}
       \{\eta,\{\xi,\phi\}\}
     \Big)\\
&=&
     \text{$(**)$左边}
\end{eqnarray*}
从而$(**)$两边相等,归纳完毕。
\end{proof}

归纳步当中主要是在验证正负号。


\section{Shuffle乘积}

对于$n\geq 1$,我们记$S_n$为$n$元对称群。

\begin{definition}($(p,q)$-Shuffle)

对于正整数$p,q$,称$\sgm\in S_{p+q}$为一个
\textbf{$p,q$-Shuffle},如果满足
%Let $S_n$ be the symmetric group, A (p,q)-Shuffle is
% a permutation $\sigma\in S_{p+q}$ such that
$$\sigma(1)<\sigma(2)<...<\sigma(p)$$
$$\sigma(p+1)<\sigma(p+2)<...<\sigma(q)$$
%Let
全体$(p,q)$-Shuffle构成的集合记为$\Sh_{p,q}$.
\end{definition}

{\color{purple}
笔者建议将“Shuffle”意译为“\textbf{洗牌}”——
因为$\Sh_{p,q}$中的置换,好比将$p+q$张扑克牌
分为$p$张、$q$张两组来洗牌。
}

容易知道,集合$\Sh_{p,q}$的元素个数为
$$\frac{(p+q)!}{p!q!}$$

回到Hochschild链复形。
我们已经知道,$\HH\downdot(A)$
是通常的光滑流形的微分形式$\Omg\updot_X$的非交换版本。
而对于微分形式$\Omg\updot_X$,
其上有外积$\wedge$使之构成分次交换代数;
我们也企图去定义非交换版本的外积。

%Let $A,A'$ be to $K$-algebras,
%$M,M'$ are $A,A'$-bimodule. We define the Shuffle product $\times$

\begin{definition}(Shuffle 乘积)
\index{shuffle product}%%这屌东西怎么翻译呀

设$A,A'$为$K$-代数,$M,M'$分别为双$A,A'$-模,
我们定义如下的运算$\times$,称之为\textbf{Shuffle乘积}:
\begin{eqnarray*}
C_p(A,M)\times C_q(A',M')&\to& C_{p+q}(A\ten A',M\ten M')\\
(m,a_1,...,a_p)\times(m',a_1',...,a_q')&\mapsto&
\sum_{\sigma\in Sh_{p,q}}
  (-1)^{|\sigma|}
  (m\ten m',\sigma(a_1,...,a_p,a_1',...,a_q'))
\end{eqnarray*}
其中$|\sgm|:=\sgn\sgm$为置换的符号,
$(m,a_1,...,a_n):=m\ten a_1\ten\cdots\ten a_n$为简单的记法;
并且置换群$S_{p+q}$在$A^{\ten (p+q)}$上的作用为
$$\sigma(a_1,...,a_q')
:=(a_{\sigma^{-1}(1)},...,a_{\sgm^{-1}(q')})$$
\end{definition}

%要取逆。。。因为是左作用%
注意在这里,群$S_{p+q}$在$A^{\ten(p+q)}$上的作用方式,
下角标中出现的是“$\sgm^{-1}$”,如此规定是为了保证
$S_{p+q}$的作用是左作用。

还要注意一点,$A,A'$以及$M,M'$并不被假定有分次结构,
$A\ten A'$在$M\ten M'$上的模作用是通常的
$$(a\ten a').(m\ten m')=(a.m)\ten(a'.m')$$
右模作用也类似,并不会出正负号,Koszul符号法则在此平凡。

还要注意,
$$C_{p+q}(A\ten A',M\ten M')
\cong (M\ten M')\ten (A\ten A')^{\ten(p+q)}$$
之中元素“$(m\ten m',a_1,...,a_p,a_1',...,a_q')$”
里面的“$a_i$”应该是$a_i\ten 1\in A\ten A'$,
以及“$a_j'$”应该是$1\ten a_j'\in A\ten A'$.

还要注意$A\ten A'$的乘法满足$a_ia_j'=a_j'a_i$交换。

\begin{prop}(Shuffle乘积与Hochschild边缘算子相容)
%The Shuffle product $\times$ is compatible with
%Hochschild differential $b$: i.e.

记号同上,则对于任意$x\in C_p(A,M)$,$y\in C_q(A',M')$,成立
$$b(x\times y)=
b(x)\times y+(-1)^{\deg x}
x\times b(y)$$
其中$b:C\downdot\to C_{\bullet-1}$
为相应Hochschild链复形各自的边缘算子。
\end{prop}

\begin{proof}当然是暴力验证了,
我们尽可能使用简练的记号。令
$$x=m \ten(a_1 ,...,a_p )$$
$$y=m'\ten(a_1',...,a_q')$$
首先看$b(x\times y)$,我们按张量缩并的位置,
以及$A\ten A'$当中的“元素类型”
(形如$1\ten A'$或是$A\ten 1$)来分类,强行将它打开:
\begin{eqnarray*}
     b(x\times y)
&=&
     b\sum_{\sgm\in\Sh_{p,q}}(-1)^{|\sgm|}
        (m\ten m')\ten \sgm(a_1,...,a_p,a_1',...,a_q')\\
&=&
     (m.a_1\ten m')\ten
     \sum_{\sgm\in\Sh_{p-1,q}}(-1)^{|\sgm|}
       \sgm(a_2,...,a_p,a_1',...,a_q')\\
& &
    +(m\ten m'.a_1')\ten
     \sum_{\sgm\in\Sh_{p,q-1}}(-1)^{|\sgm|+p}
       \sgm(a_1,...,a_p,a_2',...,a_q')\\
& &
    +\sum_{i=1}^{p+q-1}(-1)^i(m\ten m')\ten
       \Bigg[
         \sum_{1\leq\alpha<\beta\leq p
           \atop\sgm\in\Sh_{p-2,q}}
             (-1)^{|\sgm|+\alpha+\beta-1}
               \sgm(...,\underbrace{(a_{\alpha}a_{\beta})}
                        _{\text{第$i,i+1$个}},...)\\
& &
         {\color{red}
         \quad\quad
        +\sum_{1\leq\alpha \leq p,\, 1\leq\gamma\leq q
           \atop\sgm\in\Sh_{p-1,q-1}}
             (-1)^{|\sgm|+\alpha+\gamma+p-1}
               \sgm(...,\underbrace{(a_{\alpha}a_{\gamma}')}
                        _{\text{第$i,i+1$个}},...)
         }\\
& &
         {\color{red}
         \quad\quad
        +\sum_{1\leq\alpha \leq p,\, 1\leq\gamma\leq q
           \atop\sgm\in\Sh_{p-1,q-1}}
             (-1)^{|\sgm|+\alpha+\gamma+p}
               \sgm(...,\underbrace{(a_{\gamma}'a_{\alpha})}
                        _{\text{第$i,i+1$个}},...)
         }\\
& &
         \quad\quad
        +\sum_{1\leq\gamma<\delta\leq q
           \atop\sgm\in\Sh_{p,q-2}}
             (-1)^{|\sgm|+(\gamma+p)+(\delta+p)-1}
               \sgm(...,\underbrace{(a_{\gamma}'a_{\delta}')}
                        _{\text{第$i,i+1$个}},...)
       \Bigg]\\
& &
    +(-1)^{p+q}(a_n.m\ten m')
     \sum_{\sgm\in\Sh_{p-1,q}}
       (-1)^{|\sgm|+q}
       (a_1,...,a_{p-1},a_1',...,a_q')\\
& &
    +(-1)^{p+q}(m\ten a_n'.m')
     \sum_{\sgm\in\Sh_{p,q-1}}
       (-1)^{|\sgm|}
       (a_1,...,a_p,a_1',...,a_{q-1}')\\
&=&
     [m.a_1\ten(a_2,...,a_p)]\times y
    +(-1)^px\times[m'.a_1'\ten(a_2',...,a_q')]\\
& &
    +\left(
       \sum_{i=1}^{p-1}(-1)^i
         m\ten(a_1,...,a_ia_{i+1},...,a_p)
     \right)\times y\\
& &
    +(-1)^px\times
     \left(
       \sum_{i=1}^{q-1}(-1)^i
         m'\ten(a_1',...,a_i'a_{i+1}',...,a_q')
     \right)\\
& &
    +(-1)^p[a_p.m\ten (a_1,...,a_{p-1})]\times y
    +(-1)^{p+q}x\times[a_q'.m'\ten(a_1',...,a_{q-1}')]\\
&=&
    b(x)\times y+(-1)^px\times b(y)
\end{eqnarray*}
从而证毕。
\end{proof}

\begin{cor}对于$K$-代数$A,A'$,
若$M,M'$分别为双$A,A'$-模,则Shuffle乘积诱导了
链映射:%we get a chain complex map
$$C\downdot(A,M)\ten C\downdot(A',M')
\xra{\times} C\downdot(A\ten A',M\ten M')$$
并且该链映射诱导了Hochschild同调的同态%pass to homology, we get
$$H\downdot(A,M)\ten H\downdot(A',M')
\xra{\times}
H\downdot(A\ten A',M\ten M')$$
\end{cor}

\begin{proof}
仅仅是将
$$b(x\times y)=
b(x)\times y+(-1)^{\deg x}
x\times b(y)$$
换了一种说法。注意链复形张量积
$$C\downdot(A,M)\ten C\downdot(A',M')$$
的边缘算子服从与定义
\ref{分次上链复形的张量积-def}
类似的规则。Shuffle乘积在Hochschild
同调的下降的良定性也易证。
\end{proof}

特别地,当$M=A,M'=A'$时%In particular,
Shuffle乘积诱导了同态:
$$\HH\downdot(A)\ten \HH\downdot(A')
\xra{\times} \HH\downdot(A\ten A')$$

\begin{thm}(K\"{u}nneth 公式)%Forumla)

%Assume $a,A'$ are flat over $K$,
%then Shuffle product gives an isomorphism
如果$K$-代数$A,A'$作为$K$-模都是平坦的,
那么Shuffle乘积诱导的Hochschild同调之间的同态
$$\HH\downdot(A)\ten\HH\downdot(A')
\xra{\times}\HH\downdot(A\ten A')$$
为同构。
\end{thm}
\begin{proof}
代数拓扑中的标准证明。从略。
\end{proof}

注意我们总是假定$A,A'$是投射$K$-模,
从而自然满足平坦性。

%不证了,标准的证明
%%%%%%%代数映射%%%%%%%搞定

\begin{prop}
%Let $A$ be a commutative associative algebra, then
若$K$-代数$A$是交换的,则Shuffle乘积诱导了
$\HH\downdot(A)$ %is a graded commutative algebra.
的$K$-代数结构,并且$(\HH\downdot(A),\times)$
是分次交换的。
\end{prop}
这就回到了通常交换的情形了。
事实上Shuffle乘积的几何意义为微分形式的外积。

\begin{proof}对于任意的$K$-代数$A$,注意
张量积$A\ten A$也有$K$-代数结构。断言:
\begin{eqnarray*}
\pi:A\ten A&\to& A\\
a\ten b&\mapsto& ab
\end{eqnarray*}
为$K$-代数同态当且仅当$A$是交换代数。
于是当$A$交换时,考虑$K$-代数同态$A\ten A\xra{\pi}A$,
由Hochschild同调的函子性,该同态诱导了
$$\HH\downdot(A\ten A)\xra{\HH\downdot(\pi)}\HH\downdot(A)$$
再注意Shuffle乘积诱导
$$\HH\downdot(A)\ten\HH\downdot(A)\xra{\times}\HH\downdot(A\ten A)$$
将上述两个同态复合,即得到分次$K$-代数
$(\HH\downdot(A),\times)$.

容易验证此代数为分次交换的,验证如下:
对任意$p,q\geq 0$以及$a_0,a_1,...,a_p\in A$,
$a_0',a_1',...,a_q'\in A$,有
\begin{eqnarray*}
& &
     [a_0\ten(a_1,...,a_p)]
     \times[a_0'\ten(a_1',...,a_q')]\\
&=&
     (a_0a_0')\ten
     \sum_{\sgm\in\Sh_{p,q}}(-1)^{|\sgm|}
       \sgm(a_1,...,a_p;a_1',...,a_q')\\
&=&
     (a_0'\ten a_0)\ten
     \sum_{\sgm\in\Sh_{q,p}}
       (-1)^{|\sgm|+pq}
       \sgm(a_1',...,a_q';a_1,...,a_p)\\
&=&
     (-1)^{pq}
     [a_0'\ten(a_1',...,a_q')]
     \times[a_0\ten(a_1,...,a_p)]
\end{eqnarray*}
从而分次交换。
\end{proof}

%HW: 
\begin{prop}(Shuffle乘积是非交换版本的外积)
%If $A=K[x^i]$, then

若$A=K[x^1,...,x^n]$为多项式代数,则有$K$-代数同构
$$(\HH\downdot(A),\times)\cong(\Omg_A\updot,\wedge)$$
\end{prop}

\begin{proof}
回顾性质\ref{C[x^i]的循环同调}当中的双复形同态
$$\Phi:(\overline{\mcalB}\ddowndot(A),b,\mcalB)
\to(\Omg\updot_A,0,\td)$$

首先,容易验证Shuffle乘积可下降至
$$\overline{C}\downdot(A)\ten\overline{C}\downdot(A)
\xra{\times}\overline{C}\downdot(A)$$
从而我们只需要验证对于任意$x\in\overline{C}_p(A)$以及
$y\in\overline{C}_q(A)$,成立
$$\Phi(x\times y)=\Phi(x)\wedge\Phi(y)$$

对任意的$p,q\geq 0$,以及
$x=a_0\ten(\overline{a_1},...,\overline{a_p})\in\overline{C}_p(A)$,
$y=a_0'\ten(\overline{a_1}',...,\overline{a_q}';)\in\overline{C}_q(A)$,
从而
\begin{eqnarray*}
     \Phi(x\times y)
&=&
     \Phi\left(
            a_0a_0'\ten
            \sum_{\sgm\in\Sh_{p,q}}(-1)^{|\sgm|}
              \sgm(\overline{a_1},...,\overline{a_p};
                   \overline{a_1}',...,\overline{a_q}')
         \right)\\
&=&
     \frac{1}{(p+q)!}a_0a_0'
     \sum_{\sgm\in\Sh_{p,q}}(-1)^{|\sgm|}
       \sgm(\td a_1\wedge\cdots\wedge \td a_p
         \wedge \td a_1'\wedge\cdots\wedge\td a_q')\\
&=&
     \frac{1}{p!q!}
     a_0a_0'\td a_1\wedge\cdots\wedge \td a_p
     \wedge \td a_1'\wedge\cdots\wedge\td a_q'\\
& &
     \\
     \Phi(x)\wedge\Phi(y)
&=&
     \left(
       \frac{1}{p!}
       a_0\td a_1\wedge\cdots\wedge \td a_p
     \right)
     \wedge
     \left(
       \frac{1}{q!}
       a_0'\td a_1'\wedge\cdots\wedge \td a_q'
     \right)\\
&=&
     \frac{1}{p!q!}
     a_0a_0'\td a_1\wedge\cdots\wedge \td a_p
     \wedge \td a_1'\wedge\cdots\wedge\td a_q'
\end{eqnarray*}
从而得证。
\end{proof}


\section{Cup乘积}
对于$K$-代数$A$,回顾Hochschild上链复形
$$C\updot(A):=C\updot(A,A)=\bigoplus_{p\geq 0}C^p(A,A)$$
我们将介绍$C\updot(A)$上的代数结构:cup乘积。

%%%%%%%%%%%%%树%%%%%%%%%

\begin{definition}(Cup乘积)
%%%%%%%shadiao%%%%
\index{cup product}

设$A$为$K$-代数,定义\textbf{cup乘积}
$$\cup:C\updot(A)\ten C\updot(A)\to C\updot(A)$$
如下:%For 
对任意的$f\in C^p(A)$,$g\in C^q(A)$,% Define cup product
\begin{eqnarray*}
f\cup g &\in& C^{p+q}(A)\\
(f\cup g)(a_1,...,a_p;a_{p+1},...,a_q)
&=&
(-1)^{pq}f(a_1,...,a_p)g(a_{p+1},...,a_{p+q})
\end{eqnarray*}
\end{definition}

我们已经知道,Hochschild上同调是“非交换版本的多重切向量场”。
注意多重切向量场具有外积运算,而cup乘积则是“非交换版本的外积”。
我们将去说明这一点。

\begin{prop}(cup乘积与Hochschild微分的相容性)
%Cup product is compatible with Hochschild differential $\p$:

对于$K$-代数$A$,任意$f,g\in C\updot(A)$,成立
$$\p(f\cup g)=(\p f)\cup g+(-1)^{\deg f}f\cup\p g$$
\end{prop}

%%%%%%%这岂不是吃屎%%%%%%%

\begin{cor}
There is a well-defined cup product
$$\cup: H^p(A,A)\times H^{q}(A,A)\to H^{p+q}(A,A)$$
\end{cor}

HW: If $A=\bbC[x^i]$,then
$$(H\updot(A,A),\cup)\cong (PV_A,\wedge)$$

\section{Gerstenhaber 乘积}
\textbf{Gerstenhaber algebra(自己查定义)}
Gerstenhaber algebra(Wiki, HW)
(in physics, Classical BV algebra)

\begin{definition}Gershenhaber product
$$C^{p}(A,A)\times C^q(A,A)\to C^{p+q-1}(A,A)$$
$$(f,g)\to f\circ g$$

%%%%%%%Gerstenhaber%%%%%%%%%5
\end{definition}

\begin{prop}
$$\p(f\circ g)-(\p f)\circ g-(-1)^{\deg g-1}f\circ\p g
=\pm(f\cup g-(-1)^{\deg f\deg g}g\cup f)$$
$$=\text{the failure of $\circ$ being a chain map
is measured bu the commutativity of cup product}$$
\end{prop}

\begin{proof}
%%%%%%%暴力计算,需要画很多图,从略%%%%%%55
HW
\end{proof}

\begin{cor}
$(\H\updot(A,A),\cup)$ is a graded commutative algebra.
\end{cor}

\begin{proof}
Omit.
\end{proof}

%%注记:很深刻的结论。。。。%%%
%%

\begin{definition}(Cerstenhaber bracket)
$$\{f,g\}=f\circ g-(-1)^{(f-1)(g-1)}g\circ f$$
%这是非交换版本的NR括号
\end{definition}

\begin{prop}
$$\p\{f,g\}=\{\p f,g\}\pm\{f,\p g\}$$
and induces
$\{,\}$ defines on $H\updot(A,A)$.
this is  the analogue of Schouten-Nijenhuis bracket.
\end{prop}

%%%%%%%%%%%%2019.3.12第三周Mon%%%%%%%%%%%%%%%%%%%%%%%
%%%%%%%%%%%%%%%%下周一,周坚代课%%&%%%%%%%%%%%%%%%%%%
%%%%%%%%%%%%%%%%下周二,随堂考试%%%%%%%%%%%%%%%%%%%%%

%\textbf{Coalgebra and homotopy associativity}

\section{余代数与同伦结合性}

\textbf{Co-algebra}
%%%%%结合性%%%%%%%%
%%%%%%分次交换%%%%%%%%%

\begin{definition}
$C\downdot\in Mod_K^{\bbZ}$ is a graded coalgebra over $K$,
if there is a coproduct ($\deg=0$)
$$\triangle:C\to C\ten C$$
%%%coproduct%%%%%%%
\end{definition}

counit$\veps:C\to K$ satisfying
%%%%%%%%counit%%%%%%%%%

co-derivation

$\delta: C\to C$ satisfying
%%%%%co-Leibnitz rule%%%%%%%

A differential graded co-algebra is a co-algebra
$C$ with a co-derivation $\delta:C\to C$ such that
$\deg\delta=1$ and $\delta^2=0$.

co-augmentation

 (Recall:
 $A$:$K$-algebra. augmentation is an algebra morphism
$A\to K$.)

Co-algebra $(C,\triangle)$ is called co-augmentation if
there is a co-algebra map $K\to C$.
%%%%%%co-algebra的同态怎么定义?自行脑补%%%%%%%%%%

$(C,\triangle)$ is co-commutative,if
%%%%co-commutative%%%%%%

\begin{rem}
If $(C,\triangle)$ is a co-algebra, then
$(C^*,\triangle^*)$ is an algebra
$$A\ten A=C^*\ten C^*\to(C\ten C)^*\xra{\triangle^*}C^*=A$$
\end{rem}
%%%%%见图%%%%%%%5

\begin{example}
$V\in Mod_K^{\bbZ}$,
$$T^c(V):=\bigoplus_{n\geq 0}V^{\ten n}$$
$$\triangle:T(V)\to T(V)\ten T(V)$$
$$v_1\ten...\ten v_n\mapsto
\sum_{i=0}^n(v_1\ten...\ten v_i)\ten(v_{i+1}\ten...\ten v_n)$$
\end{example}

Check:$\triangle$ is a co-product, and what is its dual?

\begin{example}
$$\overline{T^c}(V):=\bigoplus_{n\geq 1}V^{\ten n}$$
$$\overline{\triangle}(v_1\ten...\ten v_n)
=
\sum_{i=1}^{n-1}(v_1\ten...\ten v_i)\ten(v_{i+1}\ten v_n)$$

$\overline{T^c}(V)$ is a co-product on $\overline{T^c}(V)$.
\end{example}

$Coder(C)$ is all the co-derivation....

\begin{prop}
$Coder(C)$ is a graded Lie algebra, where
$$[D_1,D_2]=D_1\circ D_2-(-1)^{|D_1||D_2|}D_2\circ D_1$$
\end{prop}

\textbf{Associativity.}

$$\bullet:A\ten A\to A$$
$\Longrightarrow$
$$m:A[1]\ten A[1]\to A[1]$$
$$sa_1\ten sa_2\mapsto(-1)^{|a_1|+1}s(a_1a_2)$$

Observation:
Associativity of $\bullet\Longleftrightarrow[m,m]=0$

%%%%%叼~~%%%%%







