%%%%%%%%%%%%%%2019.4.15第八周周一%%%%%%%%%%%%%%
%%%%%%%%%%%%%%%%%%Final Project%%%%%%%%%%%%%%%%%%%%%

\chapter{还没想好叫什么名字}

\textbf{Maurer-Cartan Equation}

Chevalley-Eilenberg Complex

Let $\mfkg$ be a Lie algebra, with Lie bracket 
$$[,]:\wedge^2\mfkg\to\mfkg$$

Define $C\updot(\mfkg):=\wedgeform{\bullet}\mfkg^\vee$
where $\mfkg^\vee$ be the linear dual of $\mfkg$.

("CE" is "Chevalley-Eilenberg")

we define a derivation 
$$\td_{CE}:C\updot(\mfkg)\to\mfkg$$
determined by 
$$\td_{CE}:\mfkg^\vee\to\wedgeform{2}\mfkg^\vee$$
where $\td_{CE}$ is the dual of $\wedgeform{2}\mfkg\to\mfkg$.

And extend by Leibnitz rule:
$$\td_{CE}(\afa_1\wedge\cdots\wedge\afa_k)
=
\sum_i(-1)^{i-1}
\afa_1\wedge...\td_{CE}(\afa_i)...\wedge\afa_k
$$

\begin{prop}
$$\td_{CE}^2=0$$
\end{prop}
\begin{proof}
$\iff$ Jacobi identity. Check.
%%%%%%%%%%%%%%示意图%%%%%%%%%%%%%%
\end{proof}

Another way: $C\updot(\mfkg):=\Hom(\wedgeform{\bullet}\mfkg,\bbC)$.
$$\fai:C^k(\mfkg)=\Hom(\wedgeform{k}\mfkg,\bbC)$$
$$\td_{CE}(\fai)\in C^{k+1}(\mfkg)$$
$$(\td_{CE}\fai)(e_1,...,e_{k+1})=
   \sum_{i<j}
     (-1)^{i+j-1}
     \phi([e_i,e_j],e_1,...,\hat{e_i},...,\hat{e_j},...,e_{k+1})
$$

More generally, let $M$ be a $\mfkg$-module,
\begin{definition}
$$C\updot(\mfkg,M):=\wedgeform{\bullet}\mfkg^\vee\ten M$$
\begin{eqnarray*}
\td_{CE}:M&\to&\mfkg^\vee\ten M\\
\end{eqnarray*}
is the dual of $\mfkg\ten M\to M$.

(then ,extend by Leibnitz rule)
\end{definition}

$$
  \td_{CE}(\afa\wedge...\wedge\afa_k\ten m)
=\sum\pm\afa_1\wedge...\td_{CE}(\afa_i)...\wedge\afa_k\ten M
+(-1)^k\afa_1\wedge...\wedge\afa_k\wedge\td_{CE}(m)
$$

\begin{prop}
$$\td_{CE}^2=0$$
($\iff$ $M$ is $\mfkg$-module)
\end{prop}
check...

\begin{rem}
$$C\updot=\wedgeform{\bullet}(\mfkg^\vee)\ten M$$
is a module over $C\updot(\mfkg)=\wedgeform{\bullet}\mfkg^\vee$.
Actually, this is a differential graded module!
$$C\updot(\mfkg)\curvearrowright C\updot(\mfkg,M)$$
$$(\afa,\fai)\to \afa\wedge\fai$$

$$\td_{CE}(\afa\wedge\fai)=(\td_{CE}\afa)\wedge\fai\pm\afa\wedge(\td_{CE}\fai)$$
\end{rem}

\begin{rem}
$$(C\updot(\mfkg,M),\td_{CE})$$
is a complex. Consider its cohomology:
$$H\updot(\mfkg,M):=H\updot(C\updot(\mfkg,M),\td_{CE})$$
is called Lie algebra cohomology.
\end{rem}

$$M\xra{\td_{CE}}\mfkg^\vee\ten M\xra{\td_{CE}}
M\ten\wedgeform{2}\mfkg^\vee\to\cdots$$

$\td_{CE}(m)(e)=e.m$, so 
$$H^0(\mfkg,M)=\{m\in M^\mfkg\}$$
($\mfkg$-invariant elements)

\textbf{dg case}
Let $\mfkg$ be a dgLa($\bbZ$-graded)
$$\td:\mfkg\to\mfkg$$
$\deg\td=1$

$$[,]:\wedgeform{2}\mfkg\to\mfkg$$
$\deg[,]=0$,
(graded Jacobi-identity)

\begin{definition}
$$
  C^k(\mfkg):=
    \left\{
      \begin{array}{l}
        \Hom(\Sym^k(\mfkg[1]),\bbC)  \\
        \Sym^k(\mfkg^\vee[-1])\cong\wedgeform{k}\mfkg^\vee[-k]
      \end{array}
    \right.
$$
(注意这里是分次对偶)

with CE- differential
$$\td_{CE}=\td_{\mfkg}+\td_{[,]}$$
where $\td_{\mfkg}:\mfkg^\vee[-1]\to\mfkg^\vee[-1]$ is the dual of $\td:\mfkg\to\mfkg$.

$\td_{[,]}:\mfkg^\vee[-1]\to \Sym^2(\mfkg^\vee[-1])\cong\wedgeform{2}\mfkg^\vee\to\mfkg$
is the dual of $[,]:\wedgeform{2}\mfkg\to\mfkg$
\end{definition}

$\deg(\td_{CE})=1$.

Extend to $C\updot(\mfkg)$ by Leibnitz rule,
\begin{prop}
$$\td_{CE}^2=0$$
($\iff$ dgLa structure)
\end{prop}
(check)

Similarly, $M$ is a dg-$\mfkg$-module, i.e. 
$\td_M:M\to M$ s.t.
$$\td_M(e.m)=(\td e).m\pm e.(\td_M (m))$$
we also have 
$$(C\updot(\mfkg,M),\td_{CE})$$
and $\td_{CE}^2=0$

\begin{example}
$X$ be a manifold, $\mfkg=\Gma(X,TX)=Vect_X$
$M=C^\infty(X)$ is a $\mfkg$-module, and 
$\Omg\updot(X)\subseteq C\updot(\mfkg,M)$.
\end{example}

\section{Maurer-Cartan Equation}
(连接所有结构的核心方程)

Let $(\mfkg,\td,[,])$ be a dgLa, an 
Maurer-Cartan element is  $\afa\in\mfkg_1$(i.e. $\deg\afa=1$)
satisfies the following Maurer-Cartan Equation:
$$\td\afa=\frac{1}{2}[\afa,\afa]=0$$
and define $\mcalM(\mfkg)$ be the set of all the Maurer-Cartan element...

Key point: new dgLa
$$(\mfkg^\afa,\td_\afa,[,])$$

where 
$$\td_{\afa}=\td+[\afa,-]$$
$$[,]=\text{the same}$$
(check :$\td_\afa^2=0$)

Upshot: all deformation property are described by MC equation..

\begin{example}
Let $E\to X$ be a vector bundle with flat connection $\nabla$($\nabla^2=0$)
Then there is a dgLa:
$$\mfkg=\Omg\updot(X,\End(E))$$
$$\td=\nabla$$
$$[,]\text{is the commutator in}\End(E)$$
then
$$\mcalM(\mfkg)=\{A\in\Omg^1(X,\End(E))|\nabla A+\frac{1}{2}[A,A]=0\}$$
$$\{\text{flat connections on} E\}$$

$(\nabla+A)^2=$curvature
\end{example}

\begin{example}
$X$ is a complex manifold, 
$$\mfkg=\Omg^{0,\bullet}(X,T_X^{1,0})$$
$\pbar$ is the differential. and  
$[,]$ is the Lie bracket from $(1,0)$ vector vector field.

This is a dgLa.
$$\mcalM(\mfkg)=
\{\mu\in\Omg^{0,1}(X,T_X^{1,0})|
\pbar\mu+\frac{1}{2}[\mu,\mu]=0
\}$$
Beltrami-differential, describing the deformation of complex structures.
\end{example}

$$\mu-\mu^1_{\overline{j}}\td\overline{z}^j\ten\p_i$$
locally...

$\mu\rightsquigarrow$ new complex manifold "$X_\mu$"

a local function $f$ is called holomorphic in $X_\mu$, 
$\iff$ $\pbar_jf+\sum_i\mu^i_{\overline{j}}\p_if=0$ for all $j$.

integrability $\iff$ $\pbar\mu+\frac{1}{2}[\mu,\mu]=0\iff(\pbar+\mu)^2=0$.

\begin{example}
$E\to X$ is a holomorphic vector bundle, then 
$$\mfkg:=\Omg^{0,\bullet}(X,\End(E))$$
$\pbar$ differential, and $[,]$ on $\End(E)$.
this is a dgLa.

$$\mcalM(\mfkg)=\{A\in\Omg^{0,1}(X,\End(E))|\pbar A+\frac{1}{2}[A,A]=0\}$$
(holomorphic structure on $E\to X$)

A new holomorphic vector bundle...$E_A\to X$..
a local section $s$ is holomorphic in $E_A$
$\iff$
$$\pbar s+As=0$$
\end{example}










