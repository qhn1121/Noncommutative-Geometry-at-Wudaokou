%%%%%%%%%%%%%%2019.4.15第八周周一%%%%%%%%%%%%%%
%%%%%%%%%%%%%%%%%%Final Project%%%%%%%%%%%%%%%%%%%%%

\chapter{还没想好叫什么名字}

\section{Chevalley-Eilenberg复形}
%\textbf{Maurer-Cartan Equation}
我们继续准备一些代数结构。
\begin{definition}(Chevalley-Eilenberg 复形)
\index{Chevalley-Eilenberg complex}

设$\mfkg$为李代数,记$\mfkg^\vee$为其对偶空间,
$$[,]:\wedgeform{2}\mfkg\to\mfkg$$
为其李括号。考虑如下分次交换代数
$$C\updot(\mfkg):=\wedgeform{\bullet}\mfkg^\vee
=\bigoplus_{k=0}^\infty\wedgeform{k}\mfkg^\vee$$
其乘法为外积$\wedge$,且$\deg\mfkg^\vee =1$.
定义算子
$$\td_{\CE}:\mfkg^\vee\to\wedgeform{2}\mfkg^\vee$$
为李括号$[,]$的对偶映射。
之后按关于$\wedge$的超莱布尼茨法则将$\td_{\CE}$延拓为
$$\td_{\CE}:\wedgeform{k}\mfkg^\vee\to\wedgeform{k+1}\mfkg^\vee$$
并规定$\td_{\CE}|_{\wedgeform{0}\mfkg^\vee}=0$.
称$(C\updot(\mfkg),\td_{\CE})$为\textbf{Chevalley-Eilenberg 复形}。
\label{Chevalley-Elienberg复形-李代数版本-def}
\end{definition}

%Let $\mfkg$ be a Lie algebra, with Lie bracket
%Define where $\mfkg^\vee$ be the linear dual of $\mfkg$.
%("CE" is "Chevalley-Eilenberg")

%we define a derivation
%$$\td_{CE}:C\updot(\mfkg)\to\mfkg$$
%determined by
%$$\td_{CE}:\mfkg^\vee\to\wedgeform{2}\mfkg^\vee$$
%where $\td_{CE}$ is the dual of $\wedgeform{2}\mfkg\to\mfkg$.
%And extend by Leibnitz rule:
%$$\td_{CE}(\afa_1\wedge\cdots\wedge\afa_k)
%=\sum_i(-1)^{i-1}
%\afa_1\wedge...\td_{CE}(\afa_i)...\wedge\afa_k$$

这里的李代数$\mfkg$就是通常的李代数,并非带有分次的李超代数
(后面会提到李超代数的情形)。对于$f\in\mfkg^\vee$,
$\td_{\CE}$作为李括号的对偶映射,满足:对任意$u,v\in\mfkg$,
$$\td_{\CE}(f)(u\wedge v)=\pair{f}{[u,v]}$$
“按超莱布尼茨法则延拓”即为:对任意$\omg\in\wedgeform{k}(\mfkg^\vee)$以及
$\eta\in\wedgeform{l}(\mfkg^\vee)$,成立
$$
  \td_{\CE}(\omg\wedge\eta)
= \td_{\CE}(\omg)\wedge\eta
 +(-1)^{k}\omg\wedge\td_{\CE}(\eta)
$$

\begin{prop}
上述定义的$(C\updot(\mfkg),\td_{\CE})$为微分分次代数。
\label{Chevalley-Elienberg复形-微分分次代数-prop}
\end{prop}

回顾,微分分次代数的概念最早出现于定义\ref{微分分次代数-def}.

\begin{proof}
只需要验证$\td_{CE}^2=0$,其余各条都自动成立。

\textbf{Step1}
先证明对于$f\in C^1(\mfkg)=\mfkg^\vee$,$\td_{\CE}^2(f)=0$.
我们在基下验证之。取$\mfkg$的一组基$\Bigset{e_\afa}{\afa\in\mcalI}$,
并赋以指标集$\mcalI$一个良序$\preceq$($\mfkg$可以无限维),记
$$[e_\afa,e_\beta]=c_{\afa\beta}^\gma e_\gma$$
即$c_{\afa\beta}^\gma$为李代数$\mfkg$在该基下的结构常数。
李括号的反对称性表明$c_{\afa\beta}^\gma=-c_{\beta\afa}^\gma$.

记${e^\afa|\afa\in\mcalI}$为该基的对偶基,
$\pair{e^\afa}{e_\beta}=\delta^\afa_\beta$.则由
$$
  \td_{\CE}(e^\afa)(e_\beta\wedge e_\gma)
= \pair{e^\afa}{[e_\beta,e_\gma]}
= c^\afa_{\beta\gma}
$$
可知,
$$\td_{\CE}(e^\afa)=\frac{1}{2}c_{\beta\gma}^\afa e^\beta\wedge e^\gma$$
因此有
\begin{eqnarray*}
     \td_{\CE}^2(e^\afa)
&=&
     \frac{1}{2}
     c_{\beta\gma}^\afa
     \td_{CE}(e^\beta\wedge e^\gma)\\
&=&
     \frac{1}{2}
     c_{\beta\gma}^\afa
     \left(
       \frac{1}{2}
       c^\beta_{\lmd\mu}
       e^\lmd\wedge e^\mu\wedge e^\gma
      -\frac{1}{2}
       c^\gma_{\lmd\mu}
       e^\beta\wedge e^\lmd\wedge e^\mu
     \right)\\
&=&
     \frac{1}{2}
     c^\afa_{\beta\gma}
     c^\beta_{\lmd\mu}
     e^\lmd\wedge e^\mu\wedge e^\gma\\
&=&
     \sum_{\lmd\prec\mu\prec\gma}
       \left(
         c^\afa_{\beta\gma}
         c^\beta_{\lmd\mu}
        +c^\afa_{\beta\lmd}
         c^\beta_{\mu\gma}
        +c^\afa_{\beta\mu}
         c^\beta_{\gma\lmd}
       \right)
       e^\lmd\wedge e^\mu\wedge e^\gma\\
&=&
     0
\end{eqnarray*}
其中最后一个等号利用了李括号的雅可比恒等式。
这就证明了$\td_{\CE}^2|_{C^1(\mfkg)}=0$.

\textbf{Step2}
一般地,我们要断言对任意$\omg\in C^k(\mfkg)=\wedgeform{k}\mfkg^\vee$,
$\td_{\CE}(\omg)=0$.对$k$归纳,起始步$k=1$在Step1已经证明;而
归纳步由超莱布尼茨法则几乎显然,从略。
\end{proof}

\begin{rem}注意到有自然的同构
$$C\updot(\mfkg):=\wedgeform{\bullet}(\mfkg^\vee)
\cong\Hom(\wedgeform{\bullet}\mfkg,\bbC)$$
在此意义下,我们可以内蕴地给出$\td_{\CE}$的定义:
对于任意$\fai\in C^k(\mfkg)=\Hom(\wedgeform{k}\mfkg,\bbC)$,
以及任意$u_0,u_1,...,u_k\in\mfkg$,容易证明
$$
  \td_{\CE}(\fai)(u_0\wedge u_1\wedge \cdots\wedge u_k)
=
  \sum_{1\leq i<j\leq k}
    (-1)^{i+j-1}
    \fai([u_i,u_j],u_0,...,\hat{u}_i,...,\hat{u}_j,...,u_{k})
$$
\end{rem}

%More generally, let $M$ be a $\mfkg$-module,
一般地,考虑如下推广:

\begin{definition}(Chevalley-Eilenberg 复形)

设$\mfkg$为李代数,$M$为$\mfkg$-模,则定义分次线性空间
$$C\updot(\mfkg,M):=\wedgeform{\bullet}\mfkg^\vee\ten M$$
其次数由$\deg(\mfkg^\vee)=1\,,\deg M=0$决定。
定义算子
$$\td_{\CE}: M\to \mfkg^\vee\ten M$$
为$M$的模作用$\mfkg\ten M\to M$的对偶:
具体地,对任意$m\in M$以及$u\in\mfkg$,
$$u\suobing\td_{\CE}(m)=u.m$$

再将如此定义的$\td_{\CE}$按照关于$\wedge$的超莱布尼茨法则延拓为
$$\td_{\CE}:C^k(\mfkg,M)\to C^{k+1}(\mfkg,M)$$
其中$\td_{CE}$在$\wedgeform{\bullet}\mfkg$上的作用
同定义\ref{Chevalley-Elienberg复形-李代数版本-def}.
称$(C\updot(\mfkg,M),\td_{\CE})$为Chevalley-Eilenberg 复形.
\end{definition}

容易看出,若$M=\bbC$,$\mfkg$在$M$上的作用恒为零时,
$C\updot(\mfkg,M)$即为定义
\ref{Chevalley-Elienberg复形-李代数版本-def}
当中的$C\updot(\mfkg)$.

$C\updot(\mfkg,M)$具有显然的
$(C\updot(\mfkg),\wedge)$-分次模结构。
并且由$\td_{\CE}$的定义(按超莱布尼茨法则延拓),
对任意$\omg\in\wedgeform{k}\mfkg^\vee$
以及$\eta\in\wedgeform{l}\mfkg^\vee\ten M$,成立
$$
  \td_{\CE}(\omg\wedge\eta)
= \td_{\CE}(\omg)\wedge\eta
 +(-1)^{k}\omg\wedge\td_{\CE}(\eta)
$$

%$$
%  \td_{CE}(\afa\wedge...\wedge\afa_k\ten m)
%=\sum\pm\afa_1\wedge...\td_{CE}(\afa_i)...\wedge\afa_k\ten M
%+(-1)^k\afa_1\wedge...\wedge\afa_k\wedge\td_{CE}(m)
%$$

\begin{prop}(微分分次模结构)
对于李代数$\mfkg$以及$\mfkg$-模$M$,则
$(C\updot(\mfkg,M),\td_{\CE})$为微分分次代数
$(C\updot(\mfkg),\wedge,\td_{\CE})$的\textbf{微分分次模}
(differential graded module),
\index{differential graded module\kong 微分分次模}
\label{微分分次模的定义-prop}
也就是说:\vs

(1)$C\updot(\mfkg,M)$为分次交换代数
$(C\updot(\mfkg),\wedge)$的分次模;\vs

(2)$\td_{\CE}:C^k(\mfkg,M)\to C^{k+1}(\mfkg,M)$,
并且满足$\td_{\CE}^2=0$;\vs

(3)$M$的微分结构$\td_{\CE}$与其分次$(C\updot(\mfkg),\wedge)$结构的相容性:
任意$\omg\in C^k(\mfkg)$
以及$\eta\in C^l(\mfkg,M)$,成立
$$
  \td_{\CE}(\omg\wedge\eta)
= \td_{\CE}(\omg)\wedge\eta
 +(-1)^{k}\omg\wedge\td_{\CE}(\eta)
$$
\end{prop}

%\begin{rem}
%$$C\updot=\wedgeform{\bullet}(\mfkg^\vee)\ten M$$
%is a module over $C\updot(\mfkg)=\wedgeform{\bullet}\mfkg^\vee$.
%Actually, this is a differential graded module!
%$$C\updot(\mfkg)\curvearrowright C\updot(\mfkg,M)$$
%$$(\afa,\fai)\to \afa\wedge\fai$$
%$$\td_{CE}(\afa\wedge\fai)=(\td_{CE}\afa)\wedge\fai\pm\afa\wedge(\td_{CE}\fai)$$
%\end{rem}

\begin{proof}
(1)与(3)自动成立,只验证(2):$\td_{\CE}^2=0$.
与性质\ref{Chevalley-Elienberg复形-微分分次代数-prop}
的证明过程类似,在给定基下验证。

\textbf{Step1}
取定$\mfkg$的一组基$\Bigset{e_\afa}{\afa\in\mcalI}$,
其对偶基记为$\Bigset{e^\afa}{\afa\in\mcalI}$,
取定$\mcalI$的一个良序$\preceq$,记$\{c_{\afa\beta}^\gma\}$为李代数$\mfkg$
在此基下的结构常数。再取定$M$的一组基$\Bigset{m_\afa}{\afa\in\mcalJ}$,令
$$e_\afa.m_\beta=f_{\afa\beta}^\gma m_\gma$$
即$\{f_{\afa\beta}^\gma\}_{\afa\in\mcalI\atop\beta,\gma\in\mcalJ}$
为$M$的$\mfkg$-模结构常数。容易验证
$$[e_\afa,e_\beta].m_\gma
=e_\afa.(e_\beta.m_\gma)-e_\beta.(e_\afa.m_\gma)$$
等价于
$$
  c_{\afa\beta}^\lmd
  f_{\lmd\gma}^\mu
=
  f_{\beta\gma}^\lmd
  f_{\lmd\afa}^\mu
 -
  f_{\afa\gma}^\lmd
  f_{\lmd\beta}^\mu
\eqno{(*)}
$$\vs

\textbf{Step2}
先验证$\td_{\CE}^2(M)=0$.对于基向量$m_\afa\in M$,由
$$e_\beta\suobing\td_{\CE}(m_\afa)
=e_\beta.m_\afa=f_{\beta\afa}^\mu m_\mu$$
可知
$$\td_{\CE}(m_\afa)=f_{\beta\afa}^\mu e^\beta\ten m_\mu$$
因此
\begin{eqnarray*}
     \td_{\CE}^2(m_\afa)
&=&
     f_{\beta\afa}^\mu
     \left(
       \td_{\CE}(e^\beta)\ten m_\mu
      -e^\beta\ten\td_{\CE}(m_\mu)
     \right)\\
&=&
     f_{\beta\afa}^\mu
     \left(
       \frac{1}{2}c_{\gma\lmd}^\beta
       e^\gma\wedge e^\lmd\ten m_\mu
      -
       f_{\omg\mu}^\eta
       e^\beta\wedge e^\omg\ten m_\eta
     \right)\\
&=&
     \left(
       \sum_{\gma\prec\lmd}
         f_{\beta\afa}^\mu
         c_{\gma\lmd}^\beta
         e^\gma\wedge e^\lmd\ten m_\mu
     \right)
    -
     \left(
       f_{\gma\afa}^\eta
       f_{\lmd\eta}^\mu
       e^\gma\wedge e^\lmd\ten m_\mu
     \right)\\
&=&
     \sum_{\gma\prec\lmd}
       \left(
         f_{\beta\afa}^\mu
         c_{\gma\lmd}^\beta
        -
         f_{\gma\afa}^\eta
         f_{\lmd\eta}^\mu
        +
         f_{\lmd\afa}^\eta
         f_{\gma\eta}^\mu
       \right)
       e^\gma\wedge e^\lmd\ten m_\mu\\
&=&
     0
\end{eqnarray*}
最后一步利用了$(*)$式。从而$\td_{\CE}^2|_{C^0(\mfkg,M)}=0$.\vs

\textbf{Step3}
再验证对一般的$k\geq 1$,$\td_{\CE}^2|_{C^k(\mfkg,M)}=0$.
对$k$归纳,起始步$k=0$已证;
利用$\td_{\CE}$的超莱布尼茨法则以及性质
\ref{Chevalley-Elienberg复形-微分分次代数-prop},
容易给出归纳步,从略。从而证毕。
\end{proof}

\begin{rem}(李代数上同调)

对于李代数$\mfkg$以及$\mfkg$-模$M$,注意
$(C\updot(\mfkg,M),\td_{\CE})$:
$$0\to M\xra{\td_{CE}}\mfkg^\vee\ten M\xra{\td_{\CE}}
\wedgeform{2}\mfkg^\vee \ten M\to\cdots$$
为上链复形,其上同调
$$H\updot(\mfkg,M):=H\updot(C\updot(\mfkg,M),\td_{CE})$$
称为\textbf{李代数上同调}(Lie algebra cohomology)。
\index{Lie algebra cohomology\kong 李代数上同调}
%is a complex. Consider its cohomology:
%is called Lie algebra cohomology.
\end{rem}

容易验证,
$$H^0(\mfkg,M)=\ker\left(\td_{\CE}:M\to\mfkg^\vee\ten M\right)
=\Bigset{m\in M}{g.m=0,\,\forall g\in\mfkg}$$

%$\td_{CE}(m)(e)=e.m$, so
%$$H^0(\mfkg,M)=\{m\in M^\mfkg\}$$
%($\mfkg$-invariant elements)

%上面被注释掉的三行是李思说的,我觉得不太对。


\textbf{dg case}
Let $\mfkg$ be a dgLa($\bbZ$-graded)
$$\td:\mfkg\to\mfkg$$
$\deg\td=1$

$$[,]:\wedgeform{2}\mfkg\to\mfkg$$
$\deg[,]=0$,
(graded Jacobi-identity)

\begin{definition}
$$
  C^k(\mfkg):=
    \left\{
      \begin{array}{l}
        \Hom(\Sym^k(\mfkg[1]),\bbC)  \\
        \Sym^k(\mfkg^\vee[-1])\cong\wedgeform{k}\mfkg^\vee[-k]
      \end{array}
    \right.
$$
(注意这里是分次对偶)

with CE- differential
$$\td_{CE}=\td_{\mfkg}+\td_{[,]}$$
where $\td_{\mfkg}:\mfkg^\vee[-1]\to\mfkg^\vee[-1]$ is the dual of $\td:\mfkg\to\mfkg$.

$\td_{[,]}:\mfkg^\vee[-1]\to \Sym^2(\mfkg^\vee[-1])\cong\wedgeform{2}\mfkg^\vee\to\mfkg$
is the dual of $[,]:\wedgeform{2}\mfkg\to\mfkg$
\end{definition}

$\deg(\td_{CE})=1$.

Extend to $C\updot(\mfkg)$ by Leibnitz rule,
\begin{prop}
$$\td_{CE}^2=0$$
($\iff$ dgLa structure)
\end{prop}
(check)

Similarly, $M$ is a dg-$\mfkg$-module, i.e.
$\td_M:M\to M$ s.t.
$$\td_M(e.m)=(\td e).m\pm e.(\td_M (m))$$
we also have
$$(C\updot(\mfkg,M),\td_{CE})$$
and $\td_{CE}^2=0$

\begin{example}
$X$ be a manifold, $\mfkg=\Gma(X,TX)=Vect_X$
$M=C^\infty(X)$ is a $\mfkg$-module, and
$\Omg\updot(X)\subseteq C\updot(\mfkg,M)$.
\end{example}

\section{Maurer-Cartan 方程}
(连接所有结构的核心方程)

Let $(\mfkg,\td,[,])$ be a dgLa, an
Maurer-Cartan element is  $\afa\in\mfkg_1$(i.e. $\deg\afa=1$)
satisfies the following Maurer-Cartan Equation:
$$\td\afa=\frac{1}{2}[\afa,\afa]=0$$
and define $\mcalM(\mfkg)$ be the set of all the Maurer-Cartan element...

Key point: new dgLa
$$(\mfkg^\afa,\td_\afa,[,])$$

where
$$\td_{\afa}=\td+[\afa,-]$$
$$[,]=\text{the same}$$
(check :$\td_\afa^2=0$)

Upshot: all deformation property are described by MC equation..

\begin{example}
Let $E\to X$ be a vector bundle with flat connection $\nabla$($\nabla^2=0$)
Then there is a dgLa:
$$\mfkg=\Omg\updot(X,\End(E))$$
$$\td=\nabla$$
$$[,]\text{is the commutator in}\End(E)$$
then
$$\mcalM(\mfkg)=\{A\in\Omg^1(X,\End(E))|\nabla A+\frac{1}{2}[A,A]=0\}$$
$$\{\text{flat connections on} E\}$$

$(\nabla+A)^2=$curvature
\end{example}

\begin{example}
$X$ is a complex manifold,
$$\mfkg=\Omg^{0,\bullet}(X,T_X^{1,0})$$
$\pbar$ is the differential. and
$[,]$ is the Lie bracket from $(1,0)$ vector vector field.

This is a dgLa.
$$\mcalM(\mfkg)=
\{\mu\in\Omg^{0,1}(X,T_X^{1,0})|
\pbar\mu+\frac{1}{2}[\mu,\mu]=0
\}$$
Beltrami-differential, describing the deformation of complex structures.
\end{example}

$$\mu-\mu^1_{\overline{j}}\td\overline{z}^j\ten\p_i$$
locally...

$\mu\rightsquigarrow$ new complex manifold "$X_\mu$"

a local function $f$ is called holomorphic in $X_\mu$,
$\iff$ $\pbar_jf+\sum_i\mu^i_{\overline{j}}\p_if=0$ for all $j$.

integrability $\iff$ $\pbar\mu+\frac{1}{2}[\mu,\mu]=0\iff(\pbar+\mu)^2=0$.

\begin{example}
$E\to X$ is a holomorphic vector bundle, then
$$\mfkg:=\Omg^{0,\bullet}(X,\End(E))$$
$\pbar$ differential, and $[,]$ on $\End(E)$.
this is a dgLa.

$$\mcalM(\mfkg)=\{A\in\Omg^{0,1}(X,\End(E))|\pbar A+\frac{1}{2}[A,A]=0\}$$
(holomorphic structure on $E\to X$)

A new holomorphic vector bundle...$E_A\to X$..
a local section $s$ is holomorphic in $E_A$
$\iff$
$$\pbar s+As=0$$
\end{example}

%%%%%%%%%%%%2019.4.16第八周周二%%%%%%%%%%%%%%%%%%%%%%%%%%%%%%%

\textbf{Maurer-Cartan (II)}

Last time: $\mfkg$ is a dgLa,
CE complex: $C\updot(\mfkg)=\text{两种写法,有限维是一样的}$
$$\mcalM(\mfkg)=\{\afa\in\mfkg_1|\td\afa+\frac{1}{2}[\afa,\afa]=0\}$$

Geometrically, we can think about
$$C\updot(\mfkg)=\Sym\updot(\mfkg^\vee[-1])$$
where $\mfkg^\vee[-1]$ is the dual of $\mfkg[1]$.
$C\updot(\mfkg)$ can be regarded as "functions on $\mfkg[1]$".

$\td_{CE}$ is a derivation on $C\updot(\mfkg)$, $\iff$
vector fields $X_{CE}$ on $\mfkg[1]$, $\deg X_{CE}=1$.

\begin{example}
$\mfkg$  is a Lie algebra, $\{e_\afa\}$ basis of $\mfkg$,
$\{c^\afa\}$ dual basis of $(\mfkg[1])^\vee$.
$c^\afa c^\beta=-c^\beta c^\afa$.

$\td_{CE}c^\afa=\frac{1}{2}f_{\beta\gma}^\afa c^\beta c^\gma$
$\iff$
$$X_{CE}=\frac{1}{2}f^\afa_{\beta\gma}c^\beta c^\gma \pp{c^\afa}$$

MC: $\td\afa+\frac{1}{2}[\afa,\afa]=0$.
$$\td\afa+\frac{1}{2}[\afa,\afa]=
\text{coefficients of $X_{CE}$ valued at $\afa\in\mfkg[1]$}$$
so,
$\afa\in\mcalM(\mfkg)\iff \afa=\text{zero locus of $X_{CE}$}$
\end{example}

(以上,是。。。的几何解释)

\begin{prop}
Let $\mfkg,\mfkh$ be two dgLa,
$$\fai:\mfkg\to\mfkh$$
be a homomorphism of dgLa(自己补定义,与微分、李括号都相容)

Then $\fai$ induces a map
$$\fai_*: \mcalM(\mfkg)\to\mcalM(\mfkh)$$

i.e.
$$\underline{DGLA}\xra{\mcalM}\underline{Set}$$
is a functor
\end{prop}

\begin{proof}
Obvious.
\end{proof}

$\fai:\mfkg\to\mfkh$, then
$$\mfkg[1]\xra{\fai}\mfkg[h]$$
(意淫为两个空间的映射)
$$\fai^*:\mcalO(\mfkg[1])\to\mcalO(\mfkg[1])$$
where $\mcalO(\mfkg[1]):=C\updot(\mfkg)$ is CE complex...
%%%%%%%%%%%%%%%%%%%%%%%%几何解释%%%%%%%%%%%%%%%%%%%%%%%%%

\textbf{(-1)-Symplectic geometry}
recall: $(V,\omg)$ ,where $V$ is a graded linear space, $\omg$ is a (-1)- symplectic structure.
$$\omg:\wedgeform{2}V\to\bbC$$
non-degenerated, $\deg\omg=-1$.

Given $\mcalS\in\mcalO(V)$, it defines a vector field
$X_\mcalS$ on $V$, s.t.
$$X_\mcalS=\{\mcalS,-\}$$
where $\{,\}$ is the induced poisson bracket($\deg\{,\}=1$).

$$\deg(\mcalS)=0\iff\deg(X_\mcalS)=1$$

recall: CME: $\{\mcalS,\mcalS\}=0$. it means that
$$[X_\mcalS,X_\mcalS]=0$$
if $\mcalS$ contains two terms
$$\mcalS=\mcalS_2+\mcalS_3$$
($\in \Sym^2(V^*),\Sym^3(V^*)$ respectively)

$\{\mcalS_2,-\}$ is "differential"

$\{\mcalS_3,-\}$ Lie algebra structure.

$\{\mcalS,\mcalS\}=0\Longrightarrow$ dgLa on $\mfkg$ where $\mfkg[1]=V$.

\begin{example}
Let $\mfkg$ be a Lie algebra,
$$V=\mfkg[1]\oplus\mfkg^\vee[-2]$$
(degree $-1,2$ respectively)
with natural $(-1)$- symplectic structure
$$\omg:\mfkg[1]\ten \mfkg^\vee[-2]\to\bbC$$


Let $\{e_\afa\}$ be a basis of $\mfkg$,
$$[e_\afa,e_\beta]=f^\gma_{\afa\beta}e_\gma$$

$$V^*=(\mfkg[1])^\vee\oplus(g^\vee[-2])^\vee$$

$\{c^\afa\}$ dual basis of $g^\vee[-1]$ of degree $\deg(c^\afa)=1$.

$\{u_\afa\}$ basis of $\mfkg[2]$ s.t. $\deg(u_\afa)=-2$.
$$\omg=\sum_{\afa}\td c^\afa\wedge\td u_\afa$$
\end{example}

Poisson bracket:
$$\{c^\afa,u_\beta\}=\delta^\afa_\beta$$

$$\mcalS=\sum_{\afa\beta\gma}
\frac{1}{2}f_{\afa\beta}^\gma c^\afa c^\beta u_\gma
\in\mcalO(V)=\bbC[c^\afa,u_\afa]
$$

$$\deg\mcalS=0$$

Claim: $\mcalS$ satisfies CME, $\{\mcalS,\mcalS\}=0$.

$$\mcalO(C)\cong \wedgeform{\bullet}\mfkg^\vee\ten\Sym\updot\mfkg
=C\updot(\mfkg,\Sym\updot\mfkg)
$$

$$\{\mcalS,c^\afa\}=\frac{1}{2}f^\afa_{\beta\gma}c^\beta c^\gma$$
$$\{\mcalS,u_\afa\}=f^\gma_{\afa\beta}c^\beta u_\gma$$

so,
$$X_{\mcalS}=\frac{1}{2}f^\afa_{\beta\gma}c^\beta c^\gma\pp{c^\afa}
+f^\gma_{\afa\beta}c^\beta u_\gma\pp{u_\beta}
$$

check:$\{X_\mcalS,X_\mcalS\}=0$
\vs

In general, $\mcalS\in\mcalO(V)$, s.t.
$$\mcalS=\mcalS_2+\mcalS_3+\mcalS_4+\cdots$$

$\{\mcalS_2,-\}$ ... differential

$\{\mcalS_3,-\}$... Lie bracket

$\{\mcalS_4,-\}$... is what? "3-bracket"...

$\{\mcalS_{k+1},-\}$..."$k$-bracket"

CME $\{\mcalS,\mcalS\}=0\Longrightarrow$ compatibility conditions for all "$k$-brackets"...

Homotopy Lie structure ($L_\infty$-albegra)

\begin{definition}
An $L_\infty$-algebra is a graded vector space $\mfkg$ with
a $\deg=1$ vector field $\delta$ s.t. $[\delta,\delta]=2\delta^2=0$.
\end{definition}
$$\delta\in\Hom(\Sym\updot(\mfkg[1]),\mfkg[1])$$
$$\delta=\sum_{k\geq 0}\delta_k$$
$$\delta_k\in\Hom(\Sym^k(\mfkg[1]),\mfkg[1])$$

$[\delta,\delta=0]\iff$
$$\sum\cdots=0$$

Since $\Sym^k(\mfkg[1])\cong\wedgeform{k}\mfkg[k]$, it induces
$$\ell_k:\wedgeform{k}\mfkg\to\mfkg$$
$\deg(\ell_k)=2-k$.

$[\delta,\delta]=0$用$\ell_k$来写。。。

(Jacobi identity holds up to homotopy)

Let $\mfkg$ is an $L_\infty$ algebra, define
$$\mcalM(\mfkg)=\{\afa\in\mfkg_1|
\ell_1(\afa)+\frac{1}{2}\ell_2(\afa,\afa)+\frac{1}{3!}\ell_3(\afa,\afa,\afa)=0
\}$$
(is zero locus of $\delta$ on $\mfkg[1]$)


\begin{definition}
Let $\mfkg,\mfkh$ be two $L_\infty$-algebras,
Am $L_\infty$-morphism "$\fai:\mfkg\to\mfkh$" is given by a cochain
map
$$\fai^*:C\updot(\mfkh)\to C\updot(\mfkg)$$

$$(\mfkg[1])^\vee\to(\Sym^k(\mfkg[1]))^\vee$$
$$\fai_k:\Sym^k(\mfkg[1])\to\mfkh[1]$$

\end{definition}

then we get an $L_\infty$ category.

Moreover, $\fai:\mcalM(\mfkg)\to\mcalM(\mfkh)$,
the same as before.

\begin{example}(Chern Simons theory)
$X$ is 3-manifold, $\mfkg$ Lie algebra with Trace pairing.
Fields
$$\mcalE=\Omg\updot(X)\ten\mfkg[1]$$
$$\deg(\Omg^k)=k-1$$
degree $-1$:ghost

degree 1: field

degree 2: anti-field

degree 2 anti-ghost

$$CS[\mcalA]=\int_X\frac{1}{2}Tr\mcalA\wedge\td A+\frac{1}{6}Tr\mcalA\wedge[\mcalA,\mcalA]$$
for $\mcalA\in\mcalE$

then $\{CS,CS\}=0$ holds ,
$(\Omg\updot(X)\ten\mfkg,\td,[-,-])$ is a dgLa.

$$\{CS,-\}=\delta_{BRST}$$
$$\delta_{BRST}(\mcalA)=\td \mcalA+\frac{1}{2}[\mcalA,\mcalA]$$
MC equation $\iff$ EOM
\end{example}


%%%%%%%%%%%%%2019.4.22 第九周%%%%%%%%%%%%%%%%%%

\section{Gauge theory and BRST-BV formalism}

Last time
Homotopy Lie algebra $\mfkg\quad(L_\infty$-algebra
$=$ degree $1$ square-zero vector field $\delta$ on $\mfkg[1]$.

$L_\infty$ operator
$$\{\ell_n:\wedgeform{n}\mfkg\mapsto \mfkg\}$$
with degree $\ell_n=2-n=$ components of $\delta$.

CE-complex $C\updot (\mfkg)=\widehat{\Sym}\updot(\mfkg[1]^\wedge)=\mcalO(\mfkg[1])$,
differential $\td_{CE}=\delta$.
$\td_{CE}^2=0$.

MC-equation: $\afa\in\mfkg_1$,
$$\sum_{k}\frac{1}{k!}\ell_n(\afa^k)=0$$
$\afa$ satisfies MC $\iff\afa=$ zero locus of $\delta$.

$(-1)$-shifted symplectic geometry, $\{,\}$ degree $=1$, Poisson bracket,
CME: $\{S,S\}=0\rightsquigarrow \{S,-\}$ defines a $L_\infty$-structure.
\vsp

Today: Gauge theory$\mapsto$ CME $\mapsto L_\infty$-structure.

Finite dimensional model.
Let $V$ be a finite dimensional vector space with a Lie group $G$ acts on $V$.

$V\rightsquigarrow$"space of fields"

$G\rightsquigarrow$"Gauge symmetry"

Let $f\in\mcalO(V)$ be a function(polynomial? $L^2$? )
on $V$, which is $G$-invariant.

$f\rightsquigarrow$"action function"

$G$-invariant$\rightsquigarrow$ "Gauge invariance".

$f$ can be viewed as a function on $V/G$.
We want to model
$$\int_{V/G}e^{f/\hbar}$$
Problem: $V/G$ is usually very bad(Singularities)

\begin{example}
$V=\bbC^2,G=\bbZ_2=\{1,\sgm\}$
$$\sgm:V\to V$$
$$(z_1,z_2)\mapsto(-z_1,-z_2)$$
then $V/\bbZ_2=?$
\end{example}
$$\mcalO(V/G)=\mcalO(V)^G=(\bbC[z_1,z_2])^G$$
$G$-invariant functions on $V$.Generators $u=z_1^2,v=z_1z_2,w=z_2^2$,
relations $v^2=uw$.So,
$$\mcalO(V/\bbZ_2)\cong \bbC[u,v,w]/(uw=v^2)$$
So,
$$V/\bbZ_2=\{uw=v^2\}\subseteq\bbC^3$$
this hyper-surface has a singularity at
$(u,v,w)=(0,0,0)$
(Ordinary double point)

Lesson from homological algebra.

replace "bad theory" be complex.
(derived!!!)

$$\mcalO(V)^G=H^0(\mfkg,\mcalO(V))
=H^0(C\updot(\mfkg,\mcalO(V)),\td_{CE})$$

Replace $\mcalO(V)\rightsquigarrow C\updot(\mfkg,\mcalO(V))=
\widehat{\Sym}(\mfkg[1]^\vee)\ten\mcalO(V)=\mcalO(\mfkg[1]\oplus V)$

$V/G\rightsquigarrow\mfkg[1]\oplus V$ with $\td_{CE}$

$\int_{V/G}\rightsquigarrow\int_{\mfkg[1]\oplus V}$
(BRST-formalism) $\td_{CE}=$BRST-transformation.

$f$ extends to a function on $\mfkg[1]\oplus V$,
$f\in C\updot(\mfkg,\mcalO(V))$.
$f$ is $G$-invariant$\iff \td_{CE}=0$.
\vs

Recall:
$$\int_X\rightsquigarrow\text{model by $\PV_X$}$$
$\PV(V)=\mcalO(V)\ten\wedgeform{\bullet}V^*
=\mcalO(V\ten V^*[-1])=\mcalO(T^*V[-1])$.

Apply to the construction to $\mfkg[-1]\ten V\Rightarrow$
\begin{eqnarray*}
     E
&=&
     T^*(\mfkg[1]\oplus V)[-1]\\
&=&
     (\mfkg[1]\oplus V)\oplus
     (\mfkg[1]\oplus V)^*[-1]\oplus\\
&=&
     \mfkg[1]\oplus V\oplus V^*[-1]\oplus\mfkg^*[-2]
\end{eqnarray*}
$(-1)$-symplectic of degree $-1,0,1,2$ respectively.

(ghost, field, anti-field, anti-ghost)

$\mcalO(E),\{,\}$, $\deg=1$, poisson bracket,

$\td_{CE}$ acts on $\mcalO(\mfkg[1]\oplus V)$
extends naturally to $\mcalO(E)=C\updot(\mfkg,\mcalO(V\oplus V^*[-1]\oplus \mfkg^*[-2]))$

In most cases, $\td_{CE}=\{H_{CE},-\}$, where $H_{CE}\in\mcalO(E)$,
$\deg H_{CE}=0$.

\begin{example}
$V=\{0\}$, and
$$H_{CE}=\frac{1}{2}f^\afa_{\beta\gma}c^\beta c^\gma u_\afa$$
$f$ is $G$-invariant $\iff \{H_{CE},f\}=0$

$\{f,f\}=0$ by type reason.
\end{example}

$V/G\rightsquigarrow T=\mfkg[1]\oplus V\oplus V^*[-1]\oplus \mfkg^*[-2]$

$$\mcalS^{BV}=f+H_{CE}$$
BRST-BV extension of $f$, satisfying CME $\{\mcalS^{BV},\mcalS^{BV}\}=0$

$$
  f\rightsquigarrow
  \left\{
    \begin{array}{c}
      \{H_{CE},H_{CE}\}=0\quad \text{(Lie structure)}\\
      \{H_{CE},f\}=0 \quad\text{$G$-invariance}\\
      \{f,f\}=0
    \end{array}
  \right.
$$

\begin{example}(CS-theory)(Chern-Simons)

Ordinary CS-theory on $3$-manifold $M$,
$A\in\Omg^1(X,\mfkg)$ connection $1$-form.
$$CS[A]=\int \frac{1}{2}Tr A\wedge\td A+
\frac{1}{6}Tr(A\wedge[A,A])$$
CS has a gauge symmetry
$$\text{Lie algebra(Gauge group)}=\Omg\updot(M,\mfkg)=C^\infty(M)$$
$\phi\in\Omg\updot(M,\mfkg)$,
$$\delta_\phi A=\td\phi+[A,A]\Rightarrow \delta_\phi CS=0$$
(infinitisimal transformation)
\end{example}

$$\Rightarrow E: \Omg^0(M,\mfkg)\quad
\Omg^1(M,\mfkg)\quad \Omg^2(M,\mfkg)\quad \Omg^3(M,\mfkg)$$
(ghost,field,anti-field,anti-ghost)
$$\mcalA=\phi\oplus A\oplus A^\wedge\oplus C\wedge$$

$$CS^{BV}=\int_M\frac{1}{2}Tr A\wedge A+\frac{1}{6}Tr A\wedge[A,A]$$

$$[CS^{BV},CS^{BV}]=0\iff \Omg\updot(X,\mfkg) \text{is a dgLa}$$

Special locus $A=A^\wedge =0\rightsquigarrow$
$$\frac{1}{2}\int Tr C^\vee\wedge [C,C]$$
is the analogue of $\frac{1}{2}f^\afa_{\beta\gma}c^\beta c^\gma u_\afa$

\section{AKSZ-formalism}

Super-manifold: locally ringed space $X,\mcalO_X$($X$ is a topological space)
whose structure ring$\mcalO_X$ is a graded commutative dg-algebra.

\begin{example}(de-Rham space)

$X_{dR}=(X,\Omg_X\updot)$ with $\td=$de Rham differenital.
\end{example}

\begin{example}(Vector bundle)
consider vector bundle $E\to X$, then
$$(X,\wedgeform{\bullet}E)$$
is a super-manifold, with $\td=0$
\end{example}

\begin{example}

$$B_\mfkg=(pt,C\updot(\mfkg))$$
"$pt/G$"
\end{example}

Let $X,Y$ be two super manifolds,

\textbf{Goal}: $(-1)$-symplectic structure on "$map(X,Y)$".

$X$: is equipped with $\int_X:\mcalO(X)\to\bbR$,
$\deg=-k$

(Eg:$\int\mcalO(X_{dR})\to\bbR$, $\deg=-\dim X$)

$Y$ is equipped with symplectic form of degree $k-1$,
$$\omg:\wedgeform{2}T_Y\to\mcalO_Y$$
$\deg\omg=k-1$.

AKSZ:"$map(X,Y)$" has $(-1)$-symplectic structure!

$f\in map(X,Y)$,then the "tangent map"
$$T_f map(X,Y)=\Gma(X,f^*T_Y)$$
$\Rightarrow(-1)$symplectic pairing on
$\afa,\beta\in T_fmap(X,Y)$,
$$\afa\ten\beta\mapsto\int_X\langle\afa,\beta\rangle Y\in\mcalO(X)$$
with degree $=(-k)+(k-1)=-1$.

\begin{example}(loop space)

$$S^1_{dR}\to (V,\omg)\cong \bbR^{2n}$$
where $S^1_{dR}$ is the de Rham super manifold over loop $S^1$,
and $(V,\omg)$ is an ordinary symplectic space.

$\int\Omg(S^1)\to\bbR\quad\deg=-1$

$\omg$ has degree $0=1-1$.

$$map(S_{dR},V)=\Omg\updot(S^1)\ten V$$
(-1)-symplectic
\end{example}

$(\rightsquigarrow$ Atiyah-Singer index theorem)

\begin{example}(CS-theory, again)

$M$ is a $3$-manifold,
$$M_{dR}=(M,\Omg\updot(M))\to B_\mfkg=(pt,C\updot(\mfkg))$$
$$C\updot(\mfkg)=\Sym(\mfkg[1]^\vee)$$
$$\Hom_{alg}(C\updot(\mfkg),\Omg\updot(M))
\cong\Hom_\bbC(\mfkg[-1]^\vee,\Omg\updot(M))
\cong \Omg\updot(M)\ten\mfkg[1]$$

$$\int:\mcalO(M_{dR})\to\bbR\quad \deg=3$$

$$Tr:\Sym^2\mfkg\to\bbC$$
trace pairing (Killing form)
$$\omg:\wedgeform{2}(\mfkg[-1])\to\bbC\quad\deg =2$$

$$S=\int \frac{1}{2}A\wedge\td A
+\frac{1}{6}Tr A\wedge [A,A]$$
\end{example}

%%%%%%%%%%%%%%%2019.4.23第9周%%%%%%%%%%%%%%%%%%%%%%%%%5

Today:

\textbf{Formality}

Back to deformation quantization...

$X$ be a manifold with a Poisson tensor
$P\in\Gma(X,\wedgeform{2}TX)$,
$\mcalA=C^\infty(X)$, $\{,\}$ Poisson bracket.

Quantize, $(C^\infty(X)\fps{\hbar},*)$
associative algebra, (star product),
$$\{f,g\}=\lim_{\hbar\to 0}
\frac{1}{\hbar}
\left(
  f*g-g*f
\right)$$

$X=\bbR^n$, $\mcalA=\bbR\{x^i\}$,
$$P=P^{ij}(x)\pp{x^i}\wedge\pp{x^j}$$
$$[P,P]_{SN}=0$$
if $P^{ij}$ is constant ,we have Moyal product...
In general, Kontsevich formula...

Denote
$$\PV\updot(X):=\Gma(X,\wedgeform{\bullet}TX)$$
and $[,]$ be the Schouten-Nijenhuis bracket..

$C\updot(A,A)$ Hochschild cochains,
$\p$: Hochschild differential, and $[,]_G$ be the Gerstenhaber bracket...

Recall:
$$[\PV^k,\PV^m]_{SN}\subseteq\PV^{k+m-1}$$

\begin{definition}
$$\mfkg_{\PV}\updot:=\PV\updot[1]$$

then, the Schouten-Nijenhuis bracket
$$[,]:\Sym^2(\PV\updot)\to\PV\updot$$

$$
  \rightsquigarrow
  \ell_2:\wedgeform{2}\mfkg_{\PV}\updot\to\mfkg_{\PV}\updot
$$
$\deg\ell_2=0$.
\end{definition}
So, $(\mfkg_{\PV}\updot,\ell_2)$ is a dgLa($\td=0$),
and $\mfkg^0_{\PV}=Vect_X$ is a sub Lie algebra...


What is the MC-element in $\mfkg_{\PV}\updot$??

$$MC(\mfkg_{\PV}\updot)=\{\afa\in\PV^2|[\afa,\afa]_{SN}=0\}$$
is the set of all Poisson tensor.

$C\updot(A,A)$ Hochschild $\p$, $[-,-]_G$
$$[C^k(A,A),C^m(A,A)]_G\subseteq C^{k+m-1}(A,A)$$

\begin{definition}
$$\mfkg_{Hoch}\updot=C\updot(A,A)[1]$$
$$\p\rightsquigarrow\ell_1:\mfkg_{Hoch}\updot\to\mfkg_{Hoch}\updot$$
$\deg\ell_1=1$,

$$[-,-]_G\rightsquigarrow\ell_2:\wedgeform{2}(\mfkg_{Hoch}\updot)\to\mfkg_{Hoch}\updot$$
$\deg(\ell_2)=0$.
\end{definition}

\begin{prop}
Let $(\mfkg,\td,[-,-])$ be a dgLa,
Then it induces a dgLa structure on
$$(H\updot(\mfkg,\td),[-,-])$$
$diff=0$
\end{prop}
\begin{proof}
$$\td[\afa,\beta]=[\td\afa,\beta]\pm[\afa,\td\beta]$$
so, $[-,-]$ is a cochain map.
\end{proof}

$$(\mfkg_{Hoch}\updot,\ell_1,\ell_2)\xra{\ell_1\text{-cohomology}}
(\mfkg_{PV},\ell_2)$$

\textbf{Idea:} dgLa $(\mfkg,\td,[,])$ is formal if
"it carries somehow the same information as $H\updot(\mfkg,\td)$"

$$MC(\mfkg_{Hoch}\updot)=
\{\afa\in C^2(A,A)|\p\afa+\frac{1}{2}[\afa,\afa]_G=0\}$$
Let
$$m: A^{\ten 2}\to A$$
be the product, $m\in C^2(A,A)$,
Associativity$\iff[m,m]_G=0$.
Hochschild differential $\p=[m,-]_G$.So,
MC-equation $\iff$
$$[m,\afa]_G+\frac{1}{2}[\afa,\afa]_G=0$$
$$\iff [m+\afa,m+\afa]_G=0$$
where $m+\afa\in C^2(A,A)$, %细思极恐!!

$\Rightarrow (m+\afa)$ defines a deformation of associative product on $A$.

In particular,
$$\{\afa\in C^2(A,A)\fps{\hbar}|\p\afa+\frac{1}{2}[\afa,\afa]_G=0\}=\{\text{star product}\}$$

Let $\afa=\afa_1\hbar+\afa_2\hbar^2+\cdots$

$$
  \left\{
    \begin{array}{l}
      \p\afa_1=0\quad \rightsquigarrow P=[\afa_1]\in\HH^2(A,A)=\PV^2\text{ the initial Poisson tensor}\\
      \frac{1}{2}[\afa_1,\afa_1]_G=-\p\afa_2 \quad\rightsquigarrow[P,P]_{SN}=0\\
      \cdots\cdots
    \end{array}
  \right.
$$

\textbf{Problem:} Connect
$$MC(\mfkg_{Hoch}\updot)\sim MC(\mfkg_{PV}\updot)$$
(equivalence relation???)


(Deformation functor)

$I=[0,1],t\in A$, $\Omg\updot(I)$- dg algebra, $\td_t$: de Rham differential

$$\Rightarrow \mfkg\ten\Omg\updot(I)$$
is a new dgLa, differential $\td+\td_t$,

\begin{definition}
$\afa_0,\afa_1\in MC(\mfkg)$ is said to be gauge equivalent,
if $\exists\beta\in MC(\mfkg\ten\Omg\updot(I))$, s.t.
$$\beta|_{t=0}=\afa_0$$
$$\beta|_{t=1}=\afa_1$$
\end{definition}

Remark:
\begin{eqnarray*}
\mfkg\ten\Omg\updot[I]&\xra{\text{Res $t=t_0$}}&\mfkg\\
MC&\mapsto& MC
\end{eqnarray*}

Explicitly, $\beta=\beta_0(t)+\beta_1(t)\td t$,
where $\beta_0(t)\in\mfkg^1$ and $\beta_1(t)\in\mfkg^0$.

$$(\td+\td_t)\beta+\frac{1}{2}[\beta,\beta]=0$$
$$\iff
  \left\{
    \begin{array}{l}
      \td\beta_0(t)+\frac{1}{2}[\beta_0(t),\beta_0(t)]=0
      \quad\rightsquigarrow I\to MC(\mfkg)
    \\
      \p_t\beta_0+\td\beta_1+[\beta_0,\beta_1]=0
      \quad\rightsquigarrow\text{variation of $\beta_0(t)$ is always exact at any $t$}
    \end{array}
  \right.
$$

\begin{definition}
$$Def(\mfkg)="MC(\mfkg)"/\sim$$
\end{definition}
deformation functor (local moduli)

Homework:$\sim$ us an equivalence relation(not easy!)

Homework:$\fai:\mfkg\to\hat{\mfkg}$ be a dgLa-homomorphism,
then it induces
$$\fai_*:Def(\mfkg)\to Def(\hat{\mfkg})$$

Geometry interpretation:
$MC(\mfkg)\rightsquigarrow$ zero locus of "vector field" $\delta$
$\delta^2=0$ implies "$\delta$ is tangent to MC".

Generalization:

\textbf{HW} 1. $\mfkg$ be an $L_\infty$-algebra,
Define $Def(\mfkg)=MC(\mfkg)/\sim$.
$("I_{dR}\to B_\mfkg")$

2.Let $\fai:\mfkg\to\widehat{\mfkg}$ be a homomorphism of $L_\infty$-algebra,
($\iff \fai^*: C\updot(\widehat{\mfkg})\to C\updot(\mfkg)$ map of dgLa)
$$\Rightarrow\fai_*:MC(\mfkg)\to MC(\widetilde{\mfkg})$$
$$Def(\mfkg)\to Def(\widetilde{\mfkg})$$

\begin{definition}
$$\fai:\mfkg\to\widehat{\mfkg}$$
a map of dgLa, is called a quasi-isomorphism, if
$$\fai:H\updot(\mfkg,\td)\to H\updot(\widetilde{\mfkg},\td)$$
is an isomorphism.
\end{definition}

\begin{thm}
If $\fai:\mfkg\to\widetilde{\mfkg}$ is a quasi-isomorphism of dgLa, then
$$\fai_*:Def(\mfkg)\to Def(\widetilde{\mfkg})$$
is an isomorphism.
\end{thm}
\begin{proof}
Next time
\end{proof}

\begin{thm}(Kontsevich)

$$Def(\mfkg_{Hoch}\updot)\cong Def(\mfkg_{\PV}\updot)$$
(Deformation quantization)
\end{thm}

\begin{definition}
Let $\fai:\mfkg\to\widetilde{\mfkg}$ be a morphism of $L_\infty$-algebras,
$$\fai^*:C\updot(\widetilde{\mfkg})\to C\updot(\mfkg)$$
$$\iff \Hom(\widetilde{\mfkg}^*,C\updot(\mfkg))$$
$$\iff \Hom(\Sym\updot(\mfkg[1]),\widehat{\mfkg}[1])$$
satisfying $L_\infty$-relations...

is called $L_\infty$-quasi isomorphism, if
$$\fai_1:\mfkg\to\widetilde{\mfkg}$$
induces an isomorphism of
$$H\updot(\mfkg,\ell_1)\to H\updot(\widetilde{\mfkg},\ell_1)$$
\end{definition}
In particular, if $\mfkg,\widetilde{\mfkg}$ are dgLa, then
dgLa quasi-isomorphism $\iff L_\infty$ quasi isomorphism..

\begin{thm}
If $\fai:\mfkg\to\widetilde{\mfkg}$ be an $L_\infty$ quasi-isomorphsim, then
$$\fai_*: Def(\mfkg)\cong Def(\widetilde{\mfkg})$$
\end{thm}

\begin{definition} A dgLa $(\mfkg,\td,[,])$ is called formal,
if $\exists L_\infty$-quasi isomorphism
$$\mfkg\to H\updot(\mfkg,\td)$$
\end{definition}
In particular, $Def(\mfkg)=Def(H\updot(\mfkg,\td))$

\begin{thm}(Formality Theorem, Kontsevich)

$\mfkg\updot_{Hoch}$ is formal,
and there is an explicit formula of $L_\infty$ quasi isomorphism
$\mfkg_{Hoch}\updot\to\mfkg_{\PV}\updot$

(Sum over Feymann Digrams)
\end{thm}
Reference: Kontsevich, Deformation Quantization on Poisson Manifold...

\section{形变理论}
\textbf{Deformation Theory}

Recall:$\mfkg$-dgLa,
$$MC(\mfkg)=\{\afa\in\mfkg_1|\td\afa+\frac{1}{2}[\afa,\afa]=0\}$$
$$Def(\mfkg)=MC(\mfkg)/\sim$$

$\mfkg$ is called formal, if
$$\mfkg\xra{L_{\infty}\text{quasi-isom}}H\updot(\mfkg,\td)$$

$$\mfkg_{Hoch}=(C\updot(A,A)[1],b,[-,-]_G)$$
$A=\bbC[x^i]$,
$$MC(\mfkg_{Hoch})=\text{"associated product"}$$
$$\mfkg_{\PV}=H\updot(\mfkg_{Hoch},b)$$
$$MC(\mfkg_{\PV})=\text{"Poisson bivector"}$$

Today:
$X$- is an algebraic scheme/vairety/manifold/topological space.
$$X=\Spec\qquad A=\bbC[x^1,...,x^n]/(f_1,...,f_m)$$
is the (common) "zero locus" of $f_1,f_2,...,f_m$.

\textbf{idea} Study $X$ by working at all $\{Y\to X\}$.
$Mor(-,X)$, $\underline{Sch}$- category of scheme.

$$Mor: \underline{Sch}\to\underline{Set}$$
$$X\to Mor(Y,X)$$

\begin{notation}
$R$ is a ring,
$$X(R):=Mor(\Spec R,X)\leftrightsquigarrow\Hom(\mcalO(X),R)$$
is called $R$-points of $X$.
\end{notation}

Example :$X=\Spec A$, then $X(R)=$solutions $\{f_i=0\}$ in $R$.

\begin{example}
$$X(\bbC)=\text{space of closed points of $X$}$$

$$X(\bbC[\veps]/\veps^2)=
\{\Spec(C[\veps]/\veps^2)\to X\}
=TX$$
tangent vectors of $X$...
\end{example}

\begin{notation}
$$\underline{Art}:\quad\text{Category of Artinian ring}$$
\end{notation}
Artin环就是“很肥的点”.....66666

\begin{example}
$A=\bbC[x]/x^{n+1}$ is an Artinain ring,with $\mfkm=(x)$.
\end{example}

Let $A$ be an Artinian ring, study $\Spec(A)\to X$,

\begin{definition}(Deformation functor)

The deformation functor of $X$ at $\mfkp$ is given by
$$Def_{X,\mfkp}:\underline{Art}\to\underline{Set}$$
$$A\mapsto X(A)$$
$$\Spec(\bbC)\mapsto\mfkp$$
\end{definition}

Functorial property: if $A\to B$ is a ring homomorphism in $\underline{Art}$.
(i.e. $\Spec B\to\Spec A$) ,then it induces
$$Def_{X,\mfkp}(A)\to Def_{X,\mfkp}(B)$$

\begin{thm}[Lifting Criterion for Smoothness]

$X$ is an (affine) scheme($X=\Spec R$),$\mfkp\in X$ a point.Then,
$\mfkp$ is smooth$\iff\forall$ surjective $A\to B$in $\underline{Art}$,
then $Def_{X,\mfkp}(A)\to Def_{X,\mfkp}(B)$ is surjective.
\end{thm}

$$
  \xymatrix{
    \Spec(B) \ar[r]^\fai  \ar[d]
   &X
  \\
    \Spec(A) \ar[ru]^{\exists\tilde{\fai}}
  }
$$

\begin{proof}
Difficult. Omit.
\end{proof}

\begin{example}
$X=\bbC^n$, $\mfkp=$origin, then
$$Def_{X,\mfkp}(A)=A\times A\times\cdots\times A$$
If $A\to B$ is surjective, then $A^n\to B$ is also surjective..
\end{example}

\begin{example}
$$X=\Spec(\bbC[x]/x^2)=\{x^2=0\}\subseteq\bbC$$
is a "fat point"...Consider
$$A=\bbC[\veps]/\veps^3\qquad B=\bbC[\veps]/\veps^2$$
$$A\surj B$$
$$\veps\to\veps$$
then???
\end{example}

Examples of moduli functor...

\begin{example}
Moduli space of genus $g$ curve $\mcalM_g(g>1)$.
$$Mor(X,\mcalM_g)=\text{a smooth family of genus $g$ curves over $X$}/\sim$$
\end{example}

\begin{example}Moduli space of vector bundle on $\Sgm$, $Bun(\Sgm)$.
$$Mor(X,Bun(\Sgm))=\{\text{Vector bundles on $X\times\Sgm$}\}/\sim$$
\end{example}

$\mfkg$-dgLa, a local "moduli"

(1):
$$MC:\underline{Art}\to\underline{Set}$$
$$(A,\mfkm)\to MC(\mfkg\ten \mfkm)
=\{\afa\in\mfkm\ten\mfkg_1|\td\afa+\frac{1}{2}[\afa,\afa]=0\}$$

(2):
$$Def:\underline{Art}\to\underline{Set}$$
$$(A,\mfkm)\to MC(\mfkg\ten \mfkm)/\sim$$

\begin{definition}
We define $B_\mfkg$ by
$$B_\mfkg(A):=Def(A)$$
for artinian ring $A$.
$$"\Spec A"\to B_\mfkg$$
$\iff$ an element of $Def(A)$.
\end{definition}

Recall the gauge equivalence: definie
$$\Omg_T\updot:=\bbC[t_0,t_1,\td t_0,\td t_1]/(t_0+t_1=1,\td t_0+\td t_1=0)$$
algebraic de Rham complex on $I=[0,1]$

\begin{definition}
$\afa_0,\afa_1\in MC(\mfkg\ten\mfkm)$ are called gauge equivalence, if
$\exists\beta\in MC(\mfkm\ten\mfkg\ten\Omg_I\updot)$ such that
$$\beta|_{t_0=0}=\afa_0$$
$$\beta|_{t_1=0}=\afa_1$$
then
$$Def(A)=MC(\mfkm\ten\mfkg)/\sim$$
(gauge equivalence)
\end{definition}

Tangent space of $B_\mfkg$(or $Def$ functor)

Tangent space$=Def(\bbC[\veps]/\veps^2)=MC(\bbC[\veps]/\veps^2)/\sim$

$$\afa\in MC(\bbC[\veps]/\veps^2)$$
$\afa\in\veps\ten\mfkg_1,\quad\td\afa+\frac{1}{2}[\afa,\afa]=0$.
$$\Rightarrow MC(\bbC[\veps]/\veps^2)\cong\veps\ten\ker(\mfkg_1\xra{\td}\mfkg_2)$$

Let
$$\beta\in MC(\veps\ten\mfkg\ten\Omg_I\updot)$$
$$\beta=\beta_0(t)+\beta_1(t)\td t$$
satisfies $\td\beta+\frac{1}{2}[\beta,\beta]=0$, i.e. $(\td_\mfkg+\td_t)\beta=0$
$$
  \iff
  \left\{
    \begin{array}{l}
      \td_\mfkg\beta_0=0\\
      \p_t\beta_0+\td_\mfkg\beta_1=0
    \end{array}
  \right.
$$
$$\beta|_{t=1}-\beta|_{t=0}=\int_{0}^{1}\p_t\beta_0\td t
=-\td_\mfkg\left(\int_{0}^{1}\beta_1\td t\right)$$

Claim: $\afa_0,\afa_1\in MC(\veps\ten\mfkg)$ is gauge equivalent
$\iff\afa_0-\afa_1$ is $\td_{\mfkg}$-exact.

\begin{cor}
$$Def(\bbC[\veps]/\veps^2)
\cong\frac{\ker(\mfkg_0\xra{\td_\mfkg}\mfkg_1)}
{\im(\mfkg_0\xra{\td_\mfkg}\mfkg_1)}=H^1(\mfkg,\td)$$
\end{cor}

\begin{example}$X$ is a complex manifold,
$$\mfkg=\Omg^{0,\bullet}(X,T^{1,0}_X),\pbar,[-,-]$$
$$Def(\mfkg)=\{\text{local moduli of complex structures on $X$}\}/\sim$$

$$T_{[x]\mcalM}=H^1(\Omg^{0,\bullet}(X,TX),\pbar)=H^1(X,T_X)$$
In particular, $X=\Sgm_g$ genus $g$ curve, then
$$\dim(H^1(\Sgm_g,T_{\Sgm_g}))=3g-3$$
\end{example}

%%%%%%%%%%%%%%2019.5.7%%%%%%%%%%%%%%%%%%%%%%

Last time: $\mfkg$-dgLa, MC-equation.

$$MC:\underline{Art}\to\underline{Set}$$
$$(A,\mfkm)\mapsto MC(\mfkm\ten\mfkg)$$

$$Def:\underline{Art}\to\underline{Set}$$
$$(A,\mfkm)\mapsto MC(\mfkm\ten\mfkg)/\sim$$
"$\Spec A\to B_{\mfkg}$".

Tangent space.  $Def(\bbC[\veps]/\veps^2)=H^1(\mfkg,\td)$.

\section{障碍理论}
Obstruction theory.

\begin{definition}
A functor $F:\underline{Art}\to\underline{Set}$ is called smooth,
if for any (exact) $A\to B\to B$ in $\underline{Art}$,
then $F(A)\to F(B)$ is also surjective.
\end{definition}
(Lifting criterion)

\begin{prop}
$F$ is smooth if and only if for any "square-zero extension"
$$0\to I\to \widetilde{R}\to R\to 0$$
where $R,\widetilde{R}$ in $\underline{Art}$,
$I$-ideal in $\widetilde{R}$ such that $I^2=0$.
Then we have surjective map $F(\widetilde{R})\to F(R)$.
\end{prop}

\begin{proof}
Exercise.
\end{proof}

\begin{definition}
An extension is called to minimal, if
$$0\to I\to \widetilde{R}\to R\to 0$$
and $I^2=0$ and $I\cong \bbC$.
\end{definition}

Remark: The previous proposition holds if
we replace "square extension" by minimal extension.

\begin{example}
$$A=\bbC[x]/x^{n+1}\surj B=\bbC[x]/x^2$$
\end{example}

\textbf{Obstruction theory}

Goal: Study lifting property.
$$0\to I\to\widetilde{R}\to R\to 0$$
$$0\to I\to\widetilde{\mfkm}_R\to \mfkm_R\to 0$$
is a square-zero extension, $I^2=0$, $I$ is $R$-module.
Text the surjectivity of
$Def(\widetilde{R})\to Def(R)$.

for $[\afa]\in Def(R)$, $\afa\in MC(R)$,
$\afa\in\mfkm_R\ten\mfkg_1$ satisfying
$$\td \afa+\frac{1}{2}[\afa,\afa]=0$$
Question: Can we find $\widetilde{\afa}\in\mfkm_{\widetilde{R}}\ten\mfkg$ s.t.

(1) $\widetilde{\afa}|_{\mfkm_R}=\afa$

(2) $\td\widetilde{\afa}+\frac{1}{2}[\widetilde{\afa},\widetilde{\afa}]=0$

\textbf{Step1} Choose any $\afatil$ satisfies (1).
Using the surjectivity of
$$0\to I\ten\mfkg\to\mfkm_{\Rtil}\ten\mfkg\to\mfkm_R\ten\mfkg\to 0$$
($\afatil\in \mfkm_{\Rtil}\ten\mfkg$ and $\afa\in \mfkm_R\ten\mfkg$)

\textbf{Step2} Let
$$\gma:=\td\afatil+\frac{1}{2}[\afatil,\afatil]\in\mfkm_R\ten\mfkg_2$$
Under
$$0\to I\ten\mfkg_2\to\mfkm_{\Rtil}\ten\mfkg_2\to\mfkm_R\ten\mfkg_2\to 0$$
then $\gma\in I\ten\mfkg_2$.

\textbf{Step3}Claim:
$$\td\gma+[\afa,\gma]=0$$
($\iff$ Bianchi identity)

\begin{proof}[proof of claim]
We have
$$\td\gma+[\afatil,\gma]=0$$
(Bianchi-identity)

$$
  =\td(\td\afatil+\frac{1}{2}[\afatil,\afatil])+[\afatil,\td\afatil]
   +\frac{1}{2}[\afatil,[\afatil,\afatil]]
  =0
$$

Now using $I^2=0$,
$\mfkm_{\Rtil}\cdot I=\mfkm_R\cdot I$.So,
$$\td\gma+[\afa,\gma]=0$$
\end{proof}

\textbf{Step4} $\gma$ defines an element in
$$[\gma]\in H^2(I\ten\mfkg,\td+[\afa,-])$$
Check: $(\td+[\afa,-])^2=0$.

\textbf{Step5} Show $[\gma]$ is independent of the choice of $\afatil$.
If we choose another lifting
$$
  0\to I\ten\mfkg_1\to\mfkm_{\Rtil}\ten\mfkg_1
  \to\mfkm_R\ten\mfkg_1\to 0
$$
$\afatil,\afatil'\to\afa$.
$$\afatil'=\afatil+\beta,\quad\beta\in I\ten\mfkg_1$$

$$\gma':=\td\afatil+\frac{1}{2}[\afatil',\afatil']$$
$$=\td(\afatil+\beta)+\frac{1}{2}[\afatil+\beta,\afatil+\beta]$$
$$=\td\afatil+\frac{1}{2}[\afatil,\afatil]+\td\beta+[\afatil,\beta]+\frac{1}{2}[\beta,\beta]$$
$$=\gma+\td\beta+[\afa,\beta]$$

$\Rightarrow \gma'-\gma$ is $\td+[\afa,-]$exact.
$[\gma']=[\gma]$ in $H^2(I\ten\mfkg,\td+[\afa,-])$.

\textbf{Step 6} $\afa$ lifts to a solution of $MC(\mfkm_{\Rtil}\ten\mfkg)$
$\iff[\gma]=0$ in $H^2(I\ten\mfkg,\td+[\afa,-])$.

In fact, $(\Rightarrow)$ Let $\afatil$ be a lifting, then
$\gma=\td\afatil+\frac{1}{2}[\afatil,\afatil]=0$, $[\gma]=0$.

$\Leftarrow$ If $[\gma=0]$, then $\exists\beta\in I\ten\mfkg_1$ s.t.
$$\gma+\td\beta+[\afa,\beta]=0$$
choose $\afatil=\afa+\beta$, then $\afatil$ satisfies $MC$
by the previous computation.

\textbf{Step 7} Consider minimal extension
$$0\to\bbC\to \Rtil\to R\to 0$$
$\afa\in MC(R)$. Then $\exists Ob(\afa)\in H^2(I\ten\mfkg,\td+[\afa,-])=H^2(\mfkg,\td)$
such that $Ob(\afa)=0$ in $H^2(\mfkg,\td)\iff\afa$ lifts to $MC(\Rtil)$.

\textbf{In Summary} $\mfkg$-dgLa.

(1)Tangent space $=H^1(\mfkg,\td)$.

(2)Obstruction space $=H^2(\mfkg,\td)$. then $Def$ is smooth.

\begin{example}$X$-complex manifold,
$\mcalM$- moduli space of complex structure on $X$.
$$\mfkg=\Omg^{0,\bullet}(X,TX),\pbar,[-,-]$$
Tangent space $\cong H^1(\mfkg)=H^1(X,TX)$

Obstruction space $\cong H^2(X,TX)$.

Kodaire-Spencer: $H^2(X,TX)=0\Rightarrow \mcalM$ is smooth at $[x]\in\mcalM$.
\end{example}

In particular, $X=\Sgm_g$ genus $g>2$ curve.
$$H^2(\Sgm_g,T_{\Sgm_g})=0$$
$$H^2(\Sgm_g,T_{\Sgm_g})=\text{Tangen space of $\mcalM,g$}$$
$$H^0(\Sgm_g,T_{\Sgm_g})=0$$

$\dim_{\bbC}\mcalM_g=3g-3\quad\text{by Riemann-Roch}$

Remark: Kuranishi Theory. Locally there exists a non-linear (analytic) map
$$H^1\xra{\fai} H^2$$
s.t. $\mcalM\cong\fai^{-1}(0)$.

\textbf{Deformation-Obstruction theory for QME}
Recall: $(\mcalA,\yc)$- GBV-algebra, $S_0\in\mcalA_0$
is said to satisfy CME if
$$\{S_0,S_0\}=0$$

$S\in\mcalA_0\fps{\hbar}$ is said to satisfies QME, if
$$\hbar\yc S+\frac{1}{2}\{S,S\}=0$$

Question: Given $S_0$, how to quantize to $S$?

idea: Solve order by order in $\hbar$, using obstruction theory.

$\hbar$-order: Find $S_1$.
$$\yc S_0+\{S_0,S_1\}=0$$
Let $\delta=\{S_0,-\}$, $\delta^2=0$.

(a)$\yc S_0$ is $\delta$-closed:
$$\{S_0,\yc S_0\}=\frac{1}{2}\yc\{S_0,S_0\}=0$$

(b)Solvability of $S_1\iff[\yc S_0]=0$ in $\delta$-cohomology $H^1(\mcalA,\delta)$.
(BRST-cohomology)

Assume we have find
$$S^{\leq k}=S_0+\hbar S_1+\cdots+\hbar^k S_k$$
satisfying
$$\hbar\yc S^{\leq k}+\frac{1}{2}\{S^{\leq k},S^{\leq k}\}=0\mod\hbar^{k+1}$$

Question: find $S_{k+1}$ such that
$$\hbar\yc S^{\leq k+1}+
\frac{1}{2}\{S^{\leq k+1},S^{\leq k+1}\}=0\mod\hbar^{k+2}$$

Let
$$\hbar\yc S^{\leq k}+\frac{1}{2}\{S^{\leq k},S^{\leq k}\}
=\hbar^{k+1}\mcalO_{k+1}+O(\hbar^{k+2})$$
find $S_{k+1}\hbar^{k+1}$ to kill $\mcalO_{k+1}$.

(a) $\mcalO_{k+1}$ is $\delta$-closed, $\{S_0,\mcalO_{k+1}\}=0$.

(b) Solvability of $S_{k+1}\iff \mcalO_{k+1}$ is $\delta$-exact.

(Check.)

(Hint: QME
$$\hbar \yc e^{S^{\leq k}/\hbar}
=(\hbar^k\mcalO_{k+1}+O(\hbar^{k+1}))e^{S^{\leq k}/\hbar}$$
then apply $\hbar\yc$ again.
)


%%%%%%%%%%%%%%%%%%%%%%%%%%%%%%%%%%%%%%%%%
%2019.5.13

\section{$L_\infty$-quasi-isomorphism}
Recall: 
$(\mfkg,\td,[,])$ be a dgLa, $\rightsquigarrow$ a functor
$$\underline{Art}\to \underline{Set}$$
$$(A,\mfkm)\mapsto MC(A):=
\Bigset{\afa\in\mfkm\ten\mfkg^1}{\td\afa+\frac{1}{2}[\afa,\afa]=0}$$

Deformation functor
$$\underline{Art}\to\underline{Set}$$
$$(A,\mfkm)\mapsto MC(A)/\text{Gauge equivalence}$$

Tangent space: $Def(\bbC[\veps]/\veps^2)=H^1(\mfkg,\td)$.

Obstruction space: $H^2(\mfkg,\td)$.

For each minimal extension of Artinian ring $\widetilde{A},A$
$$0\to\bbC\to\widetilde{A}\to A\to 0$$
$$Def(\widetilde{A})\to Def(A)\to H^2(\mfkg,\td)$$
$[\afa]\in Def(A)$ lifts to $[\afatil]\in Def(\widetilde{A})
\iff Ob([\afa])=0$.

Remark on gauge equivalence:
$$Def(A):=MC(A)/\sim$$
($(A,\mfkm)$ is an Artinian ring). $\afa\in MC(A)$,
$\td\afa+\frac{1}{2}[\afa,\afa]$.

Given $\phi \in \mfkm\ten\mfkg^0$,consider the transformation
$$\td+\afa\mapsto \td+\afatil=e^{\phi}(\td+\afa)e^{-\phi}=e^{\ad\phi}(\td+\afa)$$
$$\afatil=e^\phi\afa e^{-\phi}+(e^\phi\td e^{-\phi}-\td)$$
$$
  =e^{\ad\phi}\afa+
  \left(
    \frac{e^{\ad\phi}-1}{\ad\phi}
    [\phi,\td]
  \right)
$$
Precisely, define the Gauge transformation
$$\afa\to\afatil=e^{\ad\phi}\afa-\frac{e^{\ad\phi}-1}{\ad\phi}\td\phi$$

\begin{prop}(Definition)

Let $\afa,\afatil$ are gauge equivalent in $MC(A)$ if and only if
$\exists\phi\in\mfkm\ten\mfkg^0$ such that
$$\afatil=e^{\ad\phi}\afa-\frac{e^{\ad\phi}-1}{\ad\phi}\td\phi$$
\end{prop}

\begin{proof}
Not too hard. Omit.
\end{proof}

\begin{example}$A=\bbC[\veps]/\veps^2$,$\afa\in MC(A)$,
$\afa\in\veps\ten \mfkg^1,\td\afa=0$,$\phi\in\veps\ten\mfkg^0$. Then
$$e^{\ad\phi}\afa-\frac{e^{\ad\phi}-1}{\ad\phi}\td\phi
=\afa-\td\phi$$
So,
$$
  Def(\bbC[\veps]/\veps^2)
=MC(\bbC[\veps]/\veps^2)/\text{Gauge transformation}
=H^1(\mfkg,\td)
$$
\end{example}

Recall:$\mfkg$ is called an $L_\infty$-algebra, 
if $\exists\delta\in Vect_0(\mfkg[1])$
("$0$"means "vanishing at zero"),
$$\delta=\sum_{k\geq 1}\delta_k$$
$$\delta_k\in Hom(\Sym^k(\mfkg[1]),\mfkg[1])$$
("or" $\Sym^k(\mfkg[1]^*)\ten\mfkg[1]$)such that 
$\deg\delta=1$ and $\delta^2=0$.

$$\delta_k\rightsquigarrow
\ell_k:\wedgeform{k}\to\mfkg$$
$\deg\ell_k=2-k$ such that ...

In particular, $(\mfkg,\ell_1)$ is a cochain complex.

\begin{definition}
$V$ is called (shifted) $L_\infty$-algebra
($SL_\infty$-algebra) if $\exists \delta\in Vect_0(V)$
such that $\deg\delta=1,\delta^2=0$.

$\mfkg$ is $L_\infty$-algebra$\iff\mfkg[1]$ is $SL_{\infty}$-algebra.
\end{definition}

\begin{definition}
Let $(V,\delta^V)$ and $(W,\delta^W)$ be two $SL_\infty$ algebras.
An $L_\infty$-morphism is given by a formal map
$\fai: V\to W$ such that
$$\fai_*(\delta^V)|_V=\delta^W|_{\fai(V)}$$
\end{definition}

Explicitly, $\fai:V\to W$
$$\iff \fai^*:\mcalO_0(W)\to\mcalO_0(V)$$
$\mcalO_0(W):=\prod\limits_{k\geq 1}\Sym^k(W^*)$,
"$\mcalO_0$" means "vanishing at zero".
$$\iff\fai^*: W^*\xra{\text{linear}}\prod_{k\geq 1}\Sym^k(V^*)$$
$$\iff\fai\in \Hom(\prod_{k\geq 1}\Sym^k(V),W)$$
$\fai=\sum\limits_{k\geq 1}\fai_k$, 
$\fai_k\in\Hom(\Sym^k(V),W)$.

%%%%%%%%%%%%%%%本课程无期末考试%%%%%%%%%%%%%%%%%%

$$\fai_*(\delta^V)\in \Hom(\prod_{k\geq 1}\Sym^k(V),W)$$
(vector field,"push forward")

$$\delta^W|_{\fai(V)}=\cdots$$
$\fai$ is an $L_\infty$-morphism if 
(局部坐标具体表达式,挺长的)

At linearilized level, 
$$
  V\xra{\delta_1^V}V\xra{\fai_1}W= V\xra{\fai_1}W\xra{\delta_1^W}W
$$
i.e. $\fai_1:(V,\delta_1^V)\to (W,\delta_1^W)$ is a cochain map.

\begin{definition}
$\fai:(V,\delta^V)\to(W,\delta^W)$ $L_\infty$-morphism,

(1)If $\fai_1:V\to W$ is an isomorphism, then $\fai$ is called an 
$L_\infty$-isomorphism.

($\Rightarrow\fai^{-1}$ exists, also be an $L_\infty$-morphism)

(2) If $\fai:V\to W$ induces an isomorphism 
$$H(V,\delta_1^V)\to H(W,\delta_1^W)$$
$\fai$ is called an $L_\infty$-quasi isomorphism.
\end{definition}

\begin{thm}
If $\fai:V\to W$ is an $L_\infty$ quasi-isomorphism, then:

(1) $\exists\eta:W\to V$ be an $L_\infty$-isomorphism such that
$$\eta_1:H\updot(W,\delta_1^W)\to H\updot(V,\delta_1^V)$$
is the inverse of $\fai_1:H\updot(V,\delta_1^V)\to H\updot(W,\delta_1^W)$.

(2)$\fai$ induces an isomorphism of functors 
$$Def_V\to Def_W$$
\end{thm}

(We first proof a weaker version)

\begin{definition}
$V$ is a $SL_\infty$-algebra, is called minimal, if
$\delta_1^V=0$.
\end{definition}

\begin{prop}
Let $\mfkg$ be a dgLa, Let 
$$\fai_1:H=H\updot(\mfkg,\td)\to\ker\td\subseteq\mfkg$$
be a splitting of the surjective map
$$\ker\td\to H\updot(\td)=\frac{\ker\td}{\im\td}$$
Then $\fai_1$ canonically determines a pair $\delta^H,\fai$,s.t.

(1) $\delta^H$ defines an $SL_\infty$-algebra on $H[1]$

(2) $\fai:H[1]\to\mfkg[1]$ is an $L_\infty$-morphsim
such that the linear term $\fai_1:H[1]\to\mfkg[1]$ is the splitting above.

In particular, $\fai$ is an $L_\infty$ quasi-isomorphism.
\end{prop}

\begin{proof}
Consider 
$$
  \mcalE=\prod_{k\geq 1}\Sym^k(H[1]^*)\ten H[1]
         \oplus\prod_{k\geq 1}\Sym^k(H[1]^*)\ten\mfkg
$$
(two parts, $Vect_0(H[1])$ and $\mcalO_0(H[1])\ten\mfkg$)

Claim: $\mcalE$ is naturally a dgLa.

differential:$\td:\mcalO_0(H[1])\ten\mfkg\to\mcalO_0(H[1])\ten\mfkg$
is the $\mcalO_0(H[1])$-linear extension of $\td:\mfkg\to\mfkg$.

Bracket: 
$[Vect,Vect]$ is the bracket on vector fields;

$[Vect,\mcalO_0(H[1])\ten\mfkg]$ is the action 
$Vect\curvearrowright\mcalO_0$.

$[\mcalO\ten\mfkg,\mcalO\ten\mfkg]$ is the $\mcalO_0(H)$-linear extension of 
$[,]:\mfkg\times \mfkg\to\mfkg$.

(Check:$\mcalE$ is a dgLa)

$MC(\mcalE)\ni \delta^H\oplus\fai$,
$$\delta^H\in Vect_0(H[1])\,\,\deg\delta^H=1$$
$\fai\in\mcalO_0(H[1])\ten\mfkg[1]$,$\deg\fai=0$,
satisfying 
$$\td(\delta^H+\fai)+\frac{1}{2}[\delta^H+\fai,\delta^H+\fai]=0$$

$$[\delta^H,\delta^H=0]\eqno{(1)}$$
$$\td\fai+\frac{1}{2}[\fai,\fai]+[\delta^H,\fai]=0\eqno{(2)}$$

$(1)$ means that $\delta^H$ defines an $L_\infty$-structure on $H$,

$(2)$ $\fai$ defines an $L_\infty$-morphism $H[1]\to\mfkg[1]$.

Introduce a formal parameter $\hbar$,
$$\delta^H=\sum_{k\geq 1}\delta_k^H\quad
\fai=\sum_{k\geq 1}\fai_k$$

redefine

$$\delta_\hbar^H=\sum_{k\geq 1}\delta_k^H\hbar^{k-1}$$
$$\fai_\hbar=\sum_{k\geq 1}\fai_k\hbar^{k-1}$$

$$
  \iff
  \td\fai_\hbar+\frac{1}{2}[\delta_\hbar^H,\delta_\hbar^H]
  +[\delta_\hbar^H,\fai_\hbar]+\frac{\hbar}{2}[\fai_\hbar,\fai_\hbar]=0
$$

Initial data:
$$\lim_{\hbar\to 0}(\delta_\hbar^H)
=\delta_1^H=0$$
$$\lim_{\hbar\to 0}(\fai_\hbar)=\fai_1:H[1]\to\mfkg[1]$$
given.

Goal: Construct higher orders in $\hbar$.

Exercise: Initial terms defines a differential 
$$\td+[\fai,-]$$
on $\mcalE$.

(Obstruction theory)tangent space:
$H^1(\mcalE,\td+[\fai,-])$, and obstruction $H^2(\mcalE,\td+[\fai_1,-])$.

Upshot: $H\updot(\mcalE,\td+[\fai_1,-])=0$.

e.g. Use special sequence, compute $\td$-cohomology first:
$$\mcalO_0(H[1])\ten H[1]\to \mcalO_0(H[1])\ten H$$
then, $[\fai_1,-]$-cohomology are all vanishing.
\end{proof}







